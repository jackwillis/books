\chapter{Introduction to the 2018 edition}

\textsc{In Defense of Marxism} is about how major geopolitical events affect every aspect of society, including revolutionary parties. In the process of trying to catch up major events, it’s inevitable the organization falls into some degree of crisis. In this process, small differences over seemingly abstract questions can become the center of massive debates, which stretch into deeper questions of method and even how a revolutionary party organizes itself. Small engineering flaws deep within a ship’s hull may only become apparent in hurricane conditions.

The crisis documented in \emph{In Defense of Marxism} began with a political debate in the Socialist Workers Party in the United States at the beginning of World~War~II. The \SWP\ was the largest section in the newly found Fourth International, launched less than a year before. Many members were more familiar with the parts of the \emph{Transitional Program} that related to immediate class struggles in the U.S.\ than the more abstract parts like ``The \USSR\ and Problems of the Transitional Epoch,'' which characterized the Soviet Union as a “degenerated workers state”, and a concrete program to overthrow the Stalinist bureaucracy.

This somewhat abstract question became the center of a major political debate when Stalin signed a nonaggression pact with Hitler, annexed Poland and invaded Finland. Members of the \SWP\ justifiably freaked out. How could a workers’ state, degenerated or otherwise, like the Soviet Union, have expansionist foreign policy along the lines of imperialist capitalist countries, and at the same time have a totalitarian dictator, Stalin, who more closely resembled the regime of Hitler’s fascist Germany? A “minority faction” developed within the \SWP, lead by Max Shachtman, James Burnham, and Martin Abern, along the lines that these events surely show the \USSR\ is some new beast: \emph{anything but} a “degenerated workers’ state.” But beyond this there was not a new theory proposed for what is was.

From today’s vantage point, this can seem like a long-obsolete debate, lacking in contemporary relevance. After all, the \USSR\ no longer exists. There may still be some relevance in terms of how we approach Cuba or similar countries, but Stalinism doesn’t pose the same global concern as it did during World War II. However, the debate about the class character of the \USSR\ was the culmination of a wider ideological debate that had been raging throughout the left in the 1930s. A debate between Marxism and pragmatism, between the workers’ movement and the middle class intelligentsia. This ideological debate is still relevant today, and understanding the debate on the \USSR\ can help clarify these modern debates.
\noclub

Starting in 1934, the United States saw a revival of the workers’ movement and the rest of the world would soon enter into a period of revolutionary situations, especially in Spain. The United States saw particular growth in the American Trotskyist movement, which had played a leading role in the Minneapolis Teamsters’ strike. This global radicalization soon extended beyond the working class itself, attracting a layer of middle class intellectuals to revolutionary politics. Given the strength of the Trotskyist movement in the United States, a number of these intellectuals turned to the Trotskyism. Some of them, like George Novack, Felix Morrow, James Burnham, and Dwight MacDonald, explicitly joined the Trotskyist movement. Others, like Max Eastman and Sidney Hook, remained outside of any Trotskyist organization, while publicly sympathizing with their struggle.

Most of these intellectuals, however, brought in ideological baggage from their social milieux. Philosophically, the American intelligentsia was dominated by the ideology of liberal pragmatism associated with the philosopher John Dewey. Pragmatism, like Marxism, prized the role of experience as superior to passive observation or detached thought. However, they differed on other aspects of the relation between theory to practice. From a Marxist perspective, theory is the generalization of experience and this collective experience of generations of class struggle forms a necessary guide to action. From a pragmatist perspective, theory and practice are at odds with each other, and every new situation had to be approached as a blank slate.

Throughout the 1930s, debates raged on various issues between the intellectual fellow-travelers of the Trotskyist movement and the proletarian activist base. Sometimes they were on theoretical issues (the labor theory of value, the compatibility of ethics with materialism, etc.). Sometimes they were on more directly political issues (the tactics in the Spanish Revolution, the relation of Bolshevism to Stalinism, the right of free speech for fascists, etc.). For the most part, these remained isolated debates which didn’t preclude collaboration in the wider struggles on the ground, and they didn’t always have the same players on each side. But by the end of the 1930s, with the Moscow show trials, the defeat of the Spanish Revolution and the start of World War II, these isolated debates hardened into distinct ideological camps as the intellectuals began to abandon their radicalism. By the start of the Cold War, most of those intellectuals would become hardened anti-Communists, with many forming the basis of the neoconservative movement. The debate broke out within the ranks of the \SWP\ on the class character of the \USSR\ was the reflection of that wider debated within the ranks of the Trotskyist movement.

The minority put forward their take on the \USSR\ of the Soviet Union on a pragmatist basis. Burnham explicitly rejected the validity of classifying the \USSR\ as any form of workers’ state. Shachtman defended the classification of the \USSR\ as a degenerated workers’ state, but argued that this class characterization had no bearing on political program. Instead, he argued that it was only the “character of the war” that mattered, ignoring that the class character of the states involved determines the character of the war.

Trotsky realized the heart of the problem was that the minority in the \SWP, despite claiming to be more interested in “concrete” issues, were starting from \emph{idealism}, a petty bourgeois conception of society, rather than \emph{dialectical materialism}, a working class conception of society. Idealism would start with the concept of “totalitarianism” and draw comparisons between Hitler’s Germany and Stalin’s Russia on that basis. Dialectical materialism starts with the economic structure of society as key to understanding its political and social evolution, and in terms of a program for a revolutionary party, starts with the tasks of the working class.

The “dialectical” part of dialectical materialism stems from the idea that concepts and categories contain internal contradictions, and aren’t fixed. Undialectical views treat concepts like “capitalism, freedom, morals, workers’ state, etc.” as fixed entities transcending history and with rigid definitions. And when those rigid definitions can’t be met in every detail, they are discarded altogether. Because the \USSR\ didn’t fit the Platonic ideal of what a workers’ state should be, the minority in the \SWP\ concluded that the category didn’t apply, or was irrelevant. Trotsky argued that a category like “workers’ state” only applied within certain limits and the task was to determine whether the quantitative changes in the \USSR\ resulted in a qualitative change in its class character. The philosophical pragmatism of the \SWP\ minority lead them to dismiss important categories like “workers’ state” as insufficiently concrete, while putting forward a crude, idealistic conception, and decidedly non-concrete, conception of “totalitarianism”.

While acknowledging that the \USSR’s policies were a terrible new chapter in an already tragic book, Trotsky pushed back that the internal debate is not helpful to understand the tasks of the working class: the starting point for a party’s revolutionary program. The German working class’ revolutionary task was to overthrow Hitler \emph{and} the German Capitalist class that bankrolled Hitler, and implement a planned economy in Germany. The Russian working class’ political tasks are to overthrow the Stalinist bureaucracy, but \emph{preserve} the planned economy won during the Russian Revolution of 1917. Moreover, with the possibility of war between the \USSR\ and the capitalist world, the struggle to defend the \USSR\ from capitalist restoration could provide the spark moving the Soviet working class into action necessary for the political revolution against the bureaucracy.

In practice, this related to a question of how to apply the transitional method to the Stalinism world. The \SWP\ minority countered Trotsky’s views with a slogan of “simultaneous insurrection against Stalin and Hitler”. Trotsky didn’t oppose such an insurrection, should it develop. But an insurrection has to be based on mass support, and the \SWP\ minority dodged the question of how to win over the masses in the Stalinist world when faced with a threat from the capitalist world. For Trotsky, being the most resolute fighters in defense of the \USSR\ was the key to winning the Soviet masses against the bureaucracy, just as being the most resolute fighters for reforms under capitalism was the key to winning the working class as a whole against capitalism.

The question of how to define the \USSR\ spilled over into a discussion about the role of theory generally: is dialectical materialism important for answering practical questions facing the working class? This more theoretical occurred on similar lines to the more “concrete” debate on the \USSR. Burnham explicitly rejected dialectical materialism as no different than religion. Shachtman accepted the legitimacy of dialectical materialism but insisted that it had no relevance to practical politics.
\nowidow

A similar theme erupted over questions of how the \SWP\ is organized. The \emph{ideal} of democratic debate is abstracted over the question of whether or not the internal debate should be published in the party’s public newspaper. The \emph{ideal} of democracy is abstracted over the question of whether delegates to national conventions should be bound by local referendums in branches, etc. These questions on the technical aspects of party organization spilled over into wider questions of the \emph{class composition} and orientation of the party. Although there were workers in the minority and intellectuals in the majority, the minority was overwhelmingly dominated by intellectuals who had joined through the youth wing of the Socialist Party, and was politically orienting to the right-drifting intellectuals. The majority, on the other hand, was much more thoroughly rooted in the labor movement, and was oriented towards the working-class base of the mass Stalinist parties.

Throughout the process Trotsky tried to patiently explain the ideas of genuine Marxism through letters and articles to different members of the \SWP. He was equally willing to weigh in on the debate as he was willing to defend the rights of the minority faction to express their ideas without fear of expulsions. In what becomes a sharp debate, he appealed for maintaining a sense of proportion, working to avoid “unprincipled” splits over secondary questions to the problems of world revolution.

In Defense of Marxism is Trotsky’s plea that clarifying ideas is the the crucial step to defeating the powerful enemies that surround the Fourth International on all sides: reformists, capitalists, fascists and Stalinists. At points you can see his urgency. Trotsky’s last letter that concludes the book was written four days before he was murdered by a Stalinist agent in Mexico City.

Today, there are still questions about the character of countries that call themselves socialist, although the exact situation is different from during World War II. But the wider methodological issues that arose from this debate are still relevant today: What is the role of philosophy to political struggle? What is the role of the class character of the state in determining political program? How do Marxists respond to war? How should a revolutionary party relate to the middle class intelligentsia? How should a revolutionary party handle internal debate? And, most importantly, how can we adapt to new and complex situations while still retaining what is right about our past experience? All these questions retain their importance today and \emph{In Defense of Marxism} is immensely useful in dealing with these questions.

\signed{\textsc{George Martin Fell Brown}}