\chapterdated{The \USSR\ in War}{September 25, 1939}
\markright{The \fakesc{USSR} in War}

\subsection*{The German-Soviet Pact and the Character of the \USSR}

\textsc{Is it possible} after the conclusion of the German Soviet Pact to consider the \USSR\ a workers’ state? The future of the Soviet State has again and again aroused discussion in our midst. Small wonder; we have before us the first experiment in the workers’ state in history. Never before and nowhere else has this phenomenon been available for analysis. In the question of the social character of the \USSR, mistakes commonly flow, as we have previously stated, from replacing the historical fact by the programmatic norm. Concrete fact departs from the norm. This does not signify, however, that it has overthrown the norm; on the contrary, it has reaffirmed it, from the negative side. The degeneration of the first workers’ state, ascertained and explained by us, has only the more graphically shown what the workers’ state should be, what it could and would be under certain historical conditions. The contradiction between the concrete fact and the norm constrains us not to reject the norm but, on the contrary, to fight for it by means of the revolutionary road. The program of the approaching revolution in the \USSR\ is determined on the one hand by our appraisal of the \USSR, as an objective historical \emph{fact}, and on the other hand, by a \emph{norm} of the workers’ state. We do not say: “Everything is lost, we must begin all over again.” We clearly indicate those elements of the workers’ state which at the given stage can be salvaged, preserved, and further developed.

Those who seek nowadays to prove that the Soviet-German pact chan\-ges our appraisal of the Soviet State take their stand, in essence, on the position of the Comintern---to put it more correctly, on yesterday’s position of the Comintern. According to this logic, the historical mission of the workers’ state is the struggle for imperialist democracy. The “betrayal” of the democracies in favor of fascism divests the \USSR\ of its being considered a workers’ state. In point of fact, the signing of the treaty with Hitler supplies only an extra gauge with which to measure the degree of degeneration of the Soviet bureaucracy, and its contempt for the international working class, including the Comintern, but it does not provide any basis whatsoever for a reevaluation of the sociological appraisal of the \USSR.

\subsection*{Are the Differences Political or Terminological?}

Let us begin by posing the question of the nature of the Soviet state not on the abstract sociological plane but on the plane of concrete political tasks. Let us concede for the moment that the bureaucracy is a new “class” and that the present regime in the \USSR\ is a special system of class exploitation. What new political conclusions follow for us from these definitions? The Fourth International long ago recognized the necessity of overthrowing the bureaucracy by means of a revolutionary uprising of the toilers. Nothing else is proposed or can be proposed by those who proclaim the bureaucracy to be an exploiting “class.” The goal to be attained by the overthrow of the bureaucracy is the reestablishment of the rule of the Soviets, expelling from them the present bureaucracy. Nothing different can be proposed or is proposed by the Leftist critics.\footnote{We recollect that some of those comrades who are inclined to consider the bureaucracy a new class, at the same time objected strenuously to the exclusion of the bureaucracy from the Soviets.} It is the task of the regenerated Soviets to collaborate with the world revolution and the building of a socialist society. The overthrow of the bureaucracy therefore presupposes the preservation of state property and of planned economy. Herein is the nub of the whole problem.

Needless to say, the distribution of productive forces among the various branches of economy and generally the entire content of the plan will be drastically changed when this plan is determined by the interests not of the bureaucracy but of the producers themselves. But inasmuch as the question of overthrowing the parasitic oligarchy still remains linked with that of preserving the nationalized (state) property, we called the future revolution \emph{political}. Certain of our critics (Ciliga, Bruno, and others) want, come what may, to call the future revolution \emph{social}. Let us grant this definition. What does it alter in essence? To those tasks of the revolution which we have enumerated it adds nothing whatsoever.

Our critics as a rule take the facts as we long ago established them. They add absolutely nothing essential to the appraisal either of the position of the bureaucracy and the toilers, or of the role of the Kremlin on the international arena. In all these spheres, not only do they fail to challenge our analysis, but on the contrary they base themselves completely upon it and even restrict themselves entirely to it. The sole accusation they bring against us is that we do not draw the necessary “conclusions.” Upon analysis it turns out, however, that these conclusions are of a purely terminological character. Our critics refuse to call the degenerated workers’ state a `workers’ state.' They demand that the totalitarian bureaucracy be called a ruling class. The revolution against this bureaucracy they propose to consider not political but social. Were we to make them these terminological concessions, we would place our critics in a very difficult position, inasmuch as they themselves would not know what to do with their purely verbal victory.

\subsection*{Let Us Check Ourselves Once Again}

It would therefore be a piece of monstrous nonsense to split with comrades who on the question of the sociological nature of the \USSR\ have an opinion different from ours, insofar as they solidarize with us in regard to the political tasks. But on the other hand, it would be blindness on our part to ignore purely theoretical and even terminological differences, because in the course of further development they may acquire flesh and blood and lead to diametrically opposite political conclusions. Just as a tidy housewife never permits an accumulation of cobwebs and garbage, just so a revolutionary party cannot tolerate lack of clarity, confusion and equivocation. Our house must be kept clean!

Let me recall for the sake of illustration, the question of Thermidor. For a long time we asserted that Thermidor in the \USSR\ was only being prepared but had not yet been consummated. Later, investing the analogy to Thermidor with a more precise and well deliberated character, we came to the conclusion that Thermidor had already taken place long ago. This open rectification of our own mistake did not introduce the slightest consternation in our ranks. Why? Because the \emph{essence} of the processes in the Soviet Union was appraised identically by all of us, as we jointly studied day by day the growth of reaction. For us it was only a question of rendering more precise an historical analogy, nothing more. I hope that still today despite the attempt of some comrades to uncover differences on the question of the “defense of the \USSR”---with which we shall deal presently---we shall succeed by means of simply rendering our own ideas more precise to preserve unanimity on the basis of the program of the Fourth International.

\subsection*{Is It a Cancerous Growth or a New Organ?}

Our critics have more than once argued that the present Soviet bureaucracy bears very little resemblance to either the bourgeois or labor bureaucracy in capitalist society; that to a far greater degree than fascist bureaucracy it represents a new and much more powerful social formation. This is quite correct and we have never closed our eyes to it. But if we consider the Soviet bureaucracy a “class,” then we are compelled to state immediately that this class does not at all resemble any of those propertied classes known to us in the past: our gain consequently is not great. We frequently call the Soviet bureaucracy a caste, underscoring thereby its shut in character, its arbitrary rule, and the haughtiness of the ruling stratum who consider that their progenitors issued from the divine lips of Brahma whereas the popular masses originated from the grosser portions of his anatomy. But even this definition does not of course possess a strictly scientific character. Its relative superiority lies in this, that the make shift character of the term is clear to everybody, since it would enter nobody’s mind to identify the Moscow oligarchy with the Hindu caste of Brahmins. The old sociological terminology did not and could not prepare a name for a new social event which is in process of evolution (degeneration) and which has not assumed stable forms. All of us, however, continue to call the Soviet bureaucracy a bureaucracy, not being unmindful of its historical peculiarities. In our opinion this should suffice for the time being.

Scientifically and politically---and not purely terminologically---the question poses itself as follows: does the bureaucracy represent a temporary growth on a social organism or has this growth already become transformed into an historically indispensable organ? Social excrescences can be the product of an “accidental” (\ie\ temporary and extraordinary) enmeshing of historical circumstances. A social organ (and such is every class, including an exploiting class) can take shape only as a result of the deeply rooted inner needs of production itself. If we do not answer this question, then the entire controversy will degenerate into sterile toying with words.
\noclub

\subsection*{The Early Degeneration of the Bureaucracy}

The historical justification for every ruling class consisted in this---that the system of exploitation it headed raised the development of the productive forces to a new level. Beyond the shadow of a doubt, the Soviet regime gave a mighty impulse to economy. But the source of this impulse was the nationalization of the means of production and the planned beginnings, and by no means the fact that the bureaucracy usurped command over the economy. On the contrary, bureaucratism, as a system, became the worst brake on the technical and cultural development of the country. This was veiled for a certain time by the fact that Soviet economy was occupied for two decades with transplanting and assimilating the technology and organization of production in advanced capitalist countries. The period of borrowing and imitation still could, for better or for worse, be accommodated to bureaucratic automatism, \ie, the suffocation of all initiative and all creative urge. But the higher the economy rose, the more complex its requirements became, all the more unbearable became the obstacle of the bureaucratic regime. The constantly sharpening contradiction between them leads to uninterrupted political convulsions, to systematic annihilation of the most outstanding creative elements in all spheres of activity. Thus, before the bureaucracy could succeed in exuding from itself a “ruling class,” it came into irreconcilable contradiction with the demands of development. The explanation for this is to be found precisely in the fact that the bureaucracy is not the bearer of a new system of economy peculiar to itself and impossible without itself, but is a parasitic growth on a workers’ state.
\nowidow

\subsection*{The Conditions for the Omnipotence and Fall of the Bureaucracy}

The Soviet oligarchy possesses all the vices of the old ruling classes but lacks their historical mission. In the bureaucratic degeneration of the Soviet State it is not the general laws of modern society from capitalism to socialism which find expression but a special exceptional and temporary refraction of these laws under the conditions of a backward revolutionary country in a capitalist environment. The scarcity in consumers’ goods and the universal struggle to obtain them generate a policeman who arrogates to himself the function of distribution. Hostile pressure from without imposes on the policeman the role of “defender” of the country, endows him with national authority, and permits him doubly to plunder the country.

Both conditions for the omnipotence of the bureaucracy---the backwardness of the country and the imperialist environment---bear, however, a temporary and transitional character and must disappear with the victory of the world revolution. Even bourgeois economists have calculated that with a planned economy it would be possible to raise the national income of the United States rapidly to 200 billion dollars a year and thus assure the entire population not only the satisfaction of its primary needs but real comforts. On the other hand, the world revolution would do away with the danger from without as the supplementary cause of bureaucratization. The elimination of the need to expend an enormous share of the national income on armaments would raise even higher the living and cultural level of the masses. In these conditions the need for a policeman distributor would fall away by itself. Administration as a gigantic cooperative would very quickly supplant state power. There would be no room for a new ruling class or for a new exploiting regime, located between capitalism and socialism.

\subsection*{And What if the Socialist Revolution Is Not Accomplished?}

The disintegration of capitalism has reached extreme limits, likewise the disintegration of the old ruling class. The further existence of this system is impossible. The productive forces must be organized in accordance with a plan. But who will accomplish this task---the proletariat, or a new ruling class of “commissars”---politicians, administrators and technicians? Historical experience bears witness, in the opinion of certain rationalizers that one cannot entertain hope in the proletariat. The proletariat proved “incapable” of averting the last imperialist war although the material prerequisites for a socialist revolution already existed at that time. The successes of Fascism after the war were once again the consequence of the “incapacity” of the proletariat to lead capitalist society out of the blind alley. The bureaucratization of the Soviet State was in its turn the consequence of the “incapacity” of the proletariat itself to regulate society through the democratic mechanism. The Spanish revolution was strangled by the Fascist and Stalinist bureaucracies before the very eyes of the world proletariat. Finally, last link in this chain is the new imperialist war, the preparation of which took place quite openly, with complete impotence on the part of the world proletariat. If this conception is adopted, that is, if it is acknowledged that the proletariat does not have the forces to accomplish the socialist revolution, then the urgent task of the statification of the productive forces will obviously be accomplished by somebody else. By whom? By a new bureaucracy, which will replace the decayed bourgeoisie as a new ruling class on a world scale. That is how the question is beginning to be posed by those “leftists” who do not rest content with debating over words.

\subsection*{The Present War and the Fate of Modern Society}

By the very march of events this question is now posed very concretely. The second world war has begun. It attests incontrovertibly to the fact that society can no longer live on the basis of capitalism. Thereby it subjects the proletariat to a new and perhaps decisive test.

If this war provokes, as we firmly believe, a proletarian revolution, it must inevitably lead to the overthrow of the bureaucracy in the \USSR\ and regeneration of Soviet democracy on a far higher economic and cultural basis than in 1918. In that case the question as to whether the Stalinist bureaucracy was a “class” or a growth on the workers’ state will be automatically solved. To every single person it will become clear that in the process of the development of the world revolution the Soviet bureaucracy was only an \emph{episodic} relapse.

If, however, it is conceded that the present war will provoke not revolution but a decline of the proletariat, then there remains another alternative: the further decay of monopoly capitalism, its further fusion with the state and the replacement of democracy wherever it still remained by a totalitarian regime. The inability of the proletariat to take into its hands the leadership of society could actually lead under these conditions to the growth of a new exploiting class from the Bonapartist fascist bureaucracy. This would be, according to all indications, a regime of decline, signalizing the eclipse of civilization.

An analogous result might occur in the event that the proletariat of advanced capitalist countries, having conquered power, should prove incapable of holding it and surrender it, as in the \USSR, to a privileged bureaucracy. Then we would be compelled to acknowledge that the reason for the bureaucratic relapse is rooted not in the backwardness of the country and not in the imperialist environment but in the congenital incapacity of the proletariat to become a ruling class. Then it would be necessary in retrospect to establish that in its fundamental traits the present \USSR\ was the precursor of a new exploiting regime on an international scale.

We have diverged very far from the terminological controversy over the nomenclature of the Soviet state. But let our critics not protest: only by taking the necessary historical perspective can one provide himself with a correct judgment upon such a question as the replacement of one social regime by another. The historic alternative, carried to the end, is as follows: either the Stalin regime is an abhorrent relapse in the process of transforming bourgeois society into a socialist society, or the Stalin regime is the first stage of a new exploiting society. If the second prognosis proves to be correct, then, of course, the bureaucracy will become a new exploiting class. However onerous the second perspective may be, if the world proletariat should actually prove incapable of fulfilling the mission placed upon it by the course of development, nothing else would remain except openly to recognize that the socialist program based on the internal contradictions of capitalist society, ended as a utopia. It is self evident that a new “minimum” program would be required for the defense of the interests of the slaves of the totalitarian bureaucratic society.
\nowidow

But are there such incontrovertible or even impressive objective data as would compel us today to renounce the prospect of the socialist revolution? That is the whole question.

\subsection*{The Theory of “Bureaucratic Collectivism”}

Shortly after the assumption of power by Hitler, a German “left communist,” Hugo Urbahns, came to the conclusion that in place of capitalism a new historical era of “state capitalism” was impending. The first examples of this regime he named as Italy, the \USSR, Germany. Urbahns, however, did not draw the political conclusions of his theory. Recently, an Italian “left communist,” Bruno R., who formerly adhered to the Fourth International, came to the conclusion that “bureaucratic collectivism” was about to replace capitalism. (Bruno R., \emph{La bureaucratisme du monde}, Paris 1939, 350 pp.) The new bureaucracy is a class, its relations to the toilers is collective exploitation, the proletarians are transformed into the slaves of totalitarian exploiters.

Bruno R. brackets together planned economy in the \USSR, Fascism, National Socialism, and Roosevelt’s “New Deal.” All these regimes undoubtedly possess common traits, which in the last analysis are determined by the collectivist tendencies of modern economy. Lenin even prior to the October Revolution formulated the main peculiarities of imperialist capitalism as follows: Gigantic concentration of productive forces, the heightening fusion of monopoly capitalism with the state, an organic tendency toward naked dictatorship as a result of this fusion. The traits of centralization and collectivization determine both the politics of revolution and the politics of counter revolution; but this by no means signifies that it is possible to equate revolution, Thermidor, fascism, and American “reformism.” Bruno has caught on to the fact that the tendencies of collectivization assume, as a result of the political prostration of the working class, the form of “bureaucratic collectivism.” The phenomenon in itself is incontestable. But where are its limits, and what is its historical weight? What we accept as the deformity of a transitional period, the result of the unequal development of multiple factors in the social process, is taken by Bruno R. for an independent social formation in which the bureaucracy is the ruling class. Bruno R. in any case has the merit of seeking to transfer the question from the charmed circle of terminological copy book exercises to the plane of major historical generalizations. This makes it all the easier to disclose his mistake.

Like many ultra-lefts, Bruno R. identifies in essence Stalinism with Fascism. On the one side the Soviet bureaucracy has adopted the political methods of Fascism; on the other side the Fascist bureaucracy, which still confines itself to “partial” measures of state intervention, is heading toward and will soon reach complete statification of economy. The first assertion is absolutely correct. But Bruno’s assertion that fascist “anti-capitalism” is capable of arriving at the expropriation of the bourgeoisie is completely erroneous. “Partial” measures of state intervention and of nationalization in reality differ from planned state economy just as reforms differ from revolution. Mussolini and Hitler are only “coordinating” the interests of the property owners and “regulating” capitalist economy, and, moreover, primarily for war purposes. The Kremlin oligarchy is something else again: it has the opportunity of directing economy as a body only owing to the fact that the working class of Russia accomplished the greatest overturn of property relations in history. This difference must not be lost sight of.

But even if we grant that Stalinism and Fascism from opposite poles will some day arrive at one and the same type of exploitive society (“Bureaucratic Collectivism” according to Bruno R.’s terminology) this still will not lead humanity out of the blind alley. The crisis of the capitalist system is produced not only by the reactionary role of private property but also by the no less reactionary role of the national state. Even if the various fascist governments did succeed in establishing a system of planned economy at home then, aside from the, in the long run, inevitable revolutionary movements of the proletariat unforeseen by any plan, the struggle between the totalitarian states for world domination would be continued and even intensified. Wars would devour the fruits of planned economy and destroy the bases of civilization. Bertrand Russell thinks, it is true, that some victorious state may, as a result of the war, unify the entire world in a totalitarian vise. But even if such a hypothesis should be realized, which is highly doubtful, military “unification” would have no greater stability than the Versailles treaty. National uprisings and pacifications would culminate in a new world war, which would be the grave of civilization. Not our subjective wishes but the objective reality speaks for it, that the only way out for humanity is the world socialist revolution. The alternative to it is the relapse into barbarism.

\subsection*{The Proletariat and its Leadership}

We shall very soon devote a separate article to the question of the relation between the class and its leadership. We shall confine ourselves here to the most indispensable. Only vulgar “Marxists” who take it that politics is a mere and direct “reflection” of economics, are capable of thinking that leadership reflects the class directly and simply. In reality leadership, having risen above the oppressed class, inevitably succumbs to the pressure of the ruling class. The leadership of the American trade unions, for instance, “reflects” not so much the proletariat, as the bourgeoisie. The selection and education of a truly revolutionary leadership, capable of withstanding the pressure of the bourgeoisie, is an extraordinarily difficult task. The dialectics of the historic process expressed itself most brilliantly in the fact that the proletariat of the most backward country, Russia, under certain historic conditions, has put forward the most farsighted and courageous leadership. On the contrary, the proletariat in the country of the oldest capitalist culture, Great Britain, has even today the most dull witted and servile leadership.

The crisis of capitalist society which assumed an open character in July, 1914, from the very first day of the war produced a sharp crisis in the proletarian leadership. During the 25 years that have elapsed since that time, the proletariat of the advanced capitalist countries has not yet created a leadership that could rise to the level of the tasks of our epoch. The experience of Russia testifies, however, that such a leadership can be created. (This does not mean, of course, that it will be immune to degeneration.) The question consequently stands as follows: Will objective historical necessity in the long run cut a path for itself in the consciousness of the vanguard of the working class; that is, in the process of this war and those profound shocks which it must engender will a genuine revolutionary leadership be formed capable of leading the proletariat to the conquest of power?
\nowidow

The Fourth International has replied in the affirmative to this question, not only through the text of its program, but also through the very fact of its existence. All the various types of disillusioned and frightened representatives of pseudo-Marxism proceed \emph{on the contrary} from the assumption that the bankruptcy of the leadership only “reflects” the incapacity of the proletariat to fulfill its revolutionary mission. Not all our opponents express this thought clearly, but all of them---ultra-lefts, centrists, anarchists, not to mention Stalinists and social democrats---shift the responsibility for the defeats from themselves to the shoulders of the proletariat. None of them indicate under precisely what conditions the proletariat will be capable of accomplishing the socialist overturn.

If we grant as true that the cause of the defeats is rooted in the social qualities of the proletariat itself then the position of modern society will have to be acknowledged as hopeless. Under conditions of decaying capitalism the proletariat grows neither numerically nor culturally. There are no grounds, therefore, for expecting that it will sometime rise to the level of the revolutionary tasks. Altogether differently does the case present itself to him who has clarified in his mind the profound antagonism between the organic, deep going, insurmountable urge of the toiling masses to tear themselves free from the bloody capitalist chaos, and the conservative, patriotic, utterly bourgeois character of the outlived labor leadership. We must choose one of these two irreconcilable conceptions.

\subsection*{Totalitarian Dictatorship---A Condition of Acute Crisis and Not a Stable Regime}

The October Revolution was not an accident. It was forecast long in advance. Events confirmed this forecast. The degeneration does not refute the forecast, because Marxists never believed that an isolated workers’ state in Russia could maintain itself indefinitely. True enough, we expected the wrecking of the Soviet State, rather than its degeneration; to put it more correctly, we did not sharply differentiate between those two possibilities. But they do not at all contradict each other. Degeneration must inescapably end at a certain stage in downfall.
\nowidow

A totalitarian regime, whether of Stalinist or Fascist type, by its very essence can be only a temporary transitional regime. Naked dictatorship in history has generally been the product and the symptom of an especially severe social crisis, and not at all of a stable regime. Severe crisis cannot be a permanent condition of society. A totalitarian state is capable of suppressing social contradictions during a certain period, but it is incapable of perpetuating itself. The monstrous purges in the \USSR\ are most convincing testimony of the fact that Soviet society organically tends toward ejection of the bureaucracy.

It is an astonishing thing that Bruno R. sees precisely in the Stalinist purges proof of the fact that the bureaucracy has become a ruling class, for in his opinion only a ruling class is capable of measures on so large a scale.\footnote{True enough, in the last section of his book, which consists of fantastic contradictions, Bruno R. quite consciously and articulately refutes his own theory of “bureaucratic collectivism” unfolded in the first section of the book and declares that Stalinism, Fascism, and Nazism are transitory and parasitic formations, historical penalties for the impotence of the proletariat. In other words, after having subjected the views of the Fourth International to the sharpest kind of criticism. Bruno R.\@ unexpectedly returns to those views, but only in order to launch a new series or blind rumblings. We see no grounds for following in the footsteps of a writer who has obviously lost his balance. We are interested in those of his arguments by means of which he seeks to substantiate his views that the bureaucracy is a class.} He forgets however that Czarism, which was not a “class,” also permitted itself rather large scale measures in purges and moreover precisely in the period when it was nearing its doom. Symptomatic of his oncoming death agony, by the sweep and monstrous fraudulence of his purge, Stalin testifies to nothing else but the incapacity of the bureaucracy to transform itself into a stable ruling class. Might we not place ourselves in a ludicrous position if we affixed to the Bonapartist oligarchy the nomenclature of a new ruling class just a few years or even a few months prior to its inglorious downfall? Posing this question clearly should alone in our opinion restrain the comrades from terminological experimentation and overhasty generalizations.

\subsection*{The Orientation Towards World Revolution and the Regeneration of the \USSR}

A quarter of a century proved too brief a span for the revolutionary rearming of the world proletarian vanguard, and too long a period for preserving the Soviet system intact in an isolated backward country. Mankind is now paying for this with a new imperialist war; but the basic task of our epoch has not changed, for the simple reason that it has not been solved. A colossal asset in the last quarter of a century and a priceless pledge for the future is constituted by the fact that one of the detachments of the world proletariat was able to demonstrate in action how the task must be solved.

The second imperialist war poses the unsolved task on a higher historical stage. It tests anew not only the stability of the existing regimes but also the ability of the proletariat to replace them. The results of this test will undoubtedly have a decisive significance for our appraisal of the modern epoch as the epoch of proletarian revolution. If contrary to all probabilities the October Revolution fails during the course of the present war, or immediately thereafter, to find its continuation in any of the advanced countries; and if, on the contrary, the proletariat is thrown back everywhere and on all fronts---then we should doubtlessly have to pose the question of revising our conception of the present epoch and its driving forces. In that case it would be a question not of slapping a copy book label on the \USSR\ or the Stalinist gang but of re-evaluating the world historical perspective for the next decades if not centuries: Have we entered the epoch of social revolution and socialist society, or on the contrary the epoch of the declining society of totalitarian bureaucracy?

The twofold error of schematists like Hugo Urbahns and Bruno R.\@ consists, first, in that they proclaim this latter regime as having been already finally installed; secondly, in that they declare it a prolonged transitional state of society between capitalism and socialism. Yet it is absolutely self-evident that if the international proletariat, as a result of the experience of our entire epoch and the current new war proves incapable of becoming the master of society, this would signify the foundering of all hope for a socialist revolution, for it is impossible to expect any other more favorable conditions for it; in any case no one foresees them now, or is able to characterize them. Marxists do not have the slightest right (if disillusionment and fatigue are not considered “rights”) to draw the conclusion that the proletariat has forfeited its revolutionary possibilities and must renounce all aspirations to hegemony in an era immediately ahead. Twenty-five years in the scales of history, when it is a question of profoundest changes in economic and cultural systems, weigh less than an hour in the life of man. What good is the individual, who because of empirical failures in the course of an hour or a day renounces a goal that he set for himself on the basis of the experience and analysis of his entire previous lifetime? In the years of darkest Russian reaction (1907 to 1917) we took as our starting point those revolutionary possibilities which were revealed by the Russian proletariat in 1905. In the years of world reaction we must proceed from those possibilities which the Russian proletariat revealed in 1917. The Fourth International did not by accident call itself the world party of the socialist revolution. Our road is not to be changed. We steer our course toward the world revolution and by virtue of this very fact toward the regeneration of the \USSR\ as a worker’s state.

\subsection*{Foreign Policy is the Continuation of Domestic Policy}

What do we defend in the \USSR? Not that in which it resembles the capitalist countries but precisely that in which it differs from them. In Germany also we advocate an uprising against the ruling bureaucracy, but only in order immediately to overthrow capitalist property. In the \USSR\ the overthrow of the bureaucracy is indispensable for the preservation of state property. Only in this sense do we stand for the defense of the \USSR.

There is not one among us who doubts that the Soviet workers should defend the state property, not only against the parasitism of the bureaucracy, but also against the tendencies toward private ownership, for example, on the part of the Kolkhoz aristocracy. But after all, foreign policy is the continuation of policy at home. If in domestic policy we correlated defense of the conquests of the October Revolution with irreconcilable struggle against the bureaucracy, then we must do the same thing in foreign policy as well. To be sure, Bruno R.\@ proceeding from the fact that “bureaucratic collectivism” has already been victorious all along the line, assures us that no one threatens state property, because Hitler (and Chamberlain?) is as much interested, you see, in preserving it as Stalin. Sad to say, Bruno R.’s assurances are frivolous. In event of victory Hitler will in all probability begin by demanding the return to German capitalists of all the property expropriated from them; then he will secure a similar restoration of property for the English, the French, and the Belgians so as to reach an agreement with them at the expense of the \USSR; finally, he will make Germany the contractor of the most important state enterprises in the \USSR\ in the interests of the German military machine. Right now Hitler is the ally and friend of Stalin; but should Hitler, with the aid of Stalin, come out victorious on the Western Front, he would on the morrow turn his guns against the \USSR. Finally Chamberlain, too, in similar circumstances would act no differently from Hitler.

\subsection*{The Defense of the \USSR\ and the Class Struggle}

Mistakes on the question of defense of the \USSR\ most frequently flow from an incorrect understanding of the methods of “defense.” Defense of the \USSR\ does not at all mean rapprochement with the Kremlin bureaucracy, the acceptance of its politics, or a conciliation with the politics of her allies. In this question, as in all others, we remain completely on the ground of the international class struggle.

In the tiny French periodical, \emph{Que Faire}, it was recently stated that inasmuch as the “Trotskyites” are defeatists in relation to France and England they are therefore defeatists also in relation to the \USSR. In other words: If you want to defend the \USSR\ you must stop being defeatists in relation to her imperialist allies. \emph{Que Faire} calculated that the “democracies” would be the allies of the \USSR.

What these sages will say now we don’t know. But that is hardly important, for their very method is rotten. To renounce defeatism in relation to that imperialist camp to which the \USSR\ adheres today or might adhere tomorrow is to push the workers of the enemy camp to the side of their government; it means to renounce defeatism in general. The renunciation of defeatism under the conditions of imperialist war which is tantamount to the rejection of the socialist revolution---rejection of revolution in the name of “defense of the \USSR”---would sentence the \USSR\ to final decomposition and doom.
\nowidow

“Defense of the \USSR”, as interpreted by the Comintern, like yesterday’s “struggle against fascism” is based on renunciation of independent class politics. The proletariat is transformed---for various reasons in varying circumstances, but always and invariably---into an auxiliary force of one bourgeois camp against another. In contradistinction to this, some of our comrades say: Since we do not want to become tools of Stalin and his allies we therefore renounce the defense of the \USSR. But by this they only demonstrate that their understanding of “defense” coincides essentially with the understanding of the opportunists; they do not think in terms of the independent politics of the proletariat. As a matter of fact, we defend the \USSR\ as we defend the colonies, as we solve all our problems, not by supporting some imperialist governments against others, but by the method of international class struggle in the colonies as well as in the metropolitan centers.

We are not a government party; we are the party of irreconcilable opposition, not only in capitalist countries but also in the \USSR. Our tasks, among them the “defense of the \USSR”, we realize not through the medium of bourgeois governments and not even through the government of the \USSR, but exclusively through the education of the masses through agitation, through explaining to the workers what they should defend and what they should overthrow. Such a “defense” cannot give immediate miraculous results. But we do not even pretend to be miracle workers. As things stand, we are a revolutionary minority. Our work must be directed so that the workers on whom we have influence should correctly appraise events, not permit themselves to be caught unawares, and prepare the general sentiment of their own class for the revolutionary solution of the tasks confronting us.

The defense of the \USSR\ coincides for us with the preparation of world revolution. Only those methods are permissible which do not conflict with the interests of the revolution. The defense of the \USSR\ is related to the world socialist revolution as a tactical task is related to a strategic one. A tactic is subordinated to a strategic goal and in no case can be in contradiction to the latter.

\subsection*{The Question of Occupied Territories}

As I am writing these lines the question of the territories occupied by the Red Army still remains obscure. The cable dispatches contradict each other, since both sides lie a great deal; but the actual relationships on the scene are no doubt still extremely unsettled. Most of the occupied territories will doubtlessly become part of the \USSR\ in what form? Let us for a moment conceive that in accordance with the treaty with Hitler, the Moscow government leaves untouched the rights of private property in the occupied areas and limits itself to “control” after the Fascist pattern. Such a concession would have a deep going principled character and might become a starting point for a new chapter in the history of the Soviet regime: and consequently a starting point for a new appraisal on our part of the nature of the Soviet state.

It is more likely, however, that in the territories scheduled to become a part of the \USSR, the Moscow government will carry through the expropriation of the large landowners and statification of the means of production. This variant is most probable not because the bureaucracy remains true to the socialist program but because it is neither desirous nor capable of sharing the power, and the privileges the latter entails, with the old ruling classes in the occupied territories. Here an analogy literally offers itself. The first Bonaparte halted the revolution by means of a military dictatorship. However, when the French troops invaded Poland, Napoleon signed a decree: “Serfdom is abolished.” This measure was dictated not by Napoleon’s sympathies for the peasants, nor by democratic principles but rather by the fact that the Bonapartist dictatorship based itself not on feudal, but on bourgeois property relations. Inasmuch as Stalin’s Bonapartist dictatorship bases itself not on private but on state property, the invasion of Poland by the Red Army should, in the nature of the case, result in the abolition of private capitalist property, so as thus to bring the regime of the occupied territories into accord with the regime of the \USSR.

This measure, revolutionary in character---``the expropriation of the expropriators''---is in this case achieved in a military bureaucratic fashion. The appeal to independent activity on the part of the masses in the new territories – and without such an appeal, even if worded with extreme caution it is impossible to constitute a new regime---will on the morrow undoubtedly be suppressed by ruthless police measures in order to assure the preponderance of the bureaucracy over the awakened revolutionary masses. This is one side of the matter. But there is another. In order to gain the possibility of occupying Poland through a military alliance with Hitler, the Kremlin for a long time deceived and continues to deceive the masses in the \USSR\ and in the whole world, and has thereby brought about the complete disorganization of the ranks of its own Communist International. The primary political criterion for us is not the transformation of property relations in this or another area, however important these may be in themselves, but rather the change in the consciousness and organization of the world proletariat, the raising of their capacity for defending former conquests and accomplishing new ones. From this one, and the only decisive standpoint, the politics of Moscow, taken as a whole, wholly retain their reactionary character and remain the chief obstacle on the road to the world revolution.

Our \emph{general} appraisal of the Kremlin and Comintern does not, however, alter the \emph{particular} fact that the statification of property in the occupied territories is in itself a progressive measure. We must recognize this openly. Were Hitler on the morrow to throw his armies against the East, to restore “law and order” in Eastern Poland, the advanced workers would defend against Hitler these new property forms established by the Bonapartist Soviet bureaucracy.

\subsection*{We Do Not Change Our Course!}

The statification of the means of production is, as we said, a progressive measure. But its progressiveness is relative; its specific weight depends on the sum-total of all the other factors. Thus, we must first and foremost establish that the extension of the territory dominated by bureaucratic autocracy and parasitism, cloaked by “socialist” measures, can augment the prestige of the Kremlin, engender illusions concerning the possibility of replacing the proletarian revolution by bureaucratic maneuvers and so on. This evil by far outweighs the progressive content of Stalinist reforms in Poland. In order that nationalized property in the occupied areas, as well as in the \USSR, become a basis for genuinely progressive, that is to say socialist development, it is necessary to overthrow the Moscow bureaucracy. Our program retains, consequently, all its validity. The events did not catch us unaware. It is necessary only to interpret them correctly. It is necessary to understand clearly that sharp contradictions are contained in the character of the \USSR\ and in her international position. It is impossible to free oneself from those contradictions with the help of terminological sleight of hand (“Workers State”---“Not Workers State.”) We must take the facts as they are. We must build our policy by taking as our starting point the real relations and contradictions.

We do not entrust the Kremlin with any historic mission. We were and remain against seizures of new territories by the Kremlin. We are for the independence of Soviet Ukraine, and if the Byelo-Russians themselves wish---of Soviet Byelo-Russia. At the same time in the sections of Poland occupied by the Red Army, partisans of the Fourth International must play the most decisive part in expropriating the landlords and capitalists, in dividing the land among the peasants, in creating Soviets and Workers’ Committees, \etc. While so doing, they must preserve their political independence, they must fight during elections the Soviets and factory committees for the complete independence of the latter from the bureaucracy, and they must conduct revolutionary propaganda in the spirit of distrust towards the Kremlin and its local agencies.

But let us suppose that Hitler turns his weapons against the East and invades territories occupied by the Red Army. Under these conditions, partisans of the Fourth International, without changing in any way their attitude toward the Kremlin oligarchy, will advance to the forefront as the most urgent task of the hour, the military resistance against Hitler. The workers will say, “We cannot cede to Hitler the overthrowing of Stalin; that is our own task”. During the military struggle against Hitler, the revolutionary workers will strive to enter into the closest possible comradely relations with the rank and file fighters of the Red Army. While arms in hand they deal blows to Hitler, the Bolshevik-Leninists will at the same time conduct revolutionary propaganda against Stalin preparing his overthrow at the next and perhaps very near stage.

This kind of “defense of the \USSR” will naturally differ, as heaven does from earth, from the official defense which is now being conducted under the slogan: “For the Fatherland! For Stalin!” \emph{Our} defense of the \USSR\ is carried on under the slogan: “For Socialism! For the world revolution! Against Stalin!” In order that these two varieties of “defense of the \USSR” do not become confused in the consciousness of the masses it is necessary to know clearly and precisely how to formulate slogans which correspond to the concrete situation. But above all it is necessary to establish clearly just \emph{what} we are defending, just \emph{how} we are defending it, against \emph{whom} we are defending it. Our slogans will create confusion among the masses only if we ourselves do not have a clear conception of our tasks.

\subsection*{Conclusions}

We have no reasons whatsoever at the present time for changing our principled position in relation to the \USSR.

War accelerates the various political processes. It may accelerate the process of the revolutionary regeneration of the \USSR. But it may also accelerate the process of its final degeneration. For this reason it is indispensable that follow painstakingly and without prejudice these modifications which war introduces into the internal life of the \USSR\ so that we may give ourselves a timely accounting of them.

Our tasks in the occupied territories remain basically the same as in the \USSR\ itself; but inasmuch as they are posed by events in an extremely sharp form, they enable us all the better to clarify our general tasks in relation to the \USSR.

We must formulate our slogans in such a way that the workers see clearly just what we are defending in the \USSR, (state property and planned economy), and against whom we are conducting a ruthless struggle (the parasitic bureaucracy and their Comintern). We must not lose sight for a single moment of the fact that the question of overthrowing the Soviet bureaucracy is for us subordinate to the question of preserving state property in the means of production of the \USSR: that the question of preserving state property in the means of production in the \USSR\ is subordinate for us to the question of the world proletarian revolution.

\signed{L.\@ Trotsky}

\newpage

\begin{letterinfo}
  \textbf{Source:} \emph{The New International} 5, no. 11. New York, Nov.\@ 1939, pp. 325--332.
	
  \textbf{Translated:} \emph{New International}.
	
  \textbf{Reprinted:} Leon Trotsky, \emph{In Defense of Marxism}, New York 1942.
	
  \textbf{British Edition:} Leon Trotsky, \emph{In Defence of Marxism}, London 1966, pp. 3--26.
	
  \textbf{Public Domain:} Trotsky Internet Archive, 2005.
	
  \textbf{Transcription:} (for \textsc{tia}) David Walters.
	
  \textbf{Proofreading:} (for \textsc{tia}) Andy Pollack \& Einde O’Callaghan.
\end{letterinfo}