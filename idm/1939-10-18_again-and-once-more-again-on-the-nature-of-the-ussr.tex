\chapterdated{Again and Once More Again on the Nature of the \USSR}{October 18, 1939}
\markright{Again and Once More Again on the Nature of the \fakesc{USSR}}

\subsection*{Psychoanalysis and Marxism}

\textsc{Certain comrades}, or former comrades, such as Bruno R., having forgotten the past discussions and decisions of the Fourth International, attempt to explain my personal estimate of the Soviet state psychoanalytically. “Since Trotsky participated in the Russian Revolution, it is difficult for him to lay aside the idea of the workers’ state inasmuch as he would have to renounce his whole life’s cause,” \etc. I think that the old Freud, who was very perspicacious, would have cuffed the ears of psychoanalysts of this ilk a little. Naturally I would never risk taking such action myself. Nevertheless I dare assure my critics that subjectivity and sentimentality are not on my side but on theirs.

Moscow’s conduct, which has passed all bounds of abjectness and cynicism, calls forth an easy revolt within every proletarian revolutionary. Revolt engenders need for rejection. When the forces for immediate action are absent, impatient revolutionaries are inclined to resort to artificial methods. Thus arises, for example, the tactic of individual terror. More frequently resort is taken to strong expressions, to insults, and to imprecation. In the case which concerns us certain comrades are manifestly inclined to seek compensation through “terminological” terror. However, even from this point of view the mere fact of qualifying the bureaucracy as a class is worthless. If the Bonapartist riff-raff is a class this means that it is not an abortion but a viable child of history. If its marauding parasitism is “exploitation” in the scientific sense of the term, this means that the bureaucracy possesses a historical future as the ruling class indispensable to the given system of economy. Here we have the end to which impatient revolt leads when it cuts itself loose from Marxist discipline!

When an emotional mechanic considers an automobile in which, let us say, gangsters have escaped from police pursuit over a bad road, and finds the frame bent, the wheels out of line, and the motor partially damaged, he might quite justifiably say: “It is not an automobile---devil knows what it is!” Such an estimate would lack any technical and scientific value, but it would express the legitimate reaction of the mechanic at the work of the gangsters. Let us suppose, however, that this same mechanic must recondition the object which he named “devil-knows-what-it-is.” In this case he will start with the recognition that it is a damaged automobile before him. He will determine which parts are still good and which are beyond repair in order to decide how to begin work. The class-conscious worker will have a similar attitude toward the \USSR. He has full right to say that the gangsters of the bureaucracy have transformed the workers’ state into “devil-knows-what-it-is.” But when he passes from this explosive reaction to the solution of the political problem, he is forced to recognize that it is a damaged workers’ state before him, in which the motor of economy is damaged, but which still continues to run and which can be completely reconditioned with the replacement of some parts. Of course this is only an analogy. Nevertheless it is worth reflecting over.

\subsection*{“A Counter-Revolutionary Workers’ State”}

Some voices cry out: “If we continue to recognize the \USSR\ as a workers’ state, we will have to establish a new category: the counter-revolution\-ary workers’ state.” This argument attempts to shock our imagination by opposing a good programmatic norm to a miserable, mean, even repugnant reality. But haven’t we observed from day to day since 1923 how the Soviet state has played a more and more counter-revolutionary role on the international arena? Have we forgotten the experience of the Chinese Revolution, of the 1926 general strike in England and finally the very fresh experience of the Spanish Revolution? There are two completely counter-revolutionary workers’ internationals. These critics have apparently forgotten this “category.” The trade unions of France, Great Britain, the United States and other countries support completely the counterrevolutionary politics of their bourgeoisie. This does not prevent us from labeling them trade unions, from supporting their progressive steps and from defending them against the bourgeoisie. Why is it impossible to employ the same method with the counter-revolutionary workers’ state? In the last analysis a workers’ state is a trade union which has conquered power. The difference in attitude in these two cases is explainable by the simple fact that the trade unions have a long history and we have become accustomed to consider them as realities and not simply as “categories” in our program. But, as regards the workers’ state there is being evinced an inability to learn to approach it as a real historical fact which has not subordinated itself to our program.

\subsection*{Imperialism?}

Can the present expansion of the Kremlin be termed imperialism? First of all we must establish what social content is included in this term. History has known the “imperialism” of the Roman state based on slave labor, the imperialism of feudal land-ownership, the imperialism of commercial and industrial capital, the imperialism of the Czarist monarchy, \etc. The driving force behind the Moscow bureaucracy is indubitably the tendency to expand its power, its prestige, its revenues. This is the element of “imperialism” in the widest sense of the word which was a property in the past of all monarchies, oligarchies, ruling castes, medieval estates and classes. However, in contemporary literature, at least Marxist literature, imperialism is understood to mean the \emph{expansionist policy of finance capital} which has a very sharply defined economic content. To employ the term “imperialism” for the foreign policy of the Kremlin---without elucidating exactly what this signifies---means simply to identify the policy of the Bonapartist bureaucracy with the policy of monopolistic capitalism on the basis that both one and the other utilize military force for expansion. Such an identification, capable of sowing only confusion, is much more proper to petty-bourgeois democrats than to Marxists.

\newpage

\subsection*{Continuation of the Policy of Czarist Imperialism}

\noindent
The Kremlin participates in a new division of Poland, the Kremlin lays hands upon the Baltic states, the Kremlin orients toward the Balkans, Persia and Afghanistan; in other words, the Kremlin continues the policy of Czarist imperialism. Do we not have the right in this case to label the policy of the Kremlin itself imperialist? This historical-geographical argument is no more convincing than any of the others. The proletarian revolution, which occurred on the territory of the Czarist empire, attempted from the very beginning to conquer and for a time conquered the Baltic countries; attempted to penetrate Rumania and Persia and at one time led its armies up to Warsaw (1920). The lines of revolutionary expansion were the same as those of Czarism, since revolution does not change geographical conditions. That is precisely why the Mensheviks at that time already spoke of Bolshevik imperialism as borrowed from the traditions of Czarist diplomacy. The petty-bourgeois democracy willingly resorts to this argument even now. We have no reason, I repeat, for imitating them in this.

\subsection*{Agency of Imperialism?}

However, aside from the manner in which to appraise the expansionist policy of the \USSR\ itself, there remains the question of the help which Moscow provides the imperialist policy of Berlin. Here first of all, it is necessary to establish that under certain conditions---up to a certain degree and in a certain form---the support of this or that imperialism would be inevitable even for a completely healthy workers’ state---in virtue of the impossibility of breaking away from the chains of world imperialist relations. The Brest-Litovsk peace without the least doubt temporarily reinforced German imperialism against France and England. An isolated workers’ state cannot fail to maneuver between the hostile imperialist camps. Maneuvering means temporarily supporting one of them against the other. To know exactly which one of the two camps it is more advantageous or less dangerous to support at a certain moment is not a question of principle but of practical calculation and foresight. The inevitable disadvantage which is engendered as a consequence of this constrained support for one bourgeois state against another is more than covered by the fact that the isolated workers’ state is thus given the possibility of continuing its existence.
\nowidow

But there is maneuvering and maneuvering. At Brest-Litovsk the Soviet government sacrificed the national independence of the Ukraine in order to salvage the workers’ state. Nobody could speak of treason toward the Ukraine, since all the class-conscious workers understood the forced character of this sacrifice. It is completely different with Poland. The Kremlin has never and at no place represented the question as if it had been \emph{constrained} to sacrifice Poland. On the contrary, it boasts cynically of its combination, which affronts, rightfully, the most elementary democratic feelings of the oppressed classes and peoples throughout the world and thus weakens extremely the international situation of the Soviet Union. The economic transformations in the occupied provinces do not compensate for this by even a tenth part!

The entire foreign policy of the Kremlin in general is based upon a scoun\-drel\-ly embellishment of the “friendly” imperialism and thus leads to the sacrifice of the fundamental interests of the world workers’ movement for secondary and unstable advantages. After five years of duping the workers with slogans for the “defense of the democracies” Moscow is now occupied with covering up Hitler’s policy of pillage. This in itself still does not change the \USSR\ into an imperialist state. But Stalin and his Comintern are now indubitably the most valuable agency of imperialism.

If we want to define the foreign policy of the Kremlin exactly, we must say that it is the policy, of the \emph{Bonapartist bureaucracy of a degenerated workers’ state in imperialist encirclement}. This definition is not as short or as sonorous as “imperialist policy,” but in return it is more precise.

\subsection*{“The Lesser Evil”}

The occupation of eastern Poland by the Red Army is to be sure a “lesser evil” in comparison with the occupation of the same territory by Nazi troops. But this lesser evil was obtained because Hitler was assured of achieving a greater evil. If somebody sets, or helps to set a house on fire and afterward saves five out of ten of the occupants of the house in order to convert them into his own semi-slaves, that is to be sure a lesser evil than to have burned the entire ten. But it is dubious that this firebug merits a medal for the rescue. If nevertheless a medal were given to him he should be shot immediately after as in the case of the hero in one of Victor Hugo’s novels.

\subsection*{“Armed Missionaries”}

Robespierre once said that people do not like missionaries with bayonets. By this he wished to say that it is impossible to impose revolutionary ideas and institutions on other people through military violence. This correct thought does not signify of course the inadmissibility of military intervention in other countries in order to cooperate in a revolution. But such an intervention, as part of a revolutionary international policy, must be understood by the international proletariat, must correspond to the desires of the toiling masses of the country on whose territory the revolutionary troops enter. The theory of socialism in one country is not capable, naturally, of creating this active international solidarity which alone can prepare and justify armed intervention. The Kremlin poses and resolves the question of military intervention, like all other questions of its policy, absolutely independently of the ideas and feelings of the international working class. Because of this, the latest diplomatic “successes” of the Kremlin monstrously compromise the \USSR\ and introduce extreme confusion into the ranks of the world proletariat.

\subsection*{Insurrection on Two Fronts}

But if the question thus shapes itself---some comrades say---is it proper to speak of the defense of the \USSR\ and the occupied provinces? Is it not more correct to call upon the workers and peasants in both parts of former Poland to arise against Hitler as well as against Stalin? Naturally, this is very attractive. If revolution surges up simultaneously in Germany and in the \USSR, including the newly occupied provinces, this would resolve many questions at one blow. But our policy cannot be based upon only the most favorable, the most happy combination of circumstances. The question is posed thus: What to do if Hitler, before he is crushed by revolution, attacks the Ukraine before revolution has smashed Stalin? Will the partisans of the Fourth International in this case fight against the troops of Hitler as they fought in Spain in the ranks of the Republican troops against Franco? We are completely and whole-heartedly for an independent (of Hitler as well as of Stalin) Soviet Ukraine. But what to do if, before having obtained this independence, Hitler attempts to seize the Ukraine which is under the domination of the Stalinist bureaucracy? The Fourth International answers: Against Hitler we will defend this Ukraine enslaved by Stalin.
\nowidow

\subsection*{“Unconditional Defense of the \USSR”}

What does “unconditional” defense of the \USSR\ mean? It means that we do not lay any conditions upon the bureaucracy. It means that independently of the motive and causes of the war we defend the social basis of the \USSR, if it is menaced by danger on the part of imperialism.

Some comrades say: “And if the Red Army tomorrow invades India and begins to put down a revolutionary movement there shall we in this case support it?” Such a way of posing a question is not at all consistent. It is not clear above all why India is implicated. Is it not simpler to ask: If the Red Army menaces workers’ strikes or peasant protests against the bureaucracy in the \USSR\ shall we support it or not? Foreign policy is the continuation of the internal. We have never promised to support \emph{all} the actions of the Red Army which is an instrument in the hands of the Bonapartist bureaucracy. We have promised to defend only the \USSR\ as a workers’ state and solely those things within it which belong to a workers’ state.

An adroit casuist can say: If the Red Army, independently of the character of the “work” fulfilled by it, is beaten by the insurgent masses in India, this will weaken the \USSR. To this we will answer: The crushing of a revolutionary movement in India, with the cooperation of the Red Army, would signify an incomparably greater danger to the social basis of the \USSR\ than an episodical defeat of counter-revolutionary detachments of the Red Army in India. In every case the Fourth International will know how to distinguish where and when the Red Army is acting solely as an instrument of the Bonapartist reaction and where it defends the social basis of the \USSR.

A trade union led by reactionary fakers organizes a strike against the admission of Negro workers into a certain branch of industry. Shall we support such a shameful strike? Of course not. But let us imagine that the bosses, utilizing the given strike, make an attempt to crush the trade union and to make impossible in general the organized self-defense of the workers. In this case we will defend the trade union as a matter of course in spite of its reactionary leadership. Why is not this same policy applicable to the \USSR?
\nowidow

\subsection*{The Fundamental Rule}

The Fourth International has established firmly that in all imperialist countries, independent of the fact as to whether they are in alliance with the \USSR\ or in a camp hostile to it, the proletarian parties during the war must develop the class struggle with the purpose of seizing power. At the same time the proletariat of the imperialist countries must not lose sight of the interests of the \USSR’s defense (or of that of colonial revolutions) and in case of real necessity must resort to the most decisive action, for instance, strikes, acts of sabotage, \etc. The groupings of the powers since the time the Fourth International formulated this rule have changed radically. But the rule itself retains all its validity. If England and France tomorrow menace Leningrad or Moscow, the British and French workers should take the most decisive measures in order to hinder the sending of soldiers and military supplies. If Hitler finds himself constrained by the logic of the situation to send Stalin military supplies, the German workers, on the contrary, would have no reason for resorting in this concrete case to strikes or sabotage. Nobody, I hope, will propose any other solution.

\subsection*{“Revision of Marxism?”}

Some comrades evidently were surprised that I spoke in my article (``The \USSR\ in War'') of the system of “bureaucratic collectivism” as a theoretical possibility. They discovered in this even a complete revision of Marxism. This is an apparent misunderstanding. The Marxist comprehension of historical necessity has nothing in common with fatalism. Socialism is not realizable “by itself,” but as a result of the struggle of living forces, classes and their parties. The proletariat’s decisive advantage in this struggle resides in the fact that it represents historical progress, while the bourgeoisie incarnates reaction and decline. Precisely in this is the source of our conviction in victory. But we have full right to ask ourselves: What character will society take if the forces of reaction conquer?

Marxists have formulated an incalculable number of times the alternative: either socialism or return to barbarism. After the Italian “experience” we repeated thousands of times: either communism or fascism. The real passage to socialism cannot fail to appear incomparably more complicated, more heterogeneous, more contradictory than was foreseen in the general historical scheme. Marx spoke about the dictatorship of the proletariat and its future withering away but said nothing about bureaucratic degeneration of the dictatorship. We have observed and analyzed for the first time in experience such a degeneration. Is this revision of Marxism?

The march of events has succeeded in demonstrating that the delay of the socialist revolution engenders the indubitable phenomena of barbarism-chronic unemployment, pauperization of the petty bourgeoisie, fascism, finally wars of extermination which do not open up any new road. What social and political forms can the new “barbarism” take, if we admit theoretically that mankind should not be able to elevate itself to socialism? We have the possibility of expressing ourselves on this subject more concretely than Marx. Fascism on one hand, degeneration of the Soviet state on the other outline the social and political forms of a neo-barbarism. An alternative of this kind---socialism or totalitarian servitude---has not only theoretical interest, but also enormous importance in agitation, because in its light the necessity for socialist revolution appears most graphically.

If we are to speak of a revision of Marx, it is in reality the revision of those comrades who project a new type of state, “nonbourgeois” and “non-worker.” Because the alternative developed by me leads them to draw their own thoughts up to their logical conclusion, some of these critics, frightened by the conclusions of their own theory, accuse me\dots\ of revising Marxism. I prefer to think that it is simply a friendly jest.

\subsection*{The Right of Revolutionary Optimism}

\enlargethispage{1 \baselineskip}

I endeavored to demonstrate in my article ``The \USSR\ in War'' that the perspective of a non-worker and non-bourgeois society of exploitation, or “bureaucratic collectivism,” is the perspective of complete defeat and the decline of the international proletariat, the perspective of the most profound historical pessimism.

Are there any genuine reasons for such a perspective? It is not superfluous to inquire about this among our class enemies.

In the weekly of the well-known newspaper \emph{Paris-Soir} of August 31, 1939, an extremely instructive conversation is reported between the French ambassador Coulondre and Hitler on August 25, at the time of their last interview. (The source of the information is undoubtedly Coulondre himself.) Hitler sputters, boasts of the pact which he concluded with Stalin (“a realistic pact”) and “regrets” that German and French blood will be spilled.
\begin{quote}
  “But,” Coulondre objects, “Stalin displayed great double-dealing. The real victor (in case of war) will be Trotsky. Have you thought this over?”
  
  “I know,” der Fuehrer responds, “but why did France and Britain give Poland complete freedom of action?”
\end{quote}
These gentlemen like to give a personal name to the specter of revolution. But this of course is not the essence of this dramatic conversation at the very moment when diplomatic relations were ruptured. “War will inevitably provoke revolution,” the representative of imperialist democracy, himself chilled to the marrow, frightens his adversary.

“I know,” Hitler responds, as if it were a question decided long ago. “I know.” Astonishing dialogue.

Both of them, Coulondre and Hitler, represent the barbarism which advances over Europe. At the same time neither of them doubts that their barbarism will be conquered by socialist revolution. Such is now the awareness of the ruling classes of all the capitalist countries of the world. Their complete demoralization is one of the most important elements in the relation of class forces. The proletariat has a young and still weak revolutionary leadership. But the leadership of the bourgeoisie rots on its feet. At the very outset of the war which they could not avert, these gentlemen are convinced in advance of the collapse of their regime. This fact alone must be for us the source of invincible revolutionary optimism!

\begin{letterinfo}
  \firstpublished\ Leon Trotsky, \emph{In Defense of Marxism}, New York 1942.
  
  \checkedagainst\ Leon Trotsky, \emph{In Defence of Marxism}, London 1966, pp.~29--39.
\end{letterinfo}