\chapterdated[A Letter to Sherman Stanley]{A Letter to Sherman Stanley\footnote{This letter was written by Trotsky in English.}}{October 8, 1939}

\letteraddress{Dear Comrade Stanley,}

\noindent
I received your letter to O’Brien in view of his departure. The letter produced upon me a strange impression, because in opposition to your very good articles, it is full of contradictions.

I didn’t receive as yet any material about the plenum and don’t know either the text of the majority resolution or that of M.S.,\footnote{The reference is to Max Shachtman. ---\ed} but you affirm that there is not an irreconcilable opposition between the two texts. At the same time you affirm that a “disaster” approaches the party. Why? Even if there have been two \emph{irreconcilable} positions, it would signify not a “disaster” but a necessity to fight out the political struggle to the end. But if both motions represent only nuances of the same point of view expressed in the program of the 4th International, how can arise from this unprincipled (in your opinion) divergence, a catastrophe? That the majority preferred their own nuance (if it is only a nuance) is natural. But what is absolutely unnatural is that the minority proclaims: “because you, the majority, accept your own nuance and not ours we foretell catastrophe.” From the side of whom ?~?~? \dots\ And you affirm that you “look objectively upon the different groupings.” It is not my impression at all.

You write for example that from my article “one page for \emph{some reason or other} was missing.” You express in this way a very venomous suspicion towards responsible comrades. The page was missing by a regrettable nonchalance in our office here, and we sent already a new, complete text for translation.\footnote{The document, \emph{The \fakesc{USSR} in War}, arrived while the plenum of the National Committee was in session. One page was missing. The political line of this document was endorsed by the majority of the plenum. The minority raised a hue and cry about the missing page, charging, among other things, that it was being deliberately suppressed. ---\ed}

Your argument about the degenerated “workers’ \emph{empire}” seems to me not a very happy invention. “Czarist expansion program” was objected to\footnote{\ie, charged against. ---\ed} the Bolsheviks almost from the first day of the October Revolution. Even a sound workers state would tend toward expansion, and the geographical lines would inevitably coincide with the general lines of the Czarist expansion because revolutions don’t ordinarily change geographical conditions. What we object to about the Kremlin gang is not the expansion and not the geographical direction of the expansion but the bureaucratic, counter-revolutionary methods of the expansion. But at the same time because we as Marxists “look objectively” upon historic happenings we recognize that neither the Czar, nor Hitler, nor Chamberlain had or have the custom of abolishing, in the occupied countries, capitalist property, and this fact, a very progressive one, depends upon another fact; namely, that the October Revolution is not definitely assassinated by the bureaucracy, and that the last is forced by its position to take measures which we must defend in a given situation against imperialist enemies. These progressive measures are, of course, incomparably less important than the general counter-revolutionary activity of the bureaucracy: it is why we find it necessary to overthrow the bureaucracy\dots

The comrades are very indignant about the Stalin-Hitler pact. It is comprehensible. They wish to get revenge on Stalin. Very good. But today we are weak, and we cannot immediately overthrow the Kremlin. Some comrades try then to find a purely verbalistic satisfaction: they strike out from the \USSR\ the title workers’ state, as Stalin deprives a disgraced functionary of the Order of Lenin. I find it, my dear friend, a bit childish. Marxist sociology and hysteria are absolutely irreconcilable.

\signed{With best comradely greetings, \\ CRUX [\emph{Leon Trotsky}]}

\newpage

\begin{letterinfo}
	\textbf{First Published:} Leon Trotsky, \emph{In Defense of Marxism}, New York 1942.
	
	\textbf{Checked against:} Leon Trotsky, \emph{In Defence of Marxism}, London 1966, pp.~27--28.
	
	All footnotes stem from the latter edition.
\end{letterinfo}
