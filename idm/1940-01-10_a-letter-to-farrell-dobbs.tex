\chapterdated[Letter to Farrell Dobbs]{Letter to Farrell Dobbs\footnote{This letter was written by Trotsky in English.}}{January 10, 1940}

\letteraddress{Dear Friend,}

In my article\footnote{See p.~\pageref{1940-01-07_an-open-letter-to-comrade-burnham} ---J.W.} sent for translation to Wright I don’t mention at all two questions:

First, that of bureaucratic conservatism. I believe we discussed the matter a bit with you here. As a political tendency, bureaucratic conservatism represents the material interests of a certain social stratum, namely of the privileged workers’ bureaucracy in the capitalist, especially in the imperialist states and in an incomparably higher degree in the \USSR. It would be fantastic, not to say stupid to search for such roots of the “bureaucratic conservatism” of the majority. If bureaucratism and conservatism are not determined by social conditions then they represent traits in the personal characters of some leaders. Such things occur. But how explain in this case the formation of a faction? Is it a selection of conservative individualities? We have here a psychological and not a political explanation. If we admit (I personally don’t do it) that Cannon for example, has bureaucratic tendencies, then we must inevitably reach the conclusion that the majority supports Cannon \emph{in spite} of this trait and not \emph{because} of it. It signifies that the question of the social foundations of the factional fight isn’t even touched by the minority leaders.

Second, in order to compromise my “defense” of Cannon they insist that I falsely defended Molinier.\footnote{Molinier was one of the leading figures in the french Trotskyist movement. He was expelled from the movement for gross violation of discipline. ---\ed} I am the last to deny that I can commit mistakes of political nature as well as of personal appreciations. But in spite of all the argument is not very profound. I never supported the false theories of Molinier. It was namely a question of his personal character: brutality, lack of discipline and his private financial affairs. Some comrades, among them Vereecken, insisted upon immediate separation from Molinier. I insisted upon the necessity for the organization to try to discipline Molinier. But in 1934 when Molinier tried to replace the party program by “four slogans” and created a paper on this basis, I was among those who proposed his expulsion. This is the whole story. One can be of different opinion about the wisdom of my patient conduct in regard to Molinier, however I was guided of course not by the personal interests of Molinier, but by the interests of the education of the party: our own sections inherited some Comintern venom in the sense that many comrades are inclined to the abuse of such measures as expulsion, split or threats of expulsions and splits. In the case of Molinier as in the case of some American comrades (Field, Weisbord and some others), I was for a more patient attitude. In several cases I succeeded, in several others it was a failure. But I don’t regret at all my more patient attitude towards some doubtful figures in our movement. In any case, my “defense” of them was never a bloc at the expense of principles. If somebody should propose, for example, to expel Comrade Burnham, I would oppose it energetically. But at the same time, I find it necessary to conduct the most strenuous ideological fight against his anti-Marxist conceptions.

\signed{Yours fraternally, \\ \textsc{L.\ Trotsky}}
\signed{Coyoacán, D.F.}

\begin{letterinfo}
	\firstpublished\ Leon Trotsky, \emph{In Defense of Marxism}, New York 1942.
	
	\checkedagainst\ Leon Trotsky, \emph{In Defence of Marxism}, London 1966, p.~121--122.
	
	\footnoteslatter
\end{letterinfo}