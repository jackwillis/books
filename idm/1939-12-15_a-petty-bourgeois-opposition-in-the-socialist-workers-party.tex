\chapterdated{A Petty-Bourgeois Opposition in the Socialist Workers Party}{December 15, 1939}
\label{1939-12-15_a-petty-bourgeois-opposition-in-the-socialist-workers-party}

\noindent
\textsc{It is necessary} to call things by their right names. Now that the positions of both factions in the struggle have become determined with complete clearness, it must be said that the minority of the National Committee is leading a typical petty-bourgeois tendency. Like any petty-bourgeois group inside the socialist movement, the present opposition is characterized by the following features: a disdainful attitude toward theory and an inclination toward eclecticism; disrespect for the tradition of their own organization; anxiety for personal “independence” at the expense of anxiety for objective truth; nervousness instead of consistency; readiness to jump from one position to another; lack of understanding of revolutionary centralism and hostility toward it; and finally, inclination to substitute clique ties and personal relationships for party discipline. Not all the members of the opposition of course manifest these features with identical strength. Nevertheless, as always in a variegated bloc the tinge is given by those who are most distant from Marxism and proletarian policy. A prolonged and serious struggle is obviously before us. I make no attempt to exhaust the problem in this article, but I will endeavor to outline its general features.

\subsection*{Theoretical Skepticism and Eclecticism}

In the January 1939 issue of the \emph{New International} a long article was published by comrades Burnham and Shachtman, ``Intellectuals in Retreat.'' The article, while containing many correct ideas and apt political characterizations, was marred by a fundamental defect if not flaw. While polemicising against opponents who consider themselves---without sufficient reason---above all as proponents of “theory,” the article deliberately did not elevate the problem to a theoretical height. It was absolutely necessary to explain why the American “radical” intellectuals accept Marxism without the dialectic (a clock without a spring). The secret is simple. In no other country has there been such rejection of the class struggle as in the land of “unlimited opportunity.” The denial of social contradictions as the moving force of development led to the denial of the dialectic as the logic of contradictions in the domain of theoretical thought. Just as in the sphere of politics it was thought possible everybody could be convinced of the correctness of a “just” program by means of clever syllogisms and society could be reconstructed through “rational” measures. So in the sphere of theory it was accepted as proved that Aristotelian logic, lowered to the level of ``common sense,'' was sufficient for the solution of all questions.

Pragmatism, a mixture of rationalism and empiricism, became the national philosophy of the United States. The theoretical methodology of Max Eastman is not fundamentally different from the methodology of Henry Ford---both regard living society from the point of view of an “engineer” (Eastman---platonically). Historically the present disdainful attitude toward the dialectic is explained simply by the fact that the grandfathers and great-grandmothers of Max Eastman and others did not need the dialectic in order to conquer territory and enrich themselves. But times have changed and the philosophy of pragmatism has entered a period of bankruptcy just as has American capitalism.

The authors of the article did not show, could not and did not care to show, this internal connection between philosophy and the material development of society, and they frankly explained why.

\begin{quote}
  “The two authors of the present article,” they wrote of themselves, “differ thoroughly on their estimate of the general theory of dialectical materialism, one of them accepting it and the other rejecting it\dots\ There is nothing anomalous in such a situation. Though theory is doubtless always in one way or another related to practice, the relation is not invariably direct or immediate; and as we have before had occasion to remark, human beings often act inconsistently. From the point of view of each of the authors there is in the other a certain such inconsistency between ‘philosophical theory’ and political practice, which might on some occasion lead to decisive concrete political disagreement. But it does not now, nor has anyone yet demonstrated that agreement or disagreement on the more abstract doctrines of dialectical materialism necessarily affects today’s and tomorrow’s concrete political issues-and political parties, programs and struggles are based on such concrete issues. We all may hope that as we go along or when there is more leisure, agreement may also be reached on the more abstract questions. Meanwhile there is fascism and war and unemployment.”
\end{quote}

What is the meaning of this thoroughly astonishing reasoning? Inasmuch as \emph{some} people through a bad method \emph{sometimes} reach correct conclusions, and inasmuch as some people through a correct method \emph{not infrequently} reach incorrect conclusions, therefore\dots\ the method is not of great importance. We shall meditate upon methods sometime when we have more leisure, but now we have other things to do. Imagine how a worker would react upon complaining to his foreman that his tools were bad and receiving the reply: With bad tools it is possible to turn out a good job, and with good tools many people only waste material. I am afraid that such a worker, particularly if he is on piece-work, would respond to the foreman with an unacademic phrase. A worker is faced with refractory materials which show resistance and which because of that compel him to appreciate fine tools, whereas a petty-bourgeois intellectual---alas!---utilizes as his “tools” fugitive observations and superficial generalizations---until major events club him on the head.

To demand that every party member occupy himself with the philosophy of dialectics naturally would be lifeless pedantry. But a worker who has gone through the school of the class struggle gains from his own experience an inclination toward dialectical thinking. Even if unaware of this term, he readily accepts the method itself and its conclusions. With a petty bourgeois it is worse. There are of course petty-bourgeois elements organically linked with the workers, who go over to the proletarian point of view without an internal revolution. But these constitute an insignificant minority. The matter is quite different with the academically trained petty bourgeoisie. Their theoretical prejudices have already been given finished form at the school bench. Inasmuch as they succeeded in gaining a great deal of knowledge both useful and useless without the aid of the dialectic, they believe that they can continue excellently through life without it. In reality they dispense with the dialectic only to the extent they fail to check, to polish, and to sharpen theoretically their tools of thought, and to the extent that they fail to break practically from the narrow circle of their daily relationships. When thrown against great events they are easily lost and relapse again into petty-bourgeois ways of thinking.

Appealing to “inconsistency” as justification for an unprincipled theoretical bloc, signifies giving oneself bad credentials as a Marxist. Inconsistency is not accidental, and in politics it does not appear solely as an individual symptom. Inconsistency usually serves a social function. There are social groupings which cannot be consistent. Petty-bourgeois elements who have not rid themselves of hoary petty-bourgeois tendencies are systematically compelled within a workers’ party to make theoretical compromises with their own conscience.

Comrade Shachtman’s attitude toward the dialectic method, as manifested in the above-quoted argumentation, cannot be called anything but eclectical skepticism. It is clear that Shachtman became infected with this attitude not in the school of Marx but among the petty-bourgeois intellectuals to whom all forms of skepticism are proper.

\subsection*{Warning and Verification}

The article astonished me to such an extent that I immediately wrote to Comrade Shachtman:

\begin{quote}
I have just read the article you and Burnham wrote on the intellectuals. Many parts are excellent. However, the section on the dialectic is the greatest blow that you, personally, as the editor of the \emph{New International}, could have delivered to Marxist theory. Comrade Burnham says: ``I don’t recognize the dialectic.'' It is clear and everybody has to acknowledge it. But you say: ``I recognize the dialectic, but no matter; it does not have the slightest importance.'' Re-read what you wrote. This section is terribly misleading for the readers of the \emph{New International} and the best of gifts to the Eastmans of all kinds. Good! We will speak about it publicly.\nowidow
\end{quote}

My letter was written January 20, some months before the present discussion. Shachtman did not reply until March 5, when he answered in effect that he couldn’t understand why I was making such a stir about the matter. On March 9, I answered Shachtman in the following words:

\begin{quote}
  I did not reject in the slightest degree the possibility of collaboration with the anti-dialecticians, but only the advisability of writing an article together where the question of the dialectic plays, or should play, a very important role. The polemic develops on two planes: political and theoretical. Your political criticism is OK. Your theoretical criticism is insufficient; it stops at the point at which it should just become aggressive. Namely, the task consists of showing that their mistakes (insofar as they are theoretical mistakes) are products of their incapacity and unwillingness to think the things through dialectically. This task could be accomplished with a very serious pedagogical success. Instead of this you declare that dialectics is a private matter and that one can be a very good fellow without dialectic thinking.
\end{quote}

By allying himself in \emph{this} question with the anti-dialectician Burnham, Shachtman deprived himself of the possibility of showing why Eastman, Hook and many others began with a philosophical struggle against the dialectic but finished with a political struggle against the socialist revolution. That is, however, the essence of the question.

The present political discussion in the party has confirmed my apprehensions and warning in an incomparably sharper form than I could have expected, or, more correctly, feared. Shachtman’s methodological skepticism bore its deplorable fruits in the question of the nature of the Soviet state. Burnham began some time ago by constructing purely empirically, on the basis of his immediate impressions, a non-proletarian and non-bourgeois state, liquidating in passing the Marxist theory of the state as the organ of class rule. Shachtman unexpectedly took an evasive position: “The question, you see, is subject to further consideration”; moreover, the sociological definition of the \USSR\ does not possess any direct and immediate significance for our “political tasks” in which Shachtman agrees completely with Burnham. Let the reader again refer to what these comrades wrote concerning the dialectic. Burnham rejects the dialectic. Shachtman seems to accept, but\dots\ the divine gift of “inconsistency” permits them to meet on common political conclusions. \emph{The attitude of each of them toward the nature of the Soviet state reproduces point for point their attitude toward the dialectic.}

In both cases Burnham takes the leading role. This is not surprising: he possesses a method---pragmatism. Shachtman has no method. He adapts himself to Burnham. Without assuming complete responsibility for the anti-Marxian conceptions of Burnham, he defends his bloc of aggression against the Marxian conceptions with Burnham in the sphere of philosophy as well as in the sphere of sociology. In both cases Burnham appears as a pragmatist and Shachtman as an eclectic. This example has this invaluable advantage that the complete parallelism between Burnham’s and Shachtman’s positions upon two different planes of thought and upon two questions of primary importance, will strike the eyes even of comrades who have had no experience in purely theoretical thinking. The method of thought can be dialectic or vulgar, conscious or unconscious, but it exists and makes itself known.

Last January we heard from our authors: “But it does not now, nor has anyone yet demonstrated that agreement or disagreement on the more abstract doctrines of dialectical materialism necessarily affects today’s and tomorrow’s concrete political issues\dots” Nor has anyone yet demonstrated! Not more than a few months passed before Burnham and Shachtman themselves demonstrated that their attitude toward such an “abstraction” as dialectical materialism found its precise manifestation in their attitude toward the Soviet state.

To be sure it is necessary to mention that the difference between the two instances is rather important, but it is of a political and not a theoretical character. In both cases Burnham and Shachtman formed a bloc on the basis of rejection and semi-rejection of the dialectic. But in the first instance that bloc was directed against the opponents of the proletarian party. In the second instance the bloc was concluded against the Marxist wing of their own party. The front of military operations, so to speak, has changed but the weapon remains the same.

True enough, people are often inconsistent. Human consciousness nevertheless tends toward a certain homogeneity. Philosophy and logic are compelled to rely upon this homogeneity of human consciousness and not upon what this homogeneity lacks, that is, inconsistency. Burnham does not recognize the dialectic, but the dialectic recognizes Burnham, that is, extends its sway over him. Shachtman thinks that the dialectic has no importance in political conclusions, but in the political conclusions of Shachtman himself we see the deplorable fruits of his disdainful attitude toward the dialectic. We should include this example in the textbooks on dialectical materialism.

Last year I was visited by a young British professor of political economy, a sympathizer of the Fourth International. During our conversation on the ways and means of realizing socialism, he suddenly expressed the tendencies of British utilitarianism in the spirit of Keynes and others: “It is necessary to determine a clear economic end, to choose the most reasonable means for its realization," etc. I remarked: “I see that you are an adversary of dialectics.” He replied, somewhat astonished: “Yes, I don’t see any use in it.” “However,” I replied to him, “the dialectic enabled me on the basis of a few of your observations upon economic problems to determine what category of philosophical thought you belong to---this alone shows that there is an appreciable value in the dialectic.” Although I have received no word about my visitor since then, I have no doubt that this anti-dialectic professor maintains the opinion that the \USSR\ is not a workers’ state, that unconditional defense of the \USSR\ is an “out-moded” opinion, that our organizational methods are bad, etc. If it is possible to place a given person’s general type of thought on the basis of his relation to concrete practical problems, it is also possible to predict approximately, knowing his general type of thought, how a given individual will approach one or another practical question. That is the incomparable educational value of the dialectical method of thought.

\subsection*{The \textsc{abc} of Materialist Dialectics}

Gangrenous skeptics like Souvarine believe that “nobody knows” what the dialectic is. And there are “Marxists” who kowtow reverently before Souvarine and hope to learn something from him. And these Marxists hide not only in the \emph{Modern Monthly}. Unfortunately a current of Souvarinism exists in the present opposition of the \textsc{swp}. And here it is necessary to warn young comrades: Beware of this malignant infection!
\nowidow

The dialectic is neither fiction nor mysticism, but a science of the forms of our thinking insofar as it is not limited to the daily problems of life but attempts to arrive at an understanding of more complicated and drawn-out processes. The dialectic and formal logic bear a relationship similar to that between higher and lower mathematics.

I will here attempt to sketch the substance of the problem in a very concise form. The Aristotelian logic of the simple syllogism starts from the proposition that “A” is equal to “A.” This postulate is accepted as an axiom for a multitude of practical human actions and elementary generalizations. But in reality “A” is not equal to “A.” This is easy to prove if we observe these two letters under a lens---they are quite different from each other. But, one can object, the question is not of the size or the form of the letters, since they are only symbols for equal quantities, for instance, a pound of sugar. The objection is beside the point; in reality a pound of sugar is never equal to a pound of sugar---a more delicate scale always discloses a difference. Again one can object: but a pound of sugar is equal to itself. Neither is this true---all bodies change uninterruptedly in size, weight, color, etc. They are never equal to themselves. A sophist will respond that a pound of sugar is equal to itself “at any given moment.” Aside from the extremely dubious practical value of this “axiom,” it does not withstand theoretical criticism either. How should we really conceive the word ``moment?'' If it is an infinitesimal interval of time, then a pound of sugar is subjected during the course of that “moment” to inevitable changes. Or is the “moment” a purely mathematical abstraction, that is, a zero of time? But everything exists in time; and existence itself is an uninterrupted process of transformation; time is consequently a fundamental element of existence. Thus the axiom “A” is equal to “A” signifies that a thing is equal to itself if it does not change, that is, if it does not exist.

At first glance it could seem that these “subtleties” are useless. In reality they are of decisive significance. The axiom “A” is equal to “A” appears on one hand to be the point of departure for all our knowledge, on the other hand the point of departure for all the errors in our knowledge. To make use of the axiom “A” is equal to “A” with impunity is possible only within certain \emph{limits}. When quantitative changes in “A” are negligible for the task at hand then we can presume that “A” is equal to “A.” This is, for example, the manner in which a buyer and a seller consider a pound of sugar. We consider the temperature of the sun likewise. Until recently we considered the buying power of the dollar in the same way. But quantitative changes beyond certain limits become converted into qualitative. A pound of sugar subjected to the action of water or kerosene ceases to be a pound of sugar. A dollar in the embrace of a president ceases to be a dollar. To determine at the right moment the critical point where quantity changes into quality is one of the most important and difficult tasks in all the spheres of knowledge including sociology.

Every worker knows that it is impossible to make two completely equal objects. In the elaboration of bearing-brass into cone bearings, a certain deviation is allowed for the cones which should not, however, go beyond certain limits (this is called tolerance). By observing the norms of tolerance, the cones are considered as being equal. (“A” is equal to “A.”) When the tolerance is exceeded the quantity goes over into quality; in other words, the cone bearings become inferior or completely worthless.

Our scientific thinking is only a part of our general practice including techniques. For concepts there also exists “tolerance” which is established not by formal logic issuing from the axiom “A” is equal to “A,” but by dialectical logic issuing from the axiom that everything is always changing. “Common sense” is characterized by the fact that it systematically exceeds dialectical “tolerance.”

Vulgar thought operates with such concepts as capitalism, morals, freedom, workers’ state, etc. as fixed abstractions, presuming that capitalism is equal to capitalism, morals are equal to morals, etc. Dialectical thinking analyzes all things and phenomena in their continuous change, while determining in the material conditions of those changes that critical limit beyond which “A” ceases to be “A”, a workers’ state ceases to be a workers’ state.

The fundamental flaw of vulgar thought lies in the fact that it wishes to content itself with motionless imprints of a reality which consists of eternal motion. Dialectical thinking gives to concepts, by means of closer approximations, corrections, concretizations, a richness of content and flexibility; I would even say a succulence which to a certain extent brings them close to living phenomena. Not capitalism in general, but a given capitalism at a given stage of development. Not a workers’ state in general, but a given workers’ state in a backward country in an imperialist encirclement, etc.

Dialectical thinking is related to vulgar thinking in the same way that a motion picture is related to a still photograph. The motion picture does not outlaw the still photograph but combines a series of them according to the laws of motion. Dialectics does not deny the syllogism, but teaches us to combine syllogisms in such a way as to bring our understanding closer to the eternally changing reality. Hegel in his \emph{Logic} established a series of laws: change of quantity into quality, development through contradictions, conflict of content and form, interruption of continuity, change of possibility into inevitability, etc., which are just as important for theoretical thought as is the simple syllogism for more elementary tasks.

Hegel wrote before Darwin and before Marx. Thanks to the powerful impulse given to thought by the French Revolution, Hegel anticipated the general movement of science. But because it was only an \emph{anticipation}, although by a genius, it received from Hegel an idealistic character. Hegel operated with ideological shadows as the ultimate reality. Marx demonstrated that the movement of these ideological shadows reflected nothing but the movement of material bodies.

We call our dialectic, materialist, since its roots are neither in heaven nor in the depths of our “free will,” but in objective reality, in nature. Consciousness grew out of the unconscious, psychology out of physiology, the organic world out of the inorganic, the solar system out of nebulae. On all the rungs of this ladder of development, the quantitative changes were transformed into qualitative. Our thought, including dialectical thought, is only one of the forms of the expression of changing matter. There is place within this system for neither God, nor Devil, nor immortal soul, nor eternal norms of laws and morals. The dialectic of thinking, having grown out of the dialectic of nature, possesses consequently a thoroughly materialist character.

Darwinism, which explained the evolution of species through quantitative transformations passing into qualitative, was the highest triumph of the dialectic in the whole field of organic matter. Another great triumph was the discovery of the table of atomic weights of chemical elements and further the transformation of one element into another.

With these transformations (species, elements, etc.) is closely linked the question of classification, equally important in the natural as in the social sciences. Linmeus’ system (18th century), utilizing as its starting point the immutability of species, was limited to the description and classification of plants according to their external characteristics. The infantile period of botany is analogous to the infantile period of logic, since the forms of our thought develop like everything that lives. Only decisive repudiation of the idea of fixed species, only the study of the history of the evolution of plants and their anatomy prepared the basis for a really scientific classification.

Marx, who in distinction from Darwin was a conscious dialectician, discovered a basis for the scientific classification of human societies in the development of their productive forces and the structure of the relations of ownership which constitute the anatomy of society. Marxism substituted for the vulgar descriptive classification of societies and states, which even up to now still flourishes in the universities, a materialistic dialectical classification. Only through using the method of Marx is it possible correctly to determine both the concept of a workers’ state and the moment of its downfall.

All this, as we see, contains nothing “metaphysical” or “scholastic,” as conceited ignorance affirms. Dialectic logic expresses the laws of motion in contemporary scientific thought. The struggle against materialist dialectics on the contrary expresses a distant past, conservatism of the petty bourgeoisie, the self-conceit of university routinists and\dots\ a spark of hope for an after-life.

\subsection*{The Nature of the \USSR}

The definition of the \USSR\ given by comrade Burnham, “not a workers’ and not a bourgeois state,” is purely negative, wrenched from the chain of historical development, left dangling in mid-air, void of a single particle of sociology and represents simply a theoretical capitulation of pragmatism before a \emph{contradictory} historical phenomenon.

If Burnham were a dialectical materialist, he would have probed the following three questions:
\begin{enumerate}
  \item What is the historical origin of the \USSR?
  \item What changes has this state suffered during its existence?
  \item Did these changes pass from the quantitative stage to the qualitative? That is, did they create a historically necessary domination by a new exploiting class? \nowidow
\end{enumerate}
Answering these questions would have forced Burnham to draw the only possible conclusion---the \USSR\ is still a degenerated workers’ state.

The dialectic is not a magic master key for all questions. It does not replace concrete scientific analysis. But it directs this analysis along the correct road, securing it against sterile wanderings in the desert of subjectivism and scholasticism.

Bruno R.\ places both the Soviet and fascist regimes under the category of “bureaucratic collectivism,” because the \USSR, Italy and Germany are all ruled by bureaucracies; here and there are the principles of planning; in one case private property is liquidated, in another limited, etc. Thus on the basis of the \emph{relative} similarity of \emph{certain} external characteristics of \emph{different} origin, of \emph{different} specific weight, of \emph{different} class significance, a fundamental \emph{identity} of social regimes is constructed, completely in the spirit of bourgeois professors who construct categories of “controlled economy,” “centralized state,” without taking into consideration whatsoever the class nature of one or the other. Bruno R.\ and his followers, or semi-followers like Burnham, at best remain in the sphere of social classification on the level of Linnaeus in whose justification it should be remarked however that he lived before Hegel, Darwin and Marx.

Even worse and more dangerous, perhaps, are those eclectics who express the idea that the class character of the Soviet state “does not matter,” and that the direction of our policy is determined by “the character of the war.” As if the war were an independent super-social substance; as if the character of the war were not determined by the character of the ruling class, that is, by the same social factor that also determines the character of the state. Astonishing how easily some comrades forget the \textsc{abc}s of Marxism under the blows of events.

It is not surprising that the theoreticians of the opposition who reject dialectic thought capitulate lamentably before the contradictory nature of the \USSR. However the contradiction between the social basis laid down by the revolution, and the character of the caste which arose out of the degeneration of the revolution is not only an irrefutable historical fact but also a motor force. In our struggle for the overthrow of the bureaucracy we base ourselves on this contradiction. Meanwhile some ultra-lefts have already reached the ultimate absurdity by affirming that it is necessary to sacrifice the social structure of the \USSR\ in order to overthrow the Bonapartist oligarchy! They have no suspicion that the \USSR\ minus the social structure founded by the October Revolution would be a fascist regime.

\subsection*{Evolution and Dialectics}

Comrade Burnham will probably protest that as an evolutionist he is interested in the development of society and state forms not less than we dialecticians. We will not dispute this. Every educated person since Darwin has labeled himself an “evolutionist.” But a real evolutionist must apply the idea of evolution to his own forms of thinking. Elementary logic, founded in the period when the idea of evolution itself did not yet exist, is evidently insufficient for the analysis of evolutionary processes. Hegel’s logic is the logic of evolution. Only one must not forget that the concept of “evolution” itself has been completely corrupted and emasculated by university professors and liberal writers to mean peaceful “progress.” Whoever has come to understand that evolution proceeds through the struggle of antagonistic forces; that a slow accumulation of changes at a certain moment explodes the old shell and brings about a catastrophe, revolution; whoever has learned finally to apply the general laws of evolution to thinking itself, he is a dialectician, as distinguished from vulgar evolutionists. Dialectic training of the mind, as necessary to a revolutionary fighter as finger exercises to a pianist, demands approaching all problems as \emph{processes} and not as \emph{motionless categories}. Whereas vulgar evolutionists, who limit themselves generally to recognizing evolution in only certain spheres, content themselves in all other questions with the banalities of “common sense.”

The American liberal, who has reconciled himself to the existence of the \USSR, more precisely to the Moscow bureaucracy, believes, or at least believed until the Soviet-German pact, that the Soviet regime on the whole is a “progressive thing,” that the repugnant features of the bureaucracy (“well naturally they exist!”) will progressively slough away and that peaceful and painless “progress” is thus assured.

A vulgar petty-bourgeois radical is similar to a liberal “progressive” in that he takes the \USSR\ as a whole, failing to understand its internal contradictions and dynamics. When Stalin concluded an alliance with Hitler, invaded Poland, and now Finland, the vulgar radicals triumphed; the identity of the methods of Stalinism and fascism was proved They found themselves in difficulties however when the new authorities invited the population to expropriate the landowners and capitalists-they had not foreseen this possibility at all! Meanwhile the social revolutionary measures, carried out via bureaucratic military means, not only did not disturb \emph{our}, dialectic, definition of the \USSR\ as a degenerated workers’ state, but gave it the most incontrovertible corroboration. Instead of utilizing this triumph of Marxian analysis for persevering agitation, the petty-bourgeois oppositionists began to shout with criminal light-mindedness that the events have refuted our prognosis, that our old formulas are no longer applicable, that new words are necessary. What words? They haven’t decided yet themselves.

\subsection*{Defense of the \USSR}

We began with philosophy and then went to sociology. It became clear that in both spheres, of the two leading personalities of the opposition, one had taken an anti-Marxian, the other an eclectic position. If we now consider politics, particularly the question of the defense of the \USSR, we will find that just as great surprises await us.

The opposition discovered that our formula of “unconditional defense of the \USSR,” the formula of our program, is “vague, abstract and out-moded (!?).” Unfortunately they do not explain under what future “conditions” they are ready to defend the conquests of the revolution. In order to give at least an ounce of sense to their new formula, the opposition attempts to represent the matter as if up to now we had “unconditionally” defended the international policy of the Kremlin government with its Red Army and \GPU. Everything is turned upside down! In reality for a long time we have not defended the Kremlin’s international policy, not even conditionally, particularly since the time that we openly proclaimed the necessity of crushing the Kremlin oligarchy through insurrection! A wrong policy not only mutilates the current tasks but also compels one to represent his own past in a false light.

In the above-quoted article in the \emph{New International}, Burnham and Shachtman cleverly labeled the group of disillusioned intellectuals “The League of Abandoned Hopes,” and persistently asked what would be the position of this deplorable League in case of military conflict between a capitalistic country and the Soviet Union. “We take this occasion, therefore,” they wrote. “to demand from Hook, Eastman and Lyons \emph{unambiguous} declarations on the question of defense of the Soviet Union from attack by Hitler or Japan---or for that matter by England\dots” Burnham and Shachtman did not lay down any “conditions,” they did not specify any “concrete” circumstances, and at the same time they demanded an “unambiguous” reply. “\dots Would the League (of Abandoned Hopes) also refrain from taking a position or would it declare itself neutral?” they continued; “In a word, is it for the defense of the Soviet Union from imperialist attack, \emph{regardless and in spite of the Stalinist regime?}” (My emphasis.) A quotation to marvel at! And this is exactly what our program declares. Burnham and Shachtman in January 1939 stood in favor of unconditional defense of the Soviet Union and defined the significance of unconditional defense entirely correctly as “regardless and in spite of the Stalinist regime.” And yet this article was written when the experience of the Spanish Revolution had already been drained to completion. Comrade Cannon is absolutely right when he says that the role of Stalinism in Spain was incomparably more criminal than in Poland or Finland. In the first case the bureaucracy through hangman s methods strangled a socialist revolution. In the second case it gives an impulse to the socialist revolution through bureaucratic methods. Why did Burnham and Shachtman themselves so unexpectedly shift to the position of the “League of Abandoned Hopes”? Why? We cannot consider Shachtman’s super-abstract references to the “concreteness of events” as an explanation. Nevertheless, it is not difficult to find an explanation. The Kremlin’s participation in the Republican camp in Spain was supported by the bourgeois democrats all over the world. Stalin’s work in Poland and Finland is met with frantic condemnation from the same democrats. In spite of all its noisy formulas the opposition happens to be a reflection inside the Socialist Workers Party of the moods of the “left” petty bourgeoisie. This fact unfortunately is incontrovertible.

“Our subjects,” wrote Burnham and Shachtman about the League of Abandoned Hopes, “take great pride in believing that they are contributing something ‘fresh,’ that they are ‘re-evaluating in the light of new experiences,’ that they are ‘not dogmatists’ (’conservatives’? ---L.T.) who refuse to re-examine their ‘basic assumption’, etc. What a pathetic self-deception! None of them has brought to light any new facts, given any new understanding of the present or future.” Astonishing quotation! Should we not add a new chapter to their article, ``Intellectuals in Retreat?'' I offer comrade Shachtman my collaboration\dots

How is it possible that outstanding individuals like Burnham and Shachtman, unconditionally devoted to the cause of the proletariat, could become so frightened of the not so frightening gentlemen of the League of Abandoned Hopes! On the purely theoretical plane the explanation in respect to Burnham rests in his incorrect method, in respect to Shachtman in his disregard for method. Correct method not only facilitates the attainment of a correct conclusion, but, connecting every new conclusion with the preceding conclusions in a consecutive chain, fixes the conclusions in one’s memory. If political conclusions are made empirically, if inconsistency is proclaimed as a kind of advantage, then the Marxian system of politics is invariably replaced by impressionism---in so many ways characteristic of petty-bourgeois intellectuals. Every new turn of events catches the empiricist-impressionist unawares, compels him to forget what he himself wrote yesterday, and produces a consuming desire for new formulas before new ideas have appeared in his head.

\subsection*{The Soviet-Finnish War}

The resolution of the opposition upon the question of the Soviet-Finnish war is a document which could be signed, perhaps with slight changes, by the Bordigists, Vereecken, Sneevliet, Fenner Brockway, Marceau Pivert and the like, but in no case by Bolshevik-Leninists. Based exclusively on features of the Soviet bureaucracy and on the mere fact of the “invasion” the resolution is void of the slightest social content. It places Finland and the \USSR\ on the same level and unequivocally “condemns, rejects and opposes \emph{both} governments and their armies.” Having noticed, however, that something was not in order, the resolution unexpectedly and without any connection with the text adds: “In the application (!) of this perspective, the Fourth International will, of course (how marvelous is this “of course”), take into account (!) the differing economic relations in Finland and Russia.” Every word is a pearl. By “concrete” circumstances our lovers of the “concrete” mean the military situation, the moods of the masses and in the third place the opposed economic regimes. As to just how these three “concrete” circumstances will be “taken into account,” the resolution doesn’t give the slightest inkling. If the opposition opposes equally “both governments and their armies” in relation to this war, how will it “take how will it “take into account” the differences in the military situation and the social regimes? Definitely nothing of this is comprehensible.

In order to punish the Stalinists for their unquestionable crimes, the resolution, following the petty-bourgeois democrats of all shadings, does not mention by so much as a word that the Red Army in Finland expropriates large land-owners and introduces workers’ control while preparing for the expropriation of the capitalists.

Tomorrow the Stalinists will strangle the Finnish workers. But now they are giving---they are compelled to give---a tremendous impulse to the class struggle in its sharpest form. The leaders of the opposition construct their policy not upon the “concrete” process that is taking place in Finland, but upon democratic abstractions and noble sentiments.

The Soviet-Finnish war is apparently beginning to be supplemented by a civil war in which the Red Army finds itself at the given stage in the same camp as the Finnish petty peasants and the workers, while the Finnish army enjoys the support of the owning classes, the conservative labor bureaucracy and the Anglo-Saxon imperialists. The hopes which the Red Army awakens among the Finnish poor will, unless international revolution intervenes, prove to be an illusion; the collaboration of the Red Army with the poor will be only temporary; the Kremlin will soon turn its weapons against the Finnish workers and peasants. We know all this now and we say it openly as a warning. But in this “concrete” civil war that is taking place on Finnish territory, what “concrete” position must the “concrete” partisans of the Fourth International take? If they fought in Spain in the Republican camp in spite of the fact that the Stalinists were strangling the socialist revolution, all the more must they participate in Finland in that camp where the Stalinists are compelled to support the expropriation of the capitalists.

Our innovators cover the holes in their position with violent phrases. They label the policy of the \USSR\ “imperialist.” Vast enrichment of the sciences! Beginning from now on both the foreign policy of finance-capital and the policy of exterminating finance-capital will be called imperialism. This will help significantly in the clarification and class education of the workers. But simultaneously---will shout the, let us say, very hasty Stanley---the Kremlin supports the policy of finance-capital in Germany! This objection is based on the substitution of one problem for another, in the dissolving of the concrete into the abstract (the usual mistake of vulgar thought).

If Hitler tomorrow were forced to send arms to the insurrectionary Indians, must the revolutionary German workers oppose this concrete action by strikes or sabotage? On the contrary they must make sure that the insurrectionists receive the arms as soon as possible. We hope that \emph{this} is clear to Stanley. But this example is purely hypothetical. We used it in order to show that even a fascist government of finance-capital can under certain conditions be forced to support a \emph{national} revolutionary movement (in order to attempt to strangle it the next day). Hitler would never under any circumstances support a proletarian revolution for instance in France. As for the Kremlin it is at the present time forced---and this is not a hypothetical but a real situation---to provoke a social revolutionary movement in Finland (in order to attempt to strangle it politically tomorrow). To cover a given social revolutionary movement with the all-embracing term of imperialism only because it is provoked, mutilated and at the same time strangled by the Kremlin merely testifies to one’s theoretical and political poverty.

It is necessary to add that the stretching of the concept of “imperialism” lacks even the attraction of novelty. At present not only the “democrats” but also the bourgeoisie of the democratic countries describe Soviet policy as imperialist. The aim of the bourgeoisie is transparent---to erase the social contradictions between capitalistic and Soviet expansion, to hide the problem of property, and in this way to help genuine imperialism. What is the aim of Shachtman and the others? They don’t know themselves. Their terminological novelty objectively leads them away from the Marxian terminology of the Fourth International and brings them close to the terminology of the “democrats.” This circumstance, alas, again testifies to the opposition’s extreme sensitivity to the pressure of petty-bourgeois public opinion.

\subsection*{“The Organizational Question”}

From the ranks of the opposition one begins to hear more frequently: “The Russian question isn’t of any decisive importance in and of itself; the most important task is to change the party regime.” Change in regime, it is necessary to understand, means a change in leadership, or more precisely, the elimination of Cannon and his close collaborators from directing posts. These clamorous voices demonstrate that the tendency towards a struggle against “Cannon’s faction” preceded that “concreteness of events” to which Shachtman and others refer in explaining their change of position. At the same time these voices remind us of a whole series of past oppositional groups who took up a struggle on different occasions; and who, when the principled basis began to crumble under their feet, shifted to the so-called “organizational question”---the case was identical with Molinier, Sneevliet, Vereecken and many others. As disagreeable as these precedents may appear, it is impossible to pass over them.

It would be incorrect, however, to believe that the shifting of the struggle to the “organizational question” represents a simple “maneuver” in the factional struggle. No, the inner feelings of the opposition tell them, in truth, however confusedly, that the issue concerns not only the “Russian problem” but rather the entire approach to political problems in general, including also the methods of building the party. And this is in a certain sense correct.

We too have attempted above to prove that the issue concerns not only the Russian problem but even more the opposition’s method of thought, which has its social roots. The opposition is under the sway of petty-bourgeois moods and tendencies. This is the essence of the whole matter.

We saw quite clearly the ideological influence of another class in the instances of Burnham (pragmatism) and Shachtman (eclecticism). We did not take into consideration other leaders such as comrade Abern because he generally does not participate in principled discussions, limiting himself to the plane of the “organizational question.” This does not mean, however, that Abern has no importance. On the contrary, it is possible to say that Burnham and Shachtman are the amateurs of the opposition, while Abern is the unquestionable professional. Abern and only he, has his own traditional group which grew out of the old Communist Party and became bound together during the first period of the independent existence of the “Left Opposition.” All the others who hold various reasons for criticism and discontent cling to this group.

Any serious factional fight in a party is always in the final analysis a reflection of the class struggle. The Majority faction established from the beginning the ideological dependence of the opposition upon petty-bourgeois democracy. The opposition, on the contrary, precisely because of its petty-bourgeois character, does not even attempt to look for the social roots of the hostile camp.

The opposition opened up a severe factional fight which is now paralyzing the party at a very critical moment. That such a fight could be justified and not pitilessly condemned, very serious and deep foundations would be necessary. For a Marxist, such foundations can have only a class character. Before they began their bitter struggle, the leaders of the opposition were obligated to ask themselves this question: What non-proletarian class influence is reflected in the majority of the National Committee? Nevertheless, the opposition has not made the slightest attempt at such a class evaluation of the divergences. It sees only “conservatism,” “errors,” “bad methods,” and similar psychological, intellectual and technical deficiencies. The opposition is not interested in the class nature of the opposition faction, just as it is not interested in the class nature of the \USSR. This fact alone is sufficient to demonstrate the petty-bourgeois character of the opposition, with its tinge of academic pedantry and journalistic impressionism.

In order to understand what classes or strata are reflected in the factional fight, it is necessary to study the fight of both factions historically. Those members of the opposition who affirm that the present fight has “nothing in common” with the old factional struggles, demonstrate once again their superficial attitude toward the life of their own party. The fundamental core of the opposition is the same which three years ago grouped itself around Muste and Spector. The fundamental core of the Majority is the same which grouped itself around Cannon. Of the leading figures only Shachtman and Burnham have shifted from one camp to the other. But these personal shifts, important though they might be, do not change the general character of the two groups. I will not go into the historical sequence of the faction fight, referring the reader to the in every respect excellent article by Joseph Hansen, ``Organizational Methods and Political Principles.''

If we subtract everything accidental, personal and episodical, if we reduce the present groupings in struggle to their fundamental political types, then indubitably the struggle of comrade Abern against comrade Cannon has been the most consistent. In this struggle Abern represents a propagandistic group, petty-bourgeois in its social composition, united by old personal ties and having almost the character of a family. Cannon represents the proletarian party in process of formation. The historical right in this struggle---independent of what errors and mistakes might have been made---rests wholly on the side of Cannon.

When the representatives of the opposition raised the hue and cry that the “leadership is bankrupt,” “the prognoses did not turn out to be correct,” “the events caught us unawares,” “it is necessary to change our slogans,” all this without the slightest effort to think the questions through seriously, they appeared fundamentally as party defeatists. This deplorable attitude is explained by the irritation and fright of the old propagandistic circle before the new tasks and the new party relations. The sentimentality of personal ties does not want to yield to the sense of duty and discipline. The task that stands before the party is to break up the old clique ties and to dissolve the best elements of the propagandistic past in the proletarian party. It is necessary to develop such a spirit of party patriotism that nobody dare say: “The reality of the matter is not the Russian question but that we feel more easy and comfortable under Abern’s leadership than under Cannon’s.”

I personally did not arrive at this conclusion yesterday. I happened to have expressed it tens and hundreds of times in conversations with members of Abern’s group. I invariably emphasized the petty-bourgeois composition of this group. I insistently and repeatedly proposed to transfer from membership to candidacy such petty-bourgeois fellow-travelers as proved incapable of recruiting workers for the party. Private letters, conversations and admonitions as has been shown by subsequent events have not led to anything---people rarely learn from someone else’s experience. The antagonism between the two party layers and the two periods of its development rose to the surface and took on the character of bitter factional struggle. Nothing remains but to give an opinion, clearly and definitely, to the American section and the whole International. “Friendship is friendship but duty is duty”---says a Russian proverb.

The following question can be posed: If the opposition is a petty-bourgeois tendency does that signify further unity is impossible? Then how reconcile the petty-bourgeois tendency with the proletarian? To pose the question like this means to judge one-sidedly, undialectically and thus falsely. In the present discussion the opposition has clearly manifested its petty-bourgeois features. But this does not mean that the opposition has no other features. The majority of the members of the opposition are deeply devoted to the cause of the proletariat and are capable of learning. Tied today to a petty-bourgeois milieu they can tomorrow tie themselves to the proletariat. The inconsistent ones, under the influence of experience, can become more consistent. When the party embraces thousands of workers even the professional factionalists can re-educate themselves in the spirit of proletarian discipline. It is necessary to give them time for this. That is why comrade Cannon’s proposal to keep the discussion free from any threats of split, expulsions, etc., was absolutely correct and in place.

Nevertheless, it remains not less indubitable that if the party as a whole should take the road of the opposition it could suffer complete destruction. The present opposition is incapable of giving the party Marxian leadership. The majority of the present National Committee expresses more consistently, seriously and profoundly the proletarian tasks of the party than the Minority. Precisely because of this the Majority can have no interest in directing the struggle toward split-correct ideas will win. Nor can the healthy elements of the opposition wish a split---the experience of the past demonstrates very clearly that all the different kinds of improvised groups who split from the Fourth International condemned themselves to sterility and decomposition. That is why it is possible to envisage the next party convention without any fear. It will reject the anti-Marxian novelties of the opposition and guarantee party unity.

\begin{letterinfo}
  \firstpublished\ Leon Trotsky, \emph{In Defense of Marxism}, New York 1942.
  
  \checkedagainst\ Leon Trotsky, \emph{In Defence of Marxism}, London 1966, pp.~56--80.
\end{letterinfo}