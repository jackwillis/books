\chapterdated[Letter to Joseph Hansen]{Letter to Joseph Hansen\footnote{This letter was written by Trotsky in English.}}{January 18, 1940}

\letteraddress{Dear Joe,}

My article against Shachtman\footnote{See p.~\pageref{1940-01-24_from-a-scratch-to-the-danger-of-gangrene} ---J.W.} is already written. I need now to polish it for two days, and I will try to use some of the your quotations. But I wish to speak here about another more important question. Some of the leaders of the opposition are preparing a split; whereby the represent the opposition in the future as a persecuted minority. It is very characteristic of their state of mind. I believe we must answer them approximately as follows:

\begin{quote}
  You are already afraid of our future repressions? We propose to you mutual guarantees for the future minority, independently of who might be this minority, you or we. These guarantees could be formulated in four points:
  
  \begin{enumerate}
  	\item No prohibition of factions;
  	\item No other restrictions on factional activity than those dictated by the necessity for common action;
  	\item The official publications must represent, of course, the line established by the new convention;
  	\item The future minority can have, if it wishes, an internal bulletin destined for party members, or a common discussion bulletin with the majority.
  \end{enumerate}
\end{quote}

The continuation of discussion bulletins immediately after a long discussion and a convention is, of course, not a rule but an exception, a rather deplorable one. But we are not bureaucrats at all. We don’t have immutable rules. We are dialecticians also in the organizational field. If we have in the party an important minority which is dissatisfied with the decisions of the convention, it is incomparably more preferable to legalize the discussion after the convention than to have a split.

We can go, if necessary, even further and propose to them to publish, under the supervision of the new National Committee, special discussion symposiums, not only for party members, but for the public in general. We should go as far as possible in this respect in order to disarm their at least premature complaints and handicap them in provoking a split.

For my part I believe that the prolongation of the discussion, if it is channelized by the good will of both sides, can only serve in the present conditions the education of the party. I believe that the majority should make these propositions officially in the National Committee in a written form. Whatever might be their answer, the party could only win.

\signed{With best greetings, \\ \textsc{Cornell} [Leon Trotsky]}
\signed{Coyoacán, D.F.}

\begin{letterinfo}
	\firstpublished\ Leon Trotsky, \emph{In Defense of Marxism}, New York 1942.
	
	\checkedagainst\ Leon Trotsky, \emph{In Defence of Marxism}, London 1966, p.~126--127.
	
	\footnoteslatter
\end{letterinfo}