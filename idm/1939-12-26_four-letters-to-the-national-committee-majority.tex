\chapter[Four Letters to the National Committee Majority]{Four Letters to the National Committee Majority\endnote{These letters were written by Trotsky in English.}}

\addcontentsline{toc}{section}{December 26, 1939}
\markboth{December 26, 1939}{Four Letters to the National Committee Majority}
\letterdate{December 26, 1939}
\letteraddress{Dear Friends,}

I was previously disposed in favor of transmitting the discussion in the \emph{Socialist Appeal} and the \emph{New International}, but I must recognize that your arguments are very serious especially in connection with the arguments of Comrade Burnham.\endnote{The minority of the National Committee demanded that the discussion be carried in the \emph{Socialist Appeal} and the \emph{New International}. This was rejected by the majority. ---\ed}

The \emph{New International} and \emph{Socialist Appeal} are not instruments of the discussion under the control of a special discussion committee, but rather instruments of the Party and its National Committee. In the discussion bulletin the opposition can ask for equal rights with the majority, but the official party publications have the duty to defend the point of view of the Party and the Fourth International until they are changed. A discussion on the pages of the official party publications can be conducted only within the limits established by the majority of the National Committee. It is so self-evident that arguments are not necessary.

The permanent juridical guarantees for the minority surely are not borrowed from the Bolshevik experience. But they are also not an invention of Comrade Burnham; the French Socialist Party has had for a long time such constitutional guarantees which correspond completely to the spirit of envious literary and parliamentary cliques, but never prevent the subjugation of the workers by the coalition of these cliques.

The organizational structure of the proletarian vanguard must be subordinated to the positive demands of the revolutionary fight and not to the negative guarantees against their degeneration. If the Party is not fit for the needs of the socialist revolution, it would degenerate in spite of the wisest juridical stipulations. On the organizational field, Burnham shows a complete lack of revolutionary conception of the party, as he showed it on the political field in the small but very significant question of the Dies Committee. In both cases he proposes a purely negative attitude, as, in the question of the Soviet State, he gave a purely negative definition. It is not sufficient to dislike the capitalist society (a negative attitude), it is necessary to accept all the practical conclusions of a social revolutionary conception. Alas, this is not the case of Comrade Burnham.

My practical conclusions?

First, it is necessary to officially condemn before the Party the attempt to annihilate the party line by putting the party program on the same level with every innovation not accepted by the Party.

Second, if the National Committee finds it necessary to devote one issue of the \emph{New International} to the discussion (I don’t propose it now), it should be done in such a way that the reader sees where the party position is and where the attempt at revision, and that the last word remain with the majority and not with the opposition.

Third, if the internal bulletins are not sufficient, it would be possible to publish a special symposium of articles devoted to the agenda of the convention.

The fullest loyalty in the discussion, but not the slightest concession to the petty bourgeois, anarchistic spirit!

\signed{W. \textsc{Rork} [Leon Trotsky]}
\signed{Coyoacán, D.F.}

\triast
\vspace*{-1 \baselineskip}

\addcontentsline{toc}{section}{December 27, 1939}
\markboth{December 27, 1939}{Four Letters to the National Committee Majority}
\letterdate{December 27, 1939}
\letteraddress{Dear Friends,}

I must confess that your communication about the insistences of Comrades Burnham and Shachtman concerning the publication of the controversial articles in the \emph{New International} and the \emph{Socialist Appeal} in the first moment surprised me. What can be the reason, I asked myself. That they feel so sure of their position is completely excluded. Their arguments are of a very primitive nature, the contradictions between themselves are sharp and they cannot help but feel that the majority represents the tradition and the Marxist doctrine. They can’t hope to issue victorious from a theoretical fight; not only Shachtman and Abern but also Burnham must understand it. What is then the source of their thirst for publicity? The explanation is very simple: they are impatient to justify themselves before the democratic public opinion, to shout to all the Eastmans, Hooks and the others that they, the opposition, are not so bad as we. This inner necessity must be especially imperative with Burnham. It is the same kind of inner capitulation which we observed in Zinoviev and Kamenev on the eve of the October Revolution and by many “internationalists” under the pressure of the patriotic war wave. If we make abstractions from all the individual peculiarities, accidents or misunderstandings and errors, we have before us the first social patriotic sin-fall in our own party. You established this fact correctly from the beginning, but it appears to me in full clarity only now after they proclaimed their wish to announce---as the \textsc{poum}ists, Pivertists and many others---that they are not so bad as the “Trotskyites.”

This consideration is a supplementary argument against any concession to them in this field. Under the given conditions, we have the full right to say to them: you must wait for the verdict of the party and not appeal before the verdict is pronounced to the democratic patriotic judges.

I considered the question previously too abstractly, namely, only from the point of view of the theoretical fight, and from this point of view I agree completely with Comrade Goldman that we could only win. But the larger political criterion indicates that we should eliminate the premature intervention of the democratic patriotic factor in our inner party fight and that the opposition should reckon in the discussion only on their own strength as the majority does. Under these conditions the test and the selection of different elements of the opposition would have a more efficient character and the results would be more favorable for the party.

Engels spoke one time about the mood of the enraged petty-bourgeois. It seems to me that a trace of this mood can be found in the ranks of the opposition. Yesterday many of them were hypnotized by the Bolshevik tradition. They never absorbed it innerly but didn’t dare to challenge it openly. But Shachtman and Abern gave them this kind of courage and now they openly enjoy the mood of the enraged petty-bourgeois. This is the impression, for example, which I received from the last articles and letters of Stanley. He has lost totally his self-criticism and believes sincerely that every inspiration which visits his brain is worthy of being proclaimed and printed if only it is directed against the program and tradition of the party. The crime of Shachtman and Abern consists especially in having provoked such an explosion of petty-bourgeois self-satisfaction.

\signed{\textsc{W. Rork} [Leon Trotsky]}
\signed{Coyoacán, D.F.}

\begin{postscriptum}
  P.S.---It is absolutely sure that the Stalinist agents are working in our midst with the purpose to shar\-pen the discussion and provoke a split. It would be necessary to check many factional “fighters” from this special point of view.
\end{postscriptum}

\triast
\vspace*{-1 \baselineskip}

\addcontentsline{toc}{section}{January 3, 1940}
\markboth{January 3, 1940}{Four Letters to the National Committee Majority}
\letterdate{January 3, 1940}
\letteraddress{Dear Friends,}

I received the two documents of the opposition\endnote{These documents were ``The War and Bureaucratic Conservatism'' and ``What Is at Issue in the Dispute on the Russian Question.'' ---\ed}, studied that on bureaucratic conservatism and am now studying the second on the Russian question. What lamentable writings! It is difficult to find a sentence expressing a correct idea or placing a correct idea in the correct place. Intelligent and even talented people occupied an evidently false position and push themselves more and more into a blind alley.

The phrase of Abern about the “split” can have two senses: either he wishes to frighten you with a split as he did during the entry discussion\endnote{When the American Trotskyists were discussing entrance into the Socialist Party in the first months of 1936 Abern was bitterly opposed to the move. ---\ed} or he wishes really to commit political suicide. In the first case, he will of course not prevent our giving a Marxist appreciation of the opposition politics. In the second case, nothing can be done; if an adult person wishes to commit suicide it is difficult to hinder him.

The reaction of Burnham is a brutal challenge to all Marxists. If dialectics is a religion and if it is true that religion is the opium of the people, how can he refuse to fight for liberating his own party from this venom? I am now writing an open letter to Burnham on this question. I don’t believe that the public opinion of the Fourth International would permit the editor of the theoretical Marxist magazine to limit himself to rather cynical aphorisms about the foundation of scientific socialism. In any case, I will not rest until the anti-Marxist conceptions of Burnham are unmasked to the end before the Party and the International. I hope to send the open letter, at least the Russian text, the day after tomorrow.

Simultaneously, I am writing an analysis of the two documents. Excellent is the explanation why they agree to disagree about the Russian question.

I grit my teeth upon losing my time in the reading of these absolutely stale documents. The errors are so elementary that it is necessary to make an effort to remember the necessary argument from the \textsc{abc} of Marxism.

\signed{\textsc{W. Rork} [Leon Trotsky] \\ {\vskip 0.25\baselineskip} Coyoacán, D.F.}

\triast
\vspace*{-1 \baselineskip}

\addcontentsline{toc}{section}{January 4, 1940}
\markboth{January 4, 1940}{Four Letters to the National Committee Majority}
\letterdate{January 4, 1940}

\vspace*{-1\baselineskip}

\letteraddress{Dear Friends,}

I enclose a copy of my letter to Shachtman\endnote{See p.~\pageref{1939-12-20_a-letter-to-max-shachtman} ---J.W.} which I sent more than two weeks ago. Shachtman didn’t even answer me. It shows the mood into which he has pushed himself by his unprincipled fight. He makes a bloc with the anti-Marxist Burnham and he refuses to answer my letters concerning this bloc. The fact in itself is of course of doubtful importance but it has an indisputable symptomatic vein. This is my reason for sending you a copy of my letter to Shachtman.

\enlargethispage{\baselineskip}

\signed{With best wishes, \\ \textsc{L. Trotsky}}
\signed{Coyoacán, D.F.}

\begin{letterinfo}
	\firstpublished\ Leon Trotsky, \emph{In Defense of Marxism}, New York 1942.
	
	\checkedagainst\ Leon Trotsky, \emph{In Defence of Marxism}, London 1966, pp.~83--88.
\end{letterinfo}

\printendnotes[modest-chapter-endnotes]