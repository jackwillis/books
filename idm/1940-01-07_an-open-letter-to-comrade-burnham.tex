\chapterdated{An Open Letter to Comrade Burnham}{January 7, 1940}

\letteraddress{Dear Comrade:}

You have expressed as your reaction to my article on the petty-bourgeois opposition,\footnote{See p.~\pageref{1939-12-15_a-petty-bourgeois-opposition-in-the-socialist-workers-party} ---J.W.} I have been informed, that you do not intend to argue over the dialectic with me and that you will discuss only the “concrete questions.” “I stopped arguing about religion long ago,” you added ironically. I once heard Max Eastman voice this same sentiment.

\subsection*{Is There Logic in Identifying Logic with Religion?}

As I understand this, your words imply that the dialectic of Marx, Engels and Lenin belongs to the sphere of religion. What does this assertion signify? The dialectic, permit me to recall once again, is the \emph{logic of evolution}. Just as a machine shop in a plant supplies instruments for all departments, so logic is indispensable for all spheres of human knowledge. If you do not consider logic in general to be a religious prejudice (sad to say, the self-contradictory writings of the opposition incline one more and more toward this lamentable idea), then just which logic do you accept? I know of two systems of logic worthy of attention: the logic of Aristotle (formal logic) and the logic of Hegel (the dialectic). Aristotelian logic takes as its starting point immutable objects and phenomena. The scientific thought of our epoch studies all phenomena in their origin, change and disintegration. Do you hold that the progress of the sciences, including Darwinism, Marxism, modern physics, chemistry, etc., has not influenced in any way the forms of our thought? In other words, do you hold that in a world where everything changes, the syllogism alone remains unchanging and eternal? The Gospel according to St. John begins with the words: “In the beginning was the Word,” i.e., in the beginning was Reason or the Word (reason expressed in the word, namely, the syllogism). To St. John the syllogism is one of the literary pseudonyms for God. If you consider that the syllogism as immutable, i.e., has neither origin nor development, then it signifies that to you it is the product of divine revelation. But if you ac knowledge that the logical forms of our thought develop in the process of our adaptation to nature, then please take the trouble to in form us just who following Aristotle analyzed and systematized the subsequent progress of logic. So long as you do not clarify this point, I shall take the liberty of asserting that to identify logic (the dialectic) with religion reveals utter ignorance and superficiality in the basic questions of human thought.

\subsection*{Is the Revolutionist Not Obliged to Fight Against Religion?}

Let us grant however that your more than presumptuous innuendo is correct. But this does not improve affairs to your advantage. Religion, as I hope you will agree, diverts attention away from real to fictitious knowledge, away from the struggle for a better life to false hopes for reward in the Hereafter. Religion is the opium of the people. Whoever fails to struggle against religion is unworthy of bearing the name of revolutionist. On what grounds then do you justify your refusal to fight against the dialectic if you deem it one of the varieties of religion?

You stopped bothering yourself long ago, as you say, about the question of religion. But you stopped only \emph{for yourself}. In addition to you, there exist all the others. Quite a few of them. We revolutionists never “stop” bothering ourselves about religious questions, inasmuch as our task consists in emancipating from the influence of religion, not only ourselves but also the masses. If the dialectic is a religion, how is it possible to renounce the struggle against this opium within one’s own party?

Or perhaps you intended to imply that religion is of no political importance? That it is possible to be religious and at the same time a consistent communist and revolutionary fighter? You will hardly venture so rash an assertion. Naturally, we maintain the most considerate attitude toward the religious prejudices of a backward worker. Should he desire to fight for our program, we would accept him as a party member; but at the same time, our party would persistently educate him in the spirit of materialism and atheism. If you agree with this, how can you refuse to struggle against a “religion,” held, to my knowledge, by the overwhelming majority of those members of your own party who are interested in theoretical questions? You have obviously overlooked this most important aspect of the question.

Among the educated bourgeoisie there are not a few who have broken personally with religion, but whose atheism is solely for their own private consumption; they keep thoughts like these to themselves but in public often maintain that it is well the people have a religion.

Is it possible that you hold such a point of view toward your own party? Is it possible that this explains your refusal to discuss with us the philosophic foundations of Marxism? If that is the case, under your scorn for the dialectic rings a note of contempt for the party.

Please do not make the objection that I have based myself on a phrase expressed by you in private conversation, and that you are not concerned with publicly refuting dialectic materialism. This is not true. Your winged phrase serves only as an illustration. Whenever there has been an occasion, for various reasons you have proclaimed your negative attitude toward the doctrine which constitutes the theoretical foundation of our program. This is well known to everyone in the party. In the article “Intellectuals in Retreat,” written by you in collaboration with Shachtman and published in the party’s theoretical organ, it is categorically affirmed that you reject dialectic materialism. Doesn’t the party have the right after all to know just why? Do you really assume that in the Fourth International an editor of a theoretical organ can confine himself to the bare declaration: “I decisively reject dialectical materialism”---as if it were a question of a proffered cigarette: “Thank you, I don’t smoke.” The question of a correct philosophical doctrine, that is, a correct method of thought, is of decisive significance to a revolutionary party just as a good machine shop is of decisive significance to production. It is still possible to defend the old society with the material and intellectual methods inherited from the past. It is absolutely unthinkable that this old society can be overthrown and a new one constructed without first critically analyzing the current methods. If the party errs in the very foundations of its thinking it is your elementary duty to point out the correct road. Otherwise your conduct will be interpreted inevitably as the cavalier attitude of an academician toward a proletarian organization which, after all, is incapable of grasping a real “scientific” doctrine. What could be worse than that?
\nowidow

\subsection*{Instructive Examples}

Anyone acquainted with the history of the struggles of tendencies within workers’ parties knows that desertions to the camp of opportunism and even to the camp of bourgeois reaction began not infrequently with rejection of the dialectic. Petty-bourgeois intellectuals consider the dialectic the most vulnerable point in Marxism and at the same time they take advantage of the fact that it is much more difficult for workers to verify differences on the philosophical than on the political plane. This long known fact is backed by all the evidence of experience. Again, it is impermissible to discount an even more important fact, namely, that all the great and outstanding revolutionists---first and foremost, Marx, Engels, Lenin, Luxemburg, Franz Mehring---stood on the ground of dialectic materialism. Can it be assumed that all of them were incapable of distinguishing between science and religion? Isn’t there too much presumptuousness on your part, Comrade Burnham? The examples of Bernstein, Kautsky and Franz Mehring are extremely instructive. Bernstein categorically rejected the dialectic as “scholasticism” and “mysticism.” Kautsky maintained indifference toward the question of the dialectic, somewhat like Comrade Shachtman. Mehring was a tireless propagandist and defender of dialectic materialism. For decades he followed all the Innovations of philosophy and literature, indefatigably exposing the reactionary essence of idealism, neo-Kantianism, utilitarianism, all forms of mysticism, etc. The political fate of these three individuals is very well known. Bernstein ended his life as a smug petty-bourgeois democrat. Kautsky, from a centrist, became a vulgar opportunist. As for Mehring, he died a revolutionary communist.

In Russia, three very prominent academic Marxists, Struve, Bulgakov and Berdyaev began by rejecting the philosophic doctrine of Marxism and ended in the camp of reaction and the orthodox church. In the United States, Eastman, Sidney Hook and their friends utilized opposition to the dialectic as cover for their transformation from fellow travellers of the proletariat to fellow travellers of the bourgeoisie. Similar examples by the score could be cited from other countries. The example of Plekhanov, which appears to be an exception, in reality only proves the rule. Plekhanov was a remarkable propagandist of dialectic materialism, but during his whole life he never had the opportunity of participating in the actual class struggle. His thinking was divorced from practice. The revolution of 1905 and subsequently the World War flung him into the camp of petty-bourgeois democracy and forced him in actuality to renounce dialectic materialism. During the World War Plekhanov came forward openly as the protagonist of the Kantian categorical imperative in the sphere of international relations: “Do not do unto others as you would not have them do unto you.” The example of Plekhanov only proves that dialectic materialism \emph{in and of itself} still does not make a man a revolutionist.

Shachtman on the other hand argues that Liebknecht left a posthumous work against dialectic materialism which he had written in prison, Many ideas enter a person’s mind while in prison which cannot be checked by association with other people. Liebknecht, whom nobody, least of all himself, considered a theoretician, became a symbol of heroism in the world labor movement. Should any of the American opponents of the dialectic display similar self-sacrifice and independence from patriotism during war, we shall render what is due him as a revolutionist. But that will not thereby resolve the question of the dialectic method.

It is impossible to say what Liebknecht’s own final conclusions would have been had he remained at liberty. In any case before publishing his work, undoubtedly he would have shown it to his more competent friends, namely, Franz Mehring and Rosa Luxemburg. It is quite probable that on their advice he would have simply tossed the manuscript into the fire. Let us grant however that against the advice of people far excelling him in the sphere of theory he never the less had decided to publish his work. Mehring, Luxemburg, Lenin and others would not of course have proposed that lie be expelled for this from the party; on the contrary, they would have intervened decisively in his behalf had anyone made such a foolish proposal. But at the same time they would not have formed a philosophical bloc with him, but rather would have differentiated themselves decisively from his theoretical mistakes.

Comrade Shachtman’s behavior, we note, is quite otherwise. “You will observe,” he says---and this to teach the youth(!)---“that Plekhanov was an outstanding theoretician of dialectic materialism but ended up an opportunist; Liebknecht was a remarkable revolutionist but he had his doubts about dialectic materialism.” This argument if it means anything at all signifies that dialectic materialism is of no use whatsoever to a revolutionist. With these examples of Liebknecht and Plekhanov, artificially torn out of history, Shachtman reinforces and “deepens” the idea of his last year’s article, namely, that politics does not depend on method, inasmuch as method is divorced from politics through the divine gift of inconsistency. By falsely interpreting two “exceptions,” Shachtman seeks to overthrow the rule. If this is the argument of a “supporter” of Marxism, what can we expect from an opponent? The revision of Marxism passes here into its downright liquidation; more than that, into the liquidation of every doctrine and every method.

\subsection*{What Do You Propose Instead?}

Dialectic materialism is not of course an eternal and immutable philosophy. To think otherwise is to contradict the spirit of the dialectic. Further development of scientific thought will undoubtedly create a more profound doctrine into which dialectic materialism will enter merely as structural material. However, there is no basis for expecting that this philosophic revolution will be accomplished under the decaying bourgeois regime, without mentioning the fact that a Marx is not born every year or every decade. The life-and-death task of the proletariat now consists not in \emph{interpreting} the world anew but in \emph{remaking} it from top to bottom. In the next epoch we can expect great revolutionists of action but hardly a new Marx. Only on the basis of socialist culture will mankind feel the need to review the ideological heritage of the past and undoubtedly will far surpass us not only in the sphere of economy but also in the sphere of intellectual creation. The regime of the Bonapartist bureaucracy in the \USSR\ is criminal not only because it creates an ever-growing inequality in all spheres of life but also because it degrades the intellectual activity of the country to the depths of the unbridled blockheads of the \GPU.

Let us grant however that contrary to our supposition the proletariat is so fortunate during the present epoch of wars and revolutions as to produce a new theoretician or a new constellation of theoreticians who will surpass Marxism and in particular advance logic beyond the materialist dialectics. It goes without saying that all advanced workers will learn from the new teachers and the old men will have to reeducate themselves again. But in the meantime this remains the music of the future. Or am I mistaken? Perhaps you will call my attention to those works which should supplant the system of dialectic materialism for the proletariat? Were these at hand surely you would not have refused to conduct a struggle against the opium of the dialectic. But none exist. While attempting to discredit the philosophy of Marxism you do not propose anything with which to replace it.

Picture to yourself a young amateur physician who proceeds to argue with a surgeon using a scalpel that modern anatomy, neurology, etc.\ are worthless, that much in them remains unclear and incomplete and that only “conservative bureaucrats” could set to work with a scalpel on the basis of these pseudo-sciences, etc. I believe that the surgeon would ask his irresponsible colleague to leave the operating room. We too, comrade Burnham, cannot yield to cheap innuendoes about the philosophy of scientific socialism. On the contrary, since in the course of the factional struggle the question has been posed point-blank, we shall say, turning to all members of the party, especially the youth: Beware of the infiltration of bourgeois skepticism into your ranks. Remember that socialism to this day has not found higher scientific expression than Marxism. Bear in mind that the method of scientific socialism is dialectic materialism. Occupy yourselves with serious study! Study Marx, Engels, Plekhanov, Lenin and Franz Mehring. This is a hundred times more important for you than the study of tendentious, sterile and slightly ludicrous treatises on the conservatism of Cannon. Let the present discussion produce at least this positive result, that the youth attempt to embed in their minds a serious theoretical foundation for revolutionary struggle!

\subsection*{False Political “Realism”}

In your case, however, the question is not confined to the dialectic. The remarks in your resolution to the effect that you do not now pose for the decision of the party the question of the nature of the Soviet state signify in reality that you \emph{do pose} this question, if not juridically then theoretically and politically. Only infants can fail to understand this. This very statement likewise has another meaning, far more outrageous and pernicious. It means that you divorce politics from Marxist sociology. Yet for us the crux of the matter lies precisely in this. If it is possible to give a correct definition of the state without utilizing the method of dialectic materialism; if it is possible correctly to determine politics without giving a class analysis of the state, then the question arises: Is there any need whatsoever for Marxism?

Disagreeing among themselves on the class nature of the Soviet state, the leaders of the opposition agree on this, that the foreign policy of the Kremlin must be labelled “imperialist” and that the \USSR\ cannot be supported “unconditionally.” (Vastly substantial platform!) When the opposing “clique” raises the question of the nature of the Soviet state point-blank at the convention (what a crime!) you have in advance agreed\dots\ to disagree, i.e., to vote differently. In the British “national” government this precedent occurs of ministers who “agree to disagree,” i.e., to vote differently. But His Majesty’s ministers enjoy this advantage, that they are well aware of the nature of \emph{their} state and can afford the luxury of disagreement on \emph{secondary} questions. The leaders of the opposition are far less favorably situated. They permit themselves the luxury of differing on the fundamental question in order to solidarize on secondary questions. If this is Marxism and principled politics then I don’t know what unprincipled combinationism means.

You seem to consider apparently that by refusing to discuss dialectic materialism and the class nature of the Soviet state and by sticking to “concrete” questions you are acting the part of a realistic politician. This self-deception is a result of your inadequate acquaintance with the history of the past 50 years of factional struggles in the labor movement. In every principled conflict, without a single exception, the Marxists invariably sought to face the party squarely with the fundamental problems of doctrine and program, considering that only tinder this condition could the “concrete” questions find their proper place and proportion. On the other hand the opportunists of every shade, especially those who had already suffered a few defeats in the sphere of principled discussion, invariably counterpoised to the Marxist class analysis “concrete” conjunctural appraisals which they, as is the custom, formulated under the pressure of bourgeois democracy. Through decades of factional struggle this division of roles has persisted. The opposition, permit me to assure you, has invented nothing new. It is continuing the tradition of revisionism in theory and opportunism in politics.

\noclub
\enlargethispage{-1 \baselineskip}
Toward the close of the last century the revisionist attempts of Bernstein, who in England came under the influence of Anglo-Saxon empiricism and utilitarianism---the most wretched of philosophies---were mercilessly repulsed. Whereupon the German opportunists suddenly recoiled from philosophy and sociology. At conventions and in the press they did not cease to berate the Marxist “pedants,” who replaced the “concrete political questions” with general principled considerations. Read over the records of the German social-democracy toward the close of the last and the beginning of the present century---and you will be astonished yourself at the degree to which, as the French say, \emph{le mort saisit le vif} (the dead grip the living)!

You are not unacquainted with the great role played by \emph{Iskra} in the development of Russian Marxism. Iskra began with the struggle against so-called “Economism” in the labor movement and against the Narodniki (Party of the Social Revolutionists). The chief argument of the “Economists” was that \emph{Iskra} floats in the sphere of theory while they, the “Economists,” propose leading the concrete labor movement. The main argument of the Social Revolutionists was as follows: \emph{Iskra} wants to found a school of dialectic materialism while we want to overthrow Czarist autocracy. It must be said that the Narodnik terrorists took their own words very seriously: bomb in hand they sacrificed their lives. We argued with them: “Under certain circumstances a bomb is an excellent thing but we should first clarify our own minds.” It is historical experience that the greatest revolution in all history was not led by the party which started out with bombs but by the party which started out with dialectic materialism.

When the Bolsheviks and the Mensheviks were still members of the same party, the pre-convention periods and the convention itself invariably witnessed an embittered struggle over the agenda. Lenin used to propose as first on the agenda such questions as clarification of the nature of the Czarist monarchy, the analysis of the class character of the revolution, the appraisal of the stages of the revolution we were passing through, etc., Martov and Dan, the leaders of the Mensheviks, invariably objected: We are not a sociological club but a political party; we must come to an agreement not on the class nature of Czarist economy but on the “concrete political tasks.” I cite this from memory but I do not run any risk of error since these disputes were repeated from year to year and became stereotyped in character. I might add that I personally committed not a few sins on this score myself. But I have learned something since then.
\noclub

To those enamoured with “concrete political questions” Lenin invariably explained that our politics is not of conjunctural but of principled character; that tactics are subordinate to strategy; that for us the primary concern of every political campaign is that it guide the workers from the particular questions to the general, that it teach them the nature of modern society and the character of its fundamental forces. The Mensheviks always felt the need urgently to slur over principled differences in their unstable conglomeration by means of evasions whereas Lenin on the contrary posed principled questions point-blank. The current arguments of the opposition against philosophy and sociology in favor of “concrete political questions” is a belated repetition of Dan’s arguments. Not a single new word! How sad it is that Shachtman respects the principled politics of Marxism only when it has aged long enough for the archives.

Especially awkward and inappropriate does the appeal to shift from Marxist theory to “concrete political questions” sound on your lips, Comrade Burnham, for it was not I but you who raised the question of the character of the \USSR, thereby forcing me to pose the question of the method through which the class character of the state is determined. True enough, you withdrew your resolution. But this factional maneuver has no objective meaning whatsoever. You draw your \emph{political} conclusions from your \emph{sociological} premise, even if you have temporarily slipped it into your briefcase. Shachtman draws exactly the same political conclusions without a sociological premise: he adapts himself to you. Abern seeks to profit equally both from the hidden premise and the absence of a premise for his “organizational” combinations. This is the real and not the diplomatic situation in the camp of the opposition. You proceed as an anti-Marxist; Shachtman and Abern---as \emph{Platonic} Marxists. Who is worse, it is not easy to determine.

\subsection*{The Dialectic of the Present Discussion}

When confronted with the diplomatic front covering the hidden premises and lack of premises of our opponents, we, the “conservatives,” naturally reply: A fruitful dispute over “concrete questions” is possible only if you clearly specify what class premises you take as your starting point. We are not compelled to confine ourselves to those topics in this dispute which you have selected artificially. Should someone propose that we discuss as “concrete” questions the invasion of Switzerland by the Soviet fleet or the length of a tail of a Bronx witch, then I am justified in posing in advance such questions as, does Switzerland have a sea coast? Are there witches at all?

Every serious discussion develops from the particular and even the accidental to the general and fundamental. The immediate causes and motives of a discussion are of interest, in most cases, only symptomatically. Of actual political significance are only those problems which the discussion raises in its development. To certain intellectuals, anxious to indict ``bureaucratic conservatism” and to display their “dynamic spirit,” it might seem that questions concerning the dialectic, Marxism, the nature of the state, centralism are raised “artificially” and that the discussion has taken a “false” direction. The nub of the matter however consists in this, that discussion has its own objective logic which does not coincide at all with the sub jective logic of individuals and groupings. The \emph{dialectic} character of the discussion proceeds from the fact that its objective course is determined by the living conflict of opposing tendencies and not by a preconceived logical plan. The \emph{materialist} basis of the discussion consists in its reflecting the pressure of different classes. Thus, the present discussion in the \textsc{swp}, like the historic process as a whole, develops---with or without your permission, Comrade Burnham---according to the laws of dialectic materialism. There is no escape from these laws.

\subsection*{“Science” Against Marxism and “Experiments” Against Program}

Accusing your opponents of “bureaucratic conservatism” (a bare psychological abstraction insofar as no specific social interests are shown underlying this “conservatism"), you demand in your document that conservative politics be replaced by “critical and experimental politics---in a word, scientific politics.” (p.~32). This statement, at first glance so innocent and meaningless with all its pompousness, is in itself a complete exposure. You don’t speak of Marxist politics. You don’t speal of proletarian politics. You speak of “experimental,” “critical,” “scientific” politics. Why this pretentious and deliberately abstruse terminology so unusual in our ranks? I shall tell you. It is the product of your adaptation, comrade Burnham, to bourgeois public opinion, and the adaptation of Shachtman and Abern to your adaptation. Marxism is no longer fashionable among the broad circles of bourgeois intellectuals. Moreover if one should mention Marxism, God forbid, he might be taken for a dialectic materialist. It is better to avoid this discredited word. What to replace it with? Why, of course, with “science,” even with Science capitalized. And science, as everybody knows, is based on “criticism” and “experiments.” It has its own ring; so solid, so tolerant, so unsectarian, so professorial! With this formula one can enter any democratic salon.

Reread, please, your own statement once again: “In place of conservative politics, we must put bold, flexible, critical and experimental politics---in a word, scientific politics.” You couldn’t have improved it! But this is precisely the formula which all petty-bourgeois empiricists, all revisionists and, last but not least, all political adventurers have counterpoised to “narrow,” “limited,” “dogmatic” and “conservative” Marxism.

Buffon once said: The style is the man. Political terminology is not only the man but the party. Terminology is one of the elements of the class struggle. Only lifeless pedants can fail to understand this. In your document you painstakingly expunge---yes, no one else but you, Comrade Burnham---not only such terms as the dialectic and materialism but also Marxism. You are above all this. You are a man of “critical,” “experimental” science. For exactly the same reason you culled the label “imperialism” to describe the foreign policy of the Kremlin. This innovation differentiates you from the too embarrassing terminology of the Fourth International by creating less “sectarian,” less “religious,” less rigorous formulas, common to you and---oh happy coincidence---bourgeois democracy.

You want to experiment? But permit me to remind you that the workers’ movement possesses a long history with no lack of experience and, if you prefer, experiments. This experience so dearly bought has been crystallized in the shape of a definite doctrine, the very Marxism whose name you so carefully avoid. Before giving you the right to experiment, the party has the right to ask: What method will you use? Henry Ford would scarcely permit a man to experiment in his plant who had not assimilated the requisite conclusions of the past development of industry and the innumerable experiments already carried out. Furthermore experimental laboratories in factories are carefully segregated from mass production. Far more impermissible even are witch doctor experiments in the sphere of the labor movement---even though conducted under the banner of anonymous “science.” For us the science of the workers’ movement is Marxism. Nameless social science, Science with a capital letter, we leave these completely at the disposal of Eastman and his ilk.

I know that you have engaged in disputes with Eastman and in some questions you have argued very well. But you debate with him as a representative of your own circle and not as an agent of the class enemy. You revealed this conspicuously in your joint article with Shachtman when you ended up with the unexpected invitation to Eastman, Hook, Lyons and the rest that they take advantage of the pages of the \emph{New International} to promulgate their views. It did not even concern you that they might pose the question of the dialectic and thus drive you out of your diplomatic silence.

On January 20 of last year, hence long prior to this discussion, in a letter to comrade Shachtman I insisted on the urgent necessity of attentively following the internal developments of the Stalinist party. I wrote:

\begin{quote}
  It would be a thousand times more important than inviting Eastman, Lyons and the others to present their personal sweatings. I was wondering a bit why you gave space to Eastman’s last insignificant and arrogant article, he has at his disposal \emph{Harper’s Magazine}, \emph{Modern Monthly}, \emph{Common Sense}, etc. But I am absolutely perplexed that you personally \emph{invited} these people to besmirch the not-so-numerous pages of the \emph{New International}. The perpetuation of this polemic can interest some \emph{petty-bourgeois intellectuals} but not the revolutionary elements. It is my firm conviction that a certain reorganization of the \emph{New International} and the \emph{Socialist Appeal} is necessary: more distance from Eastman, Lyons, etc.; and nearer to the workers and, in this sense, to the Stalinist party.
\end{quote}

As always in such cases Shachtman replied inattentively and carelessly. In actuality, the question was resolved by the fact that the enemies of Marxism whom you invited refused to accept your invitation. This episode, however, deserves closer attention. On the one hand, you, Comrade Burnham, bolstered by Shachtman, invite bourgeois democrats to send in friendly explanations to be printed in the pages of our party organ. On the other hand, you, bolstered by this same Shachtman, refuse to engage in a debate with me over the dialectic and the class nature of the Soviet state. Doesn’t this signify that you, together with your ally Shachtman, have turned your faces somewhat toward the bourgeois semi-opponents and your backs toward your own party? Abern long ago came to the conclusion that Marxism is a doctrine to be honored but a good oppositional combination is something far more substantial. Meanwhile, Shachtman slips and slides downward, consoling himself with wise cracks. I feel, however, that his heart is a trifle heavy. Upon reaching a certain point, Shachtman will, I hope, pull himself together and begin the upward climb again. Here is the hope that his “experimental” factional politics will at least turn out to the profit of “Science.”

\subsection*{“An Unconscious Dialecticicm”}

Using as his text my remark concerning Darwin, Shachtman has stated, I have been informed, that you are an “unconscious dialectician.” This ambiguous compliment contains an iota of truth Every individual is a dialectician \emph{to some extent or other}, in most cases, unconsciously. A housewife knows that a certain amount of salt flavors soup agreeably, but that added salt makes the soup unpalatable. Consequently, an illiterate peasant woman guides her self in cooking soup by the Hegelian law of the transformation of quantity into quality. Similar examples from daily life could be cited without end. Even animals arrive at their practical conclusions not only on the basis of the Aristotelian syllogism but also on the basis of the Hegelian dialectic. Thus a fox is aware that quadrupeds and birds are nutritious and tasty. On sighting a hare, a rabbit, or a hen, a fox concludes: this particular creature belongs to the tasty and nutritive type, and---chases after the prey. We have here a complete syllogism, although the fox, we may suppose, never read Aristotle. When the same fox, however, encounters the first animal which exceeds it in size, for example, a wolf, it quickly concludes that quantity passes into quality, and turns to flee. Clearly, the legs of a fox are equipped with Hegelian tendencies, even if not fully conscious ones. All this demonstrates, in passing, that our methods of thought, both formal logic and the dialectic, are not arbitrary constructions of our reason but rather expressions of the actual inter-relationships in nature itself. In this sense, the universe throughout is permeated with “unconscious” dialectics. But nature did not stop there. No little development occurred before nature’s inner relationships were converted into the language of the consciousness of foxes and men, and man was then enabled to generalize these forms of consciousness and transform them into logical (dialectical) categories, thus creating the possibility for probing more deeply into the world about us.

The most finished expression to date of the laws of the dialectic which prevail in nature and in society has been given by Hegel and Marx. Despite the fact that Darwin was not interested in verifying his logical methods, his empiricism---that of a genius---in the sphere of natural science reached the highest dialectic generalizations. In this sense, Darwin was, as I stated in my previous article, an “unconscious dialectician.” We do not, however, value Darwin for his inability to rise to the dialectic, but for having, despite his philosophical backwardness, explained to us the origin of species. Engels was, it might be pointed out, exasperated by the narrow empiricism of the Darwinian method, although he, like Marx, immediately appreciated the greatness of the theory of natural selection. Darwin, on the contrary, remained, alas, ignorant of the meaning of Marx’s sociology to the end of his life. Had Darwin come out in the press against the dialectic or materialism, Marx and Engels would have attacked him with redoubled force so as not to allow his authority to cloak ideological reaction.

In the attorney’s plea of Shachtman to the effect that you are an “unconscious dialectician,” the stress must be laid on the word \emph{unconscious}. Shachtman’s aim (also partly unconscious) is to defend his bloc with you by degrading dialectic materialism. For in reality, Shachtman is saying: The difference between a “conscious” and an “unconscious” dialectician is not so great that one must quarrel about it. Shachtman thus attempts to discredit the Marxist method.

But the evil goes beyond even this. Very many unconscious or semi-unconscious dialecticians exist in this world. Some of them apply the materialist dialectic excellently to politics, even though they have never concerned themselves with questions of method. It would obviously be pedantic blockheadedness to attack such comrades. But it is otherwise with you, comrade Burnham. You are an editor of the theoretical organ whose task it is to educate the party in the spirit of the Marxist method. Yet you are a \emph{conscious opponent of the dialectic} and not at all an \emph{unconscious dialectician}. Even if you had, as Shachtman insists, successfully followed the dialectic in political questions, i.e., even if you were endowed with a dialectic “instinct,” we would still be compelled to begin a struggle against you, because your dialectic instinct, like other individual qualities, cannot be transmitted to others, whereas the conscious dialectic method can, to one degree or another, be made accessible to the entire party.

\subsection*{The Dialectic and Mr. Dies}

Even if you have a dialectic instinct---and I do not undertake to judge this---it is well-nigh stifled by academic routine and intellectual hauteur. What we term the class instinct of the worker, accepts with relative ease the dialectic approach to questions. There can be no talk of such a class instinct in a bourgeois intellectual. Only by \emph{consciously} surmounting his petty-bourgeois spirit can an intellectual divorced from the proletariat rise to Marxist politics. Unfortunately, Shachtman and Abern are doing everything in their power to bar this road to you. By their support they render you a very bad service, Comrade Burnham.

Bolstered by your bloc, which might be designated as the “League of Factional Abandon,” you commit one blunder after another: in philosophy, in sociology, in politics, in the organizational sphere. Your errors are not accidental. You approach each question by isolating it, by splitting it away from its connection with other questions, away from its connection with social factors, and---independently of international experience. You lack the dialectic method. Despite all your education, in politics you proceed like a witch-doctor.

In the question of the Dies Committee your mumbo-jumbo manifested itself no less glaringly than in the question of Finland. To my arguments in favor of utilizing this parliamentary body, you replied that the question should be decided not by principled considerations but by some special circumstances known to you alone but which you refrained from specifying. Permit me to tell you what these circumstances were: your ideological dependence on bourgeois public opinion. Although bourgeois democracy, in all its sections, bears full responsibility for the capitalist regime, including the Dies Committee, it is compelled, in the interests of this very same capitalism, shamefacedly to distract attention away from the too naked organs of the regime. A simple division of labor! An old fraud which still continues, however, to operate effectively! As for the workers, to whom you refer vaguely, a section of them, and a very considerable section, is like yourself under the influence of bourgeois democracy. But the average worker, not infected with the prejudices of the labor aristocracy, would joyfully welcome every bold revolutionary word thrown in the very face of the class enemy. And the more reactionary the institution which serves as the arena for the combat, all the more complete is the satisfaction of the worker. This has been proved by historical experience. Dies himself, becoming frightened and jumping back in time, demonstrated how false your position was. It is always better to compel the enemy to retreat than to hide oneself without a battle.

But at this point I see the irate figure of Shachtman rising to stop me with a gesture of protest: “The opposition bears no responsibility for Burnham’s views on the Dies Committee. This question did not assume a factional character,” and so forth and so on. I know all this. As if the only thing that lacked was for the entire opposition to express itself in favor of the tactic of boycott, so utterly senseless in this instance! It is sufficient that the leader of the opposition, who has views and openly expressed them, came out in favor of boycott. If you happened to have outgrown the age when one argues about “religion,” then, let me confess, I had considered that the entire Fourth International had outgrown the age when abstentionism is accounted the most revolutionary of policies. Aside from your lack of method, you revealed in this instance an obvious lack of political sagacity. In the given situation, a revolutionist would not have needed to discuss long before springing through a door flung open by the enemy and making the most of the opportunity. For those members of the opposition who together with you spoke against participation in the Dies Committee---and their number is not so small---it is necessary in my opinion to arrange special elementary courses in order to explain to them the elementary truths of revolutionary tactics which have nothing in common with the pseudo-radical abstentionism of the intellectual circles.

\subsection*{“Concrete Political Questions”}

The opposition is weakest precisely in the sphere where it imagines itself especially strong---the sphere of day-to-day revolutionary politics. This applies above all to you, comrade Burnham. Impotence in the face of great events manifested itself in you as well as in the entire opposition most glaringly in the questions of Poland, the Baltic states and Finland. Shachtman began by discovering a philosopher’s stone: the achievement of a simultaneous insurrection against Hitler and Stalin in occupied Poland. The idea was splendid; it is only too bad that Shachtman was deprived of the opportunity of putting it into practice. The advanced workers in eastern Poland could justifiably say: ``A simultaneous insurrection against Hitler and Stalin in a country occupied by troops might perhaps be arranged very conveniently from the Bronx; but here, locally, it is more difficult. We should like to hear Burnham’s and Shachtman’s answer to a ‘concrete political question:’ What shall we do between now and the coming insurrection?” In the meantime, the commanding staff of the Soviet army called upon the peasants and workers to seize the land and the factories. This call, supported by armed force, played an enormous role in the life of the occupied country. Moscow papers were filled to overflowing with reports of the boundless “enthusiasm” of workers and poor peasants. We should and must approach these reports with justifiable distrust: there is no lack of lies. But it is nevertheless impermissible to close one’s eyes to facts. The call to settle accounts with the landlords and to drive out the capitalists could not have failed to rouse the spirit of the hounded and crushed Ukrainian and Byelo-Russian peasants and workers who saw in the Polish landlord a double enemy.

In the Parisian organ of the Mensheviks, who are in solidarity with the bourgeois democracy of France and not the Fourth International, it was stated categorically that the advance of the Red Army was accompanied by a wave of revolutionary upsurge, echoes of which penetrated even the peasant masses of Rumania. What adds special weight to the dispatches of this organ is the close connection with the Mensheviks and the leaders of the Jewish Bund, the Polish Socialist Party and other organizations who are hostile to the Kremlin and who fled from Poland. We were therefore completely correct when we said to the Bolsheviks in eastern Poland:

\begin{quote}
  Together with the workers and peasants, and in the forefront, you must conduct a struggle against the landlords and the capitalists; do not tear yourself away from the masses, despite all their illusions, just as the Russian revolutionists did not tear themselves away from the masses who had not yet freed themselves from their hopes in the Czar (Bloody Sunday, January 22, 1905); educate the masses in the course of the struggle, warn them against naive hopes in Moscow, but do not tear yourself away from them, fight in their camp, try to extend and deepen their struggle, and to give it the greatest possible independence. Only in this way will you prepare the coming insurrection against Stalin.
\end{quote}

The course of events in Poland has completely confirmed this directive which was a continuation and a development of all our previous policies, particularly in Spain.

Since there is no principled difference between the Polish and Finnish situations, we can have no grounds for changing our directive. But the opposition, who failed to understand the meaning of the Polish events, now tries to clutch at Finland as a new anchor of salvation. “Where is the civil war in Finland? Trotsky talks of a civil war. We have seen nothing about it in the press,” and so on. The question of Finland appears to the opposition as in principle different from the question of western Ukraine and Byelo-Russia. Each question is isolated and viewed aside and apart from the general course of development. Confounded by the course of events, the opposition seeks each time to support itself on some accidental, secondary, temporary and conjunctural circumstances.

Do these cries about the absence of civil war in Finland signify that the opposition would adopt our policy if civil war were actually to unfold in Finland? Yes or no? If yes, then the opposition thereby condemns its own policy in relation to Poland, since there, despite the civil war, they limited themselves to refusal to participate in the events, while they waited for a simultaneous uprising against Stalin and Hitler. It is obvious, comrade Burnham, that you and your allies have not thought this question through to the end.

What about my assertion concerning a civil war in Finland? At the very inception of military hostilities, one might have conjectured that Moscow was seeking through a “small” punitive expedition to bring about a change of government in Helsingfors and to establish the same relations with Finland as with the other Baltic states. But the appointment of the Kuunsinen government in Terrijoki demonstrated that Moscow had other plans and aims. Dispatches then reported the creation of a Finnish “Red Army.” Naturally, it was only a question of small formations set up from above. The program of Kuusinen was issued. Next the dispatches appeared of the division of large estates among poor peasants. In their totality, these dispatches signified an attempt on the part of Moscow to organize a civil war. Naturally, this is a civil war of a special type. It does not arise spontaneously from the depths of the popular masses. It is not conducted under the leadership of the Finnish revolutionary party based on mass support. It is introduced on bayonets from without. It is controlled by the Moscow bureaucracy. All this we know, and we dealt with all this in discussing Poland. Nevertheless, it is precisely a question of civil war, of an appeal to the lowly, to the poor, a call to them to expropriate the rich, drive them out, arrest them, etc. I know of no other name for these actions except civil war.

“But, after all, the civil war in Finland did not unfold,” object the leaders of the opposition. “This means that your predictions did not materialize.” With the defeat and the retreat of the Red Army, I reply, the civil war in Finland cannot, of course, unfold under the bayonets of Mannerheim. This fact is an argument not against me but against Shachtman; since it demonstrates that in the first stages of war, at a time when discipline in armies is still strong, it is much easier to organize insurrection, and on two fronts to boot, from the Bronx than from Terrijoki.

We did not foresee the defeats of the first detachments of the Red Army. We could not have foreseen the extent to which stupidity and demoralization reign in the Kremlin and in the tops of the army beheaded by the Kremlin. Nevertheless, what is involved is only a military episode, which cannot determine our political line. Should Moscow, after its first unsuccessful attempt, refrain entirely from any further offensive against Finland, then the very question which today obscures the entire world situation to the eyes of the opposition would be removed from the order of the day. But there is little chance for this. On the other hand, if England, France and the United States, basing themselves on Scandinavia, were to aid Finland with military force, then the Finnish question would be submerged in a war between the \USSR\ and the imperialist countries. In this case, we must assume that even a majority of the oppositionists would remind themselves of the program of the Fourth International.

At the present time, however, the opposition is not interested in these two variants: either the suspension of the offensive on the part of the \USSR, or the outbreak of hostilities between the \USSR\ and the imperialist democracies. The opposition is interested only in the isolated question of the \USSR’s invasion of Finland. Very well, let us take this as our starting point. If the second offensive, as may be assumed, is better prepared and conducted, then the advance of the Red Army into the country will again place the question of civil war on the order of the day, and moreover on a much broader scale than during the first and ignominiously unsuccessful attempt. Our directive, consequently, remains completely valid so long as the question itself remains on the agenda. But what does the opposition propose in the event the Red Army successfully advances into Finland and civil war unfolds there? The opposition apparently doesn’t think about this at all, for they live from one day to the next, from one incident to another, clutching at episodes, clinging to isolated phrases in an editorial, feeding on sympathies and antipathies, and thus creating for themselves the semblance of a platform. The weakness of empiricists and impressionists is always revealed most glaringly in their approach to “concrete political questions.”
\noclub

\subsection*{Theoretical Bewilderment and Political Abstentionism}

Throughout all the vacillations and convulsions of the opposition, contradictory though they may be, two general features run like a guiding thread from the pinnacles of theory down to the most trifling political episodes. The first general feature is the absence of a unified conception. The opposition leaders split sociology from dialectic materialism. They split politics from sociology. In the sphere of politics they split our tasks in Poland from our experience in Spain---our tasks in Finland from our position on Poland. History becomes transformed into a series of exceptional incidents; politics becomes transformed into a series of improvisations. We have here, in the full sense of the term, the disintegration of Marxism, the disintegration of theoretical thought, the disintegration of politics into its constituent elements. Empiricism and its foster-brother, impressionism, dominate from top to bottom. That is why the ideological leadership, Comrade Burnham, rests with you as an opponent of the dialectic, as an empiricist, unabashed by his empiricism.

Throughout the vacillations and convulsions of the opposition, there is a second general feature intimately bound to the first, namely, a tendency to refrain from active participation, a tendency to self-elimination, to abstentionism, naturally under cover of ultra-radical phrases. You are in favor of overthrowing Hitler and Stalin in Poland; Stalin and Mannerheim in Finland. And until then, you reject both sides \emph{equally}, in other words, you withdraw from the struggle, including the civil war. Your citing the absence of civil war in Finland is only an accidental conjunctural argument. Should the civil war unfold, the opposition will attempt not to notice it, as they tried not to notice it in Poland, or they will declare that inasmuch as the policy of the Moscow bureaucracy is “imperialist” in character “we” do not take part in this filthy business. Hot on the trail of “concrete” political tasks in words, the opposition actually places itself outside the historical process. Your position, Comrade Burnham, in relation to the Dies Committee merits attention precisely because it is a graphic expression of this same tendency of abstentionism and bewilderment. Your guiding principle still remains the same: “Thank you, I don’t smoke.”

Naturally, any man, any party and even any class can become bewildered. But with the petty bourgeoisie, bewilderment, especially in the face of great events, is an inescapable and, so to speak, congenital condition. The intellectuals attempt to express their state of bewilderment in the language of “science.” The contradictory plat form of the opposition reflects petty-bourgeois bewilderment expressed in the bombastic language of the intellectuals. There is nothing proletarian about it.

\subsection*{The Petty-Bourgeoisie and Centralism}

In the organizational sphere, your views are just as schematic, empiric, non-revolutionary as in the sphere of theory and politics. A Stolberg, lantern in hand, chases after an ideal revolution, unaccompanied by any excesses, and guaranteed against Thermidor and counter-revolution; you, likewise, seek an ideal party democracy which would secure forever and for everybody the possibility of saying and doing whatever popped into his head, and which would insure the party against bureaucratic degeneration. You overlook a trifle, namely, that the party is not an arena for the assertion of free individuality, but an instrument of the proletarian revolution; that only a victorious revolution is capable of preventing the degeneration not only of the party but of the proletariat itself and of modern civilization as a whole. You do not see that our American section is not sick from too much centralism---it is laughable even to talk about it---but from a monstrous abuse and distortion of democracy on the part of petty-bourgeois elements. This is at the root of the present crisis.

A worker spends his day at the factory. He has comparatively few hours left for the party. At the meetings he is interested in learning the most important things: the correct evaluation of the situation and the political conclusions. He values those leaders who do this in the clearest and the most precise form and who keep in step with events. Petty-bourgeois, and especially declassed elements, divorced from the proletariat, vegetate in an artificial and shut-in environment. They have ample time to dabble in politics or its substitute. They pick out faults, exchange all sorts of tidbits and gossip concerning happenings among the party “tops.” They always locate a leader who initiates them into all the “secrets.” Discussion is their native element. No amount of democracy is ever enough for them. For their war of words they seek the fourth dimension. They become jittery, they revolve in a vicious circle, and they quench their thirst with salt water. Do you want to know the organizational program of the opposition? It consists of a mad hunt for the fourth dimension of party democracy. In practice this means burying politics beneath discussion; and burying centralism beneath the anarchy of the intellectual circles. When a few thousand workers join the party, they will call the petty-bourgeois anarchists severely to order. The sooner, the better.

\subsection*{Conclusions}

Why do I address you and not the other leaders of the opposition? Because you are the ideological leader of the bloc. Comrade Abern’s faction, destitute of a program and a banner, is ever in need of cover. At one time Shachtman served as cover, then came Muste with Spector, and now you, with Shachtman adapting himself to you. Your ideology I consider the expression of bourgeois influence in the proletariat.

To some comrades, the tone of this letter may perhaps seem too sharp. Yet, let me confess, I did everything in my power to restrain myself. For, after all, it is a question of nothing more or less than an attempt to reject, disqualify and overthrow the theoretical foundations, the political principles and organizational methods of our movement.

In reaction to my previous article, Comrade Abern, it has been reported, remarked: “This means split.” Such a response merely demonstrates that Abern lacks devotion to the party and the Fourth International; he is a circle man. In any case, threats of split will not deter us from presenting a Marxist analysis of the differences. For us Marxists, it is a question not of split but of educating the party. It is my firm hope that the coming convention will ruthlessly repulse the revisionists.

The convention, in my opinion, must declare categorically that in their attempts to divorce sociology from dialectic materialism and politics from sociology, the leaders of the opposition have broken from Marxism and become the transmitting mechanism for petty-bourgeois empiricism. While reaffirming, decisively and completely, its loyalty to the Marxist doctrine and the political and organizational methods of Bolshevism, while binding the editorial boards of its official publications to promulgate and defend this doctrine and these methods, the party will, of course, extend the pages of its publications in the future to those of its members who consider themselves capable of adding something new to the doctrine of Marxism. But it will not permit a game of hide-and-seek with Marxism and light-minded gibes concerning it.

The politics of a party has a class character. Without a class analysis of the state, the parties and ideological tendencies, it is impossible to arrive at a correct political orientation. The party must condemn as vulgar opportunism the attempt to determine policies in relation to the \USSR\ from incident to incident and independently of the class nature of the Soviet state.

The disintegration of capitalism, which engenders sharp dissatisfaction among the petty bourgeoisie and drives its bottom layers to the left, opens up broad possibilities but it also contains grave dangers. The Fourth International needs only those emigrants from the petty bourgeoisie who have broken completely with their social past and who have come over decisively to the standpoint of the proletariat.

This theoretical and political transit must be accompanied by an actual break with the old environment and the establishment of intimate ties with workers, in particular, by participation in the recruitment and education of proletarians for their party. Emigrants from the petty-bourgeois milieu who prove incapable of settling in the proletarian milieu must after the lapse of a certain period of time be transferred from membership in the party to the status of sympathizers.

Members of the party untested in the class struggle must not be placed in responsible positions. No matter how talented and de voted to socialism an emigrant from the bourgeois milieu may be, before becoming a teacher, he must first go to school in the working class. Young intellectuals must not be placed at the head of the intellectual youth but sent out into the provinces for a few years, into the purely proletarian centers, for hard practical work.

The class composition of the party must correspond to its class program. The American section of the Fourth International will either become proletarian or it will cease to exist.

\triast

Comrade Burnham! If we can arrive at an agreement with you on the basis of these principles, then without difficulty we shall find a correct policy in relation to Poland, Finland and even India. At the same time, I pledge myself to help you conduct a struggle against any manifestations whatsoever of bureaucratism and conservatism. These in my opinion are the conditions necessary to end the present crisis.

\signed{With Bolshevik greetings, \\ \textsc{L.\ Trotsky}}
\signed{Coyoacán, D.F.}

\begin{letterinfo}
	\firstpublished\ Leon Trotsky, \emph{In Defense of Marxism}, New York 1942.
	
	\checkedagainst\ Leon Trotsky, \emph{In Defence of Marxism}, London 1966, pp.~91--119.
\end{letterinfo}