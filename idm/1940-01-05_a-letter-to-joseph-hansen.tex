\chapterdated[A Letter to Joseph Hansen]{A Letter to Joseph Hansen\footnote{This letter was written by Trotsky in English.}}{January 5, 1940}

\letteraddress{Dear Joe,}

Thank you for your interesting information. In the case of necessity or of advisability, Jim could publish our correspondence and that with Wright concerning the split matter. This correspondence shows our firm desire to preserve the unity of the party in spite of the sharp factional struggle. I mentioned in my letter to Wright\footnote{See p.~\pageref{1939-12-18_a-letter-to-john-g-wright} ---J.W.} that even as a minority the Bolshevik wing of the party should in my opinion remain disciplined and Jim answered that he wholeheartedly agreed with that view. These two quotations are decisive for the matter.

Concerning my remarks about Finland in the article on the petty-bourgeois opposition\footnote{See p.~\pageref{1939-12-15_a-petty-bourgeois-opposition-in-the-socialist-workers-party} ---J.W.}, I will say here only a few words. Is there a principled difference between Finland and Poland---yes or no? Was the intervention of the Red Army in Poland accompanied by civil war---yes or no? The press of the Mensheviks who are very well informed thanks to their friendship with Bund and with \textsc{pps} \emph{émigrés} says openly that a revolutionary wave surrounded the advance of the Red Army. And not only in Poland but also in Rumania.

The Kremlin created the Kuusinen government with the evident purpose of supplementing the war by civil war. There was information about the beginning of the creation of a Finnish Red Army, about “enthusiasm” of poor Finnish farmers in the occupied regions where the large land properties were confiscated, and so on. What is this if not the beginning of civil war?

The further development of the civil war depended completely upon the advance of the Red Army. The “enthusiasm” of the people was evidently not hot enough to produce independent insurrections of peasants and workers under the sword of the hangman Mannerheim. The retreat of the Red Army necessarily halted the elements of the civil war at the very beginning.

If the imperialists help the Finnish bourgeoisie efficiently in defending the capitalist regime, the civil war in Finland would become for the next period impossible. But if, as is more probable, the reinforced detachments of the Red Army more successfully penetrate into the country, we will inevitably observe the process of civil war paralleling the invasion.

We cannot foresee all the military episodes, the ups and downs of purely tactical interest, but they don’t change the general “strategical” line of events. In this case as in all others, the opposition makes a purely conjunctural and impressionistic policy instead of a principled one.

(It is not necessary to repeat that the civil war in Finland as was the case in Poland would have a limited, semi-stifled nature and that it can, in the next stage, go over into a civil war between the Finnish masses and the Moscow bureaucracy. We know this at least as clearly as the opposition and we openly warn the masses. But we analyze the process as it is and we don’t identify the first stage with the second one.)

\signed{With warm wishes and greetings for all friends, \\ \textsc{L.\ Trotsky}}
\signed{Coyoacán, D.F.}

\begin{letterinfo}
	\firstpublished\ Leon Trotsky, \emph{In Defense of Marxism}, New York 1942.
	
	\checkedagainst\ Leon Trotsky, \emph{In Defence of Marxism}, London 1966, pp.~83--88.
	
	\footnoteslatter
\end{letterinfo}