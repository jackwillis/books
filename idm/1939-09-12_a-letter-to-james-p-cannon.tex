\chapterdated[Letter to James P.\@ Cannon]{Letter to James P.\@ Cannon\footnote{This letter was written by Trotsky in English.}}{September 12, 1939}

\letteraddress{Dear Jim:}

I am writing
now a study on the social character of the \USSR\ in connection with the war question. The writing, with its translation, will take at least one week more. The fundamental ideas are as follows:

\begin{enumerate}
	\item Our definition of the \USSR\ can be right or wrong, but I do not see any reason to make our definition dependent on the German-Soviet pact.
	
	\item The social character of the \USSR\ is not determined by her friendship with democracy or fascism. Who adopts such a point of view becomes a prisoner of the Stalinist conception of the People’s Front epoch.
	
	\item Who says that the \USSR\ is no more a degenerate workers’ state, but a new social formation, should clearly say what he adds to our \emph{political conclusions}.
	
	\item The \USSR\ question cannot be isolated as unique from the whole historic process of our times. Either the Stalin state is a transitory formation, it is a deformation of a workers’ state in a backward and isolated state, or “bureaucratic collectivism” (Bruno R., \emph{La bureaucratisation du monde}, Paris 1939) is a new social formation, which is replacing capitalism throughout the world (Stalinism, Fascism, New Deals, \etc.). The terminological experiments (workers’ state, not workers’ state; class, not class; \etc.) receive a sense only under this historic aspect. Who chooses the second alternative admits, openly or silently, that all the revolutionary potentialities of the world proletariat are exhausted, that the socialist movement is bankrupt, and that the old capitalism is transforming itself into “bureaucratic collectivism” with a new exploiting class.
\end{enumerate}

The tremendous importance of such a conclusion is self-expla\-na\-tory. It concerns the whole fate of the world proletariat and man\-kind. Have we the slightest right to induce ourselves by purely terminological experiments in a new historic conception which occurs to be in an absolute contradiction with our program, strategy and tactics? Such an adventuristic jump would be doubly criminal now in view of the world war when the perspective of the socialist revolution becomes an imminent reality and when the case of the \USSR\ will appear to everybody as a transitorial episode in the process of world socialist revolution.

I write these lines in haste, which explains their insufficiency, but in a week I hope to send you my more complete thesis.

\signed{
Comradely greetings,\\
\textsc{vto}\footnote{Because of the conditions of his residence in the various countries in which he lived after his exile, Trotsky often used pseudonyms in his letters. His letters were frequently signed with the name of his English secretary. ---\ed} [Leon Trotsky]
}

\begin{letterinfo}
  \firstpublished\ Leon Trotsky, \emph{In Defense of Marxism}, New York, 1942.
  
  \checkedagainst\ Leon Trotsky, \emph{In Defence of Marxism}, London 1966, pp. 1--2.
  
  \footnoteslatter
\end{letterinfo}