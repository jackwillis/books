\chapterdated[Letter to James P.\@ Cannon]{Letter to James P.\@ Cannon\footnote{This letter was written by Trotsky in English.}}{December 15, 1939}

\letteraddress{Dear Comrade Cannon,}

The leaders of the opposition have not accepted the struggle upon a principled plane up to now and will undoubtedly attempt to avoid it even in the future. It is not difficult consequently to guess what the leaders of the opposition will say in regard to the enclosed article. “There are many correct elementary truths in the article,” they will say; “we don’t deny them at all, but the article fails to answer the burning ‘concrete’ questions. Trotsky is too far from the party to be able to judge correctly. Not all petty-bourgeois elements are with the opposition, not all workers with the Majority.” Some of them will surely add that the article “ascribes” ideas to them which they have never entertained, etc.

For answers to “concrete” questions, the oppositionists want recipes from a cookbook for the epoch of imperialist wars. I don’t intend to write such a cookbook. But from our principled approach to the fundamental questions we shall always be able to arrive at a correct solution for any concrete problem, complicated though it might be. Precisely in the Finnish problem the opposition demonstrated its incapacity to answer concrete questions.

There are never factions which are chemically pure in their composition. Petty-bourgeois elements find themselves necessarily in every workers’ party and faction. The question is only who sets the tone. With the opposition the tone is set by the petty-bourgeois elements.

The inevitable accusation that the article ascribes to the opposition ideas which they never entertained is explained by the formlessness and contradictory character of the opposition’s ideas which cannot bear the touch of critical analysis. The article “ascribes” nothing to the leaders of the opposition, it only develops their ideas to the end. Naturally I can see the development of the struggle only from the side-lines. But the general traits of the struggle can often be better observed from the side-lines.

\signed{I clasp your hand warmly, \\ \textsc{L.\ Trotsky}}
\signed{Coyoacán, D.F.}

\begin{letterinfo}
	\firstpublished\ Leon Trotsky, \emph{In Defense of Marxism}, New York 1942.
	
	\checkedagainst\ Leon Trotsky, \emph{In Defence of Marxism}, London 1966.
	
	\footnoteslatter
\end{letterinfo}