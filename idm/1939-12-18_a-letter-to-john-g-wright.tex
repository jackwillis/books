\chapterdated{A Letter to John G. Wright}{December 18, 1939}
\label{1939-12-18_a-letter-to-john-g-wright}

\letteraddress{Dear Friend,}

I read your letter to Joe. I endorse completely your opinion about the necessity for a firm even implacable theoretical and political fight against the petty-bourgeois tendencies of the opposition. You will see from my last article, which will be air-mailed to you to-morrow, that I characterize the divergence’s of the opposition even more sharply than has the majority. But at the same time, I believe that the implacable ideological fight should go parallel with very cautious and wise organizational tactics. You have not the slightest interest in a split, even if the opposition should become, accidentally, a majority at the next convention. You have not the slightest reason to give the heterogeneous and unbalanced army of the opposition a pretext for a split. Even as an eventual minority, you should in my opinion remain disciplined and loyal towards the party as a whole. It is extremely important for the education in genuine party patriotism, about the necessity of which Cannon wrote me one time very correctly.

A majority composed of this opposition would not last more than a few months. Then the proletarian tendency of the party will again become the majority with tremendously increased authority. Be extremely firm but don’t lose your nerve---this applies now more than ever to the strategy of the proletarian wing of the party.

\signed{With best comradely greetings and wishes, \\ Yours, \\ \textsc{Leon Trotsky}}
\signed{Coyoacán, D.F.}

\begin{postscriptum}
  P.S.---The evils came from:
  \begin{enumerate}
    \item Bad composition especially of the most important New York branch;
    \item Lack of experience especially by the members who came over from the Socialist Party (Youth).
  \end{enumerate}
  To overcome these difficulties inherited from the past is not possible by exceptional measures. Firmness and patience are necessary.
\end{postscriptum}

\begin{letterinfo}
	\firstpublished\ Leon Trotsky, \emph{In Defense of Marxism}, New York 1942.
	
	\checkedagainst\ Leon Trotsky, \emph{In Defence of Marxism}, London 1966, pp.~81.
\end{letterinfo}