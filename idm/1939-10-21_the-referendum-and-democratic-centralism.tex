\chapterdated[The Referendum and Democratic Centralism]{The Referendum and Democratic Centralism\footnote{In the course of its factional struggle, the minority put forward the demand for a referendum on the isue in dispute concerning the USSR. The majority opposed this. Trotsky came out in support of the majority’s rejection of a referendum. ---\ed}}{October 21, 1939}

\noindent
\textsc{We demand} a referendum on the war question because we want to paralyze or weaken the centralism of the imperialist state. But can we recognize the referendum as a normal method for deciding issues in our own party? It is not possible to answer this question except in the negative.

Whoever is in favor of a referendum recognizes by this that a party decision is simply an arithmetical total of local decisions, every one of the locals being inevitably restricted by its own forces and by its limited experience. Whoever is in favor of a referendum must be in favor of imperative mandates; that is, in favor of such a procedure that every local has the right to \emph{compel} its representative at a party convention to vote in a definite manner. Whoever recognizes imperative mandates automatically denies the significance of conventions as the highest organ of the party. Instead of a convention it is sufficient to introduce a counting of local votes. The party as a centralized whole disappears. By accepting a referendum the influence of the most advanced locals and most experienced and far-sighted comrades of the capital or industrial centers is substituted for the influence of the least experienced, backward sections, etc.

Naturally we are in favor of an all-sided examination and of voting upon every question by each party local, by each party cell. But at the same time every delegate chosen by a local must have the right to weigh all the arguments relating to the question in the convention and to vote as his political judgment demands of him. If he votes in the convention against the majority which delegated him, and if he is not able to convince his organization of his correctness after the convention, then the organization can subsequently deprive him of its political confidence. Such cases are inevitable. But they are incomparably a lesser evil than the system of referendums or imperative mandates which completely kill the party as a whole.

\signed{Coyoacán, D.F. \\ October 21, 1939}

\begin{letterinfo}
	\textbf{First published:} Leon Trotsky, \emph{In Defense of Marxism}, New York, 1942.
	
	\textbf{Checked against:} Leon Trotsky, \emph{In Defence of Marxism}, London 1966, pp.~40--41.
	
	All footnotes stem from the latter edition.
\end{letterinfo}