\chapterdated[Letter to James P.\@ Cannon]{Letter to James P.\@ Cannon\footnote{This letter was inadvertently omitted from the volume \emph{In Defense of Marxism} and was printed for the first time in James P. Cannon’s \emph{The Struggle for a Proletarian Party}, pp.~98--99.}}{October 28, 1939}

\letteraddress{Dear Jim:}

Two things are clear to me from your letter of October 24:

\begin{enumerate}
  \item that a very serious ideological fight has become inevitable and politically necessary;
  \item that it would be extremely prejudicial if not fatal to connect the ideological fight with the perspective of a split, of a purge, or expulsions, and so on and so forth.
\end{enumerate}

I heard for example that Comrade Gould proclaimed in a membership meeting: “You wish to expel us.” But I don’t know what reaction came from the other side to this. I for my part would immediately protest with the greatest vehemence such suspicions. I would propose the creation of a special control commission in order to check such affirmations and rumours. If it happens that someone of the majority launches such threats I for my part would vote for a censure or severe warning.

You may have many new members or uneducated youth. They need a serious educational discussion in the light of the great events. If their thoughts at the beginning are obsessed by the perspective of personal degradation, \ie, demotions, loss of prestige, disqualifications, eliminations from Central Committee, \etc, and so on, the whole discussion would become envenomed and the authority of the leadership would be compromised.

If the leadership on the contrary opens a ruthless fight against petty-bourgeois idealistic conceptions and organizational prejudices but at the same time guarantees for the discussion itself and for the minority, the result would be not only be an ideological victory but an important growth in the authority of the leadership.

“A conciliation and a compromise at the top” on the questions which form the matter of the divergences would of course be a crime. But I for my part would propose to the minority at the top an agreement, if you wish, a compromise on the methods of discussion and parallelly on the political collaboration. For example,

\begin{enumerate}
  \item[a)] both sides eliminate from the discussion any threats, personal denigration and so on;
  \item[b)] both sides take the obligation of loyal collaboration during the discussion;
  \item[c)] every false move (threats, or rumours of threats, or a rumour of alleged threats, resignations, and so on) should be investigated by the National Committee or a special commission as a particular fact and not thrown into the discussion and so on.
\end{enumerate}

If the majority accepts such an agreement you will have the possibility of disciplining the discussion and also the advantage of having taken a good initiative. If they reject it you can at every party membership meeting present your written proposition to the minority as the best refutation of their complaints and as an example of “our regime”.

It seems to me that the last convention failed at a very bad moment (the time was not ripe) and became a kind of abortion. The genuine discussion comes some time after the convention. This signifies that you can’t avoid a convention at Christmas or so. The idea of a referendum is absurd. It could only facilitate a split on local lines. I believe that the majority in the above-mentioned agreement can propose to the minority a new convention on the basis of two platforms with all the organizational guarantees for the minority.

The convention is expensive but I don’t see any other means of concluding the present discussion and the party crisis it produces.

\signed{\textsc{J.\ Hansen} [Leon Trotsky]}

\begin{postscriptum}
  P.S.---Every serious and sharp discussion can of course lead to some desertions, departures, or even expulsions, but the whole party should be convinced from the logic of the facts that they are inevitable results occurred in spite of the best will of the leadership, and not an objective of aim of the leadership, and not the point of departure of the whole discussion. This is in my mind the decisive point of the whole matter.
\end{postscriptum}

\begin{letterinfo}
  \firstpublished\ James P. Cannon, \emph{The Struggle for a Proletarian Party}.
  
  \checkedagainst Leon Trotsky, \emph{In Defence of Marxism}, London 1966, pp.~45--47.
  
  \footnoteslatter
\end{letterinfo}