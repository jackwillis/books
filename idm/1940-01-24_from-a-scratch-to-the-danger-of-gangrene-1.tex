\newcommand{\documentnumber}[1]{
  \subsection*{\parbox{\linewidth}{\centering #1}}
}

\label{1940-01-24_from-a-scratch-to-the-danger-of-gangrene}

\chapterdated{From a Scratch---to the Danger of Gangrene}{January 24, 1940}

\noindent
\textsc{The discussion is} developing in accordance with its own internal logic. Each camp, corresponding to its social character and political physiognomy, seeks to strike at those points where its opponent is weakest and most vulnerable. It is precisely this that determines the course of the discussion and not \emph{a priori} plans of the leaders of the opposition. It is belated and sterile to lament now over the flaring up of the discussion. It is necessary only to keep a sharp eye on the role played by Stalinist provocateurs who are unquestionably in the party and who are under orders to poison the atmosphere of the discussion and to head the ideological struggle toward split. It is not so very difficult to recognize these gentlemen; their zeal is excessive and of course artificial; they replace ideas and arguments with gossip and slander. They must be exposed and thrown out through the joint efforts of both factions. But the principled struggle must be carried through to the end, that is, to serious clarification of the more important questions that have been posed. It is necessary to so utilize the discussion that it raises the theoretical level of the party.

A considerable proportion of the membership of the American section as well as our entire young International, came to us either from the Comintern in its period of decline or from the Second International. These are bad schools. The discussion has revealed that wide circles of the party lack a sound theoretical education. It is sufficient, for instance, to refer to the circumstance that the New York local of the party did not respond with a vigorous defensive reflex to the attempts at light-minded revision of Marxist doctrine and pro gram but on the contrary gave support in the majority to the revisionists. This is unfortunate but remediable to the degree that our American section and the entire International consist of honest individuals sincerely seeking their way to the revolutionary road. They have the desire and the will to learn. But there is no time to lose. It is precisely the party’s penetration into the trade unions, and into the workers’ milieu in general that demands heightening the theoretical qualification of our cadres. I do not mean by cadres the “apparatus” but the party as a whole. Every party member should and must consider himself an officer in the proletarian army.

“Since when have you become specialists in the question of philosophy?” the oppositionists now ironically ask the majority representatives. Irony here is completely out of place. Scientific socialism is the conscious expression of the unconscious historical process; namely, the instinctive and elemental drive of the proletariat to reconstruct society on communist beginnings. These organic tendencies in the psychology of workers spring to life with utmost rapidity today in the epoch of crises and wars. The discussion has revealed beyond all question a clash in the party between a petty-bourgeois tendency and a proletarian tendency. The petty-bourgeois tendency reveals its confusion in its attempt to reduce the program of the party to the small coin of “concrete” questions. The proletarian tendency on the contrary strives to correlate all the partial questions into theoretical unity. At stake at the present time is not the extent to which individual members of the majority consciously apply the dialectic method. What is important is the fact that the majority as a whole pushes toward the proletarian posing of the questions and by very reason of this tends to assimilate the dialectic which is the “algebra of the revolution.” The oppositionists, I am informed, greet with bursts of laughter the very mention of “dialectics.” In vain. This unworthy method will not help. The dialectic of the historic process has more than once cruelly punished those who tried to jeer at it.

Comrade Shachtman’s latest article, ``An Open Letter to Leon Trotsky,'' is an alarming symptom. It reveals that Shachtman refuses to learn from the discussion and persists instead in deepening his mistakes, exploiting thereby not only the inadequate theoretical level of the party, but also the specific prejudices of its petty-bourgeois wing. Everybody is aware of the facility with which Shachtman is able to weave various historical episodes around one or another axis. This ability makes Shachtman a talented journalist. Unfortunately, this by itself is not enough. The main question is what axis to select. Shachtman is absorbed always by the reflection of politics in literature and in the press. He lacks interest in the actual processes of the class struggle, the life of the masses, the inter-relationships between the different layers within the working class itself, etc. I have read not a few excellent and even brilliant articles by Shachtman but I have never seen a single commentary of his which actually probed into the life of the American working class or its vanguard.

A qualification must be made to this extent---that not only Shachtman’s personal failing is embodied therein, but the fate of a whole revolutionary generation which because of a special conjuncture of historical conditions grew up outside the labor movement. More than once in the past I have had occasion to speak and write about the danger of these valuable elements degenerating \emph{despite} their devotion to the revolution. What was an inescapable characteristic of adolescence in its day has become a weakness. Weakness invites disease. If neglected, the disease can become fatal. To escape this danger it is necessary to open a new chapter consciously in the development of the party. The propagandists and journalists of the Fourth International must begin a new chapter in their own consciousness. It is necessary to re-arm. It is necessary to make an about-face on one’s own axis: to turn one’s back to the petty-bourgeois intellectuals, and to face toward the workers.

To view as the cause of the present party crisis---the conservatism of its worker section; to seek a solution to the crisis through the victory of the petty-bourgeois bloc---it would be difficult to conceive a mistake more dangerous to the party. As a matter of fact, the gist of teh present crisis consists in the conservatism of the petty-bourgeois elements who have passed through a purely propagandistic school and who have not yet found a pathway to the road of class struggle. The present crisis is the final battle of these elements for self-preservation. Every oppositionist as an individual can, if he firmly desires, find a worthy place for himself in the revolutionary movement. As a faction they are doomed. In the struggle that is developing, Shachtman is not in the camp where he ought to be. As always in such cases, his strong sides have receded into the back ground while his weak traits on the other hand have assumed an especially finished expression. His ``Open Letter'' represents, so to speak, a crystallization of his weak traits.

Shachtman has left out a trifle: his class position. Hence his extraordinary zigzags, his improvisations and leaps. He replaces class analysis with disconnected historical anecdotes for the sole purpose of covering up his own shift, for camouflaging the contradiction between his yesterday and today. This is Shachtman’s procedure with the history of Marxism, the history of his own party, and the history of the Russian Opposition. In carrying this out, he heaps mistakes upon mistakes. All the historical analogies to which he resorts, speak, as we shall see, against him.

It is much more difficult to correct mistakes than to commit them. I must ask patience from the reader in following with me step by step all the zigzags of Shachtman’s mental operations. For my part I promise not to confine myself merely to exposing mistakes and contradictions, but to counterpoise from beginning to end the proletarian position against the petty-bourgeois, the Marxist position against the eclectic. In this way all of us perhaps may learn some thing from the discussion.

\section*{“Precedents”}

“How did we, irreconcilable revolutionists, so suddenly become a petty-bourgeois tendency?” Shachtman demands indignantly. Where are the proofs? “Wherein [has] this tendency manifested itself in the last year (!) or two among the representative spokesmen of the Minority?” (\emph{Internal Bulletin}, Vol.~2 No.~7, Jan.~1940, p.~11). Why didn’t we yield in the past to the influence of the petty-bourgeois democracy? Why during the Spanish Civil War did we\dots\ and so forth and so on. This is Shachtman’s trump argument in beginning his polemic against me and the one on which he plays variations in all keys, apparently investing it with exceptional importance. It does not so much as enter Shachtman’s mind that I can turn this very argument against him.

The opposition document, “War and Bureaucratic Conservatism,” concedes that Trotsky is right nine times out of ten, perhaps ninety-nine times out of a hundred. I understand only too well the qualified and extremely magnanimous character of this concession. The proportion of my mistakes is in reality considerably greater. How explain then the fact that two or three weeks after this document was written, Shachtman suddenly decided that Trotsky:

\enlargethispage{-2 \baselineskip}

\begin{enumerate}
  \item[a)] is incapable of a critical attitude toward information supplied him although one of his informants for ten years has been Shachtman himself;
  
  \item[b)] is incapable of distinguishing a proletarian tendency from a petty-bourgeois tendency---a Bolshevik tendency from a Menshevik tendency;
  
  \item[c)] is champion of the absurd conception of “bureaucratic revolution” in place of revolution by the masses;
    
  \item[d)] is incapable of working out a correct answer to concrete questions in Poland, Finland, etc.;
    
  \item[e)] is manifesting a tendency to capitulate to Stalinism;
    
  \item[f)] is unable to comprehend the meaning of democratic centralism---and so on \emph{ad infinitum}.
\end{enumerate}

In a word, during the space of two or three weeks Shachtman has discovered that I make mistakes ninety-nine times out of a hundred, especially where Shachtman himself happens to become involved. It occurs to me that the latest percentage also suffers from slight exaggeration---but this time in the opposite direction. In any event Shachtman discovered my tendency to replace revolution by the masses with “bureaucratic revolution” far more abruptly than I discovered his petty-bourgeois deviation.

Comrade Shachtman invites me to present proof of the existence of a “petty-bourgeois tendency” in the party during the past year; or even two-three years. Shachtman is completely justified in not wishing to refer to the more distant past. But in accordance with Shachtman’s invitation, I shall confine myself to the last three years. Please pay attention. To the rhetorical questions of my unsparing critic I shall reply with a few exact documents.

\documentnumber{I.}

\noindent
On May 25, 1937, I wrote to New York concerning the policy of the Bolshevik-Leninist faction in the Socialist Party:

\begin{quote}
  I must cite two recent documents:

  \begin{enumerate}
  	\item[a)] the private letter of ``Max'' about the convention, and
  	\item[b)] Shachtman’s article, ``Towards a Revolutionary Socialist Party.''
  \end{enumerate}

  The title of this article alone characterizes a false perspective. It seems to me established by the developments, including the last convention, that the party is evolving, not into a ``revolutionary'' party, but into a kind of \textsc{ilp}, that is, a miserable centrist political abortion without any perspective.

  The affirmation that the American Socialist Party is now ``closer to the position of revolutionary Marxism than any party of the Second or Third Internationals'' is an absolutely unmerited compliment: the American Socialist Party is only more backward than the analogous formations in Europe---the \textsc{poum}, \textsc{ilp}, \textsc{sap}, etc.\dots\ Our duty is to unmask this negative advantage of Norman Thomas and Co., and not to speak about the ``superiority [of the war resolution] over any resolution ever adopted before by the party\dots'' This is a \emph{purely literary appreciation}, because every resolution must be taken in connection with historical events, with the political situation and its imperative needs.
\end{quote}

In both of the documents mentioned in the above letter, Shachtman revealed excessive adaptability toward the left wing of the petty-bourgeois democrats---political mimicry---a very dangerous symptom in a revolutionary politician! It is extremely important to take note of his high appraisal of the “radical” position of Norman Thomas in relation to war\dots\ in Europe. Opportunists, as is well known, tend to all the greater radicalism the further removed they are from events. With this law in mind it is not difficult to appraise at its true value the fact that Shachtman and his allies accuse us of a tendency to “capitulate to Stalinism.” Alas, sitting in the Bronx, it is much easier to display irreconcilability toward the Kremlin than toward the American petty bourgeoisie.

\documentnumber{II.}

\noindent
To believe Comrade Shachtman, I dragged the question of the class composition of the factions into the dispute by the hair. Here too, let us refer to the recent past.
On October 3, 1937, I wrote to New York:

\begin{quote}
  I have remarked hundreds of times that the worker who remains unnoticed in the ``normal'' conditions of party life reveals remarkable qualities in a change of the situation when general formulas and fluent pens are not sufficient, where acquaintance with the life of workers and practical capacities are necessary. Under such conditions a gifted worker reveals a sureness of himself and reveals also his general political capabilities.
  
  Predominance in the organization of intellectuals is inevitable in the first period of the development of the organization. It is at the same time a big handicap to the political education of the more gifted workers\dots\ It is absolutely necessary at the next convention to introduce in the local and central committees as many workers as possible. To a worker, activity in the leading party body is at the same time a high political school\dots
  
  The difficulty is that in every organization there are traditional committee members and that different secondary, factional and personal considerations play a too great role in the composition of the list of candidates.
\end{quote}

I have never met either attention or interest from Comrade Shachtman in questions of this kind.

\documentnumber{III.}

\noindent
To believe Comrade Shachtman, I injected the question of Comrade Abern’s faction as a concentration of petty-bourgeois individuals artificially and without any basis in fact. Yet on October 10, 1937, at a time when Shachtman marched shoulder to shoulder with Cannon and it was considered officially that Abern had no faction, I wrote to Cannon:

\begin{quote}
  The party has only a minority of genuine factory workers \dots\ The non-proletarian elements represent a very necessary yeast, and I believe that we can be proud of the good quality of these elements \dots\ But \dots\ Our party can be inundated by non-proletarian elements and can even lose its revolutionary character. The task is naturally not to prevent the influx of intellectuals by artificial methods, \dots\ but to orientate practically all the organization toward the factories, the strikes, the unions\dots\
  
  A concrete example: We cannot devote enough or equal forces to all the factories. Our local organization can choose for its activity in the next period one, two or three factories in its area and concentrate all its forces upon these factories. If we have in one of them two or three workers we can create a special help commission of five non-workers with the purpose of enlarging our influence in these factories.
  
  The same can be done among the trade unions. We cannot introduce non-worker members in workers’ unions. But we can with success build up help commissions for oral and literary action in connection with our comrades in the union. The unbreakable conditions should be: \emph{not to command the workers but only to help them}, to give them suggestions, to arm them with the facts, ideas, factory papers, special leaflets, and so on.
  
  Such collaboration would have a tremendous educational importance from one side for the worker comrades, from the other side for the non-workers who need a solid re-education.
  
  You have for example an important number of Jewish non-worker elements in your ranks. They can be a very valuable yeast \emph{if the party succeeds by and by in extracting them from a closed milieu} and ties them to the factory workers by daily activity. I believe \emph{such an orientation would also assure a more healthy atmosphere inside the party.}
  
  One general rule we can establish immediately: a party member who doesn’t win during three or six months a new worker for the party is not a good party member.
  
  If we established seriously such a general orientation and if we verified every week the practical results, we will avoid a great danger; namely, that the intellectuals and white collar workers might suppress the worker minority, condemn it to silence, transform the party into a very intelligent discussion club but absolutely not habitable for workers.
  \nowidow
  
  The same rules should be in a corresponding form elaborated for the working and recruiting of the youth organization, \emph{otherwise we run the danger of educating good young elements into revolutionary dilettantes and not revolutionary fighters.}
\end{quote}

From this letter it is obvious, I trust, that I did not mention the danger of a petty-bourgeois deviation the day following the Stalin-Hitler pact or the day following the dismemberment of Poland, but brought it forward persistently two years ago and more. Further more, as I then pointed out, bearing in mind primarily the “non-existent” Abern faction, it was absolutely requisite in order to cleanse the atmosphere of the party, that the Jewish petty-bourgeois elements of the New York local be shifted from their habitual conservative milieu and dissolved in the real labor movement. It is precisely be cause of this that the above letter (not the first of its kind), written more than two years before the present discussion began, is of far greater weight as evidence than all the writings of the opposition leaders on the motives which impelled me to come out in defense of the “Cannon clique.”

\documentnumber{IV.}

\noindent
Shachtman’s inclination to yield to petty-bourgeois influence, especially the academic and literary, has never been a secret to me. During the time of the Dewey Commission I wrote, on October 14, 1937, to Cannon, Shachtman and Warde:

\begin{quote}
  I insisted upon the necessity to surround the Committee by delegates of workers’ groups in order to create channels from the Committee in the masses \dots\ Comrades Warde, Shachtman and others declared themselves in agreement with me on this point. Together we analyzed the practical possibilities to realize this plan \dots\ But later, in spite of repeated questions from me, I never could have information about the matter and only accidentally I heard that Comrade Shachtman was opposed to it. Why? I don’t know.
\end{quote}

Shachtman never did divulge his reasons to me. In my letter I expressed myself with the utmost diplomacy but I did not have the slightest doubt that while agreeing with me in words Shachtman in reality was afraid of wounding the excessive political sensibilities of our temporary liberal allies: in \emph{this} direction Shachtman demonstrates exceptional “delicacy.”

\documentnumber{V.}

\noindent
On April 15, 1938, I wrote to New York:

\begin{quote}
  I am a bit astonished about the kind of publicity given to Eastman’s letter in the \emph{New International}. The publication of the letter is all right, but the prominence given it on the cover, combined with the silence about Eastman’s article in \emph{Harper’s}, seems to me a bit compromising for the \emph{New International}. Many people will interpret this fact as our willingness to close our eyes on principles when friendship is concerned.
\end{quote}

\documentnumber{VI.}

\noindent
On June 1, 1938, I wrote Comrade Shachtman:

\begin{quote}
  It is difficult to understand here why you are so tolerant and even friendly toward Mr. Eugene Lyons. He speaks, it seems, at your banquets; at the same time he speaks at the banquets of the White Guards.

  This letter continued the struggle for a more independent and resolute policy toward the so-called “liberals,” who, while waging a struggle against the revolution, wish to maintain “friendly relations” with the proletariat, for this doubles their market value in the eyes of bourgeois public opinion.
\end{quote}

\documentnumber{VII.}

\noindent
On October 6, 1938, almost a year before the discussion began, I wrote about the necessity of our party press turning its face decisively toward the workers:

\begin{quote}
  Very important in this respect is the attitude of the \emph{Socialist Appeal}. It is undoubtedly a very good Marxist paper, but it is not a genuine instrument of political action \dots\ I tried to interest the editorial board of the \emph{Socialist Appeal} in this question, but without success.
\end{quote}

A note of complaint is evident in these words. And it is not accidental. Comrade Shachtman as has been mentioned already displays far more interest in isolated literary episodes of long-ago-concluded struggles than in the social composition of his own party or the readers of his own paper.

\documentnumber{VIII.}

\noindent
On January 20, 1939, in a letter which I have already cited in connection with dialectic materialism, I once again touched on the question of Comrade Shachtman’s gravitation toward the milieu of the petty-bourgeois literary fraternity.

\begin{quote}
  I cannot understand why the \emph{Socialist Appeal} is almost neglecting the Stalinist Party. This party now represents a mass of contradictions. Splits are inevitable. The next important acquisitions will surely come from the Stalinist Party. Our political attention should be concentrated on it. We should follow the development of its contradictions day by day and hour by hour. Someone on the staff ought to devote the bulk of his time to the Stalinists’ ideas and actions. We could provoke a discussion and, if possible, publish the letters of hesitating Stalinists.

  It would be a thousand times more important than inviting Eastman, Lyons and the others to present their individual sweatings. I was wondering a bit at why you gave place to Eastman’s last insignificant and arrogant article\dots\ But I am absolutely perplexed that you, personally, \emph{invite} these people to besmirch the not-so-numerous pages of the \emph{New International}. The perpetuation of this polemic can interest some \emph{petty-bourgeois intellectuals}, but not the revolutionary elements.

  It is my firm conviction that a certain reorganization of the New International and the \emph{Socialist Appeal} is necessary: more distance from Eastman, Lyons and so on; and nearer the workers and, in this sense, to the Stalinist Party.
\end{quote}

Recent events have demonstrated, sad to say, that Shachtman did not turn away from Eastman and Co.\ but on the contrary drew closer to them.

\documentnumber{IX.}

\noindent
On May 27, 1939, I again wrote concerning the character of the \emph{Socialist Appeal} in connection with the social composition of the party:

\begin{quote}
  From the mintutes I see that you are having difficulty with the \emph{Socialist Appeal}. The paper is very well done from the journalistic point of view; but it is a paper for the workers and not a workers’ paper.
  
  As it is, the paper is divided among various writers, each of whom is very good, but collectively they do not permit the workers to penetrate to the pages of the \emph{Appeal}. Each of them speaks for the workers (and speaks very well) but nobody will hear the workers. In spite of its literary brilliance, to a certain degree the paper becomes a victim of journalistic routine. You do not hear at all how the workers live, fight, clash with the police or drink whiskey. It is very dangerous for the paper as a revolutionary instrument of the party. The task is not to make a paper through the joint forces of a skilled editorial board but to encourage the workers to speak for themselves.
  
  A radical and courageous change is necessary as a condition of success.
  
  Of course it is not only a question of the paper, but of the whole course of policy. I continue to be of the opinion that you have too many \emph{petty-bourgeois boys and girls} who are very good and devoted to the party, but who do not fully realize that their duty is not to discuss among themselves, but to penetrate into the fresh milieu of workers. I repeat my proposition: Every petty-bourgeois member of the party who, during a certain time, let us say three or six months, does not win a worker for the party, should be demoted to the rank of candidate and after another three months expelled from the party. In some cases it might be unjust, but the party as a whole would receive a salutary shock which it needs very much. A very radical change is necessary.
\end{quote}

In proposing such Draconian measures as the expulsion of those petty-bourgeois elements incapable of linking themselves to the workers, I had in mind not the “defense” of Cannon’s faction but the rescue of the party from degeneration.

\documentnumber{X.}

\noindent
Commenting on skeptical voices from the Socialist Workers Party which had reached my ears, I wrote Comrade Cannon on June 16, 1939:

\begin{quote}
  The pre-war situation, the aggravation of nationalism and so on is a natural hindrance to our development and the profound cause of the depression in our ranks. But it must now be underlined that \emph{the more the party is petty-bourgeois in its composition, the more it is dependent upon the changes in the official public opinion.} It is a supplementary argument for the necessity for a courageous and active re-orientation toward the masses.

  The pessimistic reasonings you mention in your article are, of course, a reflection of the patriotic, nationalistic pressure of the official public opinion. ``If fascism is victorious in France \dots'' ``If fascism is victorious in England \dots'' and so on. The victories of fascism are important, but the death agony of capitalism is more important.
\end{quote}

The question of the dependence of the petty-bourgeois wing of the party upon official public opinion consequently was posed several months before the present discussion began and was not at all dragged in artificially in order to discredit the opposition.

\triast

\noindent
\textsc{Comrade Shachtman} demanded that I furnish “precedents” of petty-bourgeois tendencies among the leaders of the opposition during the past period. I went so far in answering, this demand as to single out from the leaders of the opposition Comrade Shachtman himself. I am far from having exhausted the material at my disposal. Two letters---one of Shachtman’s, the other mine---which are perhaps still more interesting as “precedents,” I shall cite presently in another connection. Let Shachtman not object that the lapses and mistakes in which the correspondence is concerned likewise can be brought against other comrades, including representatives of the present majority. Possibly. Probably. But Shachtman’s name is not repeated in this correspondence accidentally. Where others have committed episodic mistakes, Shachtman has evinced a tendency.

In any event, completely opposite to what Shachtman now claims concerning my alleged “sudden” and “unexpected” appraisals, I am able, documents in hand, to prove---and I believe have proved---that my article on the ``Petty-Bourgeois Opposition''\footnote{See p.~\pageref{1939-12-15_a-petty-bourgeois-opposition-in-the-socialist-workers-party} ---J.W.} did no more than summarize my correspondence with New York during the last three years. (In reality the past ten.) Shachtman has very demonstratively asked for “precedents.” I have given him “precedents.” They speak entirely against Shachtman.

\section*{The Philosophic Bloc Against Marxism}

The opposition circles consider it possible to assert that the question of dialectic materialism was introduced by me only because I lacked an answer to the “concrete” questions of Finland, Latvia, India, Afghanistan, Balochistan and so on. This argument, void of all merit in itself, is of interest however in that it characterizes the level of certain individuals in the opposition, their attitude toward theory and toward elementary ideological loyalty. It would not be amiss, therefore, to refer to the fact that my first serious conversation with Comrades Shachtman and Warde, in the train immediately after my arrival in Mexico in January 1937, was devoted to the necessity of persistently propagating dialectic materialism. After our American section split from the Socialist Party I insisted most strongly on the earliest possible publication of a theoretical organ, having again in mind the need to educate the party, first and foremost its new members, in the spirit of dialectic materialism. In the United States, I wrote at that time, where the bourgeoisie systematically in stills vulgar empiricism in the workers, more than anywhere else is it necessary to speed the elevation of the movement to a proper theoretical level. On January 20, 1939, I wrote to Comrade Shachtman concerning his joint article with Comrade Burnham, ``Intellectuals in Retreat:''

\begin{quote}
  The section on the dialectic is the greatest blow that you, personally, as the editor of the \emph{New International} could have delivered to Marxist theory \dots\ Good! We will speak about it publicly.
\end{quote}

Thus a year ago I gave open notice in advance to Shachtman that I intended to wage a public struggle against his eclectic tendencies. At that time there was no talk whatever of the coming opposition; in any case furthest from my mind was the supposition that the philosophic bloc against Marxism prepared the ground for a political bloc against the program of the Fourth International.

The character of the differences which have risen to the surface has only confirmed my former fears both in regard to the social composition of the party and in regard to the theoretical education of the cadres. There was nothing that required a change of mind or “artificial” introduction. This is how matters stand in actuality. Let me also add that I feel somewhat abashed over the fact that it is almost necessary to justify coming out in defense of Marxism within one of the sections of the Fourth International.

In his ``Open Letter,'' Shachtman refers particularly to the fact that comrade Vincent Dunne expressed satisfaction over the article on the intellectuals. But I too praised it: “Many parts are excellent.” However, as the Russian proverb puts it, a spoonful of tar can spoil a barrel of honey. It is precisely this spoonful of tar that is involved. The section devoted to dialectic materialism expresses a number of conceptions monstrous from the Marxist standpoint, whose aim, it is now clear, was to prepare the ground for a political bloc. In view of the stubbornness with which Shachtman persists that I seized upon the article as a pretext, let me once again quote the central passage in the section of interest to us:
\begin{quote}
  \dots\ nor has anyone yet demonstrated that agreement or disagreement on the more abstract doctrines of dialectic materialism necessarily affects (!) today’s and tomorrow’s concrete political issues---and political parties, programs and struggles are based on such concrete issues. (\emph{The New International}, January 1939, p.~7)
\end{quote}
Isn’t this alone sufficient? What is above all astonishing is this formula, unworthy of revolutionists: “\dots\ political parties, programs and struggles are based on such concrete issues.” What parties? What programs? What struggles? All parties and all programs are here lumped together. The party of the proletariat is a party unlike all the rest. It is not at all based upon “such concrete issues.” In its very foundation it is diametrically opposed to the parties of bourgeois horse-traders and petty-bourgeois rag patchers. Its task is the preparation of a social revolution and the regeneration of mankind on new material and moral foundations. In order not to give way under the pressure of bourgeois public opinion and police repression, the proletarian revolutionist, a leader all the more, requires a clear, far-sighted, completely thought-out world outlook. Only upon the basis of a unified Marxist conception is it possible to correctly approach “concrete” questions.

Precisely here begins Shachtman’s betrayal---not a mere mistake as I wished to believe last year; but, it is now clear, an outright theoretical betrayal. Following in the footsteps of Burnham, Shachtman teaches the young revolutionary party that “no one has yet demon strated” prestimably that dialectic materialism affects the political activity of the party. “No one has yet demonstrated,” in other words, that Marxism is of any use in the struggle of the proletariat. The party consequently does not have the least motive for acquiring and defending dialectic materialism. This is nothing else than renunciation of Marxism, of scientific method in general, a wretched capitulation to empiricism. Precisely this constitutes the philosophic bloc of Shachtman with Burnham and through Burnham with the priests of bourgeois “Science.” It is precisely this and only this to which I referred in my January 20 letter of last year.

On March 5, Shachtman replied:
\begin{quote}
  I have reread the January article of Burnham and Shachtman to which you referred, and while in the light of which you have written I might have proposed a different formulation here (!) and there (!) if the article were to be done over again, I cannot agree with the substance of your criticism.
\end{quote}
This reply, as is always the case with Shachtman in a serious situation, in reality expresses nothing whatsoever; but it still gives the impression that Shachtman has left a bridge open for retreat. Today, seized with factional frenzy, he promises to “do it again and again tomorrow.” Do what? Capitulate to bourgeois “Science”? Renounce Marxism?

Shachtman explains at length to me (we shall see presently with what foundation) the utility of this or that \emph{political bloc}. I am speaking about the deadliness of \emph{theoretical betrayal}. A bloc can be justified or not depending upon its content and the circumstances. Theoretical betrayal cannot be justified by any bloc. Shachtman refers to the fact that his article is of purely political character. I do not speak of the article but of that section which renounces Marxism. If a text book on physics contained only two lines on God as the first cause it would be my right to conclude the author is an obscurantist.

Shachtman does not reply to the accusation but tries to distract attention by turning to irrelevant matters. “Wherein does what you call my ‘bloc with Burnham in the sphere of philosophy’ differ,” he asks, “from Lenin’s bloc with Bogdanov? Why was the latter principled and ours unprincipled? I should be very much interested to know the answer to this question.” I shall deal presently with the political difference, or rather the political polar opposite between the two blocs. We are here interested in the question of Marxist method. Wherein is the difference you ask? In this, that Lenin never declaimed for Bogdanov’s profit that dialectic materialism is superfluous in solving “concrete political questions.” In this, that Lenin never theoretically confounded the Bolshevik party with parties in general. He was organically incapable of uttering such abominations. And not he alone but not a single one of the serious Bolsheviks. That is the difference. Do you understand? Shachtman sarcastically promised me that he would be “interested” in a clear answer. The answer, I trust, has been given. I don’t demand the “interest.”
\newpage

\section*{The Abstract and the Concrete; Economics and Politics}

The most lamentable section of Shachtman’s lamentable opus is the chapter, ``The State and the Character of the War.''
\begin{quote}
  “What then is our position?” asks the author. “Simply this: It is impossible to deduce \emph{directly} our policy towards a \emph{specific} war from an \emph{abstract} characterization of the class character of the state involved in the war, more particularly, from the property forms prevailing in that state. Our policy must flow from a \emph{concrete} examination of the character of the war in relation to the interests of the international socialist revolution.” (\emph{Loc. cit.}, p.~13. My emphasis)
\end{quote}
What a muddle! What a tangle of sophistry! If it is impossible to deduce our policy \emph{directly} from the class character of a state, then why can’t this be done \emph{non-directly}? Why must the analysis of the character of the state be \emph{abstract} whereas the analysis of the character of the war is \emph{concrete}? Formally speaking, one can say with equal, in fact with much more right, that our policy in relation to the \USSR\ can be deduced not from an \emph{abstract} characterization of war as “imperialist,” but only from a \emph{concrete} analysis of the character of the state in the given historical situation. The fundamental sophistry upon which Shachtman constructs everything else is simple enough: In as much as the economic basis determines events in the super-structure not \emph{immediately}; inasmuch as the mere class characterization of the state is \emph{not enough} to solve the practical tasks, therefore\dots\ therefore we can get along without examining economics and the class nature of the state; by replacing them, as Shachtman phrases it in his journalistic jargon, with the “realities of living events.” (\emph{Loc. cit.}, p.14)

The very same artifice circulated by Shachtman to justify his philosophic bloc with Burnham (dialectic materialism determines our politics not immediately, consequently\dots\ it does not \emph{in general} affect the “concrete political tasks”), is repeated here word for word in relation to Marxist sociology: Inasmuch as property forms determine the policy of a state not immediately it is possible therefore to throw Marxist sociology overboard in general in determining “concrete political tasks.”

But why stop there? Since the law of labor value determines prices not “directly” and not “immediately”; since the laws of natural selection determine not “directly” and not “immediately” the birth of a suckling pig; since the laws of gravity determine not “directly” and not “immediately” the tumble of a drunken policeman down a flight of stairs, therefore\dots\ therefore let us leave Marx, Darwin, Newton, and all the other lovers of “abstractions” to collect dust on a shelf. This is nothing less than the solemn burial of science for, after all, the entire course of the development of science proceeds from “direct” and “immediate” causes to the more remote and pro found ones, from multiple varieties and kaleidoscopic events---to the unity of the driving forces.

The law of labor value determines prices not “immediately,” but it nevertheless does determine them. Such “concrete” phenomena as the bankruptcy of the New Deal find their explanation in the final analysis in the “abstract” law of value. Roosevelt does not know this, but a Marxist dare not proceed without knowing it. Not immediately but through a whole series of intermediate factors and their reciprocal interaction, property forms determine not only politics but also morality. A proletarian politician seeking to ignore the class nature of the state would invariably end up like the policeman who ignores the laws of gravitation; that is, by smashing his nose.

Shachtman obviously does not take into account the distinction between the abstract and the concrete. Striving toward concreteness, our mind operates with abstractions. Even “this,” “given,” “concrete” dog is an abstraction because it proceeds to change, for example, by dropping its tail the “moment” we point a finger at it. Concreteness is a relative concept and not an absolute one: what is concrete in one case turns out to be abstract in another: that is, insufficiently defined for a given purpose. In order to obtain a concept “concrete” enough \emph{for a given need} it is necessary to correlate several abstractions into one---just as in reproducing a segment of life upon the screen, which is a picture in movement, it is necessary to combine a number of still photographs.

\emph{The concrete is a combination of abstraction}---not an arbitrary or subjective combination but one that corresponds to the laws of the movement of a given phenomenon.

“The interests of the international socialist revolution,” to which Shachtman appeals against the class nature of the state, represent in this given instance the vaguest of all abstractions. After all, the question which occupies us is precisely this, in what concrete way can we further the interests of the revolution? Nor would it be amiss to remember, too, that the task of the socialist revolution is to create a workers’ state. Before talking about the socialist revolution it is necessary consequently to learn how to distinguish between such “abstractions” as the bourgeoisie and the proletariat, the capitalist state and the workers’ state.

Shachtman indeed squanders his own time and that of others in proving that nationalized property does not determine “in and of itself,” “automatically,” “directly,” “immediately” the policies of the Kremlin. On the question as to how the economic “base” determines the political, juridical, philosophical, artistic and so on “superstructure” there exists a rich Marxist literature. The opinion that economics presumably determines directly and immediately the creativeness of a composer or even the verdict of a judge, represents a hoary caricature of Marxism which the bourgeois professordom of all countries has circulated time out of end to mask their intellectual impotence.\footnote{To young comrades I recommend that they study on this question the works of Engels (\emph{Anti-Dühring}), Plekhanov and Antonio Labriola. ---L.T.}

As for the question which immediately concerns us, the inter-relationship between the social foundations of the Soviet state and the policy of the Kremlin, let me remind the absent-minded Shachtman that for seventeen years we have already been establishing, publicly, the growing \emph{contradiction} between the foundation laid down by the October Revolution and the tendencies of the state “superstructure.” We have followed step by step the increasing independence of tile bureaucracy from the Soviet proletariat and the growth of its dependence upon other classes and groups both inside and out side the country. Just what does Shachtman wish to add in this sphere to the analysis already made?

However, although economics determines politics not directly or immediately, but only in the last analysis, \emph{nevertheless economics does determine politics.} The Marxists affirm precisely this in contrast to the bourgeois professors and their disciples. While analyzing and exposing the growing political independence of the bureaucracy from the proletariat, we have never lost sight of the objective social boundaries of this “independence”; namely, nationalized property supplemented by the monopoly of foreign trade.

It is astonishing! Shachtman continues to support the slogan for a political revolution against the Soviet bureaucracy. Has he ever seriously thought out the meaning of this slogan? If we hold that the social foundations laid down by the October Revolution were “automatically” reflected in the policy of the state, then why would a \emph{revolution} against the bureaucracy be necessary? If the \USSR, on the other hand, has completely ceased being a workers’ state, not a \emph{political} revolution would be required but a \emph{social} revolution. Shachtman consequently continues to defend the slogan which follows (1) from the character of the \USSR\ as a workers’ state and (2) from the irreconcilable antagonism between the social foundations of the state and the bureaucracy. But as he repeats this slogan, he tries to undermine its theoretical foundation. Is it perhaps in order to demonstrate once again the independence of his politics from scientific “abstractions”?

Under the guise of waging a struggle against the bourgeois caricature of dialectic materialism, Shachtman throws the doors wide open to historical idealism. Property forms and the class character of the state are a matter of \emph{indifference} to him in analyzing the policy of a government. The state itself appears to him an animal of indiscriminate sex. Both feet planted firmly on this bed of chicken feathers, Shachtman pompously explains to us---today in the year 1940---that in addition to the nationalized property there is also the Bonapartist filth and their reactionary politics. How new! Did Shachtman perchance think that he was speaking in a nursery.

\section*{Shachtman Makes a Bloc---Also with Lenin}

To camouflage his failure to understand the essence of the problem of the nature of the Soviet state, Shachtman leaped upon the words of Lenin directed against me on December 30, 1920, during the so-called Trade Union Discussion.

\begin{quote}
  Comrade Trotsky speaks of the workers’ state. Permit me, this is an abstraction \dots\ Our state is in reality not a workers’ state but a workers’ and peasants’ state \dots\ Our present state is such that the inclusively-organized proletariat must defend itself, and we must utilize these workers’ organizations for the defense of the workers against their state and for the defense of our state by the workers.
\end{quote}

Pointing to this quotation and hastening to proclaim that I have repeated my “mistake” of 1920, Shachtman in his precipitance failed to notice a major error in the quotation concerning the definition of the nature of the Soviet state. On January 19, Lenin himself wrote the following about his speech of December 30:

\begin{quote}
  I stated, ``our state is in reality not a workers’ state but a workers’ and peasants’ state.'' \dots\ On reading the report of the discussion, I now see that I was wrong \dots\ I should have said: ``The workers’ state is an abstraction. In reality we have a workers’ state with the following peculiar features, (1) it is the peasants and not the workers who predominate in the population and (2) it is a workers’ state with bureaucratic deformations.''
\end{quote}

From this episode two conclusions follow: Lenin placed such great importance upon the precise sociological definition of the state that he considered it necessary to correct himself in the very heat of a polemic! But Shachtman is so little interested in the class nature of the Soviet state that twenty years later he noticed neither Lenin’s mistake nor Lenin’s correction!

I shall not dwell here on the question as to just how correctly Lenin aimed his argument against me. I believe he did so incorrectly---there was no difference of opinion between us on the definition of the state. But that is not the question now. The theoretical formulation on the question of the state, made by Lenin in the above-cited quotation---in conjunction with the major correction which he himself introduced a few days later---is absolutely correct. But let us hear what incredible use Shachtman makes of Lenin’s definition:

\begin{quote}
  “Just as it was possible twenty years ago,” he writes, “to speak of the term ‘workers’ state’ as an abstraction, so is it possible to speak of the term ‘degenerated workers’ state’ as an abstraction.” (\emph{Loc. cit.}, p.~14)
\end{quote}

It is self-evident that Shachtman fails completely to understand Lenin. Twenty years ago the term “workers’ state” could not be considered in any way an abstraction in general; that is, something not real or not existing. The definition “workers’ state,” while correct in and of itself, was inadequate in relation to the particular task; namely, the defense of the workers through their trade unions, and only in this sense was it abstract. However, in relation to the defense of the \USSR\ against imperialism this self-same definition was in 1920, just as it still is today, unshakably concrete, making it obligatory for workers to defend the given state.

Shachtman does not agree. He writes:

\begin{quote}
  Just as it was once necessary in connection with the trade union problem to speak concretely of what \emph{kind} of workers’ state exists in the Soviet Union, so it is necessary to establish in connection with the present war, the \emph{degree} of degeneration of the Soviet state \dots\ And the \emph{degree} of the degeneration of the regime cannot be established by abstract reference to the existence of nationalized property, but only by observing the realities~(!) of living~(!) events~(!). Just as it was once necessary in connection with the trade union problem to speak concretely of what kind of workers’ state exists in the Soviet Union, so it is necessary to establish in connection with the present war, the degree of degeneration of the Soviet state ... And the degree of the degeneration of the regime cannot be established by abstract reference to the existence of nationalized property, but only by observing the realities~(!) of living~(!) events~(!).
\end{quote}

From this it is completely incomprehensible why in 1920 the question of the character of the \USSR\ was brought up in connection with the trade unions, i.e., particular internal questions of the regime, while today it is brought up in connection with the defense of the \USSR, that is, in connection with the entire fate of the state. In the former case the workers’ state was counterpoised to the workers, in the latter case---to the imperialists. Small wonder that the analogy limps on both legs; what Lenin counterpoised, Shachtman identifies.

