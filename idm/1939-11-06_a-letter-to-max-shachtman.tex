\chapterdated[A Letter to Max Shachtman]{A Letter to Max Shachtman\footnote{This letter was written by Trotsky in English.}}{November 6, 1939}

\letteraddress{Dear Comrade Shachtman,}

I received the transcript of your speech of October 15\footnote{This speech was delivered to a membership meeting of the New York organization of the Socialist Workers Party. It is reproduced in the \emph{Internal Bulletin}, Vol.~II No.~3, issued on November 14, 1939. ---\ed} which you sent me, and I read it, of course, with all the attention it deserves. I found a lot of excellent ideas and formulations which seemed to me in full accordance with our common position as it is expressed in the fundamental documents of the Fourth International. But what I could not find was an explanation for your attack upon our previous position as “insufficient, inadequate and outdated.”

You say that “It is the concreteness of the events which differ from our theoretical hypothesis and predictions that changes the situation” (p.~17). But unfortunately you speak about the “concreteness” of the events very abstractly so that I cannot see in what respect they change the situation and what are the consequences of these changes for our politics. You mention some examples from the past. Hence, according to you, we “saw and foresaw” the degeneration of the Third International (p.~18); but only after the Hitler victory did we find it necessary to proclaim the Fourth International. This example is not formulated exactly. We foresaw not only the degeneration of the Third International but also the possibility of its regeneration. Only the German experience of 1929--1933 convinced us that the Comintern was doomed and nothing could regenerate it. But then we changed our policy fundamentally: To the Third International we opposed the Fourth International.

But we did not draw the same conclusions concerning the Soviet state. Why? The Third International was a party, a selection of people on the basis of ideas and methods. This selection became so fundamentally opposed to Marxism that we were obliged to abandon all hope of regenerating it. But the Soviet state is not only an ideological selection, it is a complex of social institutions which continues to persist in spite of the fact that the ideas of the bureaucracy are now almost the opposite of the ideas of the October Revolution. That is why we did not renounce the possibility of regenerating the Soviet state by political revolution. Do you believe now that we must change this attitude? If not, and I am sure that you don’t propose it, where is the fundamental change produced by the “concreteness” of events?

In this connection you quote the slogan of the \emph{independent Soviet Ukraine} which, as I see with satisfaction, you accept. But you add: “As I understand our basic position it always was to oppose separatist tendencies in the Federated Soviet Republic” (p.~19). In respect to this you see a fundamental “change in policy.” But:
\begin{enumerate}
  \item The slogan of an independent Soviet Ukraine was proposed before the Hitler-Stalin pact.
  \item This slogan is only an application on the field of the national question of our general slogan for the revolutionary overthrow of the bureaucracy.
\end{enumerate}

You could with the same right say: “As I understand our basic position it was always to oppose any rebellious acts against the Soviet government.” Of course, but we changed this basic position several years ago. I don’t really see what new change you propose in this connection now.

You quote the march of the Red Army in 1920 into Poland and into Georgia and you continue: “Now, if there is nothing new in the situation, why does not the majority propose to hail the advance of the Red Army into Poland, into the Baltic countries, into Finland\dots'' (p.~20). In this decisive part of your speech you establish that something is “new in the situation” between 1920 and 1939. Of course! This newness in the situation is the bankruptcy of the Third International, the degeneracy of the Soviet state, the development of the Left Opposition, and the creation of the Fourth International. This “concreteness of events” occurred precisely between 1920 and 1939. And these events explain sufficiently why we have radically changed our position toward the politics of the Kremlin, including its military politics.

It seems that you forget somewhat that in 1920 we supported not only the deeds of the Red Army but also the deeds of the \GPU. From the point of view of our appreciation of the state there is no principled difference between the Red Army and the \GPU. In their activities they are not only closely connected but intermeshed. We can say that in 1918 and the following years we hailed the Cheka in their fight against Russian counter-revolutionaries and imperialist spies but in 1927 when the \GPU began to arrest, to exile and to shoot the genuine Bolsheviks we changed our appreciation of this institution. This concrete change occurred at least 11 years before the Soviet-German pact. That is why I am rather astonished when you speak sarcastically about “the refusal even (!) of the majority to take the same position today that we all took in 1920\dots” (p.~20). We began to change this position in 1923. We proceeded by stages more or less in accordance with the objective developments. The decisive point of this evolution was for us 1933--34. If we fail to see just what the new fundamental changes are which you propose in our policy, it doesn’t signify that we go back to 1920!

You insist especially on the necessity of abandoning the slogan for the unconditional defense of the \USSR, whereupon you interpret this slogan in the past as our unconditional support of every diplomatic and military action of the Kremlin; \ie, of Stalin’s policy. No, my dear Shachtman, this presentation doesn’t correspond to the “concreteness of events.” Already in 1927 we proclaimed in the Central Committee: “For the socialist fatherland? Yes! For the Stalinist course? No!” (\emph{The Stalin School of Falsification}, p.~177). Then you seem to forget the so-called “thesis on Clemenceau” which signified that in the interests of the genuine defense of the \USSR, the proletarian vanguard can be obliged to eliminate the Stalin government and replace it with its own. This was proclaimed in 1927! Five years later we explained to the workers that this change of government can be effectuated only by political revolution. Thus we separated fundamentally our defense of the \USSR\ as a \emph{workers’ state} from the bureaucracy’s defense of the \USSR. Whereupon you interpret our past policy as unconditional support of the diplomatic and military activities of Stalin! Permit me to say that this is a horrible deformation of our whole position not only since the creation of the Fourth International but since the very beginning of the Left Opposition.

Unconditional defense of the \USSR\ signifies, namely, that our policy is not determined by the deeds, maneuvers or crimes of the Kremlin bureaucracy but only by our conception of the interests of the Soviet state and world revolution.

At the end of your speech you quote Trotsky’s formula concerning the necessity of subordinating the defense of the nationalized property in the \USSR\ to the interests of the world revolution, and you continue: “Now my understanding of our position in the past was that we vehemently deny any possible conflict between the two\dots\ I never understood our position in the past to mean that we \emph{subordinate} the one to the other. If I understand English, the term implies either that there is a conflict between the two or the possibility of such a conflict” (p.~37). And from this you draw the impossibility of maintaining the slogan of unconditional defense of the Soviet Union.

This argument is based upon at least two misunderstandings. How and why could the interests of maintaining the nationalized property be in “conflict” with the interests of the world revolution? Tacitly you infer that the \emph{Kremlin’s} (not our) policy of defense can come into conflict with the interests of the world revolution. Of course! At every step! In every respect! However our policy of defense is not conditioned by the Kremlin’s policy. This is the first misunderstanding. But, you ask, if there is not a conflict why the necessity of subordination? Here is the second misunderstanding. We must subordinate the defense of the \USSR\ to the world revolution insofar as we subordinate a \emph{part} to a \emph{whole}. In 1918 in the polemics with Bukharin, who insisted upon a revolutionary war against Germany, Lenin answered approximately: “If there should be a revolution in Germany now, then it would be our duty to go to war even at the risk of losing. Germany’s revolution is more important than ours and we should if necessary sacrifice the Soviet power in Russia (for a while) in order to help establish it in Germany.” A strike in Chicago at this time could be unreasonable in and of itself, but if it is a matter of helping a general strike on the national scale, the Chicago workers should subordinate their interests to the interests of their class and call a strike. If the \USSR\ is involved in the war on the side of Germany, the German revolution could certainly menace the immediate interests of the defense of the \USSR. Would we advise the German workers not to act? The Comintern would surely give them such advice, but not we. We will say: “We must subordinate the interests of the defense of the Soviet Union to the interests of the world revolution.”
\nowidow

Some of your arguments are, it seems to me, answered in Trotsky’s last article, ``Again and Once More Again on the Nature of the USSR,'' which was written before I received the transcript of your speech.

You have hundreds and hundreds of new members who have not passed through our common experience. I am fearful that your presentation can lead them into the error of believing that we were unconditionally for the support of the Kremlin, at least on the international field, that we didn’t foresee such a possibility as the Stalin-Hitler collaboration, that we were taken unawares by the events, and that we must fundamentally change our position. That is not true! And independently from all the other questions which are discussed or only touched upon in your speech (leadership, conservatism, party regime and so on) we must, in my opinion, again check our position on the Russian question with all the necessary carefulness in the interest of the American section as well as of the Fourth International as a whole.

The real danger now is not the “unconditional” defense of that which is worthy of defense, but direct or indirect help to the political current which tries to identify the \USSR\ with the fascist states for the benefit of the democracies, or to the related current which tries to put all tendencies in the same pot in order to compromise Bolshevism or Marxism with Stalinism. We are the only party which really foresaw the events, not in their empirical concreteness, of course, but in their general tendency. Our strength consists in the fact that we do not need to change our orientation as the war begins. And I find it very false that some of our comrades, moved by the factional fight for a “good regime” (which they, so far as I know, have never defined), persist in shouting: “We were taken unawares! Our orientation turned out to be false! We must improvise a new line! And so on.” This seems to me completely incorrect and dangerous.

\signed{With warmest comradely greetings, \\ \textsc{lund} [Leon Trotsky]}

\signed{CC to J.P.~Cannon.}

\begin{postscriptum}
  P.S.---The formulations in this letter are far from perfect since it is not an elaborated article, but only a letter dictated by me in English and corrected by my collaborator during the dictation.
\end{postscriptum}

\signed{L.}