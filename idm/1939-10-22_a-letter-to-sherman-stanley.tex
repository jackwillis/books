\chapterdated[A Letter to Sherman Stanley]{A Letter to Sherman Stanley\footnote{This letter was written by Trotsky in English.}}{October 22, 1939}

\letteraddress{Dear Comrade Stanley,}

It is with some delay that I answer your letter of October 11.

\begin{enumerate}
  \item You say that “there can be no serious differences or disagreements” on the Russian question. If this is so, why the terrible alarm in the party against the National Committee, \ie, its majority? You should not substitute your own conceptions for that of the minority members of the National Committee who considered the question serious and burning enough to provoke a discussion just at the threshold of the war.
	
  \item I cannot agree with you that my statement does not contradict that of Comrade M.S\@. The contradiction concerns two fundamental points:
	
  \begin{enumerate}
	\item the class nature of the USSR;
	\item the defense of the USSR.
  \end{enumerate}

  On the first question, Comrade M.S. places a question mark, which signifies that he denies the old decision and postpones making a new decision. A revolutionary party cannot live between two decisions, one annihilated, the other not presented. In the question of the defense of the \USSR\ or the new occupied territories against Hitler’s (or Great Britain’s) attack, Comrade M.S. proposes a revolution against Stalin and Hitler. This abstract formula signifies negating the defense in a concrete situation. I attempted to analyze this question in a new article sent yesterday by airmail to the National Committee.
	
  \item I agree with you completely that only a serious discussion can clarify the matter, but I don’t believe that voting simultaneously for the statement of the majority and that of Comrade M.S. could conribute to the necessary clarification.
	
  \item You state in your letter that the main issue is not the Russian question but the “internal regime.” I have heard this accusation often since almost the very beginning of the existence of our movement in the United States. The formulations varied a bit, the groupings too, but a number of comrades always remained in opposition to the “regime.” They were, for example, against the entrance into the Socialist Party (not to go further into the past). However it immediately occurred that not the entrance was the “main issue” but the regime. Now the same formula is repeated in connection with the Russian question.
	
  \item I for my part believe that the passage through the Socialist Party was a salutary action for the whole development of our party and that the "regime" (or the leadership) which assured this passage was correct against the opposition which at that time represented the tendency of stagnation.
	
  \item Now at the beginning of the war a new sharp opposition arises on the Russian question. It concerns the correctness of our program elaborated through innumerable disputes, polemics, and discussions during at least ten years. Our decisions are of course not eternal. If somebody in a leading position has doubts and only doubts, it is his duty toward the party to clarify himself by fresh studies or by discussions inside the leading party bodies before throwing the question into the party---not in the form of elaborated new decisions, but in the form of doubts. Of course from the point of view of the statutes of the party, everybody, even a member of the Political Committee, has the right to do so, but I don’t believe that this right was used in a sound manner which could contribute to the amelioration of the party regime.
	
  \item Often in the past I have heard accusations from comrades against the National Committee as a whole---its lack of initiative, and so on. I am not the attorney of the National Committee and I am sure that many things have been omitted which should have been done. But whenever I insisted upon concretization of the accusations, I learned often that the dissatisfaction with their own local activity, with their own lack of initiative, was transformed into an accusation against the National Committee which was supposed to be Omniscient, Omnipresent, Omnibenevolent.
	
  \item In the present case the National Committee is accused of “conservatism.” I believe that to defend the old programmatic decision until it is replaced by a new one is the elementary duty of the National Committee. I believe that such “conservatism” is dictated by the self-preservation of the party itself.
	
  \item Thus in two most important issues of the last period comrades dissatisfied with the “regime” have had in my opinion a false political attitude. The regime must be an instrument for correct policy and not for false. When the incorrectness of their policy becomes clear, then its protagonists are often tempted to say that not this special issue is decisive but the general regime. During the development of the Left Opposition and the Fourth International we opposed such substitutions hundreds of times. When Vereecken or Sneevilet or even Molinier were beaten on all their points of difference, they declared that the genuine trouble with the Fourth International is not this or that decision but the bad regime.
	
  \item I don’t wish to make the slightest analogy between the leaders of the present opposition in our American party and the Vereeckens, Sneevliets and so on; I know very well that the leaders of the opposition are highly qualified comrades and I hope sincerely that we will continue to work together in the most friendly manner. But I cannot help being disquieted by the fact that some of them repeat the same error at every new stage of the development of the party with the support of a group of personal adherents. I believe that in the present discussion this kind of procedure must be analyzed and severely condemned by the general opinion of the party which now has tremendous tasks to fulfill.
	
\end{enumerate}

\newpage

\signed{With best comradely greetings, \\ \textsc{crux} [Leon Trotsky]}

\begin{postscriptum}
  P.S.---In view of the fact that I speak in this letter about the majority and the minority of the National Committee, especially of the comrades of the M.S. resolution, I am sending a copy of this letter to Comrades Cannon and Shachtman.
\end{postscriptum}

\begin{letterinfo}
  \textbf{First published:} Leon Trotsky, \emph{In Defense of Marxism}, New York, 1942.

  \textbf{Checked against:} Leon Trotsky, \emph{In Defence of Marxism}, London 1966, pp. 1--2.

  All footnotes stem from the latter edition.
\end{letterinfo}