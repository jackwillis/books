\chapter{The Programme of the Catalan Coalition Government}

\lettrineO{n September 7, 1936}, in a speech criticizing the Madrid coalition with the bourgeoisie, Nin had raised the slogan, ``Down With the Bourgeois Ministers,'' and the crowd had gone wild with enthusiasm. But by September 18, \emph{La Batalla} published a resolution of the Central Committee of the \textsc{poum}, accepting coalitionism:

\begin{quotation}
  The Central Committee believes now, as always, that this government must be exclusively composed of representatives of the workers’ parties and trade union organizations. But, if this point of view is not shared by the other workers’ organizations, we are willing to leave the question open, the more especially as the [Catalonian] left republican movement is of a profoundly popular nature---which distinguishes it radically from the Spanish left republican movement---and the peasant masses and workers’ sections on which it is based are moving definitely toward the revolution, influenced by the proletarian parties and organizations.
  
  The important thing is the programme, and the hegemony of the proletariat, which must be guaranteed. On one point there can be no doubt: the new government must make a declaration of unquestionable principles, affirming its intention of turning the impulse of the masses into revolutionary legality, and directing it in the sense of the socialist revolution. As for proletarian hegemony, the absolute majority of workers’ representatives will make it fully certain.
\end{quotation}

The \emph{Esquerra} leadership, hardened bourgeois politicians of twenty--thirty years’ struggle against the proletariat, was thus transformed overnight by the \POUM\ into a movement ``of a profoundly popular nature.'' And to this piece of legerdemain the \POUM\ added the hitherto unknown principle of strategy, that the way to win the leftward-moving workers and peasants in the \emph{Esquerra} was by collaborating in a government with their bourgeois leaders!

\begin{quotation}
  The working class cannot simply lay hold of the ready-made state machinery and wield it for its own purposes,
\end{quotation}
declared Marx. This was the great lesson learned from the Paris Commune:

\begin{quotation}
  \noindent
  not, as in the past, to transfer the bureaucratic and military machinery from one hand to the other, \emph{but to break it up}; and that is the precondition of any real people’s revolution on the Continent. And this is what our heroic party comrades in Paris have attempted.
\end{quotation}

What is to replace the shattered state machinery? On this, the fundamental question of revolution, the meagre experience of the Commune was fully developed by Lenin and Trotsky. Parliamentarianism was to be destroyed. In its place rise the workers’ committees in the factories, the peasants’ committees on the land, the soldiers’ committees in the army, centralized in local, regional and, finally, the national soviets. Thus, the new state, a workers’ state, is based on industrial representation, which automatically disfranchises the bourgeoisie, except as, after the consolidation of workers’ power, they individually enter productive labour and are permitted to participate in electing the soviets. Between the old bourgeois state and the new workers’ state lies a chasm over which the bourgeoisie cannot return to power except by overthrowing the workers’ state.

It was this fundamental tenet, the essence of the accumulated experience of a century of revolutionary struggle, which the \POUM\ violated in entering the Generalidad.\endnote{Those who defended this violation---Lovestoneites, Norman Thomas socialists, \textsc{ilp} etc.---thereby indicate their own future conduct in the revolutionary crisis.} They received their ministry from the hands of President Companys. The new cabinet merely continued the work of the old, and like the old, could be dismissed and replaced by a more reactionary one. Behind the protective covering of the \POUM-\CNT-\PSUC-Esquerra cabinet, the bourgeoisie would weather the revolutionary offensive, gather its shattered forces, and, with the aid of the reformists, at the ripe moment, return to full power. To this end, it was not even necessary for the bourgeoisie to participate in the cabinet. There had been ``all-workers'' cabinets in Germany, Austria, England, which had thus enabled the bourgeoisie to weather critical situations, and then kick out the workers’ ministers.

The workers’ state, the dictatorship of the proletariat, cannot exist until the old bourgeois state is destroyed. It can only be brought into existence by the direct, \emph{political} intervention of the masses, through the factory and village councils (soviets) at that point where a majority in the soviets is wielded by the workers’ party or parties which are determined to overthrow the bourgeois state. Such was the basic theoretical contribution of Lenin. Precisely this conception, however, was bowdlerized by the \POUM. The same speech of Nin calling for the dismissal of the bourgeois ministers developed a conception which could only lead to preservation of the bourgeois state.

\begin{quotation}
  Dictatorship of the proletariat. Another conception which is an object of difference with the anarchists. The proletarian dictatorship means the authority exercised by the working class. In Catalonia we can affirm that the dictatorship of the proletariat already exists. (Applause)\dots\ Not many days ago the \textsc{fai} launched a manifesto which said that it would oppose all dictatorships exercised by whatever party. We are in agreement with them. The dictatorship of the proletariat cannot be exercised by one single sector of the proletariat but by all, absolutely all. No workers’ party or union centre has the right to exercise a dictatorship. Let those present know that if the \textsc{cnt} or the Communist Party or the Socialist Party would wish to exercise a dictatorship of a party it would confront us. The dictatorship of the proletariat must be exercised by all. (\emph{La Batalla}, September 8, 1936.)
\end{quotation}

\vspace{-0.25\baselineskip}

For the dictatorship of the proletariat, as a state form, resting on the broad foundations of the network of workers’, peasants’ and combatants’ councils throughout industry, the land and the fields of battle, Nin was here substituting an entirely different conception: an agreement among the top-leaderships of the workers’ organizations jointly to assume governmental responsibility. False, and having nothing whatever in common with the Marxist conception of proletarian dictatorship! How could the proletarian dictatorship be wielded jointly with the Stalinist-democrats and the social democrats who stood for bourgeois democracy? How could party agreements be the substitute for the necessary vast network of workers’ councils?

The Leninist prediction that every real revolution gives rise to organs of dual power had been confirmed on July 19: the militia committees, supply committees, workers’ patrols, etc., etc. Leninist strategy called for the centralization of these organs of dual power into a national centre, and the taking of state power through it. The dissolution of the organs of dual power, as in Germany 1919, was called by Lenin the ``liquidation of the revolution.''

Uneasy memories of this led the \POUM\ leaders, in announcing entry into the Generalidad, to end:

\vspace{-0.25\baselineskip}

\begin{quotation}
  We are in a transition state in which the force of events has obliged us to collaborate directly in the Council of the Generalidad, along with other workers’ organizations \dots\ From the committees of workers, peasants and soldiers, for the formation of which we are pressing, will spring the direct representation of the new proletarian power.
\end{quotation}

\vspace{-0.25\baselineskip}

But this was the last swan song of the committees of dual power. For one of the first steps taken by the new cabinet of the Generalidad was to \emph{dissolve all the revolutionary committees which arose on July 19.}

The Central Committee of the Militias was dissolved and its powers turned over to the Ministries of Defence and Internal Security. The local militia and anti-fascist committees, almost invariably proletarian in composition, which had been ruling the towns and villages, were dissolved and replaced by municipal administrations composed in the same proportion as the cabinet (Esquerra 3, \CNT\ 3, \PSUC\ 2, Peasants Union, \POUM, and Accio Catala, the right-wing bourgeois organization, 1 each). Then to make sure that no revolutionary organ had been overlooked, an additional decree was passed which deserves full quotation:

\begin{quote}
  \textsc{\textsf{Article 1.}}\quad There are dissolved in all Catalonia the local committees, whatever be the name or title they bear, as well as all those local organisms which may have risen to down the subversive movement, with cultural, economic or any other species of aims.
  
  \textsc{\textsf{Article 2.}}\quad Resistance to dissolving them will be considered as a fascist act and its instigators delivered to the Tribunals of Popular Justice.
  
  (October 9, 1936.)
\end{quote}

The dissolution of the committees marked the first great advance of the counter-revolution. It removed the nascent soviet danger and enabled the bourgeois state to begin retrieving in every sphere the power which had fallen from its hands on July 19. Completely disoriented, the \POUM\ made no attempt to harmonize its previous call for committees with its sanction for their dissolution two weeks later. On the other hand, there remained in the hands of the bourgeoisie its traditional lever, the parliament. For the \POUM\ did not even get, in return for participation in the government, a decree dissolving the parliament. On the contrary, the financial decrees of the new cabinet carried the usual article requiring an accounting to the Catalan parliament. Parliament is dead, the \POUM\ assured the workers, but the government it sat in did not say so. True, unlike Caballero, Companys dared not convene parliament for many months, but this legal instrument of bourgeois domination remained intact. The meeting of the parliamentary deputation on April 9, 1937, during a ministerial crisis, scared the \CNT\ back into the government. And after the May Days, having defeated the workers, Companys convened the parliament which the \POUM\ had sworn was dead!

One more important step for consolidating the power of the bourgeois state was carried out on October 27, 1936: a decree disarming the workers.

\begin{quote}
  \textsc{\textsf{Article 1.}}\quad All long arms [e.g.\ rifles, machine guns, etc.] to be found in the hands of citizens shall be delivered to the municipalities or recovered by them, in a period of eight days after publication of this decree. Such arms shall be deposited in the Artillery Headquarters and the Ministry of Defence in Barcelona, in order to take care of the needs of the front.
  
  \textsc{\textsf{Article 2.}}\quad At the end of the cited period those who retain such armament will be considered as fascists and judged with the rigour which their conduct deserves.
  
  (\emph{La Batalla}, October 28, 1936.)
\end{quote}

The \POUM\ and \CNT\ published this decree without a single word of explanation to their following!

Thus the salvation of the bourgeois state was achieved. The \POUM, having been utilized during the critical months, was kicked out of the government in a cabinet reorganization, December 12, 1936. The \CNT\ with its great following was utilized longer, particularly since it increasingly adapted itself to the domination of the bourgeoisie, and was, therefore, kicked out only in July of the next year. But the power which the \POUM\ and \CNT\ had enabled the government to arrogate to itself remained in the government’s hands.

\section{The Economic Programme of the Coalition}

Apart from the ``workers’ majority,'' the \POUM\ justified entry because of the ``socialist orientation'' of the government’s economic programme. This criterion was utterly false, for revolutionary Marxism has always made clear that the necessary precondition to socialist economics is the dictatorship of the proletariat.

The Bolsheviks in 1917 were even prepared, on the basis of a workers’ state, to permit the continued existence, for a period, of private industry in certain fields, modified by workers’ control of production. Precisely in those fields of economic life in which the Bolsheviks acted first, however, the Catalan coalition did not act: nationalization of the banks and of the land.

Finance capital, even in backward Spain as elsewhere, dominates all other forms of capital. Yet, all that the coalition agreed to, in Point~VIII of its economic programme, was: ``Workers' control of banking enterprises until arriving at the nationalization of banking.'' ``Workers' control'' in practice meant merely guarding against disbursements of funds to fascist sympathizers and unauthorized persons. ``Until'' put off nationalization of banking indefinitely---nothing was ever done about it. This vast lever meant, as the next months proved, that the collectivized industries were at the mercy of those who could withhold credits. Precisely through this means, the bourgeois state, month by month, was to whittle down the economic power of the working class.

The Bolsheviks had \emph{nationalized} the land and granted control of it to the local soviets: that meant the \emph{end of private property in land}. The peasant need not enter the collectives; he was, however, no longer able to buy and sell land, and no creditor could seize it.\endnote{Louis Fischer, with ignorance fortified by impudence, argues against the Spanish collectives that in Russia collectivization came many years after the revolution. He leaves out the little ``detail'' that Lenin’s first decree was the nationalization of the land and the end of private property in land.}

The ``radical'' Catalan programme, ``the collectivization of great rural properties and respect for small agrarian property,'' concealed a reactionary perspective: land could still be bought and sold. Even more important: according to the Catalan autonomy statute, the central government had the last word on economic questions involving all Spain, and it had only authorized seizure of \emph{fascist-owned estates}. The coalition ``ignored'' the discrepancy between the two decrees. The \POUM\ did not have sense enough to bring the discrepancy out into the open and force the central government to formally recognize the Catalan decree, or have the Generalidad declare its full autonomy in economic questions. That meant: once the bourgeoisie recovered its strength, the Madrid decree on the land would prevail.

On October 24, a long and intricate decree was promulgated, concretizing the government’s conception of ``collectivization of the great industries, public services and transport.'' Before entry into the government, the \POUM\ had criticized industrial ``collectivization,'' pointing out that the unions, and even the workers in individual factories, were treating them as their own property. ``Syndicalist capitalism'' was making of the factories merely a form of producers’ co-operatives, in which the workers divided the profits. But industry could be run efficiently only as a national entity, together with all banking facilities and a monopoly of foreign trade. Now the \POUM\ accepted ``collectivization,'' which was nothing more than producers’ co-operatives, though real planning was impossible without banking and trade monopolies. The ``control of foreign trade'' which was promised never materialized. The \POUM’s proposal to include in the decree an ``Industrial and Credit Bank of Catalonia to attend to the needs and requirements of collectivized industry,'' was rejected. Thus, the foundations were laid for cutting to pieces the industries seized by the workers.

Another deadly blow to the ``collectivized'' factories was the arrangement for compensation to their former owners. Contrary to popular thought, the question of compensation for confiscated property is not excluded in advance for revolutionary Marxists. If the bourgeoisie would not resist, Lenin offered to arrange for partial compensation. The \POUM\ correctly concluded that the Spanish bourgeoisie had either gone over to Franco already or were---those in the Loyalist area---in no position but to accept the ``opportunity to work, or if unable to work, social insurance, under the same conditions as other workers.'' (The question of compensation to foreign capitalists was not at issue, since all correctly agreed this had to be recognized; but under cover of this abstractly correct formula, the government was soon to ``compensate'' foreigners-by giving them back their factories! The rest of the coalition, including the anarchists, rejected the \POUM\ proposal. Nor did they arrange for definite norms of compensation. Nor did compensation---as it did in the case of foreign capital---rest on the government.

Instead, ``the inventoried credit balance of any firm'' would ``be placed at the credit of the beneficiary [former owner] `as a social compensation,''' and ``the compensation for Spanish owners shall be suspended for later determination.'' In plain English, this meant that compensation would be a charge on the collectivized enterprise, i.e., on the workers involved, and the amount of it was left for a later time; with the re-construction of the bourgeois power, the bourgeoisie would levy against the workers’ enterprises in favour of the former owners on the sole criterion of how far the bourgeoisie dared saddle the workers with enforced payments of interest on capital debt. If the government grew strong enough, the former owners would go on clipping their coupons and receiving their dividends, just as before. The \POUM\ correctly termed this question ``fundamental;'' but stayed in the coalition government, nevertheless.

The collectivization decree provided for intervention in each factory of a government agent as part of the Factory Council. In all enterprises employing over 500, its director had to be approved by the government. Once elected by the workers in the factory, the Factory Council remained for two years in office, except for outright dereliction of duty, thus ``freezing'' the political composition of the councils and making it impossible for a revolutionary party to win control of the factories. The General Councils, embracing a whole industry, were even less flexible, eight out of twelve members being appointed by the leaders of the \UGT\ and \CNT, and presided over by representatives of the government. These measures, ensuring no ``revolt from below,'' were approved by all, including the \POUM.

Is it not obvious that the economic programme of the Generalidad merely accepted some of the gains already made by the workers themselves, and combined them with a series of political and economic measures which would eventually wipe out those gains? Yet, for this and a seat in the cabinet, the \POUM\ sold its chance to lead the Spanish revolution. By its blanket acceptance of the governmental programme, the \CNT\ revealed the complete bankruptcy of anarchism as a road to the social revolution.\endnote{After the May Days, the Generalidad repudiated the legality of the decree collectivizing industry.}

\section{The International Policy of the Coalition}

Like their counterparts in Madrid, the Esquerra and \PSUC\ looked to the League of Nations and the ``great democracies'' for succour. Nor was the \CNT\ much better. Juan Peiró, after the fall of the Caballero government, naively declared that the \CNT\ had been assured that the moderate government programme was meant for foreign consumption only.\endnote{``\dots the International bourgeoisie refused to supply us with those requirements (arms). It was a tragic moment: we had to create the impression that the masters were not the revolutionary committees but rather the legal government; failing that, we should not have received anything at all \dots\ We must needs adapt ourselves to the inexorable circumstances of the moment, that is to say, accept governmental collaboration\dots'' (Garcia Oliver, ex-Minister of Justice, speech in Paris, text published by the anarchist \emph{Spain and the World}, July 2, 1937)

``Spain offers all liberal and democratic nations of the world the opportunity of undertaking a strong offensive against the fascist forces, and if this means war, they must accept it before it is too late. They must not wait until fascism has perfected its war machine.'' (\emph{Official English Edition}~107, Dec.~8, 1936, Generalidad Commissariat of Propaganda)

Federica Montseny (outstanding \CNT\ leader): ``I believe that a people of such great intelligence (England) will realize that the establishment of a fascist state to the south of France... would be directly against its interests. The fate of the world as well as the outcome of this war depends on England...'' (\emph{Ibid.}~108, Dec.~10, 1936)}

This undoubtedly explains why the \CNT\ sent no organized delegations abroad to campaign among the workers.

The \POUM\ too fell victim to this opportunist policy. Despite its abstractly correct understanding of the reactionary international role of the Soviet bureaucracy, and its criticism of the failure of Stalin to sell arms to Spain during the first three crucial months, the \POUM\ failed to understand the fact that the Soviet note of October 7, 1936---``if violation is not halted immediately, it will consider itself free from any obligation resulting from the agreement''---did not mean leaving the non-intervention committee, and in no sense guaranteed sufficient arms shipments to turn the tide.

\begin{quotation}
  There is no doubt that the recent step of the Soviet Government in breaking the non-intervention pact will be of extraordinary political consequence. It is probably the most important political event since the commencement of the civil war,
\end{quotation}
said \emph{La Batalla}. Even worse, the \POUM’s perspective was that French imperialism would send arms:

\begin{quotation}
	How will the French Government reply to this new situation? Will it keep its attitude of neutrality? This would mean utter unpopularity and discredit. Blum would fall from power in the midst of general condemnation \dots\ We do not believe that Lion Blum would commit such a colossal blunder. Seeing that the only obstacle in the way of the correction of his policy was the Soviet Government’s attitude, the change in the latter should determine a complete change in Blum’s policy. (\emph{La Batalla}, October 11, 1936.)
\end{quotation}

Here, as everywhere, the \POUM\ had lost its bearings. It is not accidental that during its ministerial months, it sent no delegations abroad to campaign among the advanced workers.