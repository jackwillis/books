\chapter{Only Two Roads}

\lettrineS{ixteen months} of civil war have conclusively demonstrated that all the roads pointed out to the Spanish people reduce themselves to but two choices. One is the road we point: revolutionary war against fascism. All other paths lead into the road marked out by Anglo--French imperialism.

Anglo--French imperialism has demonstrated that it has no intention whatsoever of aiding the Loyalists to victory. Even the Stalinized \emph{New Republic} (October~27, 1937) was at long last compelled to admit: ``It is clear that by now the concern of France and England over a fascist victory in Spain has become~-- if it was not from the start~-- a~completely secondary consideration.\kn\kn''

The Spanish question is but one factor in the conflict of interests among the imperialist powers, and will be finally ``settled''~-- if the imperialists of both camps have their way~-- only when they come to the point of a general settlement of all questions, i.e., the imperialist~war.

Having most to lose, the Anglo--French bloc holds back from the war, although it must eventually fight to hold its own. Until that moment it avoids decisive showdowns, in Spain as elsewhere. It permitted a trickle of aid to the Loyalists from the Soviet Union because it did not want a victory for Franco while his Italo--German allies dominated his regime. British interests have employed the interim to arrange with Burgos for joint exploitation of the British-owned Bilbao region. The first week of November, Chamberlain announced establishment of formal relations with Franco (as small coin to antifascist sentiment, the diplomatic and consular officials were designated merely as ``agents''), while Eden assured Parliament that a Franco victory would not mean a regime hostile to Britain. Thus, the masters of the Anglo--French bloc prepared themselves for a Franco victory.

Whatever fears the Anglo--French bloc may have had about a Fran\-co victory, it never wanted a Loyalist victory. An early victory would have been followed by social revolution. Even now, after six months of repressions by the Negr\'in government, the Anglo--French rulers still doubt that a Loyalist victory would not be followed by social revolution. They are correct. For the millions of \CNT\ and \UGT\ workers, held in leash by the civil war, at its victorious completion would shatter the bourgeois limits of the People’s Front. Moreover, an imminent Loyalist victory would be such a blow to Italo--German prestige that it would be countered by invasion of Spain on an imperialist scale of warfare and an attempt to bottle up the Mediterranean. The danger to the ``lifeline of empire'' of the Anglo--French bloc would bring war on the immediate order of the day. Anglo--French desire to postpone the war thus led directly to opposition to Loyalist victory.

The only reason the Anglo--French bloc did not openly court Franco was that it dared not abandon its chief advantage in the coming war: the myth of democratic war against fascism, by which the proletariat is being mobilized to support the imperialist war.

The main preoccupation of Anglo--French imperialism from the first was: how postpone the war\kn, maintain the democratic myth, and yet begin to edge Hitler and Mussolini out of Spain? The answer was also obvious: a compromise between the Loyalist and fascist camps. As early as December~17, 1936, ``Augur'' semi-officially stated that English agents were working for a local armistice in the North, while French agents were doing likewise in Catalonia. Even the social patriot, Zyromski, stated in \emph{Populaire} (March~3, 1937):

\begin{quotation}
  Moves can be seen that are aiming at concluding a peace which would signify not only the end of the Spanish revolution, but also the total loss of the social victories already achieved.  
\end{quotation}

The Caballero Socialist, Luis Araquist\'ain, Ambassador to France from September 1936 to May 1937, afterward declared:

\begin{quotation}
  We have counted too much, in illusion and hope, on the London committee, that is to say, on the aid of the European democracies. Now is the hour to realize that we can expect nothing decisive from them in our favor, and from one of them much against us, at the least. (\emph{Adelante,} July~18, 1937).
\end{quotation}

The Negr\'in government put itself entirely in the hands of the Anglo--French bloc; and Negr\'in’s speeches, notably that to the Cortes on October~1, in dwelling on the need for preparing for peace, and his speech after the fall of Gij\'on, revealed that his government was ready to carry out the Anglo--French proposals for compromise.

The face of Negr\'in was turned not to the battle fronts but to London and Paris. The government’s orientation was summed up succinctly by the pro-Loyalist Matthews, after the fall of Gij\'on: ``On the whole, there is more discouragement here over the London discussion than what has happened in the north.'' Matthews continued:

\begin{quotation}
  There was a passage in Premier Negr\'in’s broadcast last ni\-ght which so perfectly expressed the government’s opinion that it deserves to be put on record: ``Once more our foreign enemies are trying to take advantage of the \emph{ingenuous candor of European democracies} by fine subtleties\dots. I am now warning the free countries of the world, for our cause is their cause. \emph{Spain will accept any means of reducing the anguish of this country,} but let the democracies not be seduced by the Machiavellianism of their worst enemies, and let them not again be the victim of a base decision.\kn\kn'' (\emph{New York Times,} October~24, 1937).
\end{quotation}

Truly that passage perfectly expressed the government’s opinion. Were not the consequences of this policy so tragic for the masses, one would roar with laughter at the picture of the ``ingenuous candor'' of perfidious Albion\kn\footnote{The oldest name of Great Britain.} and the Quai d’Orsay.\kn\kn\footnote{The location of the French Ministry of Foreign Affairs.} Fearing that he was to be abandoned altogether\kn, Negr\'in was thus begging his imperialist mentors to remember that he ``will accept any means of reducing the anguish of this country.\kn\kn'' Had he not already proved that by his repressions of the workers?\kp\kp%
\endnote{%
\indexLFischer
``Chautemps reflects the bourgeois and fascist dislike of Valencia. He, therefore, constantly urges Valencia to moderate its action and emphasize the democratic character of the regime.\kn\kn'' This testimony is Louis Fischer’s! (\emph{The Nation,} October~16, 1937).
}

That the Loyalist government had already agreed to support a compromise with the fascists is attested to, not only by authoritative revolutionary and bourgeois sources, but also by a Stalinist source:

\indexLFischer
\begin{quotation}
  A Spanish government representative who attended King George~VI’s coronation outlined to Foreign Minister, Eden, Valencia’s plan for ending the civil war. A truce was to be declared. All foreign troops and volunteers serving on both sides would then be immediately withdrawn from Spain. During the truce no battle lines would be shifted. Non-Spaniards having been eliminated, Great Britain, France, Germany\kn, Italy and the Soviet Union were to devise a scheme, which the Spanish government \emph{pledged itself in advance to accept,} whereby the will of the Spanish nation regarding its political and social future might be authoritatively ascertained. (Louis Fischer\kn, \emph{The~Nation,} September~4, 1937).
\end{quotation}

At the best, such an arrangement would mean a plebiscite under the supervision of the European powers. With Franco in possession of territory including more than half of the Spanish people, and with the Italo--German and Anglo--French blocs competing for Franco’s friendship, one can imagine the outcome of the plebiscite: unity of the bourgeois elements in both Spanish camps in a Bonapartist regime, decked out at the beginning with formal democratic rights, but ruling the masses primarily through the armed might of Franco’s~armies.

Such was the end of the road pointed out by the Anglo--French imperialists and already accepted by the Negr\'in government. There were still objective difficulties in the way: Franco hoped to win everything and was encouraged to fight on by Italy and Germany. But this much was clear. If not a complete Franco victory, to which England and France were already reconciled, then the best that can come from Anglo--French ``aid'' was a joint regime with the fascists.


Stalin might find this a bitter pill to swallow. However a compromise with the fascists would be dressed up, it would, nevertheless, be a terrible blow to Stalinist prestige throughout the world. But rather than break with the main objective of Soviet policy~-- the winning of an alliance with Anglo--French imperialism~-- Stalin was ready to submit to a settlement dictated by them. He would ``find a formula.\kn\kn'' The same arguments which were used to justify Soviet entry into the non-intervention committee, if accepted, would justify the final act of treachery against the Spanish people.

\medskip

Let us recall those shabby arguments.

\begin{quotation}
  The Soviet Union emphatically was not in favor of the non-intervention agreement. With sufficient support from the socialist parties, the labor and antifascist movements of the world, besides the support of the communist parties, the Soviet Union would have been able to stop the non-interven\-tion move in its track.\endnote{Harry Gannes, \emph{How the Soviet Union Helps Spain,} November~1936. This was the official Stalinist apology for supporting the London committee.}\label{en:GannesUSSR}
\end{quotation}

Need we remind anyone that Stalin never tried to rally the world labor movement before he endorsed the non-intervention scheme? If the Stalin regime was powerless to stop the bandits, did it have to join them? The Stalinists understood quite well the role of England: 

\begin{quotation}
  The Baldwin cabinet gauged its international action to retain the goodwill of the prospective fascist dictators of Spain [and] \dots\ to prevent a victory by the People’s Front\dots. Sufficient has appeared \dots\ to make positive the assertion that Britain has come to its own agreement with General Franco.\kn\endnotemark[\ref{en:GannesUSSR}]
\end{quotation}

But what mattered the fate of Spain, the future of the European revolution? All that weighed nothing in Stalin’s scale as against the tenuous friendship of imperialist France:

\begin{quotation}
  The Soviet Union could not come to an open clash with Blum on the non-intervention pact because that would have played into the hands of Hitler and the pro-Nazi faction in the London Tory cabinet which was trying to provoke just such a state of affairs.\endnotemark[\ref{en:GannesUSSR}]  
\end{quotation}

\noindent
Therefore?

\smallskip

\noindent
Pretend that the non-intervention committee has its uses:

\begin{quotation}
  Rather than to allow collusion between the Nazis and the Tory ministers to confront Spain, the Soviet Union strove to do all it could \emph{within the non-intervention committee} to stop fascist arms from being shipped to Spain!{\kern 0.5pt}\endnotemark[\ref{en:GannesUSSR}]  
\end{quotation}

Likewise, we have no doubt, Stalin will strive to do all he can \emph{within the committee of compromise} to get an equitable arrangement for the participation of the Loyalists in the joint regime with the fascists.

\begin{sloppypar}
Precisely in these last months, when the Anglo--French scheme was taking final shape\kn, Stalin found a new alibi to supplement those provided by the Franco--Soviet pact and ``collective security,\kn\kn'' with which to push the Loyalists into still greater dependence on the Anglo--French bloc. Louis Fischer gave it crudely enough:
\end{sloppypar}

\begin{quotation}
  The Spanish war has assumed such large dimensions and is lasting so long that Russia alone, especially if it must help China also, cannot bear the burden. Some other nation or nations must contribute\dots. If England would save Spain from Franco, Russia would perhaps be ready and able to save China from Japan. (\emph{The Nation,} October~16, 1937).
\end{quotation}

Thus, China becomes an alibi for not decisively aiding Spain, while Spain remains an alibi for not saving China! ``If England would save Spain from Franco\dots.''

The Spanish people were also directed down the road of Anglo-French imperialism by the Communist International, of course, and by the Labor and Socialist International. Apart from pious gestures of organizing fundraising, the two internationals have called only for the workers to get ``their'' democratic governments to come to Spain’s aid. The ``international proletariat'' is called upon to ``compel the fulfillment of its main demands on behalf of the Spanish people, which are the immediate withdrawal of the interventionist armed forces of Italy and Germany; the lifting of the blockade; the recognition of all the international rights of the lawful Spanish government; the application of the statutes of the League of Nations against the fascist aggressors.\kn\kn'' (\emph{Daily Worker,} July~19, 1937).

All these ``demands'' are calls for \emph{governmental actions.} Since the French Socialists and British Labourites knew that serious governmental action could only take place in the event of war\kn, and since their capitalist masters made plain they were not yet ready for war\kn, they objected to too-precipitate nudges from the Comintern. Their accusation of warmongering, Dimitrov could only answer by terming ``unworthy speculation on the antiwar sentiments of the masses at large!'' But the Socialists and Labourites were at one with the Stalinists in putting the fate of the Spanish people in the hands of ``their'' governments. For both had already pledged to support their capitalists in the coming war.

\dinkus

\begin{sloppypar}
  Whence would come the leadership to organize the Spanish masses in implacable struggle against the betrayal of Spain?
\end{sloppypar}

\medskip

That leadership could scarcely come from the ruling group in the \CNT\kn, not the least of whose crimes was its failure to steel the workers against illusions about Anglo--French aid. The very manifesto of July~17, 1937, addressed to the world proletariat, declaring, ``there is only one salvation: your aid,\kn\kn'' launched a slogan perfectly acceptable to the bourgeois--Stalinist bloc: ``Put pressure on your governments to adopt decisions favorable to our struggle.\kn\kn'' Roosevelt’s Chicago speech was acclaimed by the \CNT\ press. According to \emph{Solidaridad~Obrera} (October~7), it proved that ``democratic unity in Europe will be achieved only through energetic action against fascism.\kn\kn''

The \CNT\ leaders clung to their old course, merely asking the face-saving formula of ``antifascist front'' to be substituted for Popular Front, in their return to the government. Many of the local anarchist papers, close to the masses, reflected their outrage at the conduct of the leadership. One wrote:

\begin{quotation}
  To read a great part of the \CNT\ and anarchist press of Spain makes one indignant or give vent to tears of rage. Hundreds of our comrades were massacred in the streets of Barcelona during the May fighting, through the treason of our allies in the antifascist struggle; in Castille alone, almost a hundred comrades have been cowardly assassinated by the communists; other comrades have been assassinated by the same party in other regions; public and covert campaigns of de\-fam\-a\-tion and lies of all kinds are conducted against anarchism and the \CNT\kn, in order to poison and twist the spirit of the masses against our movement. And in the face of these crimes our press continues speaking of unity\kn, of political decency; asking loyalty among, all, calm, serenity\kn, sincerity\kn, spirit of sacrifice, and all those sentiments we are the only ones to believe and feel and which only serve for the other political sectors to cover their ambitions and treason\dots. Not to tell the truth from now on would be to betray ourselves and the proletariat. (\emph{Ideas,} Bajo~Llo\-bre\-gat, September~30, 1937).
\end{quotation}

\noindent
But the conduct of the \CNT\ leadership grew even more shameful.

The rage of the masses after the fall of Santander forced the Stalinists to utter some mollifying words, calling for a cessation of the campaign against the \CNT\kn. Whereupon even the most left of the big \CNT\ papers immediately hailed ``the rectification that undoubtedly has begun to be produced in the politics of the Communist Party.'' (\emph{\CNT,} October~6). The fall of Gij\'on, isolating the government still more from the masses, led to negotiations for \CNT\ support. All complaints forgotten, the \CNT\ leaders hastened to declare their readiness to enter the government!

Of the \UGT\ leaders, even less need be said. They had uttered not a word in defense of the \POUM\kn. Caballero made not a single public speech for five months, while the Stalinists prepared to split the \UGT\kn. The pact for united action, signed by the \CNT\ and \UGT\ on July~9, which could have organized the defense of the elementary rights of the workers\kn, remained stillborn. Though obviously representing a majority of the provincial federations of the socialist party\kn, the Caballero group went no further than to protest against the actions of the unrepresentative Prieto National Committee. Rather than an ally, the \UGT\ leaders simply further weakened the already impotent \CNT\ leaders.

Of the \POUM\kn, one can no longer speak of it as an entity. It was riv\-en irrevocably. All the blows of the leadership had been directed against the left wing, while the right wing had been courted and flattered. \emph{El~Comunista} of Valencia had openly flouted the party decisions, holding to a flagrantly People’s Front line, moving steadily toward Stalinism. Finally\kn, a week before the outlawing of the party, the Central Committee was driven to publish a resolution (\emph{Juventud~Comunista,} June~10) declaring: ``The enlarged Central Committee \dots\ has agreed to propose to the Congress the summary expulsion of the fractional group that in Valencia has worked against the revolutionary policy of our dear party.\kn\kn''

The party congress was never held. Scheduled for June~19, it was preceded by the June~16 raids. The \POUM\ was entirely unprepared for illegal work, as the sweeping success of the raids indicated. Had the congress been held, it would have found the chief centers of the party, Barcelona and Madrid, aligned to the left against the leadership. One left-wing group called for condemnation of the London Bureau and for the creation of a new, fourth International. The other declared: ``It has become clear that there does not exist in our revolution a real Marxist party of the vanguard.\kn\kn''

It was not, then, to the existing organizations as such that one could look for new leadership to prevent a compromise with the fascists. Fortunately, events had only passed the leaders by. Among the masses of the \CNT\ and \UGT\ arose new cadres seeking a way out.

Of special significance were the Friends of Durruti, for they represented a conscious break with the anti-statism of traditional anarchism. They explicitly declared the need for democratic organs of power, juntas or soviets, in the overthrow of capitalism, and the necessary state measures of repression against the counterrevolution. Outlawed on May 26, they had soon reestablished their press. Despite the threefold illegality of government, Stalinists and \CNT\ leadership, the \emph{Amigo del Pueblo} voiced the aspirations of the masses. \emph{Libertad,} also published illegally\kn, was another dissident anarchist organ. Numerous local anarchist papers, as well as the voice of the Libertarian Youth and many local \FAI\ groups, were raised against the capitulation of the \CNT\ leaders. Some still took the hopeless road of ``No More Governments.\kn\kn'' But the development of the Friends of Durruti was a harbinger of the future of all revolutionary workers in the \CNT-\FAI.

\begin{sloppypar}
  The masses of the \UGT\ and left socialists had long indicated their impatience with the pusillanimity of their leadership. But the first overt sign of revolutionary crystallization came only in October\kn, when over five hundred youth withdrew from the ``United Youth'' to rebuild a revolutionary socialist youth organization.
\end{sloppypar}

Simultaneously, the split in the \UGT\kn, forced by the Stalinists, effectively awoke many left-wing workers to the problem of saving their unions from the Stalinist destroyers. In this struggle were inescapably posed all the fundamental problems of the Spanish revolution, the nature of class-struggle unionism, the role of the revolutionary party among the masses. From it would crystallize forces for the new party of the revolution.

Here, then, was the Herculean task of the Bolshevik--Leninists. These Fourth Internationalists, condemned to illegality by the \POUM\ leadership even at the height of the revolution, organized by those expelled from the \POUM\ only in the spring of 1937\kn, seeking a way to the masses, must help to fuse the left wing of the \POUM\kn, the revolutionary socialist youth and the politically awakened \CNT\ and \UGT\ workers, to create the cadres of the revolutionary party in Spain. Could that party, if based on revolutionary foundations, be any but a party standing on the platform of the Fourth International?

Where else, indeed, would it seek for international comradeship and collaboration? The Second and Third Internationals were the organs of the betrayers of the Spanish people. Nor was it an arbitrary act, when the left wing of the \POUM\ called for repudiation of the London Bureau, the so-called ``International Bureau for Revolutionary Socialist Unity.\kn\kn'' For this center to which the \POUM\ had been affiliated, had sabotaged the struggle against Stalin’s frame-up system to which the \POUM\ had fallen victim.

\begin{sloppypar}
  While the \POUM\ itself had from the first denounced the Moscow trials and propagated a ``Trotskyist analysis,\kn\kn'' the London Bureau had worked in the opposite direction. It had refused to collaborate in a commission of inquiry into the Moscow trials. Why? Brockway~-- then launching a joint \textsc{ilp–cp} ``Unity Campaign''~-- blurted out the reason: it would ``cause prejudice in Soviet circles.\kn\kn''
  Brockway therefore proposed \dots\ a commission to investigate Trotskyism! When taxed with this course, Brockway defended himself~by impugning the character of the Commission of Inquiry headed by John Dewey.
\end{sloppypar}

Meanwhile, the London Bureau was blowing up. The \textsc{sap} (Socialist Workers Party of Germany) had first attacked the Moscow trials, but soon abandoned criticism of Stalinism, signing a joint pact for a People’s Front in Germany. \emph{Juventud Comunista} (June~3) reported the split of the London Youth Bureau: ``The \textsc{sap} Youth had placed itself in a Stalinist and reactionary position\dots. The Youth of the \textsc{sap} had signed one of the most shameful documents which the history of the German workers’ movement has known.\kn\kn'' On the very day the \POUM\ leadership was arrested as Gestapo agents, \emph{Julio,} \PSUC\ youth organ (June~19) under the headline, ``Trotskyism is Synonymous with Counter-Revolution,\kn\kn'' had hailed the policies of the \textsc{ilp} and \textsc{sap} youth groups and proudly pointed out that the Swedish affiliate of the London Bureau was steadily approaching a Stalinist People’s Front policy.

Even fouler was the position of those other ``allies'' of the \POUM\kn, the Brandler–Lovestone groups. For a decade they had defended the Stalinist bureaucracy’s every crime, on the basis of a false distinction between Stalin’s policies in the Soviet Union and the erroneous policies of the Comintern elsewhere. When Zinoviev and Kamenev were done to death, these lawyers for Stalinism had hailed the dread deed as a vindication of Soviet justice. They had likewise defended the second Moscow trial in February 1937. I was myself present at a public meeting in the Lovestone headquarters when Bertram Wolfe apologized because a \POUM\ representative had just called the trials frameups! Only after the execution of the Red Generals had the Lovestone group~-- with no explanation~-- begun to reverse its course. In ten years they had done their utmost to aid Stalin in pinning the appellation counterrevolutionary on the Trotskyists, and even when they had been driven to accepting Trotsky’s analysis of the Stalin purge, these foreheads of brass remained as ever the implacable enemies of the resurgence of the revocation in Russia as, indeed, elsewhere. As the \textsc{sap,} the Swedish affiliate, etc., move out of one end of the London Bureau, they are being replaced by the Brandler–Lovestone movement. The change has scarcely brought improvement!

How did this International Bureau for Revolutionary Socialist Unity prepare for the defense of the \POUM? Its meeting of June~6, 1937\kn, adopted two resolutions (p.~\pageref{fig:intlbureau}).

\begin{figure}[t]
\begin{quote}
  \textsf{\textsc{Resolution~1}} \quad
  The \POUM\ alone has recognized and proclaimed the necessity of transforming the anti-fascist struggle into a fight against capitalism under the hegemony of the proletariat. This is the real reason for the ferocious attacks and calumnies of the Communist party allied with the capitalist forces in the Popular Front against the \POUM\kn.

  \textsf{\textsc{Resolution~2}} \quad
  Every measure directed against the revolutionary working class of Spain is at the same time a measure in the interests of French and British imperialism and a step towards compromise with the fascists.
  
  \vskip -0.25\baselineskip

  ~~~~~~In this hour of danger we appeal to all working class organizations of the whole world, and \emph{particularly} to the Second \emph{and Third International} \kp\dots\ let us at last take up a unified stand against all these treacherous maneuvers of the world bourgeoisie. [Author's italics.]
\end{quote}
\caption{Resolutions adopted by the International Bureau for Revolutionary~Socialist Unity on June~6, 1937}
\label{fig:intlbureau}
\end{figure}

One resolution for the left, one for the semi-Stalinist right~-- that is, the London Bureau.\kn\kn%
\endnote{%
%
In the June~4 issue of the \emph{New Leader,} the \textsc{ilp} leader\kn, Fenner Brockway\kn, gave the \POUM\ some advice at this critical juncture. Some revealing extracts:

\begin{quotation}
  It is important that \POUM\kn, together with other workers’ forces, should concentrate on the fight against Franco\dots. The Spanish Communist Party had justifiably criticized the absence of coordination at the front and the bad organization of the armed forces. \POUM\ must be careful not to appear to resist proposals which will facilitate efficiency in the fight against Franco but that does not mean that it must accept without protest a return to the reactionary structure of the old army.
\end{quotation}

This kind of advice a week before the \POUM\ was outlawed! That the task of the \POUM\ was relentless, implacable struggle against the government, placing no confidence whatsoever in the \CNT\ and \UGT\ leaders, to issue united front proposals for concrete, daily defense of elementary workers’ rights, and to immediately combine legal with illegal work~-- this was naturally beyond Brockway. The same issue carries a letter from the \textsc{ilp} representative in Spain, McNair\kn, to the Stalinist leader\kn, Dutt, beginning:

\begin{quotation}
  It is painful for me to be compelled to enter into controversy with a comrade of the CP in view of the desire I have to see unity among the working class parties\dots. I still hold the point of view\dots\ “the important thing to have in mind is that the Unity Campaign in Britain should engender unity in Spain rather than allow Spanish disunity to break up the Unity Campaign in Britain\dots.”
\end{quotation}

}

\medskip

``But are not the principles you propose for the regroupment of the Spanish masses, are they not intellectual constructions to which the masses will feel alien? And is it not too late?\kp''

No! We revolutionists are the only practical people in the world. For we merely articulate the fundamental aspirations of the masses, indeed, what they are already saying in their own way. We merely clarify the nature of the instrumentalities, above all, the nature of the revolutionary party and the workers’ state, which the masses need to achieve what they want. It is never too late for the masses to begin to hew their road to freedom. Pessimism and skepticism are luxuries for the few. The masses have no other choice except to fight for their lives and the future of their children.

If our analysis has not illuminated the inner forces of the Spanish revolution, let us recall a few words of Durruti on the battlefield of Aragon, when he was leading the ill-armed militias in the only substantial advance of the whole civil war. He was no theoretician, but an activist leader of masses. All the more significantly do his words express the revolutionary outlook of the class conscious workers. The \CNT\ leaders have buried these words deeper than they buried Durruti! But let us remember them:

\vfill

\begin{quotation}
  For us it is a question of crushing fascism once and for all. Yes, and in spite of the government.
  
  No government in the world fights fascism to the death. When the bourgeoisie sees power slipping from its grasp, it has recourse to fascism to maintain itself. The liberal government of Spain could have rendered the fascist elements powerless long ago. Instead it temporized and compromised and dallied. Even now at this moment, there are men in this government who want to go easy with the rebels. You never can tell, you know~-- he laughed~-- the present government might yet need these rebellious forces to crush the workers’ movement\dots.
  
  We know what we want. To us it means nothing that there is a Soviet Union somewhere in the world, for the sake of whose peace and tranquility the workers of Germany and China were sacrificed to fascist barbarism by Stalin. We want the revolution here in Spain, right now\kn, not maybe after the next European war. We are giving Hitler and Mussolini far more worry today with our revolution than the whole Red Army of Russia. We are setting an example to the German and Italian working class how to deal with fascism.
  
  I do not expect any help for a libertarian revolution from any government in the world. Maybe the conflicting interests in the various imperialisms might have some influence on our struggle. That is quite possible. Franco is doing his best to drag Europe into the conflict. He will not hesitate to pitch Germany against us. But we expect no help, not even from our own government in the last analysis.
\end{quotation}

\vfill

\newpage

``You will be sitting on top of a pile of ruins if you are victorious,\kn\kn'' said Van~Paassen.

\smallskip

Durruti answered:

\begin{quotation}
  We have always lived in slums and holes in the wall. We will know how to accommodate ourselves for a time, for\kn, you must not forget, we can also build. It is we who built theses palaces and cities, here in Spain and in America and everywhere. We, the workers, we can build others to take their place. And better ones. We are not in the least afraid of ruins. We are going to inherit the earth. There is not the slightest doubt about that. The bourgeoisie might blast and ruin its own world before it leaves the stage of history. We carry a new world, here, in our hearts. That world is growing this minute.\kn%
  \endnote{Interview of Durruti with Pierre Van~Paassen, \emph{Toronto Star,} September~1936.}
\end{quotation}

\vfill

\thispagestyle{empty}

\begin{flushright}
  \rule{0.5\textwidth}{0.8pt}
  
  \bigskip
  
  \textsf{Felix Morrow} \\ 10 November 1937
\end{flushright}