\chapter{Glossary}

\section{General}

\begin{multicols}{2}
  \small
  \RaggedRight
  \setlength{\parskip}{0.25\baselineskip}

\textbf{Assault Guard} • \emph{Guardia de Asalto}

\bigskip

\textbf{Asturian Commune} • \emph{Comuna Asturiana}

Uprising of workers in 1934, drowned in blood, with over \textls[40]{5,000} murdered in reprisals.

\bigskip

\textbf{\emph{caciques}}

Landlords who acted like feudal barons.

\bigskip

\textbf{\emph{carabineros}}

\bigskip

\textbf{Civil Guard} • \emph{Guardia Civil}

\bigskip

\textbf{\emph{Cortes}}

The Spanish parliament.

\bigskip

\textbf{\emph{Generalitat}}

Catalan regional government.

\bigskip

\textbf{republicans}

A general term for parties supporting the Front government. Several bourgeois parties had the word “republican” in their name.

\bigskip

\textbf{Spanish Legion} • \emph{Legión Española}

\end{multicols}

\section{Political organizations}

\begin{multicols}{2}
  \small
  \RaggedRight
  \setlength{\parskip}{0.25\baselineskip}
	
  \textbf{\textsc{ceda}} • \emph{Confederación Española de Derechas Autónomas} • Spanish Confederation of Autonomous Right-wing Groups
  
  \bigskip

  \textbf{\textsc{cnt}} • \emph{Confederación Nacional del Trabajo} • National Confederation of Labor

  \bigskip

  \textbf{Comintern} • Communist International

  The body linking together CPs internationally. By the 1930s it had been reduced to the role of border guard for Stalin’s regime in Russia.
  
  \bigskip

  \textbf{\ERC} • \emph{Esquerra Republicana de Catalunya} • Republican Left of Catalonia
  
  \bigskip
  
  \textbf{\textsc{fai}} • \emph{Federación Anarquista Ibérica} • Iberian Anarchist Federation
  
  \bigskip

  \textbf{Falange} • \emph{Falange Española de las Juntas de Ofensiva Nacional-Sindicalista} • Spanish Phalanx of the Councils of the National-Syndicalist Offensive
  
  \bigskip

  \textbf{Friends of Durruti} • \emph{Agrupación de los Amigos de Durruti}
  
  \bigskip

  \textbf{\textsc{gepci}} • \emph{Federación Catalana de Gremios y Entidades de Pequeños Comerciantes e Industriales} • Catalan Federation of Small Business Owners and Manufacturers
  
  \bigskip

  \textbf{\textsc{irmc}} • International Revolutionary Marxist Centre • International Bureau for Revolutionary Socialist Unity • “3\,½ International” • “London Bureau”
  
  \bigskip

  \textbf{IR} • \emph{Izquierda Republicana} • Republican Left

  \bigskip

  \textbf{\textsc{pce}} • \emph{Partido Comunista de España} • Communist Party of Spain
  
  \bigskip
  
  \textbf{Popular Front} • People’s Front • \emph{Frente Popular}
  \bigskip

  \textbf{\textsc{poum}} • \emph{Partido Obrero de Unificación Marxista} • Workers’ Party of Marxist Unification

  \bigskip

  \textbf{\textsc{prr}} • \emph{Partido Republicano Radical} • Radical Republican Party

  \bigskip

  \textbf{\textsc{psoe}} • \emph{Partido Socialista Obrero Español} • Spanish Socialist Workers' Party

  \bigskip

  \textbf{\textsc{psuc}} • \emph{Partit Socialista Unificat de Catalunya} • Unified Socialist Party of Catalonia

  \bigskip

  \textbf{\textsc{ugt}} • \emph{Unión General de Trabajadores} • General Workers’ Union

  The second biggest trade union federation led by the Socialist Party (\textsc{psoe}).

\end{multicols}

\section{Individuals}

\section{Publications}

\begin{multicols}{2}
  \small
  \RaggedRight
  %\setlength{\parskip}{0.25\baselineskip}
  
  \begin{minipage}{\linewidth}
  \textbf{\emph{El Amigo del Pueblo}} • \\
  \emph{Friend of the People}
  
  [May 1937--Feb 1938; Belchite, Zaragoza, Aragón]
  Published (illegal\-ly) twelve times by the Friends of Durruti. 
\end{minipage}

  %
  \bigskip

  \begin{minipage}{\linewidth}
  \textbf{\emph{La Batalla}}
  • \emph{The Battle}

  [Dec 1922--May 1937; Barcelona, Catalonia] \POUM\ monthly organ.
\end{minipage}

  %
  \bigskip
  
  \begin{minipage}{\linewidth}
  \textbf{\emph{Claridad}}
  • \emph{Clarity}

  [July 1935--March 1940; Madrid]
  Daily newspaper; became official \UGT\ organ in 1937; left socialist–leaning.
 
\end{minipage}
  
  %
  \bigskip
  
  \begin{minipage}{\linewidth}
  	\textbf{\emph{El Liberal}}
 	• \emph{The Liberal}
  	
  	[1901--c.~1937; Bilbao, Biscay, Basque Country]
  	Independent newspaper owned by Indalecio Prieto.
  \end{minipage}
  

  %
  \bigskip

  \begin{minipage}{\linewidth}
  	\textbf{\emph{Mundo Obrero}}
  	• \emph{Workers' World}
  	
  	[Aug 1930--; Madrid]
  	\textsc{pce} fortnightly organ.
  \end{minipage}
  
  %
  \bigskip
  
  \begin{minipage}{\linewidth}
  \textbf{\emph{El Socialista}}
  • \emph{The Socialist}

  [1886--; Madrid] \textsc{psoe} organ; right socialist–leaning during the war.
  
\end{minipage}

  %
  \bigskip
  
  \begin{minipage}{\linewidth}
  \textbf{\emph{Solidaridad Obrera}}
  • \emph{Workers’ Solidarity}

  [1907--; Barcelona, Catalonia] \CNT\ organ. Largest Spanish newspaper in circulation during the war.
  
\end{minipage}

\end{multicols}