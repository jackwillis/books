\chapter{Glossary}

\section{General}

\begin{multicols}{2}
  \RaggedRight
  \setlength{\parskip}{0.25\baselineskip}

\textbf{Assault Guard} / Guardia de Asalto

\bigskip

\textbf{Asturian Commune} / Comuna Asturiana

Uprising of workers in 1934, drowned in blood, with over 5,000 murdered in reprisals.

\bigskip

\textbf{Caciques}

Landlords who acted like feudal barons.

\bigskip

\textbf{Carabineros}

\bigskip

\textbf{Civil Guard} / Guardia Civil

\bigskip

\textbf{Cortes}

The Spanish parliament.

\bigskip

\textbf{Generalitat}

Catalan regional government.

\bigskip

\textbf{Republicans}

A general term for parties supporting the Front government. Several bourgeois parties had the word “republican” in their name.

\bigskip

\textbf{Spanish Legion} / Legión Española

\end{multicols}

\section{Political organizations}

\section{Individuals}

\section{Publications}

\begin{multicols}{2}
  \RaggedRight
  \setlength{\parskip}{0.25\baselineskip}
  
  \textbf{\emph{El Amigo del Pueblo}}
  / \emph{Friend of the People}
  
  Published (illegally) twelve times by the Friends of Durruti. 
  
  May 1937--Feb 1938
  
  Belchite, Zaragoza, Aragón

  %
  \bigskip

  \textbf{\emph{La Batalla}}
  / \emph{The Battle}

  \POUM\ monthly organ.
  
  Dec 1922--May 1937
  
  Barcelona, Catalonia

  %
  \bigskip
  
  \textbf{\emph{Claridad}}
  / \emph{Clarity}

  Daily newspaper; became official \UGT\ organ in 1937; left socialist–leaning.
  
  July 1935--March 1940
  
  Madrid
  
  %
  \bigskip
  
  \textbf{\emph{El Liberal}}
  / \emph{The Liberal}

  Independent newspaper owned by Indalecio Prieto.
  
  1901--c.~1937
  
  Bilbao, Biscay\kn, Basque Country

  %
  \bigskip

  \textbf{\emph{Mundo Obrero}}
  / \emph{Workers' World}

  \textsc{pce} fortnightly organ.
  
  Aug 1930--
  
  Madrid
  
  %
  \bigskip
  
  \textbf{\emph{El Socialista}}
  / \emph{The Socialist}

  \textsc{psoe} organ; right socialist–leaning during the war.
  
  1886--
  
  Madrid

  %
  \bigskip
  
  \textbf{\emph{Solidaridad Obrera}}
  / \emph{Workers’ Solidarity}

  \CNT\ organ. Largest Spanish newspaper in circulation during the war.
  
  1907--
  
  Barcelona, Catalonia

\end{multicols}