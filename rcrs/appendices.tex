\chapter{Glossary}

\section{General remarks}

\begin{multicols}{2}
  \RaggedRight
  \setlength{\parskip}{0.25\baselineskip}

\textbf{Assault Guard} / Guardia de Asalto

\bigskip

\textbf{Asturian Commune} / Comuna Asturiana

Uprising of workers in 1934, drowned in blood, with over 5,000 murdered in reprisals.

\bigskip

\textbf{Caciques}

Landlords who acted like feudal barons.

\bigskip

\textbf{Carabineros}

\bigskip

\textbf{Civil Guard} / Guardia Civil

\bigskip

\textbf{Cortes}

The Spanish parliament.

\bigskip

\textbf{Generalitat}

Catalan regional government.

\bigskip

\textbf{Republicans}

A general term for parties supporting the Front government. Several bourgeois parties had the word “republican” in their name.

\bigskip

\textbf{Spanish Legion} / Legión Española

\end{multicols}

\section{Political organizations}

\begin{multicols}{2}
  \RaggedRight
  \setlength{\parskip}{0.25\baselineskip}
	
  \textbf{\textsc{ceda}} / Confederación Española de Derechas Autónomas / Spanish Confederation of Autonomous Right-wing Groups
  
  \bigskip

  \textbf{\textsc{cnt-fai}} / Confederación Nacional del Trabajo–Federación Anarquista Ibérica / National Confederation of Labor–Iberian Anarchist Federation

  \bigskip

  \textbf{Comintern} / Communist International

  The body linking together CPs internationally. By the 1930s it had been reduced to the role of border guard for Stalin’s regime in Russia.
  
  \bigskip

  \textbf{\ERC} / Esquerra Republicana de Catalunya / Republican Left of Catalonia
  
  \bigskip

  \textbf{Falange} / Falange Española de las Juntas de Ofensiva Nacional-Sindicalista / Spanish Phalanx of the Councils of the National-Syndicalist Offensive
  
  \bigskip

  \textbf{Friends of Durruti} / Agrupación de los Amigos de Durruti
  
  \bigskip

  \textbf{\textsc{gepci}} / Federación Catalana de Gremios y Entidades de Pequeños Comerciantes e Industriales / Catalan Federation of Small Business Owners and Manufacturers
  
  \bigskip

  \textbf{\textsc{irmc}} / International Revolutionary Marxist Centre / International Bureau for Revolutionary Socialist Unity / “3 ½ International” / “London Bureau”
  
  \bigskip

  \textbf{IR} / Izquierda Republicana / Republican Left

  \bigskip

  \textbf{\textsc{pce}} / Partido Comunista de España / Communist Party of Spain
  
  \bigskip
  
  \textbf{Popular Front} / People’s Front / Frente Popular
  \bigskip

  \textbf{\textsc{poum}} / Partido Obrero de Unificación Marxista / Workers’ Party of Marxist Unification

  \bigskip

  \textbf{\textsc{prr}} / Partido Republicano Radical / Radical Republican Party

  \bigskip

  \textbf{\textsc{psoe}} / Partido Socialista Obrero Español / Spanish Socialist Workers' Party

  \bigskip

  \textbf{\textsc{psuc}} / Partit Socialista Unificat de Catalunya / Unified Socialist Party of Catalonia

  \bigskip

  \textbf{\textsc{ugt}} / Unión General de Trabajadores / General Workers’ Union

  The second biggest trade union federation led by the Socialist Party (\textsc{psoe}).

\end{multicols}

\section{Individuals}

\section{Referenced publications}

\begin{multicols}{2}
  \RaggedRight
  \setlength{\parskip}{0.25\baselineskip}
  
  \textbf{\emph{El Amigo del Pueblo}}
  / \emph{Friend of the People}
  
  Published (illegally) twelve times by the Friends of Durruti. 
  
  May 1937--Feb 1938
  
  Belchite, Zaragoza, Aragón

  %
  \bigskip

  \textbf{\emph{La Batalla}}
  / \emph{The Battle}

  \POUM\ monthly organ.
  
  Dec 1922--May 1937
  
  Barcelona, Catalonia

  %
  \bigskip
  
  \textbf{\emph{Claridad}}
  / \emph{Clarity}

  Daily newspaper; became official \UGT\ organ in 1937; left socialist–leaning.
  
  July 1935--March 1940
  
  Madrid
  
  %
  \bigskip
  
  \textbf{\emph{El Liberal}}
  / \emph{The Liberal}

  Independent newspaper owned by Indalecio Prieto.
  
  1901--c.~1937
  
  Bilbao, Biscay\kn, Basque Country

  %
  \bigskip

  \textbf{\emph{Mundo Obrero}}
  / \emph{Workers' World}

  \textsc{pce} fortnightly organ.
  
  Aug 1930--
  
  Madrid
  
  %
  \bigskip
  
  \textbf{\emph{El Socialista}}
  / \emph{The Socialist}

  \textsc{psoe} organ; right socialist–leaning during the war.
  
  1886--
  
  Madrid

  %
  \bigskip
  
  \textbf{\emph{Solidaridad Obrera}}
  / \emph{Workers’ Solidarity}

  \CNT\ organ. Largest Spanish newspaper in circulation during the war.
  
  1907--
  
  Barcelona, Catalonia

\end{multicols}