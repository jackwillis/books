% Basic layout
\documentclass[
  12pt,
  pagesize,
  paper = 6in:9in,
  DIV = 12,
  openany
]{scrbook}

% Show bad (overfull) boxes
\overfullrule=5pt

% Encoding
\usepackage[T1]{fontenc}
\usepackage[utf8]{inputenc}
\usepackage[english]{babel}

% Use Alegreya font super family
\usepackage[osf]{Alegreya}
\usepackage[osf]{AlegreyaSans}

% Microtypography
\usepackage[final, tracking = true, kerning = true, spacing = {}, stretch = 15, shrink = 15]{microtype}

\SetTracking{encoding = {*}, shape = {sc, scit}}{40}

\SetExtraKerning[unit=space]{
  encoding={*},
  family={*},
  series={*},
  size={footnotesize,small,normalsize}
}{
  F={-125,0},
  P={-125,0},
  L={-75,25},

  {–}={75,75},
  {—}={500,500},
  {“}={0,200},
  {”}={200,0},
  {‘}={-50,100},
  {’}={100,-50},
  {,}={-200,0},
  {.}={0,0}
}

% small caps kerning
\SetExtraKerning[unit=space]{
  encoding={*},
  family={*},
  series={*},
  shape={sc,scit},
  size={footnotesize,small,normalsize}
}{
  c={-75,0},
  n={0,75},
  a={-75,0}
}

\newcommand{\kn}{\kern -0.5pt} % negative kerning
\newcommand{\kp}{\kern 0.5pt} % positive kerning

% Chapter dots -- https://tex.stackexchange.com/q/292882
\KOMAoption{chapterentrydots}{true}

% Hyperlink styles
\usepackage[setpagesize = false, pdfpagelayout = TwoPageRight]{hyperref}
\usepackage{xcolor}
\hypersetup{
	colorlinks,
	linkcolor = {red!50!black},
	citecolor = {red!50!black},
	urlcolor = {blue!80!black}
}

% Arbitrary font sizes are used by titlepage.tex
\usepackage{anyfontsize}

% Provides framed environment
\usepackage{framed}

\usepackage{ellipsis}

\usepackage{ragged2e} % provides \FlushLeft (\flushleft with hyphenation)

\usepackage{enumitem}

\usepackage{multicol}

% Provides `\endnote`
\usepackage[totoc = chapter]{enotez}

% Footnotes use roman numerals to distinguish from endnotes
\renewcommand{\thefootnote}{\roman{footnote}}

% Section (sub-chapter) counters should be Roman numerals -- Arabic numerals look too technical

% Provides \scalebox
\usepackage{graphicx}

% Custom roman numeral kerning

% RN = roman numeral
\newcommand{\textrn}[1]{\scalebox{1.3}[0.9]{\textrm{#1}}}

\newcommand{\fancyromannumeral}[1]{%
  \ifcase#1 \ensuremath{\emptyset}%
  \or I%
  \or I\kern -1.5pt I%
  \or I\kern -1.5pt I\kern -1.5pt I%
  \or I\kern -1pt V%
  \or V%
  \or V\kern -1pt I%
  \or V\kern -1pt I\kern -1.5pt I%
  \or V\kern -1pt I\kern -1.5pt I\kern -1.5pt I%
  \or I\kern -1pt X%
  \or X%
  \or X\kern -1pt I%
  \or X\kern -1pt I\kern -1.5pt I%
  \else \romannumeral{#1}%
  \fi%
}

\renewcommand*{\sectionformat}{\textrn{\fancyromannumeral{\number\value{section}}}\enskip}
\renewcommand{\thesection}{\roman{section}}

% https://tex.stackexchange.com/a/102313/160204
\renewcommand*{\chapterformat}{\thechapter\enskip}

% Drop caps
\usepackage{lettrine}

\newcommand{\lettrineA}[1]{\lettrine[lines=3, slope=4pt, findent=-3pt]{A}{#1}}
\newcommand{\lettrineE}[1]{\lettrine[lines=3, findent=0pt, nindent=2pt]{E}{#1}}
\newcommand{\lettrineF}[1]{\lettrine[lines=3, slope=-7pt, nindent=0pt, findent=2pt]{F}{#1}}
\newcommand{\lettrineI}[1]{\lettrine[lines=3, findent=2pt, nindent=1pt]{I}{#1}}
\newcommand{\lettrineL}[1]{\lettrine[lines=3, findent=-6pt, nindent=12pt]{L}{#1}}
\newcommand{\lettrineM}[1]{\lettrine[lines=3]{M}{#1}}
\newcommand{\lettrineO}[1]{\lettrine[lines=3, findent=0pt, nindent=6pt, slope=-3pt]{O}{#1}}
\newcommand{\lettrineP}[1]{\lettrine[lines=3, findent=3pt, nindent=3pt, slope=-9pt]{P}{#1}}
\newcommand{\lettrineS}[1]{\lettrine[lines=3]{S}{#1}}
\newcommand{\lettrineT}[1]{\lettrine[lines=3, findent=2pt, nindent=1pt]{T}{#1}}

% Blockquote styling

\usepackage{etoolbox}

% Keep track of whether we're in the endnotes section
\newtoggle{endnotes}

\AtBeginEnvironment{quotation}{\iftoggle{endnotes}{\footnotesize}{\small}}
\AtBeginEnvironment{quote}{\iftoggle{endnotes}{\footnotesize}{\small}}

% The nowidow package provides \nowidow and \noclub
\usepackage{nowidow}

% PDF metadata
\usepackage[a-1b]{pdfx}
\begin{filecontents*}{\jobname.xmpdata}
  \Title{Revolution and Counter-Revolution in Spain}
  \Author{Felix Morrow}
  \Subject{Spanish civil war}
  \Keywords{Spanish civil war\sep Spain\sep Marxism\sep Trotskyism}
  \Publisher{Madison Socialist Alternative}
  \PublicationType{book}
  \Copyright{Creative Commons Attribution-ShareAlike 4.0 International}
  \CopyrightURL{https://creativecommons.org/licenses/by-sa/4.0/}
\end{filecontents*}

% Text helpers
\newcommand{\POUM}{\textsc{poum}}
\newcommand{\CNT}{\textsc{cnt}}
\newcommand{\ERC}{\textsc{erc}}
\newcommand{\FAI}{\textsc{fai}}
\newcommand{\UGT}{\textsc{ugt}}
\newcommand{\GEPCI}{\textsc{gepci}}
\newcommand{\GPU}{\textsc{gpu}}
\newcommand{\PSUC}{\textsc{psuc}}
\newcommand{\USSR}{\textsc{ussr}}
\newcommand{\CEDA}{\textsc{ceda}}
\newcommand{\JCW}{\textsc{jcw}}

\newcommand{\separatorline}{\begin{center}\rule{0.75\textwidth}{.4pt}\end{center}}
\newcommand{\dinkus}{\medskip {\centering\noindent\( \ast \quad \ast \quad \ast\)\par} \medskip}
\newcommand{\dinkette}{\begin{center}\( \ast \)\end{center}}

% Indexing -- provides \index, \see, and \printendnotes
\usepackage{makeidx}
\makeindex % tells TeX to generate files necessary for `makeindex`

% https://tex.stackexchange.com/q/318472/160204
\newcommand\gobbleone[1]{}
\newcommand*{\seeonly}[2]{\ (\emph{\seename} #1)}

\newcommand{\indexbienionegro}{\index{Bienio negro@\emph{Bienio negro}}}
\newcommand{\indexCNT}{\index{CNT@\CNT}}
\newcommand{\indexEsquerra}{\index{ERC@\ERC}}
\index{Esquerra \seeonly{\ERC}|gobbleone}
\newcommand{\indexGEPCI}{\index{GEPCI@\GEPCI}}
\newcommand{\indexGPU}{\index{GPU@\GPU}}
\newcommand{\indexPOUM}{\index{POUM@\POUM}}
\newcommand{\indexPSUC}{\index{POUM@\PSUC}}
\newcommand{\indexUGT}{\index{UGT@\UGT}}
\newcommand{\indexUSSR}{\index{USSR@\USSR}}

\newcommand{\indexClaridad}{\index{Claridad@\emph{Claridad}}}
\newcommand{\indexLaBatalla}{\index{La Batalla@\emph{La Batalla}}}
\newcommand{\indexLibertad}{\index{Libertad@\emph{Libertad}}}
\newcommand{\indexMundoObrero}{\index{Mundo Obrero@\emph{Mundo Obrero}}}
\newcommand{\indexNYT}{\index{New York Times@\emph{New York Times}}}
\newcommand{\indexSolidaridadObrera}{\index{Solidaridad Obrera@\emph{Solidaridad Obrera}}}

\newcommand{\indexAHitler}{\index{Hitler, Adolf}}
\newcommand{\indexALerroux}{\index{Lerroux, Alejandro}}
\newcommand{\indexANin}{\index{Nin Perez, Andreas}}
\newcommand{\indexBBolloten}{\index{Bolloten, Burnett}}
\newcommand{\indexBMussolini}{\index{Mussolini, Benito}}
\newcommand{\indexDBarrio}{\index{Barrio, Diego@Barrio, Diego Mart\'inez}}
\newcommand{\indexFFranco}{\index{Franco, Francisco}}
\newcommand{\indexGRobles}{\index{Robles, Gil}}
\newcommand{\indexIPrieto}{\index{Prieto, Indalecio}}
\newcommand{\indexJGiral}{\index{Giral, Jose@Giral, Jos\'e}}
\newcommand{\indexJHernandez}{\index{Hernandez, Jose@Hernandez, Jos\'e}}
\newcommand{\indexJNegrin}{\index{Negrin, Juan@Negr\'in y L\'opez, Juan}}
\newcommand{\indexLCaballero}{\index{Caballero, Largo}}
\newcommand{\indexLCompanys}{\index{Companys, Lluis@Companys, Llu\'is}}
\newcommand{\indexLFischer}{\index{Fischer, Louis}}
\newcommand{\indexMAzana}{\index{Azana, Manuel@Azaña, Manuel}}
\newcommand{\indexPCE}{\index{Communist Party of Spain}}
\newcommand{\indexSPozas}{\index{Pozas Perea, Sebastian@Pozas Perea, Sebastián}}
\newcommand{\indexVVidali}{\index{Vidali, Vittorio}}


\begin{document}

\frontmatter

  \pdfbookmark[0]{Title}{title}
  % Full title
\begin{titlepage}
	\setlength{\parindent}{0pt}
	
	\vspace*{\fill}
	
	{\sffamily\bfseries\fontsize{42}{42}\selectfont
		Revolution\,\textmd{and} \\
		Counter-Revolution \\
		\textmd{in}\,Spain
		\par}
	
	\vspace{63pt}
	
	{\fontsize{21}{21}\selectfont
		\textsc{Felix Morrow}
		\par}
	
	\vspace*{\fill}
	
\end{titlepage}

% Copyright information
{
	\thispagestyle{empty}
	\setlength{\parindent}{0em}
	\setlength{\parskip}{0.5em}
	\sloppy
	
	\vspace*{\fill}
	\vspace*{\fill}
	
	Originally published in 1938 by Pioneer Publishers, New York, NY.
	
	Thanks to the Marxists Internet Archive
	for making this work freely available online.\footnote{\href{https://www.marxists.org/archive/morrow-felix/1938/revolution-spain/}{https://www.marxists.org/archive/morrow-felix/1938/revolution-spain/}}
	Transcribed by Ted Crawford in 2003
	and proofread by Einde O’Callaghan in August 2015.
	
	This edition was designed and proofread by Jack Willis in 2018.
	
	\textsc{pdf}s and source code for this edition are available online\footnote{\href{https://www.attac.us/books/rcrs/}{https://www.attac.us/books/rcrs/}}
	under a Creative Commons Attribution-Share-Alike 4.0 International License.\footnote{\href{https://creativecommons.org/licenses/by-sa/4.0/}{https://creativecommons.org/licenses/by-sa/4.0/}}
	
	Typeset in 12pt Alegreya and {\AlegreyaSans Alegreya Sans} using \LaTeX\ document preparation system.
	
	\vspace*{\fill}
}
  
  \pdfbookmark[0]{Contents}{contents}
  \microtypesetup{protrusion=false}
    \tableofcontents
  \microtypesetup{protrusion=true}

  \chapter{Biographical notes}

This book was first published in 1938 by Pioneer in the United States.
An expanded edition which contained Morrow’s much shorter \emph{Civil War in Spain} pamphlet was published in the United States by Path\-finder Press in 1974 with a new index and copies may still be available from them. Note the reference to \emph{Civil War in Spain} by Morrow in Note~[\ref{en:CivilWarReference}].

The text of the following book is that of the second edition of 1963 published by New Park Publications together with the introduction written at that date by the late Tom Kemp. At the time, and for some years afterwards, the Socialist Labour League in Britain, who controlled New Park, was politically allied with the Socialist Workers Party in the United States, who controlled Pioneer/Pathfinder. The type of this edition was reset with English spellings etc. and the book was given its own index. Since it is now out of print, Index Books, the successor to New Park Publications, is consequently delighted that this important text is available, if only on the web, as they have no plans to reprint it.

It has been transcribed by Ted Crawford in 2003 and any errors and omissions are his responsibility. Notes have been placed at the end of the chapters and numbered rather than placed at the bottom of the page and the very occasional typographical errors removed. Any notes added are clearly stated to have been done by him and signed \ERC.

\begin{flushright}
	Ted Crawford, 2003 \\
	(for the Marxists Internet Archive)
\end{flushright}

\chapter{Foreword}

\lettrineF{or the} new generations moving into political life, a study of the Spanish Civil War of 1936–39 contains valuable lessons. This book, hammered out in time with the dramatic events it describes, stands in direct line with the great writings of Marx and Engels on the Revolutions of 1848 and the Paris Commune and Trotsky’s \emph{History of the Russian Revolution.} No doubt later historical research has been able to uncover from the record a more detailed account of the facts, and the reports of participants have helped to throw light on the motives of many of those involved. The basic interpretation made by Morrow in the light of Marxism retains all its validity and offers an invaluable key to one of the decisive events of our epoch. In reprinting this book, which has become a rarity, a valuable service is being rendered to the working-class movement and to students of politics. Needless to say, while Morrow’s work has so well withstood the test of time there is not a single part of the voluminous literature produced by the Stalinists which could be reprinted today without courting ridicule.

The Spanish workers and peasants, in July 1936, made a revolution but they could not complete it or maintain the positions won. Their revolution was defeated and counterrevolution, in the shape of the bourgeois republic, by its triumph, prepared the way for the bloodier triumph of Franco. This defeat was one of a long series which issued from the betrayals of social democracy and Stalinism. It reflected, above all, the failure to build in time the necessary revolutionary leadership, a party of the Bolshevik type. This failure was not peculiarly Spanish but was part of an international crisis of proletarian leadership which called for the formation of the Fourth International and is still with us today.
\nowidow

During these years the Spanish proletariat was in the vanguard of the international struggle of its class. It made a revolution in the teeth of the attempt by the fascist military junta to establish a dictatorship on behalf of the landed oligarchy, big business and the church, which saw in the crushing of the working-class movement the only way to secure their privileges. Before this determined threat the bourgeois state, wearing the trappings of the Popular Front, panicked and disintegrated. The army and police, upon which its existence depended, joined the rebellion or prepared to do so. The response of the workers, as well as of the peasants in many parts of the country, was to set up their own organs of power and to carry out a social revolution. In this way they sought, spontaneously, to give concrete form to the promise of the Popular Front government which they had elected in February 1936.

Morrow describes graphically, as others have since done in still greater detail, how this revolution was made. But he does more, he explains the significance of the dual power which thus came into being and defines the character of this revolution. According to the Communist Parties, applying the line of the Seventh World Congress of the Comintern\endnote{B.~Bolloten, \emph{The Grand Camouflage} (London 1961), p. 42.}\label{en:BollotenGrandCamouflageP42}\index{Communist International} it was a bourgeois democratic revolution from which should issue a national democratic state. True, the bourgeois democratic revolution in Spain had been delayed and was half completed, but it cannot be doubted that capitalism was firmly established and that the state was a bourgeois state. When the workers took over the factories and the peasants took over the land and established collectives, their armed militias conquered power by overcoming the rebel troops. ``Shorn of the repressive organs of the state,'' writes Bolloten{\indexBBolloten}, who is no Marxist, ``the government of Jos\'e Giral{\indexJGiral} possessed the nominal power, but not the power itself, for this was split into countless fragments.''\endnotemark[\ref{en:BollotenGrandCamouflageP42}]

The failure of the workers and peasants to complete their victory can be explained by two main factors. First there was the influence of the anarchist leadership (\textsc{fai-cnt}) which responded to the revolutionary impulse of the masses with the pusillanimity of the petty-bourgeois and presented the world with the unique spectacle of anarchist ministers in a bourgeois government. These leaders were a powerful obstacle on the way to building the revolutionary party which the situation demanded; though hard experience pressed this lesson home on many anarchist workers and especially the youth, it was a lesson learned too late. The way was thus clear for counterrevolution. The spearhead of this counterrevolution was of the most insidious kind because it came decked out as Marxism and Communism. The Spanish Communist Party\indexPCE, at the beginning of 1936, was still a relatively small party. As a member of the Popular Front coalition it made itself, after July, the protagonist of the restoration of republican, i.e., bourgeois institutions, including a well-armed police and disciplined army to supersede the workers’ guards and militias. In this way it gained rapidly in strength; ``from the outset,'' writes Bolloten.\endnote{B. Bolloten, \emph{op. cit.}, p. 87, and whole of chapter 6.}

``The Communist Party appeared before the distraught middle classes not only as a defender of property, but as the champion of the Republic and the orderly processes of government''~-- which in the circumstances meant counterrevolution. It also became the most determined champion of the policy of win the war first which rallied to its side many former supporters of the other republican parties. As the Republic became increasingly dependent upon Russian military supplies so was the power of the Communist Party enhanced, or appeared to be.

In fact, it was not the Spanish Communists whose power grew but the direct power of the Soviet bureaucracy in Spain. As Jes\'us Hern\'andez, Communist Minister of Education\index{Ministries!Education} in the Negr\'in Government has subsequently revealed,\endnote{J.~Hernandez, \emph{La Grande Trahison} (Paris 1953), French translation of \emph{Yo fui un ministro de Stalin} (Mexico City 1953.) Hernandez{\indexJHernandez} was Communist Minister of Education in the government of Caballero, whose downfall he helped to bring about, then in that of Negr\'in.} the policy of the party was determined, often over the protests of its more honest leaders, by loyal agents of the policy of the Stalinized Comintern~-- Togliatti, Vidali{\indexVVidali} (alias Contreras), Orlov and others. This policy was not determined by the needs of the Spanish revolution but by the diplomatic requirements of the Kremlin which had no interest in revolution in the Iberian peninsula. Stalin settled for a limited commitment in Spain, hoping that the ``democracies''~-- France and England~-- would intervene to stop the spread of fascism and ready to pull out as it became increasingly clear that this was not likely. To this end, then, not only was the Communist Party set on a counterrevolutionary course, but methods and organization of the \textsc{nkvd} were transplanted to Spain. It was concerned primarily with tracking down and exterminating genuine revolutionaries, operating quite independently of the government, and even of the Spanish Communist Party. Its most spectacular success, following the provocation which led to the May Days in Barcelona in 1937, was the hounding of the \textsc{poum} and the murder of Andreas Nin.{\indexANin} Hernandez has described how Nin was first tortured to the limits of human endurance and then, according to a scenario devised by Vidali, ``liberated'' by supposed agents of the Gestapo disguised as members of the International Brigade, leaving behind ``evidence'' indicating that he was a German spy.\endnote{Hernandez, \emph{op. cit.}, chapter 5.}

The energy with which this repression was carried on against left-wing militants indicated the deep concern of the Stalinists to destroy the Spanish revolution. But they could only play their counterrevolutionary role successfully by appearing themselves in revolutionary guise when occasion demanded; otherwise they could not have won mass support, including from many workers, and made an international impact. This was particularly so at the time of the battle for Madrid. As recent historians put it,

\begin{quotation}
  The history of the defense of Madrid shows also that in certain circumstances the Communist Party{\indexPCE} is capable, not only of making an appeal to revolutionary traditions such as those of the Russian October Revolution or of the Red Army, but also of using really revolutionary methods, in a word, appearing, in the eyes of large masses of people, as an authentically revolutionary party. Many Spanish or foreign militants experienced in the defense of the capital a revolutionary epic in which the anti-fascist label was purely provisional. Against the mercenaries from Germany or Italy, they wished to be fighters in the international proletarian revolution. Many among them fought the revolution for the time being, with the conviction that it was nothing more than a tactical withdrawal of a provisional kind, and that at the end of the anti-fascist struggle the World Communist Revolution would be~found.\endnote{P. Broué and E. Témime, \emph{La révolution et la guerre d’Espagne} (Paris 1961), p. 213. This is the best of the recent books on the Spanish Revolution. Unfortunately it is not available in an English translation. [A translation was published in 1972 by Faber and Faber. \textemdash{}\ERC]}
\end{quotation}

This cynical abuse of genuine revolutionary convictions was not least among the crimes of Stalinism in this period and it is necessary to recall that many of those who fought in Spain, or who backed the Republican Government, did so because they were deluded into believing that by so doing they supported the revolutionary cause.

The weight of this sentiment was so great that the task of those who, under the inspiration of Trotsky, stood against the tide and spoke up clearly against the Stalinist betrayal was virtually insuperable. All the more reason why, now that the counterrevolutionary policies of Stalin can be seen more clearly for what they were, works like Morrow’s should be studied.

It was from the ranks of the \textsc{poum}~-- so often wrongly described as Trotskyists, including by some who ought to know better\endnote{Including H.~Thomas, in his work \emph{The Spanish Civil War} which, although distorting the political significance of the struggle, is the fullest account so far available in English. Thus on p.~71 he produces the following curious piece of reasoning: ``Although not Trotskyist in the sense of being strict followers of Trotsky (they were not affiliated to the Fourth International), these men could justifiably be regarded as such since they were Marxist opponents of Stalin who shared Trotsky’s general views: permanent resolution\index{Trotskyism} abroad, working class collectives at home.'' The \textsc{poum} leaders did not regard themselves as Trotskyists, nor did the Trotskyists outside Spain regard them as such, \dots\ therefore, Thomas accepts the Stalinist characterization of them as Trotskyists!}~-- that the chance of building a genuine revolutionary leadership was greatest. It is thus necessary to understand the nature of the \textsc{poum} and the reasons for its failures and capitulation both as displayed in this book and as explained in Trotsky’s pamphlet \emph{The Lessons of Spain}. The \textsc{poum} had a sufficient basis in the industrial working class in Catalonia to have given the necessary leadership. In order to do so, however, it was not sufficient to give lip-service to the Permanent Revolution or to make a literary exposure of the evils of Stalinism. It was also necessary to struggle against the reformist and anarchist leaders, however respected or powerful or personally honest, they may have~been.

``Instead of mobilizing the masses against the reformist leaders, including the Anarchists,'' ran Trotsky’s indictment,

\begin{quotation}
  \noindent
  \dots\ the \textsc{poum} tried to convince these gentlemen of the superiority of socialism over capitalism. This tuning fork gave the pitch to all the articles and speeches of the \textsc{poum} leaders. In order not to quarrel with the Anarchist leaders they did not form their own nuclei and in general did not conduct any kind of work inside the \textsc{cnt}.
  
  Evading sharp conflicts, they did not carry on revolutionary work in the republican army. They built instead ``their own'' trade unions and ``their own'' militia which guarded ``their own'' institutions or occupied ``their own'' section of the front.\index{Isolationism} By isolating the revolutionary vanguard from the class, the \textsc{poum} rendered the vanguard impotent and left the class without leadership.
  
  Politically the \textsc{poum} remained throughout far closer to the People’s Front, for whose left wing it provided the cover, than to Bolshevism. That the \textsc{poum} nevertheless fell victim to bloody and base repressions was due to the fact that the People’s Front could not fulfill its mission, namely, to stifle the socialist revolution~-- except by cutting off, piece by piece, its own left flank.
\end{quotation}

The \emph{salami tactic}\index{Salami tactic} was not invented by R\'akosi; it was used in Spain and the \textsc{poum} was its chief victim. The \textsc{poum} provided an object lesson in showing how ``left'' parties can, through the vice of centrism, oppose a barrier to the solution of the crisis of leadership.

\smallskip

As Trotsky continued,

\begin{quotation}
  Contrary to its own intentions, the \textsc{poum} proved to be, in the final analysis, the chief obstacle on the road to the creation of a revolutionary party.
\end{quotation}

The supporters of the Fourth International in Spain were but a handful, some were foreigners and they lacked roots in the working class. By the time that they found a hearing in the \textsc{poum}, or from workers under anarchist influence~-- especially the Friends of Durruti who, according to Morrow, ``explicitly declared the need for democratic organs of power, juntas or soviets, in the overthrow of capitalism, and the necessary state measures of repression against the counterrevolution''~-- the most militant sections of the working class had been defeated and divided. After May~1937, the victory of the counter-revolution was assured. From that time the last act was played out in the somber tones of tragedy and against a backcloth of defeat and betrayal.

\dinkus

It is one of the merits of Morrow’s book that he shows in detail how the bourgeois nature of the Republican government made it incapable of waging a revolutionary war, the only type of war which stood any chance of overcoming the forces of international fascism. It could not promise liberation to the Moors, it could not offer agrarian reform to the peasants in Franco territory, it could not appear as the liberator of the working class nor could it appeal to the fascist troops with revolutionary propaganda. The deliberate cutting off of arms and supplies to sectors of the front held by \textsc{cnt} and \textsc{poum} militias had its counterpart in the superb armament and equipment of Assault Guards and army units kept in the rear for ``security'' purposes.\endnote{On this point the mordant record in G.~Orwell’s \emph{Homage to Catalonia}~-- the best personal account of the war~-- now available as a Penguin book.} Morrow recounts how on several sectors of the front the way was opened to the fascist forces by the betrayals of republican officers and police. He shows how the navy, an important weapon in republican hands, was reduced to impotence out of consideration for foreign powers.

Those who argue that a revolutionary seizure of power was out of the question and that the restoration of republican institutions was necessary to defeat Franco are answered by implication throughout Morrow’s work.%
\endnote{[At this point, the \textsc{mia} edition reads: ``They are dealt with more specifically on pages 82--91 which rank among the most important in the book and to which, therefore, the reader is especially referred.''
However, those page numbers are not relevant to this edition. —\JCW]}
But the most convincing refutation of the argument for ``anti-fascism'' is its complete practical failure and the demoralization which accompanied and resulted from its failure. The military defeat followed the counterrevolution with tragic inevitability once the ``democracies'' showed that they were not interested in getting involved in a war to stop Spain going fascist and Stalin necessarily pulled out his forces to cut his losses. The undignified scramble of the agents of the Comintern and the \textsc{nkvd} to find places on planes and boats leaving Spain in the final stages of the war was a fitting conclusion to their work.\endnote{J. Hernandez, \emph{op. cit.}, chapters 8 and 9.} 

Unfortunately for some of them, Stalin decided that they knew to much, or had been too shaken by the experiences; Rosenberg, Antonov-Ovseenko and Kolsov were among those who perished on their return to Russia. Nor can it have been accidental that the post-war purges in Eastern Europe included among their victims a remarkably high proportion of communists who had seen service in~Spain.

The Franco regime, erected on the ruins of the revolution of 1936, has endured to this day. In the Second World War it leant on the Axis powers, but without formally participating on their side. Its fate, in 1945, hung in the balance; but the betrayals of the Social Democrats and Communists in Western Europe decided the future of the Spanish people for a further lengthy period. The regime has, needless to say, solved none of the problems of backward Spain. Despite talk of ``agrarian reform'' the distribution of land ownership remains much as it was in 1936. Since the mid-1950s especially, the Spanish working class has provided, through continued resistance to the regime, including a number of great strikes, evidence of its vitality. The regime depends, as it has always done, upon the support of the other capitalist powers; by now that means mainly the United States. Even so, its future remains in question. There has been frequent talk of monarchist restoration. Reactionaries like Gil Robles{\indexGRobles} offer themselves as an alternative. Sections of the Church have taken their distance from the regime. No doubt, among many sections of the population, democratic illusions are rife. For many workers and peasants long years of poverty and repression have brought their toll of demoralization from which only the younger elements are entirely~free.

\begin{sloppypar}
The question of the forthcoming Spanish revolution remains posed. The crisis of the leadership still has to be solved. In the meantime, the Communist Party, still able to convince many that it is a revolutionary party, prepares new betrayals. In the policy of ``peaceful coexistence'' and the parliamentary road~-- as though there can be a peaceful path in keeping with totalitarian Spain!~-- it preaches ``national reconciliation.'' It is willing to collaborate with all and sundry, including former supporters of Franco, in a broad coalition to restore ``democracy'' in Spain. It therefore repeats, for the present period, precisely the same role that it played in 1936--39. A study of this book can forearm against such a policy, a policy which can only lead the workers and peasants of Spain to fresh defeats.
\end{sloppypar}

\begin{flushright}
  7th March, 1963 \\
  T. Kemp
\end{flushright}

\mainmatter

  \chapter{Why the Fascists Revolted}

\lettrineA{t dawn}, July 17, 1936, General Franco assumed command of the Moors and Legionnaires of Spanish Morocco, and issued a manifesto to the army and the nation to join him in establishing an authoritarian state in Spain. In the next three days, one by one, almost all of the fifty garrisons in Spain declared for fascism. The basic sections of the capitalists and landlords, having already participated in Franco’s conspiracy, fled into fascist-held territory or out of Spain either before or after the rising. It was clear immediately that this rising had nothing in common with the \emph{pronunciamento} movements whereby the Spanish army had so often supported one bourgeois faction against another. It was not a ``handful of generals,'' but the ruling class as a whole, which was directing its armed minions in an assault, above all, against the economic, political and cultural organizations of the working class.

Franco’s programme is identical in fundamentals with that of Mussolini and Hitler. Fascism is a special form of reaction, the product of the period of capitalist decline. To see this fully one has only to compare Franco’s regime with that of the monarchy. The last Alfonso’s record is a bloody account of massacres of peasants and workers, of terrorism and assassination of proletarian leaders. Yet side by side with the systematic measures of repression, the monarchy permitted a restricted existence to economic and political organizations of the working class and to municipal and national organs of parliamentary democracy. Even under the Primo de Rivera dictatorship (1923--1930), the Socialist Party and the \textsc{ugt} led a legal existence; indeed, Largo Caballero, head of the \textsc{ugt}, was a Councillor of State under Rivera. In other words, even the reactionary monarchy sought part of its mass support in the organized proletariat, through the mediation of reformist leaders like Prieto and Caballero. Similarly, a system of legal unions and social-democratic parties existed in the empires of Wilhelm and Franz Joseph. Even under Tsar Nicholas there was a measure of legality for unions, co-operatives and the labour press, in which the Bolsheviks were able to work though themselves illegal: \emph{Pravda} had a circulation of 60,000 in 1912--1914.

As contrasted to these reactionary regimes, the special character of fascism consists in its extirpation of any and all independent organizations of the working class. Capitalism in decline finds impossible even the most elementary concessions to the masses. One by one, those capitalist countries which arrive at a complete impasse, take the road of fascism.

Italy, a ``victor'' in the World War, weakly developed in basic industries, could not compete with more advanced countries in the imperialist race for markets. Strangling in its economic contradictions, the capitalist class of Italy could find a way out only over the broken bones of the workers’ organizations. The hordes of the ``petty bourgeoisie gone mad,'' organized and uniformed by Mussolini, trained as hoodlums, were finally unleashed for the special task of crushing the workers’ organizations.

The bourgeoisie does not light-mindedly take to fascism. The Nazi movement of Germany had almost no bourgeois support in its putsch of 1923. In the ensuing decade, it secured financial support only from a few individual capitalists until 1932. The bourgeoisie of Germany hesitated for a long time before it accepted the instrumentality of Hitler; for fifteen years it preferred to lean on the social-democratic leaders. But at the height of the world economic crisis, technically advanced Germany, handicapped by the Versailles Treaty in its imperialist conflicts with England, France and America, could ``solve'' its crisis temporarily, in capitalist terms, only by destroying the workers’ organization which had existed for three-fourths of a century.

Fascism is that special form of capitalist domination which the bourgeoisie finally resorts to when the continued existence of capitalism is incompatible with the existence of organized workers. Fascism is resorted to when the concessions, which are a product of the activities of trade unions and political parties of labour, become an intolerable burden on the capitalist rulers, hence intolerable to the further existence of capitalism. For the working class, at this point, the issue is inexorably posed for immediate solution: either fascism or socialism.

Spanish capitalism had arrived at this point when Franco revolted. His movement, though incorporating the remnants of Spain’s feudal aristocracy, is no more ``feudal'' in basic social character than was that of Mussolini or Hitler.

The chief industry of Spain, agriculture, accounting for over half the national income, almost two-thirds of exports and most, of the government’s internal revenue, with 70 per cent of the population living on the land, was in desperate straits. The division of the land was the worst in Europe: one-third owned by great landowners, in some cases in estates covering half a province; another third held by more numerous owners but also in large estates; only one-third owned by peasants, and most of this in primitively equipped farms of five hectares\footnote{One hectare equals 2.4 acres.} or less of extraordinarily dry, poor land, insufficient to support their families, and necessitating day labour on the big estates to eke out an existence. Thus, most of the five million peasant families were dependent on sharecropping or employment on the big estates.

Spanish agriculture was conducted by primitive methods. Its yield per hectare was the lowest in Europe. Increased productivity required capital investment in machinery and fertilizers, employment of technicians, re-training of the peasants. From the landowners’ standpoint it was cheaper to continue primitive methods at the expense of the peasantry. The one recent period of good prices for produce, the war years, 1914--1918, which gave Spanish agriculture a temporary opportunity to profiteer in the word market, instead of being used to improve the land, was capitalized into cash via mortgages obtained by the landowners. Driven out of the world market after the war, Spanish agriculture collapsed. The general agricultural crisis, first preceding and then part of the world crisis, aggravated by the tariff barriers raised against Spanish agriculture by England and France, led to widespread unemployment and starvation.

Precisely at the depth of the crisis, in 1931, the rise of the republic gave a new impetus to the organization of agricultural workers’ unions. The resultant wage raises seem pitiful enough. A good wage was six pesetas (seventy-five cents) a day. But even this was a deadly menace to the profits of the Spanish landowners, in the epoch of the decline of European agriculture. The great plains of South America and Australia were providing wheat and beef to Europe at prices which were dealing European agriculture a blow incomparably more serious than that dealt by the produce of North America during the epoch of capitalist expansion. \emph{Thus, the existence of agricultural workers’ unions and peasants’ organizations was incompatible with the further existence of landed capitalism in Spain}.

The landowners got a breathing space during the \emph{bienio negro}, the ``two black years,'' of September 1933 to January 1936, when the reactionary governments of Lerroux-Gil Robles terrorized the masses and put down the revolt of October 1934. During that period, day wages on the land fell to two or three pesetas. But the masses soon rallied. Gil Robles’ attempt to build a mass fascist organization failed, both through his own ineptitude and under the blows of the workers. The Asturian Commune of October 1934, though crushed by Moors and Legionnaires, became the inspiration of the masses, and Lerroux-Gil Robles gave way to the Popular Front in February 1936, rather than wait for a more decisive onslaught by the proletariat. The agrarian workers and peasants built even more formidable unions from February to July 1936, and the precarious condition of agricultural profits drove the landowners and their allies, the Catholic hierarchy and the banks, to a speedy resort to arms to destroy the workers’ organizations.

The capitalists in industry and transportation were likewise at an impasse.

The era of expansion of Spanish industry had been short: 1898--1918. The very development of Spanish industry in the war years became a source of further difficulties. The end of the war meant that Spain’s industry, infantile and backed by no strong state power, soon fell behind in the imperialist race for markets. Even Spain’s internal market could not long be preserved for her own industry. Primo de Rivera’s attempt to preserve it by insurmountable tariff walls brought from France and England retaliation against Spanish agriculture. The resultant agricultural crisis caused the internal market for industry to collapse. In 1931, this country of twenty-four millions had nearly a million unemployed workers and peasants, heads of families; before the end of 1933 the number was a million and a half.

With the end of the \emph{bienio negro}, the workers’ economic struggles took on extraordinary scope. Conscious of having freed themselves from the domination of Gil Robles by their own efforts, the masses did not wait for Azaña to fulfill his promises. In the four days between the elections of February 1936 and Azaña’s hasty assumption of the premiership, the masses effectively carried out the amnesty by tearing open the jails. Nor did the workers wait for the government decree, and for the decision as to its constitutionality---which came from the Court of Constitutional Guarantees only on September 6, nearly two months after Franco’s revolt!---to get back the jobs of those dismissed after the October 1934 revolt. In the shops and factories the workers took along those dismissed, and put them back to work. Then, beginning with a general strike on April 17, 1936, in Madrid, there began a great movement of the masses, often including political demands, but primarily for better wages and conditions.

We can only roughly indicate the magnitude of the great strike wave. The strikes covered both the cities and the rural districts. Every city and province of any importance had at least one general strike during February--July 1936. Nearly a million were on strike on June 10, a half million on June 20, a million on June 24, over a million during the first days of July.

Spanish capitalism could scarcely hope to solve its problems by expanding its markets for manufactured goods. That road was closed to it externally by the great imperialist powers. Internally, the only way to expand was to create a prosperous landed peasantry, but that meant dividing the land. The city capitalist and landed proprietor were often one and the same person, or bound together by family ties. In any event, the summit of Spanish capitalism, the banks, were inextricably bound up with the interests of the landowners, whose mortgages they held. No real road of development was open to Spanish capitalism. But it could temporarily solve its problems in one way: by destroying the trade unions which were endangering its profits.

Bourgeois democracy is that form of capitalist state which leans on the support of the workers, secured through the reformist leaders. The capitalists of Spain concluded that democracy was intolerable, and that meant that bourgeois democracy and reformism were finished in Spain.

Mussolini declared he had saved Italy from Bolshevism. Unfortunately, the truth is that the workers’ post-war upsurge had already receded, thereby facilitating Mussolini’s assumption of power. Hitler said the same, at a time when the workers were hopelessly divided and disoriented. Franco had need of the same myth for justifying his resort to arms. What was true, in Italy, Germany and now in Spain, is that democracy could no longer exist. Precisely the fact that fascism had to seize power, even though there was no immediate danger of a proletarian revolution, is the most conclusive evidence that democracy was finished.

Franco’s rebellion left only two alternatives: either fascism would conquer, or the working class, rallying the peasantry by giving it the land, would destroy fascism and with it the capitalism in which it is rooted.

The Stalinists and social-democrats, seeking theoretical justification for their collaboration with the liberal bourgeoisie, declare that the roots of fascism in Spain are feudal. For the Stalinists, this is an entirely new theory, concocted \emph{ad hoc}. Spanish fascism is no more feudal than is Italian. The backwardness of industry in both countries cannot be overcome within the capitalist framework, since neither can compete with the advanced industrial countries in the epoch of declining world markets. They could only secure temporary stabilization by cutting their labour costs below the European level, and to do this required crushing every form of labour organization. Spanish agriculture is backward and ``feudal'' in its operating methods. But land has been bought, sold and mortgaged, like any other commodity, for two centuries. Hence, the land question is a \emph{capitalist} question.
\noclub

The Stalinist journalist, Louis Fischer, writes:

\begin{quotation}
  Strangely enough, Spain’s small industrialist class support\-ed the reactionary position taken by the landlords. The industrialists should have welcomed a land reform which would create a home market for their goods. But they believed that more than economics was involved. They feared that the granting of land to the peasantry would rob the owning classes of political power. The manufacturers, therefore, who should have encouraged the republic in its attempts to stage a peaceful revolution which would have enriched the country, actually leagued themselves with the backward looking landlords to prevent all amelioration and reform. (\emph{The War in Spain}, published by \emph{The Nation}.)
\end{quotation}

It does not occur to Fischer to wonder whether landlord and capitalist are not often one and the same, or of the same family, or whether the manufacturer, dependent on the banks, is not fearful for the banks’ mortgages on the land. But even as Fischer poses the problem, the answer is clear. The manufacturer fears the diminution of the political power of the owning classes. Why? Because the weakening of the police power permits the workers in his factory to organize and make inroads into his profits. Fischer’s own laboured explanation gives the show away. Spanish fascism is the weapon not of ``feudalism'' but of capitalism. It can be fought successfully by the working class and the peasantry, and by them alone.
  \chapter{Bourgeois “Allies” in the People's Front}

\lettrineT{he stake of} the workers’ parties and unions in the struggle against fascism was clear: their very existence was at stake. As Hitler and Mussolini before him, Franco{\indexFFranco} would physically destroy the leadership and active cadres of their organizations and leave the workers in forced disunity, each an atom at the mercy of concentrated capital. The struggle against fascism was, therefore, a life and death question\index{Fascism}\index{Socialism or barbarism} not only to the masses of the workers but also even to the reformist leaders of the workers. But this is not the same as to say that these leaders knew how to fight against fascism. Their most fatal error was their assumption that their bourgeois allies in the People’s Front were equally vitally concerned to fight against fascism.

\index{Bourgeois--Stalinist bloc}
Azaña's Republican Left\footnote{Izquierda Republicana}, Martinet Barrio's Republican Union\footnote{Unión Republicana}, and Companys' Catalan Left\footnote{Esquerra Republicana de Catalunya (\textsc{erc}); referred to as ``Esquerra''} had compacted with the Socialist and Communist parties and the \textsc{ugt}---with the tacit consent of the anarchists, whose masses voted for the joint tickets---in the elections of February 16, 1936. The Basque Nationalists had also joined. These four bourgeois groups, therefore, found themselves on the other side of the barricades from the big bourgeoisie on July 17. Could they be depended on to cooperate loyally in the struggle against fascism?

\index{Bourgeoisie}
We said no, because no vital interest of the liberal bourgeoisie was menaced by the fascists. The workers were in danger of losing their trade unions, without which they would starve. What comparable loss was faced by the liberal bourgeoisie? Undoubtedly, in a totalitarian state, the professional politicians would have to find other professions; the liberal bourgeois press would go under (if we grant that the bourgeois politicians and journalists would not go over to Franco entirely).
\noclub

\index{Fascism}\index{Popular Front}\index{Bourgeoisie}
Both Italy and Germany have demonstrated that fascism refuses to become reconciled to individual democratic politicians; some are jailed, others must emigrate. But all these constitute minor inconveniences. The basic strata of the liberal bourgeoisie go on as before the advent of fascism. If they do not share the special favors extended by the fascist state to those capitalists who had joined the fascists before victory, they do share the advantages of low wages and curtailed social services. Only to the same extent as all other capitalists are they subject to the fascist exactions, via party or government, which are the stiff price which capitalism pays for the services of fascism. The liberal bourgeoisie of Spain had only to look at Germany and Italy to be reassured for the future. While the trade union officials have been extirpated, the liberal bourgeoisie has found elbow room in which to be assimilated.
What is at work here is a class criterion: fascism is the enemy primarily of the working class. \emph{Hence, it is absolutely false and fatal to assume that the bourgeois elements in the Popular Front have a crucial stake in prosecuting seriously the struggle against fascism.}

Second, our proof that Azaña, Barrio, Companys and their ilk cannot be loyal allies of the working class, rested not merely on deductive analysis, but upon concrete experience---the record of these worthies. Since the socialists and Stalinists in the People’s Front have suppressed the facts about their allies, we must give some space to this question.

From 1931 to 1934, the Comintern called Azaña{\indexMAzana} a fascist, which was certainly incorrect, although it accurately pointed to his systematic oppression of the masses.

As late as January 1936, the Comintern\index{Communist International} said of him:

\begin{quotation}
  The Communist Party knows the danger of Azaña just as well as the socialists who collaborated with him when he was in power. They know that he is an enemy of the working class \dots\ But they also know that the defeat of the \textsc{ceda} (Gil Robles) would automatically bring with it a certain amount of relief from the repression, for a time at least. (\emph{Inprecor}, Vol. 15, p. 762.)
\end{quotation}

The last phrase is an admission that repression will come from the direction of Azaña himself. And it did come, as Josh Diaz, Secretary of the Communist Party\index{Communist Party of Spain}, was compelled to admit just before the civil war broke out:

\begin{quotation}
  The government, which we are loyally supporting in the measure that it completes the pact of the Popular Front, is a government that is commencing to lose the confidence of the workers, and I say to the left republican government that its road is the wrong road of April 1931. (\emph{Mundo Obrero}{\indexMundoObrero}, July 6, 1936.)
\end{quotation}

One must recall the ``wrong road of April 1931'' to realize what an admission the Stalinists were making after all their attempts to differentiate the coalition government of 1931 from the Popular Front government of 1936. The coalition of 1931 had promised land to the peasants and gave them none because the land could not be divided without undermining capitalism. The coalition of 1931 had refused the workers unemployment relief. Azaña, as Minister of War,\index{Ministries!War} had not touched the reactionary officer caste of the army, and had enforced the infamous law under which all criticism of the army by civilians was an offence against the state. As Premier, Azaña had left the swollen wealth and power of the church hierarchy intact. Azaña had left Morocco in the hands of the Legionnaires and Moorish mercenaries. Only against the workers and peasants had Azaña been stern. The annals of 1931--33 are the annals of his government’s repressions of the workers and peasants. Elsewhere\endnote{\emph{The Civil War in Spain}, September 1936, Pioneer Publishers.}\label{en:CivilWarReference} I have told this story at length.

Azaña, as \emph{Mundo Obrero} admitted, proved no better as head of the Popular Front government of February--July 1936. Again his regime rejected the idea of distribution of the land, and put down the peasantry when it attempted to seize it. Again the church remained in full control of its great wealth and power. Again Morocco remained in the hands of the Foreign Legionnaires\index{Spanish Legion} until they finally took it over completely on July 17. Again strikes\index{Strikes} were declared illegal, modified martial law imposed, workers’ demonstrations and meetings broken up. Suffice it to say that in the critical last days, after the assassination of the fascist leader, Calvo Sotelo, the working class headquarters were ordered closed. \emph{The day before the fascist outbreak the labor press appeared with gaping white spaces where the government censorship had lifted out editorials and sections of articles warning against the \emph{coup d’\'etat}!}\index{Censorship}\index{Early warning}

In the last three months before July 17, in desperate attempts to stop the strike movement, hundreds of strikers were arrested en masse, local general strikes declared illegal and regional headquarters of the \textsc{ugt} and \textsc{cnt} closed for weeks at a time.

The most damning insight into Azaña was provided by his attitude toward the army. Its officer caste was disloyal to the core toward the republic. These pampered pets of the monarchy had seized every opportunity since 1931 to wreak bloody vengeance against the workers and peasants upon whom the republic rested. The atrocities they committed in crushing the revolt of October 1934 were so horrible that criminal punishment of those responsible was one of Azaña’s campaign promises. But he brought not a single officer to trial in the ensuing months.

Mola, director of Public Safety\index{Ministries!Public Safety, Madrid} of Madrid under the dictatorship of Berenguer--Mola who had fled on the heels of Alfonso while the streets echoed with the masses’ cries of ``¡Abajo Mola!''---this Mola was restored to his generalship in the army by Azaña, and despite his close association with Gil Robles in the \emph{bienio negro}, was military commander of Navarre at the moment of the fascist revolt, and became chief strategist of the Franco armies. Franco, Goded, Queipo de Llano---all had similarly malodorous records of disloyalty to the republic and yet Azaña left the army in their hands. More, he demanded that the masses submit to them.

Colonel Julio Mangada, now fighting in the anti-fascist forces, who had been court-martialled and driven out of the army by these generals because of his republicanism, is authority for the fact that he had repeatedly informed Azaña, Martinez Barrio, and other republican leaders of the plans of the generals. In April 1936, Mangada published a thoroughly documented pamphlet which not only exposed the fascist plot, but proved conclusively that President Azaña was fully informed of the plot when, on March 18, 1936, upon the demand of the general staff, his government gave the army a clean bill of health. Referring to ``rumors insistently circulating concerning the state of mind of the officers and subalterns of the army,'' the Government of the Republic has ``learned with sorrow and indignation of the unjust attacks to which the officers of the army have been~subjected.''

Azaña’s cabinet not only repudiated these rumors, describing the military conspirators as ``remote from all political struggle, faithful servitors of the constituted power and guarantee of obedience to the popular will,'' but declared that only a criminal and tortuous desire to undermine it [the army] can explain the insults and verbal and written attacks which have been directed against it.'' And finally, ``the Government of the Republic applies and will apply the law against anyone who persists in such an unpatriotic attitude.''

No wonder that the reactionary leaders praised Azaña. On April 3, 1936, Azaña made a speech promising the reactionaries that he would stop the strikes and seizures of the land\index{Reprivatization}{\indexMAzana}. Calvo Sotelo praised it: ``It was the expression of a true conservative. His declaration of respect for the law and the constitution should make a good impression on public opinion.''

``I support 90 percent of the speech,'' declared the spokesman of Gil Robles’ organization. ``Azaña is the only man capable of offering the country security and defense of all legal rights,'' declared Ventosa, spokesman of the Catalan landowners\index{Landlords}. They praised Azaña, for he was preparing the way for them.

Though the army was ready to rebel in May 1936, many reactionaries doubted whether this was possible as yet. Azaña pressed upon them his solution: let the reformist leaders stop the strikes. His offer was accepted. Miguel Maura, representing the extreme right industrialists and landowners, called for a strong regime of ``all republicans and those socialists not contaminated by revolutionary lunacy.'' So, having been elevated to the presidency, Azaña offered the premiership to the right-wing socialist, Prieto. The Stalinists, the Catalan Esquerra{\indexEsquerra}, the Republican Union of Barrio, as well as the reactionary bourgeoisie, supported Azaña’s candidate.

The left socialists, however, prevented Prieto{\indexIPrieto} from accepting\ldots\ For the reactionary bourgeoisie, the Prieto premiership would have been, at most, a breathing space to prepare further. Having, failed to secure it, they plunged into civil war.

Such was the record of Azaña’s Republican Left. That of the other liberal bourgeois parties was, if anything, worse. Companys’ Catalan Esquerra had ruled Catalonia since 1931.{\indexLCompanys} Its Catalonian nationalism served to hold in leash the more backward strata of peasants while Companys used armed force against the \textsc{cnt}. On the eve of the October 1934 revolt he had reduced the \textsc{cnt} to semi-legal status, with hundreds of its leaders jailed. It was this situation which had moved the \textsc{cnt} so unwisely to refuse to join the revolt against Lerroux--Gil Robles, declaring that Companys was just as much a tyrant; while Companys, faced with the choice between arming the workers and knuckling under to Gil Robles, had chosen the latter.\endnote{The Estat Catala\index{Estat Catala}, a split-off from the Esquerra, combining extreme separatism with anti-labor hooliganism, had provided its khaki-shirt members for strike-breaking{\index{Strike breaking}}; had disarmed workers during the October 1934 revolt. This organization, too, after July 19, turned up in the ``anti-fascist'' camp!}

As for the Republican Union\index{Republican Union} of Martinez Barrio\indexDBarrio, it was nothing more than reconstituted remnants of Gil Robles’ allies, Lerroux’s Radicals. Barrio himself had been Lerroux’s chief lieutenant, and had served as one of the premiers in the \emph{bienio negro}, putting down with great cruelty an anarchist rising in December 1933. He had perspicaciously left the sinking ship of the Radicals when it became clear that the crushing of the October 1934 revolt had failed to stem the masses, and made his debut as an ``anti-fascist'' in 1935 by signing a petition for amnesty for political prisoners. When Lerroux{\indexALerroux} fell after a financial scandal his following turned to Barrio.
\nowidow

The fourth of the bourgeois parties, the Basque Nationalists\index{Basque Nationalists}, had collaborated closely with the extreme reactionaries of the rest of Spain until Lerroux had sought to curb ancient provincial privileges. Catholic, led by the big landowners and capitalists of the four Basque provinces---the Basque Nationalists had supported Gil Robles in crushing the Asturian Commune\index{Asturian Commune} of October 1934. They were from the first uncomfortable in their alliance with the workers’ organizations. That they did not go over to the other side of the barricades immediately, is to be explained by the fact that the Biscayan region\index{Basque Country!Biscay} was a traditional sphere of influence of Anglo--French imperialism and as such hesitated to enter the alliance with Hitler and Mussolini.

These, then, were the ``loyal,'' ``reliable,'' ``honourable'' allies of the Stalinist--reformist\index{Communist Party of Spain} leaders in the struggle against fascism. If in peacetime the liberal bourgeoisie had refused to touch the land, the church or the army, because they did not want to undermine the foundations of private property, was it conceivable that now, with arms in hand, the liberal bourgeoisie would loyally support a war to the finish against reaction? If Franco’s army was crushed, what would happen to the liberal bourgeoisie which in the last analysis had maintained its privileges only because of the army? Precisely because of these considerations, the Franco forces moved boldly, taking it for granted that Azaña and Companys would come to heel. Precisely because of these considerations, Azaña and the liberal bourgeoisie \emph{did attempt to come to terms with Franco}.

The Stalinists and reformists, compromised by their People’s Front policy, have connived with the liberal bourgeoisie in suppressing almost completely from the outside world the cold facts which reveal the treachery of which Azaña and his associates were guilty in the first days of the revolt. But here are the indisputable facts.
\nowidow

On the morning of July 17, 1936, General Franco{\indexFFranco}, having seized Morocco{\index{Morocco}}, radioed his manifesto to the Spanish garrisons, instructing them to seize the cities. Franco’s communications were received at the naval station near Madrid\index{Madrid} by a loyal operator and promptly revealed to the Minister of the Navy\index{Ministries!Navy}, Giral. But the government did not divulge the news in any form until the morning of the 18th and then it issued only a reassuring note:
\nowidow

\begin{quotation}
  The Government declares that the movement is exclusively limited to certain cities of the protectorate Zone [Morocco] and that nobody, absolutely nobody on the Peninsula [Spain], has added to such an absurd undertaking.
\end{quotation}

Later that day, at 3 \textsc{pm}, when the government had full and positive information of the scope of the rising, including the seizure of Seville, Navarre and Saragossa, it issued a note which said:

\begin{quotation}
  \begin{sloppypar}
  The Government speaks again in order to confirm the absolute tranquillity of the whole Peninsula. The Government acknowledges the offers of support which it has received [from the workers’ organizations] and, while being grateful for them, declares that the best aid that can be given to the Government is to guarantee the normality of daily life, in order to set a high example of serenity and of confidence in the means of the military strength of the State.
  \end{sloppypar}
  
  Thanks to the foresighted means adopted by the authorities, a broad movement of aggression against the republic may be deemed to have been broken up; it has found no assistance in the Peninsula and has only succeeded in securing followers in a fraction of the army in Morocco\ldots
  
  These measures, together with the customary orders to the forces in Morocco who are laboring to overcome the rising, permit us to affirm that the action of the Government will be sufficient to re-establish normality. (\emph{Claridad}{\indexClaridad}, July 18, 1936.)
\end{quotation}

This incredibly dishonest note was issued to justify the government’s refusal to arm the workers, as the trade unions had requested. But this was not all. At 5:20 and again at 7:20 \textsc{pm}, the government issued similar notes, the last declaring that ``in Seville\ldots there were acts of rebellion by the military elements that were repelled by the forces in the service of the government.'' Seville had then been in the hands of Queipo de Llano for most of the day.

Having deceived the workers about the true state of affairs, the cabinet went into an all-night conference. Azaña{\indexMAzana} had Premier Ca\-sa\-res Quiroga, a member of his own party, resign, and replaced him with the more ``respectable'' Barrio{\indexDBarrio}, and the night was spent hunting up bourgeois leaders outside the Popular Front who could be induced to enter the cabinet. With this rightist combination, Azaña made frantic attempts to contact the military leaders and come to an understanding with them. The fascist leaders, however, took the overtures as a sure sign of their victory, and refused Azaña any form of face-saving compromise. They demanded that the republicans step aside for an open military dictatorship. Even when this became known to Azaña and the cabinet ministers, they took no steps to organize resistance. Meanwhile, garrison after garrison, apprised of the government’s paralysis, took heart and unfurled the banner of rebellion.

Thus, for two decisive days, the rebels marched on while the government besought them to save its face. It made no move to declare the dissolution of rebellious regiments, to declare the soldiers absolved from obeying their officers. The workers, remembering the \emph{bienio negro}, remembering the fate of the proletariat of Italy and Germany, were clamouring for arms. Even the reformist leaders were knocking at the doors of the presidential palace, beseeching Azaña and Giral to arm the workers. In the vicinity of the garrisons, the unions had declared a general strike to paralyze the rebellion. But folded arms would not be sufficient to face the enemy. Grim silence enveloped the Montana barracks in Madrid. The officers there, in accordance with the plan of the rising, were waiting for the garrisons surrounding Madrid to reach the city, when they would join forces. Azaña and Giral and their associates waited helplessly for the blow to fall.

And, indeed could it be otherwise? The camp of Franco was saying: We, the serious masters of capital, the real spokesmen of bourgeois society, tell you that democracy must be finished, if capitalism is to live.\index{Socialism or barbarism} Choose, Azaña, between democracy and capitalism. What was deeper in Azaña and the liberal bourgeoisie? Their ``democracy'' or their capitalism? They gave their answer by bowing their heads before the onward marching ranks of fascism.

On the afternoon of July 18, the chief worker-allies of the bourgeoisie, the National Committees of the Socialist and Communist parties, issued a joint declaration:

\begin{quotation}
  The moment is a difficult one, but by no means desperate. The Government is certain that it has sufficient resources to overcome the criminal attempt\ldots\ In the eventuality that the resources of the Government be not sufficient the republic has the solemn promise of the Popular Front, which gathers under its discipline the whole Spanish proletariat, resolved serenely and dispassionately to intervene in the struggle as quickly as its intervention is to be called for\ldots\ The Government commands and the Popular Front\index{Popular Front} obeys.
\end{quotation}

But the government never gave the signal!

\medskip

Fortunately, the workers did not wait for it.
  \chapter{The Revolution of July 19}

\index{Barcelona}\index{working class}\index{Revolution of July 19 (1936)}
\lettrineT{he Barcelona} proletariat prevented the capitulation of the republic to the fascists. On July~19, almost barehanded, they stormed the first barracks successfully. By 2~\textsc{pm} the next day they were masters of Barcelona.

\indexCNT\index{anarchism}\index{Goded, Manuel}\index{Revolution of July 19 (1936)}
It was not accidental that the honor of initiating the armed struggle against fascism belongs to the Barcelona proletariat. Chief seaport and industrial center of Spain, concentrating in it and the surrounding industrial towns of Catalonia nearly half the industrial proletariat of Spain, Barcelona has always been the revolutionary vanguard. The parliamentary reformism of the socialist-led \textsc{ugt} had never found a foothold there. The united socialist and Stalinist parties (the \textsc{psuc}) had fewer members on July 19 than the \textsc{poum}. The workers were almost wholly organized in the \textsc{cnt}, whose suffering and persecution under both the monarchy and republic had imbued its masses with a militant anti-capitalist tradition, although its anarchist philosophy gave it no systematic direction. But, before this philosophy was to reveal its tragic inadequacy, the \textsc{cnt} reached historic heights in its successful struggle against the forces of General~Goded.

\index{Assault Guards}\index{battle}\index{armament of the working class!refusal to arm}\index{Barcelona}\index{Revolution of July 19 (1936)}
As in Madrid, the Catalan government refused to arm the workers. \textsc{cnt} and \textsc{poum} emissaries, demanding arms, were smilingly informed they could pick up those dropped by wounded Assault Guards.
But \textsc{cnt} and \textsc{poum} workers during the afternoon of the 18th were raiding sporting goods stores for rifles, construction jobs for sticks of dynamite, fascist homes for concealed weapons. With the aid of a few friendly Assault Guards, they had seized a few racks of government rifles. (The revolutionary workers had painstakingly gathered a few guns and pistols since 1934.) That -- and as many motor vehicles as they could find -- was all the workers had, when at five o’clock on the morning of the 19th, the fascist officers began to lead detachments from the barracks.

\index{Goded, Manuel}\index{Revolution of July 19 (1936)}
Isolated engagements before paving-stone barricades led to a general engagement in the afternoon. And here political weapons more than made up for the superior armament of the fascists. Heroic workers stepped forward from the lines to call upon the soldiers to learn why they were shooting down their fellow toilers. They fell under rifle and machine gun fire, but others took their place. Here and there a soldier began shooting wide. Soon, bolder ones turned on their officers. Some nameless military genius~-- perhaps he died then~-- seized the moment and the mass of workers abandoned their prone positions and surged forward. The first barracks were taken. General Goded was captured that afternoon. With arms from the arsenals the workers cleaned up Barcelona. Within a few days, all Catalonia was in their hands.

\index{Madrid}\index{left socialists}\index{Revolution of July 19 (1936)}
Simultaneously, the Madrid proletariat was mobilizing. The left socialists distributed their scant store of arms, saved from October 1934. Barricades went up on key streets and around the Montaña barracks. Workers’ groups were looking for reactionary leaders. At dawn of the 19th the first militia patrols took their places. At midnight the first shots were exchanged with the barracks. But it was not until the next day, when the great news came from Barcelona, that the barracks were stormed.

\index{Valencia}\index{armament of working class}\index{Revolution of July 19 (1936)}
Valencia, too, was soon saved from the fascists. Refused arms by the governor appointed by Azaña, the workers prepared to face the troops with barricades, cobblestones and kitchen knives -- until their comrades within the garrison shot the officers and gave arms to the~workers.

\index{Asturian Commune}\index{Revolution of July 19 (1936)}
The Asturian miners, fighters of the Commune of October 1934, outfitted a column of five thousand dynamiters for a march on Mad\-rid. It arrived there on the 20th, just after the barracks had been taken, and took up guard duty in the streets.

\index{M\'alaga@Malaga}\index{Revolution of July 19 (1936)}
In M\'alaga, strategic port opposite Morocco, the ingenious workers, unarmed at first, had surrounded the reactionary garrison with a wall of gasoline-fired houses and barricades.

\index{spontenaity}\index{working class}\index{Revolution of July 19 (1936)}
In a word, without so much as by-your-leave to the government, the proletariat had begun a war to the death against the fascists. The initiative had passed out of the hands of the republican bourgeoisie.

\index{workers' militias}\index{Revolution of July 19 (1936)}
Most of the army was with the fascists. It must be confronted by a new army. Every workers’ organization proceeded to organize militia regiments, equip them, and send them to the front. The government had no direct contact with the workers’ militia. The organizations presented their requisitions and payrolls to the government, which handed over supplies and funds which the organizations distributed to the militias. Such officers as remained in the Loyalist camp were assigned as ``technicians'' to the militias. Their military proposals were transmitted to the militiamen through the worker officers. Those Civil and Assault Guards still adhering to the government soon disappeared from the streets. In the prevailing atmosphere the government was compelled to send them to the front. Their police duties were taken over by worker police and militiamen.

\index{sailors' committees}\index{Revolution of July 19 (1936)}
The sailors, traditionally more radical than soldiers, saved a good part of the fleet by shooting their officers. Elected sailors’ committees took over control of the Loyalist fleet, and established contact with the workers’ committees on shore.

\index{armed workers' committees}\index{customs officers}\index{Revolution of July 19 (1936)}
Armed workers’ committees displaced the customs officers at the frontiers. A union book or red party card was better than a passport for entering the country. Few reactionaries managed to get out through the workers’ cordon.
\nowidow

\index{revolutionary economic measures}
The revolutionary military measures were accompanied by revolutionary economic measures against fascism. Why this happened, if the world-historical scheme called merely for ``defense of the republic,'' the Stalinist--democrats have yet to explain.

\index{Catalonia}\index{Barcelona}\index{workers' committees}\indexCNT\indexUGT\index{weapons production}\index{transportation}\index{dual power}\index{factory control}\index{Revolution of July 19 (1936)}
Especially was this true in Catalonia where, within a week from July~19, transport and industry was almost entirely in the hands of \textsc{cnt} workers’ committees, or where workers belonged to both, \textsc{cnt--ugt} joint committees. The union committees systematically took over, re-established order and sped up production for wartime needs. Through national industries stemming from Barcelona, the same process spread to Madrid, Valencia, Alicante, Almeria and M\'alaga although never becoming as universal as in Catalonia. In the Basque provinces, however, where the big bourgeoisie had declared for the~democratic republic, they remained masters in the factories. A \textsc{cnt--ugt} committee took charge of all transportation in Spain. Soon factory delegations were going abroad to arrange for exports and imports.

\index{peasantry}\index{land reform}\index{Revolution of July 19 (1936)}
The peasants needed no urging to take the land. They had been trying to take it since 1931; but Casas Viejas, Castilblanco, Yeste, were honored names of villages where the peasants had been massacred by Azaña’s troops because they had taken land. Now Azaña could not stop them. As the news came from the cities, the peasants spread over the land. Their scythes and axes took care of any government official or republican landowner injudicious enough to bar their way.

\index{peasant committees}\index{Revolution of July 19 (1936)}
In many places, permeated by anarchist and left socialist teachings, the peasants organized directly into collectives. Peasant committees took charge of feeding the militias and the cities, giving or selling directly to the provisioning committees, militia columns and the trade unions.

\index{Revolution of July 19 (1936)}
Everywhere the existing governmental forms and workers’ organizations proved inadequate as methods of organizing the war and revolution. Every district, town and village created its militia committees, to arm the masses and drill them. The \textsc{cnt--ugt} factory committees, directing all the workers, including those never before organized, developed a broader scope than the existing trade union organizations. The old municipal administrations disappeared, to be replaced, generally, by agreed-upon committees giving representation to all the anti-fascist parties and unions. But in these the Esquerra and Republican Left politicians seldom appeared. They were replaced by workers and peasants who, though still adhering to the republican parties, followed the lead of the more advanced workers who sat with them.
\index{workers' committees}\indexCNT\indexUGT\index{dual power}\index{factory control}

\index{Central Committee of Anti-Fascist Militias of Catalonia}\index{Catalonia}\index{armament of working class}\index{Revolution of July 19 (1936)}
The most important of these new organs of power was the Central Committee of Anti-Fascist Militias of Catalonia, organized July 21. Of its fifteen members, five were anarchists from the \textsc{cnt} and \textsc{fai}, and these dominated the Central Committee. The \textsc{ugt} had three members, despite its numerical weakness in Catalonia, but the anarchists hoped thereby to encourage similar committees elsewhere. The \textsc{poum} had one, the Peasant Union (Rabassaires) one, and the Stalinists (\textsc{psuc}) one. The bourgeois parties had four.

\index{workers' militias}\index{workers' committees}\index{worker police}\index{ministries!supplies}\index{dual power}\index{Revolution of July 19 (1936)}
Unlike a coalition government which in actuality rests on the old state machine, the Central Committee, dominated by the anarchists, rested on the workers’ organizations and militias. The Esquerra and those closest to it -- the Stalinists and the \textsc{ugt} -- merely tagged along for the time being. The decrees of the Central Committee were the only law in Catalonia. Companys unquestioningly obeyed its requisitions and financial orders. Beginning presumably as the center for organizing the militias, it inevitably had to take on more and more governmental functions. Soon it organized a department of worker police; then a department of supplies, whose word was law in the factories and seaport.

\index{battle}\index{Catalonia}\index{Aragon}\index{army of social liberation@``army of social liberation''}\index{fortress of the revolution@``fortress of the revolution''}\index{armament of working class}\index{Revolution of July 19 (1936)}
In those months in which the Central Committee existed, its military campaigns were inextricably bound up with revolutionary acts. This is evident from its campaign in Aragon, on which the Catalan militias marched within five days. They conquered Aragon as an army of social liberation. Village anti-fascist committees were set up, to which were turned over all the large estates, crops, supplies, cattle, tools, etc., belonging to big land owners and reactionaries. Thereupon the village committee organized production on the new basis, usually collectives, and created a village militia to carry out socialization and fight reaction. Captured reactionaries were placed before the general assembly of the village for trial. All property titles, mortgages, debt documents in the official records, were consigned to the bonfire. Having thus transformed the world of the village, the Catalonian columns could go forward, secure in the knowledge that every village so dealt with was a fortress of the revolution!
\index{battle}\index{Catalonia}\index{army of social liberation@``army of social liberation''}\index{fortress of the revolution@``fortress of the revolution''}

\index{Stalinism}\index{battle}\index{Catalonia}\index{Aragon}\index{Zaragoza}\index{dual power}\index{factory control}\index{workers' committees}\index{Zaragoza}\indexCNT\indexPOUM\index{Revolution of July 19 (1936)}
Much malicious propaganda has been spread by the Stalinists concerning the alleged weakness of the military activity of the anarchists. The hasty creation of militias, the organization of war industry, were inevitably haphazard in all unaccustomed hands. But in those first months, the anarchists, seconded by the \textsc{poum}, made up for much of their military inexperience by their bold social policies. In civil war, politics is the determining weapon. By taking the initiative, by seizing the factories, by encouraging the peasantry to take the land, the \textsc{cnt} masses crushed the Catalonian garrisons. By marching into Aragon as social liberators, they roused the peasantry to paralyze the mobility of the fascist forces. In the plans of the generals, Zaragoza, seat of the War College and perhaps the biggest of the army garrisons, was to have been for Eastern Spain what Burgos became in the west. Instead, Zaragoza was immobilized from the first days.

\index{worker police}\index{factory control}\index{peasant committees}\index{worker committees}\index{dual power}\index{Esquerra}\indexCNT\index{Revolution of July 19 (1936)}
Around the Central Committee of the militias rallied the multitudinous committees of the factories, villages, supplies, food, police, etc., inform joint committees of the various anti-fascist organizations, in actuality wielding an authority greater than that of its constituents. After the first tidal wave of revolution, of course, the committees revealed their basic weakness: they were based on mutual agreement of the organizations from which they drew their members, and after the first weeks, the Esquerra, backed by the Stalinists, recovered their courage and voiced their own program. The \textsc{cnt} leaders began to make concessions detrimental to the revolution. From that point on, the committees could have only functioned progressively by abandoning the method of mutual agreement and adopting the method of majority decisions by democratically elected delegates from the militias and factories.

\index{Madrid}\index{Valencia}\index{workers' militias}\index{worker police}\index{factory control}\index{workers' committees}\index{dual power}\index{district committees}\index{Revolution of July 19 (1936)}
The Valencia and Madrid regions also developed a network of anti-fascist joint militia committees, worker patrols, factory committees, and district committees to wipe out the reactionaries in the cities and send the militia to the front.

\index{revolutionary economic measures}\index{armament of working class}\index{Revolution of July 19 (1936)}
Thus, side by side with the official governments of Madrid and Catalonia there had arisen organs predominantly worker-controlled, through which the masses organized the struggle against fascism. In the main, the military, economic and political struggle was proceeding independently of the government and, indeed, in spite of~it.

\index{dual power}\index{October Revolution}\index{Revolution of July 19 (1936)}
How are we to characterize such a regime? In essence, it was identical with the regime which existed in Russia from February to November 1917 -- a regime of dual power. One power, that of Azaña and Companys, without an army, police or other armed force of its own, was already too weak to challenge the existence of the other. The other, that of the armed proletariat, was not yet conscious enough of the necessity to dispense with the existence of the power of Azaña and Companys. This phenomenon of dual power has accompanied all proletarian revolutions. It signifies that the class struggle is about to reach the point where either one or the other must become undisputed master. It is a critical balancing of alternatives on a razor edge. A long period of equilibrium is out of the question; either one or the other must soon prevail! The \emph{revolution of July~19} was incomplete, but that it was a revolution is attested to by its having created a regime of dual power.
  \chapter{Toward a Coalition with the Bourgeoisie}

\lettrine[lines=3, findent=3pt, nindent=0pt]{I}{n every other} period of dual power---Russia of February to November 1917, Germany of 1918 to 1919 are the most important---the bourgeois government continued to exist, thanks only to the entry into it of representatives of the reformist workers’ organizations, who thereby became the main prop for the bourgeoisie. The Mensheviks and Social Revolutionaries not only defended the Provisional Government within the Soviets but also sat with the bourgeois ministers in the government. Ebert and Scheidemann wielded a majority in the Soldiers’ and Workers’ Councils but simultaneously sat in the government. In Spain, however, for seven critical weeks, no workers’ representative entered the cabinet.

Not that the bourgeoisie did not want them there, nor that workers’ leaders were not available and willing! On the evening of July 19, when full confirmation of the workers’ conquest of Barcelona arrived, Azaña finally abandoned the attempt to form a ``peace cabinet'' under Barrio. Giral became premier. Azaña and Giral asked Prieto and Caballero to enter the cabinet. Prieto was more than willing. Caballero refused Giral’s proposal, and Prieto dared not enter without him.

In Catalonia, during the last days of July, Companys took three Stalinist leaders into his cabinet. But in three days they were forced to resign, at the demand of the anarchists, who denounced their entry as disruption of the leading role of the Central Committee of Militias.

Thus, for seven weeks, the bourgeois governments remained isolated from the masses, unprotected by reformist ministers. Nor did the conduct of the republicans enhance their prestige. The more cowardly functionaries fled to Paris. The \textsc{cnt} \emph{Solidaridad Obrera} published day after day a \emph{Gallery of Illustrious Men}, of republicans who had fled. The government had in its possession one of the largest gold reserves outside those of the big imperialist powers---over six hundred millions in dollars---yet made no effort during those first two months to purchase arms abroad. It praised France’s attempts to organize ``non-intervention.'' It cried out against the workers’ seizure of the factories and organization of war production. It denounced the district committees and worker-patrols which were cleansing the rear-guard of reactionaries.

The Catalan-bourgeois regime, led by the astute Companys (he had once been a lawyer for the \textsc{cnt} and had a keen knowledge of the workers’ movement), riding a revolutionary upsurge far more intense than that in Madrid, behaved much more cleverly than Azaña-Giral. In the first red weeks it sanctioned without question all steps taken by the workers. But it was even more isolated in Barcelona than was the Madrid cabinet.

The Madrid and Barcelona governments lacked the indispensable instrument of sovereignty: armed force. The regular army was with Franco. The regular police no longer had any real independent existence, having been swallowed up in the flood of armed workers. Though itself denuded of its police, most of whom had either volunteered or been sent, under workers’ pressure, to the front, the Madrid bourgeoisie had looked askance at the official status conceded to the worker-leadership of the militias by the Catalan government. The discreet explanation offered by the Esquerra leader, Jaime Miravittles, tells volumes:

\begin{quotation}
  The Central Committee of Militias was born two or three days after the [subversive] movement, in the absence of any regular public force and when there was no army in Barcelona. For another thing, there were no longer any Civil or Assault Guards. For all of them had fought so arduously, united with the forces of the people, that now they formed part of the same mass and had remained mixed up with that. In these circumstances, weeks went by without it being possible to reunite and regroup the dispersed forces of the Assault and Civil Guards. (\emph{Heraldo de Madrid}, September 4, 1936.)
\end{quotation}

Yet, the fact is, despite the rise of dual power, despite the scope of the power of the proletariat in the militias and their control of economic life, the workers’ state remained embryonic, atomized, scattered in the various militias and factory committees and local anti-fascist defence committees jointly constituted by the various organizations. It never became centralized in nationwide Soldiers’ and Workers’ Councils, as it had been in Russia in 1917, in Germany in 1918–19. Only when dual power assumes such organizational proportions is there put on the order of the day the choice between the prevailing regime and a new revolutionary order of which the Councils become the state form. The Spanish revolution never rose to this point despite the fact that the real power of the proletariat was far greater than the power wielded by the workers in the German revolution or, indeed, than that wielded by the Russian workers before November. Locally and in each militia column, the workers ruled; but at the top there was only the government! This paradox has a simple explanation: there was no revolutionary party in Spain, ready to drive through the organization of soviets boldly and single-mindedly.

But isn’t it a far cry from the failure to create the organs to overthrow the bourgeoisie, to the acceptance of the role of class collaboration with the bourgeoisie? Not at all. In a revolutionary period the alternatives are poised on a razor-edge: either one or the other. Every day is as a decade in peacetime. Today’s ``realism'' becomes tomorrow’s avenue to collaboration with the bourgeoisie. Civil war is raging. The liberal bourgeoisie offers to co-operate in fighting the fascists. It is obvious that the workers should accept their aid. What are the limits of such cooperation? The ``sectarian'' Bolsheviks, in the struggle against Kornilov, set exceedingly sharp limits. Above all, they gathered power in the hands of the soviets.
\nowidow

In the very heat of the struggle against the Kornilov counterrevolution in September 1917, when Kerensky and the other bourgeois ministers in the coalition government were certainly shouting for smashing Kornilov, as much as Azaña and Companys were declaiming against Franco, the Bolsheviks warned the workers that the Provisional Government was impotent and that only the soviets could defeat Kornilov. In a special letter to the Central Committee of the Bolsheviks, Lenin castigated those who uttered ``phrases about the defence of the country, about supporting the Provisional Government.''

``We will fight, we are fighting against Kornilov, even as Kerensky’s troops do, but we do not support Kerensky,'' said Lenin. ``On the contrary, we expose his weakness. There is the difference. It is rather a subtle difference, but it is highly essential and one must not forget it.''

And there was not the slightest thought of waiting until the struggle against Kornilov was over before taking state power. On the contrary, declared Lenin, ``even tomorrow events may put power into our hands, and then we shall not relinquish it.''\endnote{\emph{Works}, Vol. XXI, Book I, p. 137}

Lenin was ready to collaborate with Kerensky himself in a military-technical union. But with this pre-condition, already existing: the masses organized in class organs, democratically elected, where the Bolsheviks could contend for a majority.

Without developing soviets---workers’ councils---it was inevitable that even the anarchists and the \textsc{poum} would drift into governmental collaboration with the bourgeoisie. For what does it mean, in practice, to refuse to build soviets in the midst of civil war? It means to recognize the right of the liberal bourgeoisie to \emph{govern} the struggle, i.e., to dictate its social and political limits.

Thus it was that all the workers’ organizations, without exception, drifted closer and closer to the liberal bourgeoisie. In the intervening weeks, Azaña and Companys recovered their nerve, as they saw that the inroads of the workers were not to be consummated by the overturn of the state power. Azaña gathered together every officer who, caught behind the lines, proclaimed himself for the republic. At first the officers could deal with the militias only through the militia committees. But the Bolshevik method of using the technical knowledge of the officers without giving them power over the soldiers, can be employed only during the height of the transition from dual power to a workers’ state, or by a soviet regime. Little by little the officers pushed their way to direct command.

The government’s control of the treasury and of the banks---for the workers, including the anarchists, had stopped short at the banks, merely instituting a form of workers’ control which was little more than guarding against disbursements to fascists and encouraging capital loans to collectivized factories---gave it a powerful lever in encouraging the considerable number of foreign-owned enterprises (which had not been seized), in placing governmental representatives in the factories, in intervening in foreign trade, in providing room for quick growth to small factories and shops and traders that had been spared from collectivization. Madrid, controlling the gold reserves, used them as an unanswerable argument in Catalonia in instances where Companys proved powerless. Under contemporary capitalism, finance capital dominates manufacturing and transportation. This law of economics was not abrogated because the workers had seized the factories and railroads. All that the workers had done in seizing these enterprises was to transform them into producers’ co-operatives, still subject to the laws of capitalist economics. Before they could be freed from these laws, all industry and land, together with bank capital and gold and silver reserves, would have to become the property of a workers’ state. But this required overthrowing the bourgeois state. The manipulation of finance capital to curb the workers’ movement is a phase of the Spanish struggle which will deserve the most careful and detailed study, and undoubtedly will provide new insights into the nature of the bourgeois state. This weapon was openly unleashed in its full force much later, but even in the first seven weeks its guarded use enabled the regime to recover much lost ground.

In the very first weeks the government, feeling its way, returned to the use of one of the instruments of state power most hated by the workers, the press censorship. It was hated particularly because of the government’s use of it during the last days before the fascist rebellion, when socialist and anarchist warnings against the imminent civil war were deleted. Azaña hastened to assure the press that the censorship would be limited to military news; but this was merely a bridge to general censorship. The unreserved supporters of the Popular Front, the Stalinists and Prieto Socialists, agreed without a murmur. An objectionable feature in the Stalinist \emph{Mundo Obrero} of August 20 led to suppression of the issue. Caballero’s \emph{Claridad} grumblingly acceded. The anarchists and the \textsc{poum} followed. Only the Madrid organ of the Anarchist Youth refused entry to the censor. But censorship was not a separate problem: it would inevitably be the prerogative of the state power.

In August, the \textsc{cnt} entered the Basque Defence Junta which was not a military organization at all, but a regional government in which the Basque big bourgeois party held the posts of finance and industry. This, the first time in history that anarchists participated in a government, was reported by the anarchist press without explanation. A great opportunity was presented to the \textsc{poum} to win the \textsc{cnt} workers to struggle for a workers’ state, but the \textsc{poum} made no issue of the Basque government---for the \textsc{poum} acted identically in Valencia.

The Popular Executive, with bourgeois participation, was constituted in Valencia as a regional government, and here the \textsc{poum} entered too. In those days the \textsc{poum}’s central organ, \emph{La Batalla}, was calling for an all-workers’ government in Madrid and Barcelona: the contradiction between this slogan and the Valencia step was passed by without comment.

Formed within two days of the uprising as a military centre, the Central Committee of the Catalan Militias began to undertake collaboration with the bourgeoisie in economic activities as well. Transformation of the Central Committee into a body of democratically elected delegates from the factories and militia columns would have given it more power and authority and, at the same time, would have reduced the role of the bourgeoisie to its actual strength in the militias and factories. This was the only way out of the dilemma. But the \textsc{cnt} was blind to the problem, and the \textsc{poum} kept silent.

Finally, on August 11, the Council of Economy was formed on the initiative of Companys to centralize economic activity. Here it was, despite the bait of a radical economic programme, an undisguised question of socio-economic collaboration under the hegemony of the bourgeoisie. But the \textsc{cnt} and \textsc{poum} entered it.

Thus, in every sphere, the bourgeoisie edged its way back. Thus, the workers were carried, step by step, toward governmental coalition with the bourgeoisie.

To understand this process clearly, we must now examine more closely the political conceptions of the workers’ organizations.
  \chapter{Politics of the Spanish Working Class}

\section{The Right-Wing Socialists}

\lettrineP{rieto, Negrin, \&\ Besteiro} clung consistently to the theory that Spain had before it a considerable period of capitalist development. Besteiro and others had disgraced themselves by denouncing the workers’ resort to arms in October 1934. But Prieto, Negrin and their main associates had comported themselves at least as well as Caballero in the Asturian fighting and the general strike without, however, changing their political perspective. They had carried the party, in spite of left-wing opposition, into the electoral coalition of February 1936. The left socialists had, however, prevented Prieto’s entry into the cabinet. Prieto had clearly indicated that, were the left wing successful in gaining control of the party, he was prepared to fuse with Azaña’s Republican Left. He had gone so far, in the months before the civil war, as to join with Azaña in denouncing the strike epidemic. In their political conceptions the right-wing socialists were, indeed, simply petty-bourgeois republicans who, in the struggle against the monarchy, had correctly estimated that mass support could be had only by socialist colouration. In the republican arena opened in 1931, the first test had revealed them to be blood-brothers of Azaña.

Himself a Basque industrialist of considerable wealth, Prieto’s \emph{El Liberal} of Bilbao was one of the most influential organs among the bourgeoisie. Decades of class collaboration had given him the full confidence of the Basque bourgeoisie. More than any other figure, Prieto provided the link which united the Catholic, narrow-minded Basque capitalists, Azaña’s cosmopolitan and cynical intellectuals, and the Stalinist forces. Callous, ruthless, able, Prieto had none of the subjective fears of the Scandinavian and British Labour party leaders. He recognized the full significance of the policy embarked on by Stalin when the civil war began, and thereafter greeted the Stalinist spokesmen as ideological brothers.

\section{The Stalinists}

The political programme of Stalinism in 1936 appeared very different from its ultra-left denunciations of Azaña, Prieto, Caballero, the anarchists, as ``fascists'' and ``social fascists'' in 1931. Yet the basic policy remained the same. In 1936, as in 1931, the Stalinists wanted no proletarian revolution in Spain.

Walter Duranty, unofficial apologist for the Kremlin, in 1931 described its attitude:

\begin{quotation}
  The first Soviet comment on the events in Spain appears in the leading editorial in \emph{Pravda} today, but the organ of the Russian Communist Party seems none too jubilant over the prospects of the revolutionary struggle which it clearly expects will follow Alfonso’s downfall\ldots
  
  The unexpectedly pessimistic tone of \emph{Pravda}\ldots\ Perhaps is to be explained by Soviet anxiety lest the events in Spain disturb European peace during the present critical period of the five-year plan. Rightly or wrongly, it is believed here that the peace of Europe hangs literally on a thread, that the accumulation of armaments and national hatreds are much greater than before the war and make the present situation no less dangerous than the Spring of 1914, and that Spanish fireworks might easily provoke a general conflagration. (\emph{New York Times}, May 17, 1931.)
  
  Paradoxically enough, it appears that Moscow is not overdelighted by this circumstance---in fact it may almost be said that if the Spanish revolution ``swings left'' as Moscow now expects, Moscow will be more embarrassed than pleased\ldots
  
  \ldots For, first, the Soviet Union is excessively and perhaps unduly nervous about a war danger and ``views with alarm'' any event anywhere that may upset the European status quo\ldots\ Secondly, the Kremlin’s policy today stands much more on the success of socialist construction in Russia than upon world revolution\ldots\ (\emph{New York Times}, May 18, 1931.)
\end{quotation}

In 1931, the Kremlin had secured its goal by a policy of non-col\-lab\-or\-ation with the rest of the proletarian parties. The communists were thus isolated from the mass movement by union-splitting, no united front of organizations, attacks on other working-class meetings, etc. In 1931, the Kremlin had wanted nothing but maintenance of the status quo in Europe. In 1936, however, the Comintern adopted a new perspective, embodied in the Seventh Congress. The new course was to maintain the status quo as long as possible, this time not merely by preventing revolutions, but by active class collaboration with the bourgeoisie in the ``democratic'' countries. This collaboration was designed, in the event of war breaking out, to provide Russia with England. and France as its allies. The price Russia was offering to pay for an alliance with Anglo-French imperialism was the subordination of the proletariat to the bourgeosie. ``Socialism in a single country'' had revealed its full meaning as ``no socialism anywhere else.''

Lenin and the Bolsheviks were realists enough to allow the Soviet state to utilize conflicts between various capitalist powers even to the extent of using one against another in the event of war. Even more fundamental to their revolutionary politics, however, was the doctrine that, whatever the Soviet’s military alliances, the proletariat in every country has the unalterable duty to oppose its ``own'' bourgeoisie in war, to overthrow it in the course of the war, and to replace it with a workers’ revolutionary government which is the only possible real ally of the Soviet Union.

This fundamental tenet of Marxism was repudiated by the Seventh Congress of the Comintern. The French Communist party was already openly proclaiming its readiness to support its bourgeoisie in the coming war. Despite this, England’s coolness had largely negated the Franco-Soviet pact. Even under Blum the pact had not yet led to conferences between the two general staffs. The Spanish Civil War provided the Kremlin with an opportunity to prove once and for all to both French and English imperialists that, not only did the Kremlin intend to encourage no revolution, it was prepared to take the lead in crushing one that had nevertheless started.

Apparently not even all of the foreign Stalinist correspondents in Barcelona realized, in the first days of the civil war, that the Comintern had actually set itself the task of unravelling this all but completed revolution. On July 22, the London \emph{Daily Worker} carried a leading article:

\begin{quotation}
  In Spain, Socialists and Communists fought shoulder to shoulder in armed battle to defend their trade unions and political organizations, to guard the Spanish Republic and to defend democratic liberties so that they could advance towards a Spanish Soviet Republic.
\end{quotation}

And the same day, its Barcelona representative, Frank Pitcairn, wired: ``Red Militia Crushes Fascists, Triumph in Barcelona.''

\begin{quotation}
  The united working-class forces have already gained the upper hand. Streets here are being patrolled by cars filled with armed workers who are preserving order and discipline. Preparations are going forward for the organization of a permanent workers’ militia.
\end{quotation}

The Spanish Stalinists, however, joined Prieto and Azaña in appeals to the workers not to seize property. The Stalinists were the first to submit their press to the censorship. They were the first to demand liquidation of the workers’ militias, and the first to hand their militiamen over to Azaña’s officers. The civil war was not two months old when they began---what the government did not dare until nearly a year later---a murderous campaign against the \textsc{poum} and the Anarchist Youth. The Stalinists demanded subordination to the bourgeoisie, not merely for the period of the civil war, but afterward as well.

\begin{quotation}
  It is absolutely false,
\end{quotation}
declared Jesus Hernandez, editor of \emph{Mundo Obrero} (August 6, 1936),

\begin{quotation}
  \noindent
  that the present workers’ movement has for its object the establishment of a proletarian dictatorship after the war has terminated. It cannot be said we have a social motive for our participation in the war. We communists are the first to repudiate this supposition. We are motivated exclusively by a desire to defend the democratic republic.
\end{quotation}

\emph{L’Humanité}, organ of the French Communist Party, early in August published the following statement:

\begin{quotation}
  The Central Committee of the Communist Party of Spain requests us to inform the public, in reply to the fantastic and tendentious reports published by certain newspapers that the Spanish people are not striving for the establishment of the dictatorship of the proletariat, but know only one aim: the defence of the republican order, while respecting property.
\end{quotation}

As the months passed, the Stalinists adopted an even firmer stand against anything but a capitalist system. Josh Diaz, ``beloved leader'' of the Spanish party, at a plenary session of the Central Committee, March 5, 1937, declared:

\begin{quotation}
  If in the beginning the various premature attempts at ‘socialization’ and ‘collectivization,’ which were the result of an unclear understanding of the character of the present struggle, might have been justified by the fact that the big landlords and manufacturers had deserted their estates and factories and that it was necessary at all costs to continue production, now on the contrary they cannot be justified at all. At the present time, when there is a government of the Popular Front, in which all the forces engaged in the fight against fascism are represented, such things are not only not desirable, but absolutely impermissible. (\emph{Communist International}, May 1937)
\end{quotation}

Recognizing that the danger of a proletarian revolution came first of all from Catalonia, the Stalinists concentrated enormous resources in Barcelona. Having practically no organization of their own there, they recruited into their service the conservative labour leaders and petty-bourgeois politicians, by way of a fusion of the Communist Party of Catalonia with the Catalan section of the Socialist Party, the {Socialist Union} (a nationalist organization limited to Catalonia), and {Catala Proletari}, a split-off from the bourgeois Esquerra. The fusion, the {Unified Socialist Party of Catalonia} ({\textsc{psuc}}), affiliated to the Comintern. It had only a few thousand members at the beginning of the civil war but unlimited funds and hordes of Comintern functionaries. It took over the moribund Catalan section of the \textsc{ugt} and, when the Generalidad decreed compulsory syndicalization of all employees, recruited the most backward workers and clerks, who preferred this respectable institution to the radical \textsc{cnt}. But the biggest mass base of the Stalinists in Catalonia was a federation of traders, small businessmen, and manufacturers, the {Federaciones de Gremios y Entiadades de Pequefios Comerciantes e Industriales} ({\textsc{gepci}}), which in July was dubbed a union and affiliated to the Catalan \textsc{ugt}. The so-called Catalan section operated in complete independence from the Caballero-controlled National Executive of the \textsc{ugt}. Thereupon, as the chief and most vigorous defenders of the bourgeoisie, the \textsc{psuc} recruited heavily from the Catalan Esquerra.

The Stalinists followed a similar course in the rest of Spain. From the first, the {\textsc{cnt} Agricultural Union} and the {\textsc{ugt} Peasant and Land Workers’ Federation}---both supporting collectivization of the land---accused the Stalinists of organizing separate ``unions'' of the richer peasants who were opposed to the collectives. The Stalinist party grew more rapidly than any other organization, for it opened its doors wide. Dubious bourgeois elements flocked to it for protection. As early as August 19 and 20, 1936, Caballero’s organ, \emph{Claridad}, accused the Stalinist `Alliance of Anti-Fascist Writers' of harbouring reactionaries.\endnote{H.N. Brailsford, the British Socialist and People’s Frontist, says: The Communist Party ``is no longer primarily a party of the industrial workers or even a Marxist party'' and ``this development must be permanent. This prediction I base on the social composition of the C.P. both in Catalonia and in Spain.'' (\emph{New Republic}, June 9, 1937)}

When, after three long months of boycott, in the third week of October the first Soviet planes and guns finally arrived, the Communist party---which up to then had been on the defensive, unable to counter the sharp criticism of the \textsc{poum} on Stalin’s refusal to send arms---received a terrific impetus. Thenceforward its proposals were inextricably linked with the threat that Stalin would send no more planes and arms. Ambassador Rosenberg, in Madrid and Valencia, and Consul-General Antonov-Ovseyenko in Barcelona, made political speeches plainly indicating their preferences. When, at the November celebration of the anniversary of the Russian Revolution in Barcelona (at a parade participated in by all the bourgeois parties!), Ovseyenko ended his speech with ``Long live the Catalan people and its hero, President Companys,'' the workers were left in no doubt on which class the Kremlin was placing its approval.\endnote{One extraordinary incident deserves reporting. On November 27, 1936, \emph{La Batalla} was able to demonstrate that the \textsc{cnt}, \textsc{ugt}, Socialist Party, Republican Left---all favoured \textsc{poum} representation in the Madrid Defence Junta; yet the \textsc{poum} was not represented. How was it possible for Stalinist opposition alone to prevent the \textsc{poum}, with its militia columns on all fronts, from being represented? Could the Stalinists alone wield a veto? The answer was that the Soviet Embassy had intervened. ``It is intolerable that, on account of the aid they furnish us, they should attempt to impose upon us definite political norms, definite vetoes, to intervene and even to direct our politics,'' complained \emph{La Batalla}. The Madrid Defence Council incident, Ovseyenko’s November speech, Rosenberg’s addresses, were the public incidents which aroused the \textsc{poum}; through their cabinet post in the Generalidad they knew of ever more serious incidents to which they could not refer while in the government.

Consul-General Ovseyenko’s note to the press, answering the \textsc{poum}, probably has no parallel in all previous diplomatic history. It read like an editorial in \emph{Mundo Obrero}, denouncing the ``fascist manoeuvres'' of the \textsc{poum}, as an ``enemy of the Soviet Union.'' But before the year was up, Ovseyenko went further. On December 7, the \textsc{poum} called upon the Generalidad to offer a place of asylum to Leon Trotsky. Before the Generalidad could answer, the Soviet Consul-General declared to the press (\emph{La Prensa} reported it here) that if Trotsky were permitted to enter Catalonia, the Soviet government would cut off all aid to Spain. Truly, bureaucratic despotism could go no further!}

We have only sketched the Stalinist policy sufficiently to place it in the picture. We shall see it grow more openly, ruthlessly counter-revolutionary, in the ensuing year.

\section{Caballero, the Left Socialists, and the \textsc{ugt}}

Largo Caballero belonged to the same generation as Prieto. Both had reached middle age under the monarchy and modelled themselves upon the German Social Democrats of the right wing. As head of the \textsc{ugt}, Caballero had silently accepted the suppression of the Anarchist-led \textsc{cnt} by Primo de Rivera. More, he had sanctioned it by accepting a state councillorship from the dictator. He had joined the 1931–1933 coalition cabinet as Minister of Labour and had sponsored a law continuing Rivera’s mixed arbitration boards to settle strikes. ``We shall introduce compulsory arbitration. Those workers’ organizations which do not submit to it will be declared outside the law,'' he declared on July 23, 1931. Under his ministry, it was unlawful to strike for political demands or without ten days’ written notice to the employer. No trade union or any other labour meeting could be held without police witnesses present. Side by side with Prieto, Caballero had defended the repressions of the land-hungry peasants, the thousands of political arrests.

After the collapse of the 1931–33 coalition, a strong left wing developed, first in the Socialist Youth, demanding a re-orientation of the party. In 1934, Caballero unexpectedly declared for it. He had, said his friends, read Marx and Lenin for the first time after being ousted from the government. Nevertheless, Caballero’s group made no serious preparations for the October 1934 uprising. In Madrid, their chief stronghold, the rising never went beyond a general strike. On trial for inciting the insurrection---he was acquitted---Caballero denied the charge.

On record against coalitions, and for proletarian revolution, Caballero nevertheless agreed to the electoral coalition of February 1936 and supported Azaña’s cabinet in the Cortes on all basic questions. Caballero’s position, in effect, was that he would not repeat his rble as Minster of Labour in the 1931–33 coalition but that he would support Azaña from outside the cabinet, thereby being free to criticize. Ths was scarcely revolutionary irreconcilability. It was merely a form of critical loyalty, offering no threat to the bourgeois regime. During the February–July (1936) strike wave, Caballero incurred sharp criticism, both from the \textsc{cnt} and his own ranks, for discouraging strikes. An ardent advocate of fusion of the socialist and communist parties, he was mainly responsible for the fusion of the socialist and Stalinist youth. He had recouped his standing with the left wing of the party, however, by leading the fight to prevent Prieto from accepting the premiership. In the ensuing struggle, Prieto’s Executive had outlawed \emph{Claridad} (Caballero’s paper), reorganized pro-Caballero party districts, and indefinitely postponed the party convention. A split would have come, but the civil war intervened and, for the sake of presenting a picture of harmony, the Caballero forces had conceded to Prieto the national centre of the party.

At the height of the workers’ movement during the first weeks of the civil war, Caballero came into sharp collision with the Azaña-Prieto-Stalinist bloc. So long as discipline in barracks, management of feeding, lodging and payrolls, were in the hands of the workers’ organizations, and the militias freely organized discussions on political questions, the bourgeois-military caste could have no hope of securing real supremacy. Accordingly, the government, as a first feeler, called for enlistment of ten thousand reserve soldiers as a separate force under direct government control. The Stalinists defended the proposal. ``Some comrades have wished to see in the creation of the new voluntary army something like a menace to the role of the militias,'' said \emph{Mundo Obrero}, August 21. The Stalinists denied the very possibility and ended: ``Our slogan, today as yesterday, is the same for this. Everything for the People’s Front and everything through the People’s Front.''

This thoroughly reactionary position was effectively exposed by the \textsc{ugt} organ, \emph{Claridad}:

\begin{quotation}
  To think of another type of army to be substituted for those who are actually fighting and who in certain ways control their own revolutionary action, is to think in counter-revolutionary terms. That is what Lenin said (\emph{State and Revolution}): ``Every revolution, after the destruction of the state apparatus, shows us how the governing class tries to re-establish special bodies of armed men at `its' service, and how the oppressed class attempts to create a new organization of a type capable of serving not the exploiters but the exploited.''
\end{quotation}

Nevertheless, Caballero and the rest of the left-socialist leadership, in those critical early weeks, drew closer to Azaña, Prieto and the Stalinists. The dual power was proving a cumbersome and inadequate method of organizing the struggle against the fascist forces. Only two alternatives presented themselves inexorably: either join a coalition government, or replace the bourgeois power entirely by a workers’ regime.

Here, however, programmatic errors showed their terrible practical results. In April 1936 the leading group of the left socialists, the Madrid organization, had adopted a new programme, declaring for the dictatorship of the proletariat. What organizational form would it take? Luis Araquistain, Caballero’s theoretician, argued that Spain needed no soviets. The April programme had consequently embodied in it the conception that ``the organ of the proletarian dictatorship will be the Socialist party.'' But the left socialists had been prevented by Prieto’s postponement of the congress from assuming formal control of the party, and had desisted from further struggle for control when the civil war broke out. Furthermore, according to their programme, they would have to wait until the party included a majority of the proletariat. This programmatic failure to provide for united action through workers’ councils (soviets) in which socialists, communists, anarchists, \textsc{poum}ists, etc., would be gathered together with the deepest layers of the masses, this distorted notion of the lessons of the Russian Revolution, was a fatal error for the left socialists to make, and especially in Spain, with its anarchist traditions. They were saying precisely what the anarchist leaders had been accusing both communists and revolutionary socialists of meaning by the proletarian dictatorship.

The road to the proletarian dictatorship lay clearly before the proletariat. What was needed was to give the factory committees, militia committees, peasant committees, a democratic character, by having them elected by all of the workers in each unit; to bring together these elected delegates in village, city, regional councils, which in turn would send elected delegates to a national congress. True, the soviet form would not of itself solve the whole problem. A reformist majority in the executive committee would decline the assumption of state power. But the workers could still find in the soviets their natural organs of struggle until the genuinely revolutionary elements in the various parties banded together to win a revolutionary majority in the congress, and establish a workers’ state.

The road lay clearly before the proletariat but, not accidentally, the programme for that road was not the heritage of the left socialists. Caballero would criticize, grumble, excoriate, but he offered no alternative to the coalition with the bourgeoisie. Finally, he became its head.

\section[The \textsc{cnt-fai}]{\textsc{cnt-fai}: The National Confederation of Labour and the Anarchist Iberian Federation}

The followers of Bakunin had older roots in Spain than the Marxists. The \textsc{cnt} had been traditionally anarchist in leadership. The tide of the October Revolution had, for a short time, overtaken the \textsc{cnt}. It had sent a delegate to the Comintern Congress in 1921. The anarchists had then resorted to organized fraction work and recaptured it. Thenceforward, while continuing its traditional epithets against political parties, Spanish anarchism had in the \textsc{fai} a highly centralized party apparatus through which it maintained control of the \textsc{cnt}.

Ferociously persecuted by Alfonso and Primo de Rivera to the point where it actually dissolved for a time, the \textsc{cnt} from 1931 on, commanded an undisputed majority in the industrial centres of Catalonia and strong movements elsewhere. After the civil war began, it undoubtedly was larger than the \textsc{ugt} (some of whose biggest sections lay in fascist territory.)

Hitherto, in the history of the working class, anarchism had never been tested on a grand scale. Now, leading great masses, it was to have a definitive task.

Anarchism has consistently refused to recognize the distinction between a bourgeois and workers’ state. Even in the days of Lenin and Trotsky, anarchism denounced the Soviet Union as an exploiters’ regime. Precisely the failure to distinguish between a bourgeois and proletarian state had already led the \textsc{cnt}, in the honeymoon days of the revolution of 1931, to the same kind of opportunist errors as are always made by reformists---who also, in their way, make no distinction between bourgeois and workers’ states. Overcome by the ``fumes of the revolution,'' the \textsc{cnt} had benevolently greeted the bourgeois republic: ``Under a regime of liberty, the bloodless revolution is still more possible, still easier than under the monarchy.'' (\emph{Solidaridad Obrero}, April 23, 1931.) By October 1934 it swung to the equally false extreme of refusing to join with the republicans and socialists in the armed struggle against Gil Robles (with the honourable exception of the \textsc{cnt} regional organization in Asturias.)

Now, in the far more powerful fumes of the ``Revolution of July 19,'' when the accustomed boundary lines between bourgeoisie and proletariat were smeared over for the time being, the anarchists’ traditional refusal to distinguish between a bourgeois and workers’ state led them slowly, but decisively, into the ministry of a bourgeois state.

The false anarchist teachings on the nature of the state, it might seem, should logically have led them to refuse governmental participation in any event. Already running Catalan industry and the militias, however, the anarchists were in the intolerable position of objecting to the necessary administrative co-ordination and centralization of the work they had already begun. Their anti-statism \emph{as such} had to be thrown off. What \emph{did} remain, to wreak disaster in the end, was their failure to recognize the distinction between a workers’ and a bourgeois state.

Class collaboration, indeed, lies concealed in the heart of anarchist philosophy. It is hidden, during periods of reaction, by anarchist hatred of capitalist oppression. But, in a revolutionary period of dual power, it must come to the surface. For then the capitalist smilingly offers to share in building the new world. And the anarchist, being opposed to ``all dictatorships,'' including dictatorship of the proletariat, will require of the capitalist merely that he throw off the capitalist outlook, to which he agrees, naturally, the better to prepare the crushing of the workers.

There is a second fundamental tenet in anarchist teaching which led in the same direction. Since Bakunin, the anarchists had accused the Marxists of over-estimating the importance of state power, and had characterized this as merely the reflection of the petty-bourgeois intellectual’s pre-occupation with lucrative administrative posts. Anarchism calls upon the workers to turn their backs on the state and seek control of the factories as the real source of power. The ultimate sources of power (property relations) being secured, the state power will collapse, never to be replaced. The Spanish anarchists thus failed to understand that it was only the collapse of the state power, with the defection of the army to Franco, which had enabled them to seize the factories and that, if Companys and his allies were allowed the opportunity to reconstruct the bourgeois state, they would soon enough take the factories away from the workers. Intoxicated with their control of the factories and the militias, the anarchists assumed that capitalism had \emph{already} disappeared in Catalonia. They talked of the ``new social economy,'' and Companys was only too willing to talk as they did, for it blinded them and not him.

\section{The \textsc{poum}}

Here was a rare opportunity for even a small revolutionary party. Soviets cannot be built at will. They can be organized only in a period of dual power, of revolutionary upheaval. But in the period which calls for them, a revolutionary party can further their creation, in spite of the opposition of the most powerful reformist parties. In Russia, the Mensheviks and Social Revolutionaries, particularly after July, sought to siphon off the strength of the soviets into the government, sought to discourage their functioning or the creation of new ones, without success despite the fact that these reformists still wielded a majority in the soviets. In Germany, the social-democratic leadership sought, even more determinedly since they had the Russian lessons fresh before them, to prevent the creation of the workers’ and soldiers’ councils. In Spain, the direct hostility of the Stalinists and Prieto, the ``theoretical'' opposition of Caballero and the anarchists, would have been of no avail, for the basic units of the soviets were already there, in the factory, militia and peasant committees, and needed only democratization and bringing together in the localities. A single example, in \textsc{poum} controlled industrial towns like Lerida or Gerona, of delegates elected in every factory and shop, joining with delegates from the workers’ patrols and the militias to create a workers’ parliament which would function as the ruling body of the area, would have electrified Catalonia and initiated an identical process everywhere.

The \textsc{poum} was the one organization which seemed suited to undertake the task of creating the soviets. Its leaders had been the founders of the communist movement in Spain. It had, however, basic weaknesses. Its majority came from the {Workers’ and Peasants’ Bloc of Maurin}, whose cadre had collaborated with Stalin in the 1924–28 period in sending the Communist Party of China into the bourgeois Kuomintang \emph{bloc of four classes}; in creating farmer-labour and ``two-class'' parties ``of workers and farmers'' (a fancy name for a bloc with reformists and the liberal bourgeoisie)---and, in a word, in the whole opportunist course of those disastrous years. Maurin and his followers had broken with the Comintern not on these basic questions but on other issues---the Catalonian national question, etc.---when the Comintern had turned to dual unionism, ‘social fascism,’ etc., in 1929. Moreover, the fusion of the Maurinists with the former {Communist Left} (Trotskyists), led by Andres Nin and Juan Andrade---whose previous failure to sharply differentiate themselves from the Maurinist ideology had been the subject of years of controversy with the International Left Opposition---was an unprincipled amalgamation, in which the Communist Left elements had adopted a joint programme which was simply Maurin’s old conceptions, of which Trotsky had said as early as June 1931:

\begin{quotation}
  All that I have written in my latest work, \emph{The Spanish Revolution in Danger}, against the official policy of the Comintern in the Spanish question, applies entirely to the Catalonian Federation (Workers and Peasants bloc)\ldots\ it represents a pure ``Kuomintangism'' transported to Spanish soil. The ideas and the methods against which the Opposition fought implacably when it was a question of the Chinese policy of the Kuomintang, find their most disastrous expression in the Maurin programme\ldots\ A false point of departure during a revolution is inevitably translated in the course of events into the language of defeat. (\emph{The Militant}, August 1, 1931)
\end{quotation}

The first fruits of the fusion had scarcely been reassuring. After months of campaigning against a coalition with the bourgeoisie, the \textsc{poum} had overnight entered the electoral coalition of February 1936. It renounced the coalition after the elections, but on the very eve of the civil war (\emph{La Batalla}, July 17) called for ``an authentic Government of the Popular Front, with the direct [ministerial] participation of the Socialist and Communist Parties'' as a means to ``complete the democratic experience of the masses'' and hasten the revolution-an absolutely false slogan, having nothing in common with the Bolshevik method of demonstrating the necessity of the workers’ state and the impossibility of reforming the bourgeois state by forcing the reformists to assume governmental power \emph{without} the bourgeois ministers.

Nevertheless, many had the hope that the \textsc{poum} would take the lead in organizing the soviets. Nin now stood at the head of the party. He had been in Russia during the early years of the Russian Revolution, a leader of the {Red International of Labour Unions}. Would he not resist the provincialism of the Maurinist cadres? The \textsc{poum} workers, better trained politically than the anarchists, played a great role, entirely out of proportion to their numbers in the first revolutionary weeks, in seizing the land and factories. From a party of about 8,000 on the eve of civil war, the \textsc{poum} grew quickly, though remaining primarily a Catalonian organization. In the first months it quadrupled its numbers. Even more quickly its influence grew, as evidenced by the fact that it recruited over ten thousand militiamen under its banner.

The rising tide of coalitionism, however, engulfed the \textsc{poum}. The theoretical premises for it were already there, in the Maurinist ideology, to which Nin had signed his name in the fusion. The \textsc{poum} leadership clung to the \textsc{cnt}. Instead of boldly contending with the anarcho-reformists for the leadership of the masses, Nin sought illusory strength by identifying himself with them. The \textsc{poum} sent its militants into the smaller and heterogeneous Catalan \textsc{ugt} instead of contending for leadership of the millions in the \textsc{cnt}. It organized \textsc{poum} militia columns, circumscribing its influence, instead of sending its forces into the enormous \textsc{cnt} columns where the decisive sections of the proletariat were already gathered. \emph{La Batalla} recorded the tendency of \textsc{cnt} unions to treat collectivized property as their own. It never attacked the anarcho-syndicalist theories which created the tendency. In the ensuing year, it never once made a principled attack on the anarcho-reformist leadership, not even when the anarchists acquiesced in the expulsion of the \textsc{poum} from the Generalidad. Far from leading to united action with the \textsc{cnt}, this false course permitted the \textsc{cnt-fai} leadership, with perfect impunity, to turn its back on the \textsc{poum}.

More than once, in the days of Marx and Engels, and in the first revolutionary years of the Comintern, a weak national leadership had been corrected by its international collaborators. But the \textsc{poum}’s international connections stood to the right of the Spanish party. The {International Committee of Revolutionary Socialist Unity}---chiefly the \textsc{ilp} of England and the \textsc{sap} of Germany---issued a manifesto to the Spanish proletariat on August 17, 1936, which did not contain a single word of criticism of the Popular Front. The \textsc{sap} was shortly to go over to Popular Frontism itself, while the \textsc{ilp} embraced the Communist party in a ``Unity Campaign.'' Such were the ideological brethren for whom Nin and Andrade had renounced ``Trotskyism,'' the movement for the Fourth International. True enough, the Fourth Internationalists were small organizations compared to the reformist parties of Europe. But they offered the \textsc{poum} the rarest and most precious form of aid: a consistent Marxist analysis of the Spanish events and a revolutionary programme to defeat fascism. Nin was more ``practical,'' and abdicated the opportunity to lead the Spanish revolution.
  \chapter{Program of the Caballero Coalition Government}

\lettrineI{s it necessary}, at this late date, to explain that the cabinet of three Caballero men, three Prieto men, two Stalinists, and five bourgeois ministers, which was established on September 4, 1936, was a bourgeois government, a typical cabinet of class collaboration?

Apparently it is still necessary, for as late as May 9, 1937, a resolution of the National Executive Committee of the Socialist Party, USA,\endnote{This likely refers to the Socialist Party of America since the \textsc{spusa} was founded in 1973. ---\JCW} characterized this regime as ``a provisional revolutionary government.''

In turning over the premiership, Giral said:

\begin{quotation}
  I remain as a cabinet minister in order to demonstrate that the new government is an amplification of the old from the moment in which the president of the resigning government continues forming part of the new.
\end{quotation}

Caballero succinctly enough summarized his government’s programme to the Cortes:

\begin{quotation}
  This government was constituted, all those forming it previously renouncing the defence of our principles and particular tendencies, in order to remain united on one sole aspiration: to defend Spain in her struggle against fascism. (\emph{Claridad}, October 1, 1936.)
\end{quotation}

Certainly, Caballero had renounced his principles, but not the bourgeoisie and the Stalinists. For the common ground on which they joined with Caballero to form the government was the continuation of the old bourgeois order.

The programmatic declaration of the new cabinet had nothing in it which the previous government could not have signed. Point II is its essence:

\begin{quotation}
  The ministerial programme signifies essentially, the firm decision to assure triumph over the rebellion, co-ordinating the forces of the people, through the required unity of action. To that is subordinated every other political interest, putting to a side the ideological differences, since at present there can be no other task than that of assuring the smashing of the insurrection. (\emph{Claridad}, September 5, 1936.)
\end{quotation}

Not one word about the land! Not one word about the factory committees! And, as ``the representatives of the people,'' these ``democrats'' convened the former Cortes elected on February 16 by the electoral agreement which had given a majority to the bourgeoisie on the joint slate!

A few weeks before assuming the premiership, Caballero had been inveighing (through \emph{Claridad}\kp) against separating the revolution from the war. He had protested against the displacement of the militias. Now he became the leader in reconstructing the bourgeois state. What had happened?

We need not speculate on what went on his mind. The observable change, reflected in \emph{Claridad,} was that instead of depending on the working class of Spain and on international working class aid, Caballero now put his hopes on winning the aid of the ``great democracies,'' Anglo-French imperialism.

On September 2, in an interview with the Havas Agency, Prieto had declared himself ``pleased that the French government had taken the initiative in the proposals for nonintervention,'' although ``it had not had the full value that France wishes to give it.''
``Each day is more urgent for France to work with great energy to avoid dangers for all.''

\begin{quotation}
  Why does the \textsc{cnt} act as if we were finding ourselves before a completed revolution?
\end{quotation}
complained \emph{El Socialista}:

\begin{quotation}
  Our geographic law is not that of immense Russia, by no means. And we have to take into account the attitude of the states that surround us, in order to determine our own attitude. Let not everything rest on spiritual force nor on reason, but on knowing how to renounce four in order to gain a hundred. We still hope that the estimate of the Spanish events made by certain democracies will be changed, and it would be a pity, a tragedy, to compromise these possibilities by pushing the velocity of the revolution, which at present does not conduct us to any positive solution. (\emph{El Socialista}, October 5, 1936.)
\end{quotation}

The classical social democrats of the Prieto school could thus say, quite plainly, what the ``Spanish Lenin,'' Caballero, and the ex-Le\-ni\-nists, the Stalintern, had to obscure: they were currying favour with Anglo-French imperialism by strangling the revolution. As late as August 24, Caballero had hoped that Hitler’s intransigence would block the formation of the non-intervention committee. But with Hitler’s embargo on arms shipments on that date, and the Soviet’s declaration of adherence, it was clear that the Spanish blockade would be of long duration. The question was sharply posed: either fight the non-intervention blockade and denounce Blum and the Soviet Union for backing it, or accept the Stalinist perspective of gradually winning away France and England from the blockade by demonstrating the bourgeois respectability and stability of the Spanish Government. In other words, either accept the perspective of proletarian revolution and the necessity of arousing the international proletariat to aid Spain and spread the revolution to France, or accept class collaboration in Spain and abroad. When the alternatives became inescapable, Caballero chose the latter. Within a few days, his comrade, Alvarez del Vayo, was off to grovel at the feet of the imperialists in the League of Nations.

Caballero understood quite well that to arouse the Spanish masses to supreme efforts, it was necessary to offer them a programme of social reconstruction. A circular order to the political commissioners at the front from Caballero’s War Ministry emphasizes:

\begin{quotation}
  It is necessary to convince the fighters who are defending the republican regime with their lives that at the termination of the war the organization of the state will undergo a profound modification. From the present we shall go on to a structure, socially, economically, and juridically, all for the benefit of the working masses. We should try to imbue such conceptions in the spirit of the troops by means of simple and plain examples. (\emph{Gaceta de la Republica}, October 17, 1936.)
\end{quotation}

But the masses, Caballero presumably hoped, could be inspired with words, while the hard-headed imperialists of England and France would be content only with deeds.

\emph{To rouse the peasantry to struggle}, to provide their best sons for the war, not as sullen, demoralized conscripts but as lionhearted soldiers, to raise the food and fibres necessary to feed and clothe the army and the rear---that could only be done by giving the land to the peasantry, land to those who till it, the land as a national possession given in usufruct to the working farmer. Propaganda for liberty, etc., is absurdly insufficient. These are not your American or French farmers, who have already some land, enough to live on without being hungry:

\begin{quotation}
  Misery is still appalling in Estremadura, Albacete, Andalusia, Caceres and Ciudad Real. It is by no means a figure of speech when it is said that peasants are dying of hunger. There are villages in Hurdes, in La Mancha where the peasants, in absolute despair, revolt no longer. They eat roots and fruit. The events in Yeste [land seizures] are dramas of hunger. In Navas de Estena, about thirty miles from Madrid, forks and beds are unknown. The villagers’ chief diet consists of a soup made from bread, water, oil and vinegar.
\end{quotation}

These words are not those of a Trotskyist agitator but the involuntary testimony of a Stalinist functionary (\emph{Inprecorr}, August 1, 1936). How can one seriously hope to rouse these depths, except by the one act which can convince them of a new era: give them the land. Can one expect them to ``defend the republic''---that republic of Azaña---that had shot them like dogs for seizing land or stored grain?

Now the peasants and agricultural workers had seized land---not everywhere yet---but still had no assurance that the government was not permitting it merely as a provisional measure for the war which it would attempt to annul afterward. What the peasants wanted was a general decree nationalizing the land throughout Spain, giving it in usufruct to those who till it, so that no usurer can ever take it away from them. Equally, the tillers of the soil wanted the power to assure their land tenure, and that could only be a government of their flesh and blood workers’ and peasants’ regime.

Does it require much insight to see what effect such a land decree would have on the fascist forces? Not only on the land-hungry peasants in the fascist areas but, above all, among the sons of the peasants who constituted the ranks of the fascist armies, deceived by their officers as to the causes of the conflict. A few airplane loads of leaflets spread on the fascist fronts, announcing the land decree, would be worth an army of a million men. No single other move of the Loyalist side could sow more demoralization and decomposition in the fascist forces.

But thirty years as a ``responsible leader'' had left its mark too deeply. The inner forces of the masses had been too long an object of concern and fear to Caballero, something which he had to curb and channelize into safe boundaries. The land decree of October 7, 1936, merely sanctioned division of estates belonging to known fascists; other wealthy landowners, peasant exploiters, etc., remained untouched. The aroused hopes of the peasantry were smothered.

The \textsc{ugt} workers in the factories, shops, and railroads were setting up their factory committees, taking over the plants. What would Caballero have to say to them? In Valencia and Madrid the government swiftly intervened, placing government directors in charge who confined the factory committees to routine activity. Not until February 23, 1937, was a comprehensive decree on the industries adopted (issued over the name of Juan Peiro, the anarchist Minister of Industry). It gave the workers no security for the future regime in industry; established strict intervention by the government. ``Workers’ control,'' by its terms, proved little more than a collective contract, such as, for example, operates in shops dealing with the {Amalgamated Clothing Workers’ Union} in America---that is, no real measure of workers’ control at all.

Caballero had denounced the Giral cabinet for building an army outside the workers’ militias and for re-building the old Civil Guard. (The great ``Caballero'' column on the Madrid front, in its uncensored paper, had called for direct resistance to Giral’s proposal.) Now Caballero put his prestige behind the Giral plans. The conscription decrees followed the traditional forms, gave no place to soldiers’ committees. That meant reviving the bourgeois army, with supreme power in the hands of a military caste.

\subsection*{Freedom for Morocco?}

Delegations of Arabs and Moors came to the government, pleading for a decree. The government would not budge. The redoubtable Abd-el-Krim, exiled by France, sent a plea to Caballero to intervene with Blum. so that he might be permitted to return to Morocco to lead an insurrection against Franco. Caballero would not ask, and Blum would not grant. To rouse Spanish Morocco might endanger imperialist domination throughout Africa.
\nowidow

Thus Caballero and his Stalinist allies set their faces as flint against revolutionary methods of struggle against fascism. In due time, at the end of October, came their reward: a modicum of army supplies from Stalin. In the ensuing months, came more supplies, particularly after great defeats: after the encirclement of Madrid, after the fall of Malaga, after the fall of Bilbao, supplies enough to save the Loyalists for the moment, but never enough to permit them to carry through a really sustained offensive which might lead to the total collapse of Franco.

What was the political logic behind this careful turning-off-and-on of the spigot of supplies? If it were a question of Soviet Russia’s limited resources that still does not explain, for example, why all the planes that were to go to Spain could not have been massed for a decisive struggle in one period. The explanation of the spigot is not technical but political. Enough was given to prevent early defeat of the Loyalists and the consequent collapse of Soviet prestige in the international working class. And this fitted in, at bottom, with Anglo-French policy, which did not desire an immediate Franco victory. But not enough was given desire facilitate a victorious conclusion from which might issue-once the spectre of Franco was gone---a Soviet Spain.

Such was the programme of the ``provisional revolutionary government'' of Caballero. Nothing was added or subtracted from it with the entry of the \textsc{cnt} ministers on November 4, 1936. By then the ``great democracies'' had had an opportunity, observing the \textsc{cnt} in the Catalonian government formed on September 26, to be reassured about the ``responsibility'' of these anarchists.

There was one troublesome point: the anarchist-controlled Council of Defence of Aragon, comprising the territory wrested from the fascists by the Catalonian militias on the Aragon front, had a fearful reputation as an arch-revolutionary body. The price of four cabinet seats for the \textsc{cnt} was some reassurance on Aragon. Accordingly, on October 31, the Aragon Council met with Caballero.
\nowidow

\begin{quotation}
	The object of our visit,
\end{quotation}
declared the Council’s president, Joaquim Ascaso,

\begin{quotation}
  \noindent
  has been to pay our respects to the head of the government and to assure him of our attachment to the government of the people. We are disposed to accept all the laws it passes and we, in our turn, ask the Minister for all the help we need. The Aragon Council is formed of elements from the Popular Front so that all the forces upholding the government are represented in it.
\end{quotation}

\separatorline

\begin{quotation}
  Interviews with President Azaña, with President Companys, and with Largo Caballero,
\end{quotation}
added a Generalidad statement of November 4,

\begin{quotation}
  \noindent
  have destroyed any suspicions that might have arisen that the government which has been constituted [in Aragon] was of an extremist character, unrelated to the other governmental organs of the republic and opposed to the government of Catalonia.
\end{quotation}
  
That day the anarchists took their seats in Caballero’s cabinet.
  \chapter{The Programme of the Catalan Coalition Government}

\lettrineO{n September 7, 1936}, in a speech criticizing the Madrid coalition with the bourgeoisie, Nin had raised the slogan, ``Down With the Bourgeois Ministers,'' and the crowd had gone wild with enthusiasm. But by September 18, \emph{La Batalla} published a resolution of the Central Committee of the \textsc{poum}, accepting coalitionism:

\begin{quotation}
  The Central Committee believes now, as always, that this government must be exclusively composed of representatives of the workers’ parties and trade union organizations. But, if this point of view is not shared by the other workers’ organizations, we are willing to leave the question open, the more especially as the [Catalonian] left republican movement is of a profoundly popular nature---which distinguishes it radically from the Spanish left republican movement---and the peasant masses and workers’ sections on which it is based are moving definitely toward the revolution, influenced by the proletarian parties and organizations.
  
  The important thing is the programme, and the hegemony of the proletariat, which must be guaranteed. On one point there can be no doubt: the new government must make a declaration of unquestionable principles, affirming its intention of turning the impulse of the masses into revolutionary legality, and directing it in the sense of the socialist revolution. As for proletarian hegemony, the absolute majority of workers’ representatives will make it fully certain.
\end{quotation}

The \emph{Esquerra} leadership, hardened bourgeois politicians of twenty--thirty years’ struggle against the proletariat, was thus transformed overnight by the \POUM\ into a movement ``of a profoundly popular nature.'' And to this piece of legerdemain the \POUM\ added the hitherto unknown principle of strategy, that the way to win the leftward-moving workers and peasants in the \emph{Esquerra} was by collaborating in a government with their bourgeois leaders!

\begin{quotation}
  The working class cannot simply lay hold of the ready-made state machinery and wield it for its own purposes,
\end{quotation}
declared Marx. This was the great lesson learned from the Paris Commune:

\begin{quotation}
  \noindent
  not, as in the past, to transfer the bureaucratic and military machinery from one hand to the other, \emph{but to break it up}; and that is the precondition of any real people’s revolution on the Continent. And this is what our heroic party comrades in Paris have attempted.
\end{quotation}

What is to replace the shattered state machinery? On this, the fundamental question of revolution, the meagre experience of the Commune was fully developed by Lenin and Trotsky. Parliamentarianism was to be destroyed. In its place rise the workers’ committees in the factories, the peasants’ committees on the land, the soldiers’ committees in the army, centralized in local, regional and, finally, the national soviets. Thus, the new state, a workers’ state, is based on industrial representation, which automatically disfranchises the bourgeoisie, except as, after the consolidation of workers’ power, they individually enter productive labour and are permitted to participate in electing the soviets. Between the old bourgeois state and the new workers’ state lies a chasm over which the bourgeoisie cannot return to power except by overthrowing the workers’ state.

It was this fundamental tenet, the essence of the accumulated experience of a century of revolutionary struggle, which the \POUM\ violated in entering the Generalidad.\endnote{Those who defended this violation---Lovestoneites, Norman Thomas socialists, \textsc{ilp} etc.---thereby indicate their own future conduct in the revolutionary crisis.} They received their ministry from the hands of President Companys. The new cabinet merely continued the work of the old, and like the old, could be dismissed and replaced by a more reactionary one. Behind the protective covering of the \POUM-\CNT-\PSUC-Esquerra cabinet, the bourgeoisie would weather the revolutionary offensive, gather its shattered forces, and, with the aid of the reformists, at the ripe moment, return to full power. To this end, it was not even necessary for the bourgeoisie to participate in the cabinet. There had been ``all-workers'' cabinets in Germany, Austria, England, which had thus enabled the bourgeoisie to weather critical situations, and then kick out the workers’ ministers.

The workers’ state, the dictatorship of the proletariat, cannot exist until the old bourgeois state is destroyed. It can only be brought into existence by the direct, \emph{political} intervention of the masses, through the factory and village councils (soviets) at that point where a majority in the soviets is wielded by the workers’ party or parties which are determined to overthrow the bourgeois state. Such was the basic theoretical contribution of Lenin. Precisely this conception, however, was bowdlerized by the \POUM. The same speech of Nin calling for the dismissal of the bourgeois ministers developed a conception which could only lead to preservation of the bourgeois state.

\begin{quotation}
  Dictatorship of the proletariat. Another conception which is an object of difference with the anarchists. The proletarian dictatorship means the authority exercised by the working class. In Catalonia we can affirm that the dictatorship of the proletariat already exists. (Applause)\dots\ Not many days ago the \textsc{fai} launched a manifesto which said that it would oppose all dictatorships exercised by whatever party. We are in agreement with them. The dictatorship of the proletariat cannot be exercised by one single sector of the proletariat but by all, absolutely all. No workers’ party or union centre has the right to exercise a dictatorship. Let those present know that if the \textsc{cnt} or the Communist Party or the Socialist Party would wish to exercise a dictatorship of a party it would confront us. The dictatorship of the proletariat must be exercised by all. (\emph{La Batalla}, September 8, 1936.)
\end{quotation}

\vspace{-0.25\baselineskip}

For the dictatorship of the proletariat, as a state form, resting on the broad foundations of the network of workers’, peasants’ and combatants’ councils throughout industry, the land and the fields of battle, Nin was here substituting an entirely different conception: an agreement among the top-leaderships of the workers’ organizations jointly to assume governmental responsibility. False, and having nothing whatever in common with the Marxist conception of proletarian dictatorship! How could the proletarian dictatorship be wielded jointly with the Stalinist-democrats and the social democrats who stood for bourgeois democracy? How could party agreements be the substitute for the necessary vast network of workers’ councils?

The Leninist prediction that every real revolution gives rise to organs of dual power had been confirmed on July 19: the militia committees, supply committees, workers’ patrols, etc., etc. Leninist strategy called for the centralization of these organs of dual power into a national centre, and the taking of state power through it. The dissolution of the organs of dual power, as in Germany 1919, was called by Lenin the ``liquidation of the revolution.''

Uneasy memories of this led the \POUM\ leaders, in announcing entry into the Generalidad, to end:

\vspace{-0.25\baselineskip}

\begin{quotation}
  We are in a transition state in which the force of events has obliged us to collaborate directly in the Council of the Generalidad, along with other workers’ organizations \dots\ From the committees of workers, peasants and soldiers, for the formation of which we are pressing, will spring the direct representation of the new proletarian power.
\end{quotation}

\vspace{-0.25\baselineskip}

But this was the last swan song of the committees of dual power. For one of the first steps taken by the new cabinet of the Generalidad was to \emph{dissolve all the revolutionary committees which arose on July 19.}

The Central Committee of the Militias was dissolved and its powers turned over to the Ministries of Defence and Internal Security. The local militia and anti-fascist committees, almost invariably proletarian in composition, which had been ruling the towns and villages, were dissolved and replaced by municipal administrations composed in the same proportion as the cabinet (Esquerra 3, \CNT\ 3, \PSUC\ 2, Peasants Union, \POUM, and Accio Catala, the right-wing bourgeois organization, 1 each). Then to make sure that no revolutionary organ had been overlooked, an additional decree was passed which deserves full quotation:

\begin{quote}
  \textsc{\textsf{Article 1.}}\quad There are dissolved in all Catalonia the local committees, whatever be the name or title they bear, as well as all those local organisms which may have risen to down the subversive movement, with cultural, economic or any other species of aims.
  
  \textsc{\textsf{Article 2.}}\quad Resistance to dissolving them will be considered as a fascist act and its instigators delivered to the Tribunals of Popular Justice.
  
  (October 9, 1936.)
\end{quote}

The dissolution of the committees marked the first great advance of the counter-revolution. It removed the nascent soviet danger and enabled the bourgeois state to begin retrieving in every sphere the power which had fallen from its hands on July 19. Completely disoriented, the \POUM\ made no attempt to harmonize its previous call for committees with its sanction for their dissolution two weeks later. On the other hand, there remained in the hands of the bourgeoisie its traditional lever, the parliament. For the \POUM\ did not even get, in return for participation in the government, a decree dissolving the parliament. On the contrary, the financial decrees of the new cabinet carried the usual article requiring an accounting to the Catalan parliament. Parliament is dead, the \POUM\ assured the workers, but the government it sat in did not say so. True, unlike Caballero, Companys dared not convene parliament for many months, but this legal instrument of bourgeois domination remained intact. The meeting of the parliamentary deputation on April 9, 1937, during a ministerial crisis, scared the \CNT\ back into the government. And after the May Days, having defeated the workers, Companys convened the parliament which the \POUM\ had sworn was dead!

One more important step for consolidating the power of the bourgeois state was carried out on October 27, 1936: a decree disarming the workers.

\begin{quote}
  \textsc{\textsf{Article 1.}}\quad All long arms [e.g.\ rifles, machine guns, etc.] to be found in the hands of citizens shall be delivered to the municipalities or recovered by them, in a period of eight days after publication of this decree. Such arms shall be deposited in the Artillery Headquarters and the Ministry of Defence in Barcelona, in order to take care of the needs of the front.
  
  \textsc{\textsf{Article 2.}}\quad At the end of the cited period those who retain such armament will be considered as fascists and judged with the rigour which their conduct deserves.
  
  (\emph{La Batalla}, October 28, 1936.)
\end{quote}

The \POUM\ and \CNT\ published this decree without a single word of explanation to their following!

Thus the salvation of the bourgeois state was achieved. The \POUM, having been utilized during the critical months, was kicked out of the government in a cabinet reorganization, December 12, 1936. The \CNT\ with its great following was utilized longer, particularly since it increasingly adapted itself to the domination of the bourgeoisie, and was, therefore, kicked out only in July of the next year. But the power which the \POUM\ and \CNT\ had enabled the government to arrogate to itself remained in the government’s hands.

\section{The Economic Programme of the Coalition}

Apart from the ``workers’ majority,'' the \POUM\ justified entry because of the ``socialist orientation'' of the government’s economic programme. This criterion was utterly false, for revolutionary Marxism has always made clear that the necessary precondition to socialist economics is the dictatorship of the proletariat.

The Bolsheviks in 1917 were even prepared, on the basis of a workers’ state, to permit the continued existence, for a period, of private industry in certain fields, modified by workers’ control of production. Precisely in those fields of economic life in which the Bolsheviks acted first, however, the Catalan coalition did not act: nationalization of the banks and of the land.

Finance capital, even in backward Spain as elsewhere, dominates all other forms of capital. Yet, all that the coalition agreed to, in Point~VIII of its economic programme, was: ``Workers' control of banking enterprises until arriving at the nationalization of banking.'' ``Workers' control'' in practice meant merely guarding against disbursements of funds to fascist sympathizers and unauthorized persons. ``Until'' put off nationalization of banking indefinitely---nothing was ever done about it. This vast lever meant, as the next months proved, that the collectivized industries were at the mercy of those who could withhold credits. Precisely through this means, the bourgeois state, month by month, was to whittle down the economic power of the working class.

The Bolsheviks had \emph{nationalized} the land and granted control of it to the local soviets: that meant the \emph{end of private property in land}. The peasant need not enter the collectives; he was, however, no longer able to buy and sell land, and no creditor could seize it.\endnote{Louis Fischer, with ignorance fortified by impudence, argues against the Spanish collectives that in Russia collectivization came many years after the revolution. He leaves out the little ``detail'' that Lenin’s first decree was the nationalization of the land and the end of private property in land.}

The ``radical'' Catalan programme, ``the collectivization of great rural properties and respect for small agrarian property,'' concealed a reactionary perspective: land could still be bought and sold. Even more important: according to the Catalan autonomy statute, the central government had the last word on economic questions involving all Spain, and it had only authorized seizure of \emph{fascist-owned estates}. The coalition ``ignored'' the discrepancy between the two decrees. The \POUM\ did not have sense enough to bring the discrepancy out into the open and force the central government to formally recognize the Catalan decree, or have the Generalidad declare its full autonomy in economic questions. That meant: once the bourgeoisie recovered its strength, the Madrid decree on the land would prevail.

On October 24, a long and intricate decree was promulgated, concretizing the government’s conception of ``collectivization of the great industries, public services and transport.'' Before entry into the government, the \POUM\ had criticized industrial ``collectivization,'' pointing out that the unions, and even the workers in individual factories, were treating them as their own property. ``Syndicalist capitalism'' was making of the factories merely a form of producers’ co-operatives, in which the workers divided the profits. But industry could be run efficiently only as a national entity, together with all banking facilities and a monopoly of foreign trade. Now the \POUM\ accepted ``collectivization,'' which was nothing more than producers’ co-operatives, though real planning was impossible without banking and trade monopolies. The ``control of foreign trade'' which was promised never materialized. The \POUM’s proposal to include in the decree an ``Industrial and Credit Bank of Catalonia to attend to the needs and requirements of collectivized industry,'' was rejected. Thus, the foundations were laid for cutting to pieces the industries seized by the workers.

Another deadly blow to the ``collectivized'' factories was the arrangement for compensation to their former owners. Contrary to popular thought, the question of compensation for confiscated property is not excluded in advance for revolutionary Marxists. If the bourgeoisie would not resist, Lenin offered to arrange for partial compensation. The \POUM\ correctly concluded that the Spanish bourgeoisie had either gone over to Franco already or were---those in the Loyalist area---in no position but to accept the ``opportunity to work, or if unable to work, social insurance, under the same conditions as other workers.'' (The question of compensation to foreign capitalists was not at issue, since all correctly agreed this had to be recognized; but under cover of this abstractly correct formula, the government was soon to ``compensate'' foreigners-by giving them back their factories! The rest of the coalition, including the anarchists, rejected the \POUM\ proposal. Nor did they arrange for definite norms of compensation. Nor did compensation---as it did in the case of foreign capital---rest on the government.

Instead, ``the inventoried credit balance of any firm'' would ``be placed at the credit of the beneficiary [former owner] `as a social compensation,''' and ``the compensation for Spanish owners shall be suspended for later determination.'' In plain English, this meant that compensation would be a charge on the collectivized enterprise, i.e., on the workers involved, and the amount of it was left for a later time; with the re-construction of the bourgeois power, the bourgeoisie would levy against the workers’ enterprises in favour of the former owners on the sole criterion of how far the bourgeoisie dared saddle the workers with enforced payments of interest on capital debt. If the government grew strong enough, the former owners would go on clipping their coupons and receiving their dividends, just as before. The \POUM\ correctly termed this question ``fundamental;'' but stayed in the coalition government, nevertheless.

The collectivization decree provided for intervention in each factory of a government agent as part of the Factory Council. In all enterprises employing over 500, its director had to be approved by the government. Once elected by the workers in the factory, the Factory Council remained for two years in office, except for outright dereliction of duty, thus ``freezing'' the political composition of the councils and making it impossible for a revolutionary party to win control of the factories. The General Councils, embracing a whole industry, were even less flexible, eight out of twelve members being appointed by the leaders of the \UGT\ and \CNT, and presided over by representatives of the government. These measures, ensuring no ``revolt from below,'' were approved by all, including the \POUM.

Is it not obvious that the economic programme of the Generalidad merely accepted some of the gains already made by the workers themselves, and combined them with a series of political and economic measures which would eventually wipe out those gains? Yet, for this and a seat in the cabinet, the \POUM\ sold its chance to lead the Spanish revolution. By its blanket acceptance of the governmental programme, the \CNT\ revealed the complete bankruptcy of anarchism as a road to the social revolution.\endnote{After the May Days, the Generalidad repudiated the legality of the decree collectivizing industry.}

\section{The International Policy of the Coalition}

Like their counterparts in Madrid, the Esquerra and \PSUC\ looked to the League of Nations and the ``great democracies'' for succour. Nor was the \CNT\ much better. Juan Peiró, after the fall of the Caballero government, naively declared that the \CNT\ had been assured that the moderate government programme was meant for foreign consumption only.\endnote{``\dots the International bourgeoisie refused to supply us with those requirements (arms). It was a tragic moment: we had to create the impression that the masters were not the revolutionary committees but rather the legal government; failing that, we should not have received anything at all \dots\ We must needs adapt ourselves to the inexorable circumstances of the moment, that is to say, accept governmental collaboration\dots'' (Garcia Oliver, ex-Minister of Justice, speech in Paris, text published by the anarchist \emph{Spain and the World}, July 2, 1937)

``Spain offers all liberal and democratic nations of the world the opportunity of undertaking a strong offensive against the fascist forces, and if this means war, they must accept it before it is too late. They must not wait until fascism has perfected its war machine.'' (\emph{Official English Edition}~107, Dec.~8, 1936, Generalidad Commissariat of Propaganda)

Federica Montseny (outstanding \CNT\ leader): ``I believe that a people of such great intelligence (England) will realize that the establishment of a fascist state to the south of France... would be directly against its interests. The fate of the world as well as the outcome of this war depends on England...'' (\emph{Ibid.}~108, Dec.~10, 1936)}

This undoubtedly explains why the \CNT\ sent no organized delegations abroad to campaign among the workers.

The \POUM\ too fell victim to this opportunist policy. Despite its abstractly correct understanding of the reactionary international role of the Soviet bureaucracy, and its criticism of the failure of Stalin to sell arms to Spain during the first three crucial months, the \POUM\ failed to understand the fact that the Soviet note of October 7, 1936---``if violation is not halted immediately, it will consider itself free from any obligation resulting from the agreement''---did not mean leaving the non-intervention committee, and in no sense guaranteed sufficient arms shipments to turn the tide.

\begin{quotation}
  There is no doubt that the recent step of the Soviet Government in breaking the non-intervention pact will be of extraordinary political consequence. It is probably the most important political event since the commencement of the civil war,
\end{quotation}
said \emph{La Batalla}. Even worse, the \POUM’s perspective was that French imperialism would send arms:

\begin{quotation}
	How will the French Government reply to this new situation? Will it keep its attitude of neutrality? This would mean utter unpopularity and discredit. Blum would fall from power in the midst of general condemnation \dots\ We do not believe that Lion Blum would commit such a colossal blunder. Seeing that the only obstacle in the way of the correction of his policy was the Soviet Government’s attitude, the change in the latter should determine a complete change in Blum’s policy. (\emph{La Batalla}, October 11, 1936.)
\end{quotation}

Here, as everywhere, the \POUM\ had lost its bearings. It is not accidental that during its ministerial months, it sent no delegations abroad to campaign among the advanced workers.
  \chapter[Revival of the Bourgeois State]{The Revival of \\ the Bourgeois State \\ \bigskip \textsc{\LARGE September 1936--April 1937}}

\section{Economic Counter-Revolution}

\lettrineT{he} eight months after the workers’ representatives entered the Madrid and Barcelona cabinets saw the proletarian conquests in the economic field slowly whittled down. Controlling the treasury and the banks, the government was able to force its will on the workers by the threat of withdrawing credits.

In Catalonia, the chief industrial center, the process moved more slowly but to the same end. Some fifty-eight financial decrees of the Generalidad in January sharply restricted the scope of activity of the collectivized factories. On February~3, for the first time, the Generalidad dared decree illegal the collectivization of an industry~-- dairy products. During the April ministerial crisis, the Generalidad annulled workers’ control over the customs by refusing to certify workers’ ownership of material that had been exported and was being tied up in foreign courts by suits of former owners; henceforth the factories and agricultural collectives exporting goods were at the mercy of the government.

Comorera, \PSUC\ chieftain, had taken over the Ministry of Supplies on December 15, when the \POUM\ was ousted from the cabinet. On January 7, he decreed dissolution of the workers’ supply committees which had been purchasing food from the peasants. Into this breach poured the speculators and traders of the \textsc{gepci} (Corporation and Units of Petty Merchants and Manufacturers)~-- holding \UGT\ cards!~-- and the resultant hoarding and rise of food prices led to widespread malnutrition. Each family received ration cards but supplies were not rationed according to the number of persons served by each depot. In the workers’ suburbs of Barcelona, long queues stood throughout the day, supplies often exhausted before the end of the queues was reached, while in the bourgeois districts there was plenty. The privately owned restaurants had ample supplies for those who could pay the price. Milk was unobtainable for workers’ children but purchasable in the restaurants. Though bread (at a fixed price) was often not to be had, cake (price uncontrolled) was always to be bought.

On the sixth anniversary of the republic (April 14, boycotted by \FAI\kn, \CNT\kn, and \POUM), the Esquerra and Stalinist demonstrations were overshadowed by women’s demonstrations against the food prices. Yet the Stalinists put to political use even their crimes. The masses were given to understand that \PSUC\ and \UGT\ membership would get them better rations. Anonymous stickers blamed the collectivized farms and transportation for the price rises.

Vicente Uribe, Stalinist Minister of~Agriculture, played the same role here as a Stalinist Minister of~Agriculture had played in the Wang Jingwei regime of 1927, in Wuhan, in fighting the peasants. Uribe’s department dismantled collectives, organized former landowners to whom their lands were returned as state ``co-admin\-istrat\-ors,'' prevented the collectives from selling their produce without the use of middlemen.

A national campaign for ``state control'' and ``municipalization'' of industry laid the basis for wresting all control from the factory committees. The economic counterrevolution proceeded, however, comparatively slowly. For the bourgeois--Stalinist bloc understood, as the anarchists did not, that the necessary precondition for destroying the workers’ economic conquests was the crushing of the workers’ militias and police, and the disarming of the workers in the rearguard. But force alone was insufficient to achieve this end. Force had to be combined with propaganda.

\section{Censorship}

To facilitate the success of their own propaganda, the bourgeois--reformist bloc resorted, through the government, to systematic curtailment of the \CNT-\FAI--\POUM\ press and radio. The \POUM\ was the chief victim. While it was still in the Generalidad, the Catalan \emph{Hoja Official} boycotted all mention of \POUM\ meetings and radio broadcasts. On February 26, the Generalidad forbade a \CNT--\POUM\ mass meeting in Tarragona. On March 5, \emph{La Batalla} was fined 5,000 pesetas and refused a bill of particulars on the charge of disobeying the military censor. On March 14, \emph{La Batalla} was suspended for four days, this time openly for a political editorial. At the same time the Generalidad refused to the \POUM\ use of the official radio station for broadcasts. The \POUM\ dailies in Lerida, Gerona, etc., were constantly harassed.

The deadliest blows to the \POUM\ in this period, however, were delivered outside Catalonia. The Stalinist-controlled Madrid Defense Junta in January permanently suspended \emph{\POUM,} a weekly. The same authority suspended and confiscated the presses of \emph{El Combatiente Rojo,} \POUM\ militia daily, on February 10, and shortly thereafter suspended the \POUM\ radio station, closing it permanently in April.

The Junta also refused to permit the \POUM\ Youth (Juventud Comunista Ib\'erica) to publish \emph{La Antorcha,} the official prohibition cynically stating that ``the \textsc{jci} needs no press.'' \emph{Juventud Roja,} Valencian \POUM\ Youth organ, was submitted to severe political censorship in March. The only \POUM\ organ untouched was \emph{El Comunista} of\kp\ Valencia, weekly organ of the ferociously anti-Trotskyist, half-Stalinist right wing.

Another important field of work among the masses was closed to the \POUM\ when the \POUM\ Red Aid was excluded at the demand of the \PSUC\ from the Permanent Committee of Aid for Madrid. The \CNT\kn, in the name of unity, agreed to this criminal act, which became national in scope in April, when the \POUM\ Red Aid was excluded from participating in Madrid Week.

This sketchy outline of governmental outlawry of \POUM\ activities \emph{before May} conclusively refutes the Stalinist claim that the \POUM\ was persecuted for its participation in the May events.

The censorship against the \POUM\ was carried out by cabinets in which sat \CNT\ ministers. Only the Anarchist Youth, Juventud Libertaria, publicly protested. But the \CNT\ press also was systematically harassed. Does history record another instance of cabinet ministers submitting to repression of their own press?

The \FAI\ daily, \emph{Nosotros} of Valencia, was suspended indefinitely on February 27 for an article attacking Caballero’s war policy. On March 26, the Basque Government suspended \emph{\CNT\ del Norte,} arrested the editorial staff and the \CNT\ Regional Committee~-- and gave the presses to the Basque Communist party. Various issues of \emph{\CNT} and \emph{Castilla Libre,} both of Madrid, were suppressed April 11--18. \emph{Nosotros} was again suspended on April 16.

Censorship and suspension were formal measures. At least equal\-ly efficacious were the ``informal'' measures whereby the \CNT-\FAI--\POUM\ newspaper packets ``failed'' to arrive at the front or arrived weeks late. Meanwhile enormous editions of the Stalinist and bourgeois press, untouched by the censor and always delivered, were distributed free to the \CNT\kn, \UGT\ and \POUM\ militias. The government radio stations were always at the service of the Nelkens and Pasionarias. Almost all the so-called political commissioners at the front were Stalinists and bourgeois. Thus deceit supplemented naked force.

\section{The Police}

In the first months after July 19, police duties were almost entirely in the hands of the workers’ patrols in Catalonia and the ``militias of the rearguard'' in Madrid and Valencia. But the opportunity permanently to dissolve the bourgeois police slipped by.

Under Caballero, the Civil Guard was rechristened the `National Republican Guard.' The remnants of this and the Assault Guards were gradually withdrawn from the front. Those who had gone over to Franco were more than replaced by new men.

The most extraordinary step in reviving the bourgeois police was the mushroom growth of the hitherto small customs force, the Carabineros, under Finance Minister Negr\'in, into a heavily armed praetorian guard of 40,000.\endnote{``A reliable police force is being built up quietly but surely. The Valencia government discovered an ideal instrument for this purpose in the Carabineros. These were formerly customs officers and guards, and always had a good reputation for loyalty. It is reported on good authority that 40,000 have been recruited for this force, and that 20,000 have already been armed and equipped\dots. The anarchists have already noticed and complained about the increased strength of this force at a time when we all know there’s little enough traffic coming over the frontiers, land or sea. They realize that it will be used against them.'' (James Minifie, \emph{New York Herald Tribune,} April~28, 1937).}

On February 28, the Carabineros were forbidden to belong to a political party or a trade union or to attend their mass meetings. The same decree was extended to the Civil and Assault Guards thereafter. That meant quarantining the police against the working class. The hopelessly disoriented anarchist ministers voted for this measure on the ground that it would stop Stalinist proselytizing!

By April, the militias were finally pushed out of all police duties in Madrid and Valencia.

In the proletarian stronghold of Catalonia, this process ran into the determined opposition of the \CNT\ masses. There was also an ``unfortunate incident'' which slowed up the bourgeois scheme. The first Chief of Police for all Catalonia, appointed by the cabinet~-- Andr\'e Reberter~-- proved to be one of the ringleaders in a plot to assassinate the \CNT\ leaders, establish an independent Catalonia and make a separate peace with Franco.\endnote{CNT’s intelligence service had discovered the plot and \emph{Solidaridad Obrera} published the facts on November 27 and 28. At first it was scoffed at by the Stalinists and the Esquerra; but they were forced to order an investigation. As a result, it was found that the chief forces in the plot were those of the separatist Estat Catala, a khaki-shirt organization, a split-off from the Esquerra, and its secretary-general and over a hundred leading members were arrested. Chief of Police Reberter, Estat Catala member, was executed upon conviction. Casanovas, President of the Catalan Parliament, ``at first toyed with the plot, then rejected it,'' said an official explanation. Casanovas was permitted to go to France~-- and to return to political life in Barcelona after the May Days!} His exposure strengthened the position of the workers’ patrols, largely manned by the \CNT\kn.
\nowidow

But then the patrols were attacked from within. The \PSUC\ ordered its members to withdraw (most of them did not, and were expelled from the \PSUC). The Esquerra also withdrew from the patrols. Thereafter all the usual Stalinist methods of defamation were directed at the patrols, loudest when the patrols arrested \PSUC\ and \textsc{gepci} businessmen for hoarding and profiteering on food.

On March 1, a Generalidad decree unified all police into a single state-controlled corps, its members prohibited from association with trade unions and parties and to be chosen by seniority. This meant abolition of the workers’ patrols and the barring of their members from the unified police. Apparently the \CNT\ ministers voted for the decree. But the resultant outcry of the Catalan masses led the \CNT\ to join the \POUM\ in declaring they would refuse to submit to it. On March 15, nevertheless, the Minister of Public Order, Jaime Ayguade, attempted unsuccessfully to suppress forcibly workers’ patrols in the outlying districts of Barcelona. This question was one of those leading to the dissolution of the Catalan cabinet on March 27. But there was no change when the new cabinet, again with \CNT\ ministers, convened on April 16. Ayguad\'e continued his attempts to disarm the patrols, while the \CNT\ ministers sat in the cabinet, their papers contenting themselves with warning the workers against provocation.

\section{Liquidation of the Militias}

There could, of course, be no hope of reviving a stable bourgeois regime so long as the organization and administrative responsibility for the armed forces was in the hands of the unions and workers’ parties, which presented payrolls, requisitions, etc., to the Madrid and Catalan governments, and stood between the militias and the governments.

The Stalinists early sought to set an ``example'' by handing their militias over to government control, helping to institute the salute, supremacy of officers behind the lines, etc. ``No discussion, no politics in the army,'' cried the Stalinist press, meaning of course no working-class discussion or politics.

The example was wasted on the \CNT\ masses. At least a third of the armed forces were \CNT\ members, suspicious of the officers sent by the government, relegating them to the status of ``technicians'' and barring them from interfering in the social and political life of the militias. The \POUM\ had 10,000 militiamen who acted likewise. The \POUM\ reprinted for distribution in the militias the original \emph{Red Army Manual} of Trotsky, providing for a democratic internal regime and political life in the army. The Stalinist campaign for wiping out the internal democratic life of the militias, under the slogan of ``unified command,'' was countered by the simple and unanswerable question: why does a unified command necessitate reestablishing the old barracks regime and the supremacy of a bourgeois officer caste?

But the government eventually had its way. The militarization and mobilization decrees passed in September and October with \CNT\ and \POUM\ consent provided conscription of regular regiments ruled by the old military code. Systematic selection of candidates for the officers’ schools gave preponderance to the bourgeoisie and Stalinists, and these manned the new regiments.

When the first drafts of the new army were ready and sent to the front, the government pitted them against the militias, demanding reorganization of the militias in a similar mold. By March the government had largely succeeded on the Stalinist-controlled Madrid front. On the Aragon and Levante fronts, manned chiefly by \CNT-\FAI\ and \POUM\ militias, the government prepared the liquidation of the militias by a ruthless, systematic policy of withholding arms. Only after reorganization, the militias were informed, would they be given adequate arms for an offensive on these fronts. Yet the sheer mass of the \CNT\ militias prevented the government from attaining its objectives until after the May days, when Azaña’s ex--Minister of~War, General Pozas, took over the Aragon front.

In the last analysis, however, the government’s final success came not from its own efforts so much as from the politically false character of the \CNT--\POUM\ demand for a ``unified command under the control of the workers’ organizations.''

The Stalinists and their ``non-party'' publicists of the stripe of Louis Fischer and Ralph Bates have deliberately perverted the facts of the controversy between the \POUM--\CNT\ and the government on army reorganization. The Stalinists make it appear that the  \POUM--\CNT\ wanted to retain the loosely organized militias as against an efficiently centralized army. This is a lie made out of whole cloth, as may be demonstrated by a thousand articles in the \POUM--\CNT\ press of the time calling for a disciplined army under unified command. The real issue was: who will control the army, bourgeoisie or working class? Nor did the \POUM--\CNT\ alone raise this question. In opposing Giral’s original scheme for a special army, the \UGT\ organ, \emph{Claridad,} had declared:

\begin{quotation}
  We must take care that the masses and the leadership of the armed forces, which should be above all the people in arms, should not escape from our hands. (August 20, 1936).
\end{quotation}

That was the real issue. The bourgeoisie won out because the \UGT\kn, \POUM\kn, and \CNT-\FAI\ made the hopeless error of seeking a proletarian-controlled army within a bourgeois state. So much were they for centralization and unified command that they voted for the governmental decrees which in the ensuing months served to wipe out all workers’ control of the army. \UGT\kn, \POUM\kn, and \CNT's consent to these decrees was not the least of their crimes against the working class.

Their slogan for a unified command under control of the workers’ organizations was false because it provided no method of achieving that goal. The demand which should have been raised, from the first day of the war, was for amalgamation of all the militias and the few existent regiments into a single force, with democratic election of soldiers’ committees in each unit, centralized in a national election of soldiers’ delegates to a national council. As new regiments were conscripted, their soldiers’ committees would have entered the local and national councils. Thus, in drawing the armed masses into daily political life, bourgeois control of the armed forces could have been effectively prevented.

The \POUM\ had a wonderful opportunity to demonstrate the efficacy of this method. On the Aragon front it had for eight months direct organizational control over some 9,000 militiamen. It had an unparalleled opportunity to educate them politically, to elect soldiers’ committees among them as an example to the rest of the militias, then to demand amalgamation in which its trained forces would have been a powerful leaven. Nothing was done. The \POUM\ press carried stories of representatives of the Aragon front meeting in congress. These meetings were nothing but gatherings of appointees of the national office. In fact, the \POUM\ \emph{forbade} election of soldiers’ committees. Why? Among other reasons was the fact that opposition to the \POUM’s opportunist politics was rife in the ranks and the bureaucracy feared that creation of the committees would provide the necessary arena in which the Left Opposition might conquer.

The simple, concrete slogan of elected soldiers’ committees was the only road for securing proletarian control of the army. This slogan, moreover, could only be a transitional step. For a worker-controlled army could not exist indefinitely side by side with the bourgeois state. If the bourgeois state continued to exist, it would inevitably destroy workers’ control of the army.

The \POUM--\CNT--\UGT\ proponents of workers’ control raised neither the concrete slogan nor had they any program for displacing the bourgeois state. Their basic orientation, therefore, doomed to impotence their opposition to bourgeois domination of the army.

\newpage
\bigskip

\section{Disarming of the Workers in the Rear}

In the revolutionary days following July 19, the Madrid and Catalan governments had perforce sanctioned the arming of the workers who had already armed themselves. Workers’ organizations were empowered to issue arms permits to their members. For the workers it was not only a question of guarding against the counter-revolutionary attempts of the government, but the daily necessity of protecting the peasants’ committees against reactionaries, guarding the factories, railroads, bridges, etc., against fascist bands, protecting the coast from raids, ferreting out hidden fascist nests.

In October came the first disarming decree providing for delivery of all rifles and machine guns to the government. In practice, it was interpreted to allow the workers’ organizations to continue issuing permits for long arms to industrial guards and peasant committees. But it set the fatal precedent.

On February 15, the central government ordered the collection of \emph{all} long arms as well as all small arms not held by permission. On March 12, the cabinet ordered the workers’ organizations to collect large and small arms from their members and to surrender them within forty-eight hours. This order was applied directly to Catalonia on April 17. National Republican Guards began officially to disarm workers on sight in the streets of Barcelona. Three hundred workers~-- \CNT\ members holding organization permits~-- were thus disarmed by police during the last week in April.

The pretext that the arms were needed at the front was a bare-faced lie, as any worker could see with his own eyes. For while the workers were being deprived of rifles and revolvers, some of them in the possession of the \CNT\ since the days of the monarchy, the cities were being filled with the rebuilt police forces, armed to the teeth with new Russian rifles, machine guns, artillery, and armoured cars.

\newpage

\bigskip

\section[Extra-Legal Methods of Repression]{Extra-Legal Methods of Repression: The Spanish \textsc{gpu}}

On December 17, 1936, \emph{Pravda,} Stalin’s personal organ, declared: ``As for Catalonia, the purging of the Trotskyists and the An\-archo-Syn\-di\-cal\-ists has begun; it will be conducted with the same energy with which it was conducted in the \USSR\kn.''

The ``legal methods,'' however, moved too slowly. They were supplemented by organized terrorist bands, equipped with private prisons and torture chambers, termed ``preventoriums.'' The worthies recruited for this work beggar description: ex-members of the fascist \textsc{ceda}, Cuban gangsters, brothel racketeers, passport forgers, sadists.\endnote{\emph{Cultura Proletaria,} \textsc{ny} anti-fascist paper, published a report from Cuba: ``The CT \dots\ sent 27 ex-officers of the old army who have nothing in common with workers and are mercenaries formerly in Machado’s service\dots. On its last trip the Mexique took an expedition of these fake militia (with a few exceptions), among them went the three \'Alvarez brothers, former Machado gunmen active in breaking the Bahia strike. On the 29th of this month \dots\ `Sargento del Toro' sails, too, as a communist militiaman. He is a full-fledged assassin of the Machado days, bodyguard of the President of the Senate in that period. He was one of those who helped massacre workers in a demonstration here on August 27.''

The former Valencia Secretary of the \textsc{ceda} is now in the CP. Even Louis Fischer admits that ``bourgeois generals and politicians, and many peasants who approve the CP’s policy of protecting small property holders have joined \dots\ essentially their new political affiliation reflects a despair of the old social system as well as a hope to salvage one or two of its remnants.''

An apt description, as Anita Brenner points out, of the social group which swelled Hitler’s ranks. For further details on the Spanish \textsc{gpu} and the repressions, see Anita Brenner’s excellent article and ``Dossier of Counterrevolution,'' \emph{Modern Monthly,} September 1937.} Spawned by the petty-bourgeois composition of the Communist party, nurtured by its counterrevolutionary program, these organized bands of the Spanish \textsc{gpu} exhibited toward the workers the ferocity of Hitler’s bloodhounds, for like them, they were trained to exterminate revolution.

Rodriguez, \CNT\ member and Special Commissioner of Prisons, in April formally charged José Cazorla, Stalinist Central Committee member and Chief of Police under the Madrid Junta, and Santiago Carillo, another Central Committee member, of illegally seizing workers arrested by Cazorla but acquitted by the popular tribunals, and ``taking said acquitted parties to secret jails or sending them into communist militia battalions in advanced positions to be used as `fortifications.'\kp\kp'' The \CNT\ in vain demanded a formal investigation of its charges. Only when it was established that Cazorla’s gang, as a sideline, was working with racketeers who were releasing important fascists from prison without official sanction, was Cazorla removed. He was simply replaced by Carrillo, another Stalinist, and the extralegal \textsc{gpu} and its private prisons continued as before.

``It is becoming clear that the Chekist\endnote{The anarchists refer to the \GPU. In general, they blind themselves to the vast gulf between the \emph{Cheka,} which ruthlessly suppressed the White Guards and their associates in the early period of the Russian revolution, and the Stalinist \emph{\GPU\kn,} which ruthlessly suppresses and assassinates proletarian revolutionists.} organizations recently discovered in Madrid \dots\ are directly linked with similar centers operating under a unified leadership and on a preconceived plan of national scope,''
wrote \emph{Solidaridad Obrera} on April 25, 1937.
On April 8, the \CNT\kn, armed with proofs, had finally forced the arrest of a gang of Stalinists in Murcia, and the removal of the civil governor for maintaining private prisons and torture chambers. On March 15, sixteen \CNT\ members had been murdered by Stalinists in Villanueva de Alcardete in Toledo Province. The \CNT\ demand for punishment was countered by \emph{Mundo Obrero}’s defense of the murderers as revolutionary anti-fascists. The subsequent judicial investigation established that an all-Stalinist gang, including the Communist party’s mayors of Villanueva and Villamayor, operating as a ``Defense Committee,'' had murdered political enemies, looted, levied tribute, and forcibly raped the defenseless women of the area. Five of the Stalinists were condemned to death, eight others sentenced to prison.

The organized gangsterism of the Spanish \GPU\ has been established in the Spanish Government’s own courts of law. We limit ourselves here to juridically established instances. But the \CNT\ press is filled with hundreds of instances in which the ``legal'' counterrevolution was supplemented by the \GPU\ in Spain.
  \chapter{The Counter-Revolution and the Masses}

\lettrineI{t would be a libel} on the socialist and anarchist-led masses to think that they were not alarmed by the advance of the counter-revolution. Discontent, however, is not enough. It is necessary also to know the way out. Without a firm, well-developed strategy for repelling the counter-revolution and leading the masses to state power, discontent can accumulate indefinitely and only issue in sporadic, desperate lunges which are doomed to defeat. In other words, the masses require a revolutionary leadership.

Especially in the ranks of the \CNT\ and \FAI\ the discontent was enormous. It seeped out in hundreds of articles and letters in the anarchist press. Though the anarchist ministers in Valencia and the Generalidad voted for the governmental decrees or submitted to them without public protest, their press did not dare defend the governmental policies directly. As governmental repressions increased, the pressure of the \CNT\ workers on their leadership increased.

On March 27, the \CNT\ ministers withdrew from the Catalonian government. The ensuing ministerial crisis lasted three whole weeks.

``We cannot sacrifice the revolution to unity,'' declared the \CNT\ press.

``No more concessions to reformism.''

``Unity has been maintained until now on the basis of our concessions.''

``We can retreat no further.''

Precisely what the \CNT\ leadership now proposed, however, was a mystery. Companys neatly punctured their postures by a summary of the ministerial course since December, demonstrating that the \CNT\ ministers had voted for everything---the disarming of the workers, the army mobilization and reorganization decrees, dissolution of the workers’ patrols, etc. Stop this humbug and come back to work, Companys was saving. And as a matter of fact, the \CNT\ ministers were ready to come back at the end of the first week. At this point, however, the Stalinists demanded a further capitulation: the organizations providing ministers should sign a joint declaration pledging themselves to carry out a stated series of tasks. The \CNT\ ministers protested that the usual ministerial declaration after constituting the new cabinet would be sufficient---the Stalinist proposal would have left the \CNT\ ministers absolutely naked before the masses. Thus the ministerial crisis dragged out two more weeks.

There then ensued a little by-play which amounted to nothing more than a division of labour, whereby the \CNT\ leaders were bound more strongly than ever to the Generalidad. Companys assured the \CNT\ that he agreed with them and not with the Stalinists, and offered his services to ``force'' the Stalinists to relinquish their demand. At the same time, Premier Tarradellas, Companys’ lieutenant, defended the administration of the war industries (run by the \CNT) against an attack in the \PSUC\ organ, \emph{Treball}, which he termed the ``most arbitrary falsehoods.'' For these little services, the \CNT\ abjectly gave Companys unconditional political support:

\begin{quotation}
  We declare publicly that the \CNT\ is to be found at the side of the president of the Generalidad. Luis Companys, whom we have accorded whatever facilities have been required for the solution of the political crisis. We stand by the president who, without any kind of servile praise---a proceeding incompatible with the morale of our revolutionary movement---knows that he can count on our most profound respect and our most sincere support. (\emph{Solidaridad Obrera}, April 15, 1937, p. 12)
\end{quotation}

Companys, of course, managed to persuade the Stalinists to relinquish the demand for a pact, and on April 16, the ministerial crisis was ``resolved.'' The new cabinet, like its predecessor, provided a majority for the bourgeoisie and the Stalinists, and, of course, differed in nothing from the previous one.

The masses of the \CNT\ could not be so ``flexible.'' They had a heroic tradition of struggle to the death against capitalism. Even more insistently, the revival of the bourgeois state was taking place on their backs. Inflation and the uncontrolled manipulation of prices by the businessmen ``mediating'' between the peasantry and the city masses now led to perpendicular price rises. In this period the rise of prices is the leitmotif of all activity. The press is full of the problem. The condition of the masses was growing daily more intolerable, and the \CNT\ leaders showed them no way out.

Many voices now cried for a return to the traditional apoliticism of the \CNT. ``No More Governments!'' Local \CNT\ papers broke discipline and took up this refrain. It was counsel of unthinking despair.

Far more significant was the rise of the Friends of Durruti. In the name of the martyred leader, a movement rose which had assimilated the need for political life, but rejected collaboration with the bourgeoisie and reformists. The Friends of Durruti were organized to wrest leadership from the \CNT\ bureaucracy. In the last days of April, they plastered Barcelona with their slogans---an open break with the \CNT\ leadership. These slogans included the essential points of a revolutionary programme: all power to the working class, and democratic organs of the workers, peasants and combatants, as the expression of the workers’ power.

The Friends of Durruti represented a deep ferment in the libertarian movement. On April 1, a manifesto of the Libertarian Youth of Catalonia (\emph{Ruta}, April 1, 1937) had denounced the ``United Socialist Youth'' [Stalinists], who first assisted the revaluation of the Azaña stock-fallen so low in the first days of the revolution when he tried to flee the country-and who called to the Unified Catholic Youth and even to those who were sympathetic to fascism; stigmatized the bourgeois-Stalinist bloc as ``supporting openly all the intentions of the English and French governments to encircle the Spanish revolution;'' excoriated the counter-revolutionary assaults on the publishing houses and radio station of the \POUM\ in Madrid. 

It pointed out that:

\begin{quotation}
  \dots\ arms are denied to the Aragon front because it is definitely revolutionary, in order to be able afterward to throw mud at the columns operating on that front.
	
  \dots\ the Central Government boycotts Catalan economy in order to force us to renounce our revolutionary conquests.
	
  \dots\ the sons of the people are sent to the front, but for counter-revolutionary ends the uniformed forces are being kept in the rear.
  
  \dots\ they have gained ground for a dictatorship---not pro\-le\-ta\-ri\-an!---but bourgeois.
\end{quotation}

Clearly differentiating the Anarchist Youth from the \CNT\ ministers, the manifesto concluded:

\begin{quotation}
  We are firmly decided not to be responsible for the crimes and betrayals of which the working class is being made the object\dots\ We are ready to return, if that is necessary, to the underground struggle against the deceivers, against the tyrants of the people and the miserable merchants of politics.
\end{quotation}

An editorial in the same issue of Ruta declared:

\begin{quotation}
  Let not certain comrades come to us with appeasing words. We shall not renounce our struggle. Official automobiles and the sedentary life of the bureaucracy do not dazzle us.
\end{quotation} 

This from the official organization of the anarchist youth!

Not in a day or a month, however, does a regroupment take place. The \CNT\ had a long tradition and the discontent of its masses would evolve only at a slow pace into an organized struggle for a new leadership and a new programme. Particularly was this true because no revolutionary party existed to encourage this development.

\section*{The \POUM’s Answer to the Counter-Revolution}

An abyss was opening up between the \CNT\ leaders and the masses within the \CNT\ movement. Would the \POUM\ step into the breach and place itself at the head of the militant masses?

The prevalence of a wide tendency in \CNT\ ranks to go back to traditional apoliticism was an annihilating criticism of the \POUM, which had done nothing to win these workers to revolutionary-political life. Also with no aid from the \POUM\ leadership, a genuinely revolutionary current was crystallizing in the Friends of Durruti and the Libertarian Youth. If the \POUM\ was ever to strike out independently of the \CNT\ leadership, this was the moment!

The \POUM\ did nothing of the sort. On the contrary, in the ministerial crisis of March 26--April 16, it revealed that it had learned nothing whatever from its earlier participation in the Generalidad. The Central Committee of the \POUM\ adopted a resolution declaring:

\begin{quotation}
  There is needed a government that would canalize the aspirations of the masses, giving a radical and concrete solution to all the problems by way of the creation of a new order that would be guarantor of the revolution and of the victory at the front. This Government can only be a government formed by representatives of all the political and trade union organizations of the working class which would propose as immediate tasks the realization of the following programme. (\emph{La Batalla}, March 30)
\end{quotation}

The proposed fifteen-point programme is not a bad one---for a revolutionary government. But the absurdity of proposing it to a government which by definition includes the Stalinists and the Esquerra-controlled Union of Rabassaires (independent peasants) is indicated immediately by the last point on the programme: the convocation of a congress of delegates of the unions, peasants and combatants which will in turn elect a permanent workers’ and peasants’ government.

For six months the \POUM\ had been saying that the Stalinists were organizing the counter-revolution. How, then, could the \POUM\ propose to collaborate with them in the government and in convoking the congress? From this proposal the workers could only conclude that the \POUM’s characterization of the Stalinists had been so much factional talk, and would henceforth take no \POUM\ charges against the Stalinists as being seriously meant.

And Companys and his Esquerra? A new cabinet must receive a mandate from Companys, and the \POUM\ proposed no break with this law. Was it conceivable that Companys would agree to a government which would convoke such a congress? Here, too, the masses could only draw the conclusion that the \POUM’s declaration of the necessarily counter-revolutionary role of the Esquerra of Companys was not seriously meant.
  \chapter[The May Days]{The May Days: Barricades in Barcelona}

\lettrineE{ven more than} before the civil war, Catalonia was the chief economic centre of Spain; and these economic forces were now in the hands of the workers and peasants (\emph{so they thought}). The entire textile industry of Spain was located here. Its workers now provided clothing and blankets for the armies and the civilian population, and the vitally needed goods for export. With Bilbao’s iron and steel mills virtually cut off from the rest of Spain, the metal and chemical workers of Catalonia had, by the most heroic diligence, created a great war industry to equip the anti-fascist armies. The agricultural collectives, raising the greatest crops in Spanish history, were feeding the armies and cities and providing citrus fruit for export. The \CNT\ seamen were carrying away the exports which gave Spain credits abroad and were bringing home precious cargoes for use in the struggle against Franco. The masses of the \CNT\ were holding the Aragon and Teruel fronts; they had sent Durruti and the pick of their militias to save Madrid in the nick of time. The Catalonian proletariat, in a word, was the backbone of the anti-fascist forces, and knew it.

What is more, its power had been acknowledged, after July 19, even by Companys. The Catalan president, addressing the \CNT-\FAI\ in the July Days, had said:

\begin{quotation}
  You have always been severely persecuted and I, with much pain but forced by political realities, I, who was once with you, later saw myself obliged to oppose and persecute you. Today you are the masters of the city and of Catalonia, because you alone vanquished the Fascist soldiers. I hope you will not find it distasteful that I should now remind you that you did not lack the aid of the few or many men of my party and of the Guards\dots\ You have conquered, and all is in your power. If you do not need or want me as President tell me now, and I shall become another soldier in the anti-fascist fight. If, on the contrary, you believe me when I say that I would abandon this post to victorious fascism only as a corpse, perhaps, with my party comrades and my name and prestige, I can serve you.
\end{quotation}

Consequently, the alarm and rage of the Catalonian masses at the counter-revolutionary inroads were the emotions of freed men and masters of their fate in danger again of enslavement. Submission without a fight was out of the question!

On April 17---the day after the \CNT\ ministers rejoined the Ge\-ne\-ra\-li\-dad---a force of Carabineros arrived in Puigcerda and demanded that the \CNT\ worker-patrols there surrender control of the customs. While \CNT\ top-leaders hurried to Puigcerda to arrange a peaceful solution---i.e., to cajole the workers into surrendering control of the border---Assault and Civil Guards were sent to Figueras and other towns throughout the province to wrest police control from the workers’ organizations. Simultaneously, in Barcelona, the Assault Guards proceeded to disarm workers on sight, in the streets. During the last week of April they reported three hundred thus disarmed. Collisions between the workers and the Guards took place nightly. Truckloads of Guards would disarm solitary workers. The workers retaliated. Workers who refused to submit were shot. Guards were picked off in turn.

On April 25, a \PSUC\ trade union leader, Roldan Cortada, was assassinated in Molins de Llobregat. Who killed him is not known to this day. The \CNT\ denounced the murder and proposed an investigation. The \POUM\ pointed out that, significantly enough, Cortada had been a supporter of Caballero before the fusion and had been known to disapprove of the pogrom-spirit being generated by the Stalinists. But the \PSUC\ squeezed the opportunity dry, denouncing the ``uncontrollables,'' ``hidden fascist agents,'' etc.

On April 27, the \CNT\ and \POUM\ representatives appeared at Cortada’s funeral---and found it a demonstration of the forces of the counter-revolution. For three and a half hours the ``funeral''---\PSUC\ and government soldiers and police gathered from far and wide and armed to the teeth---marched through the workers’ districts of Barcelona. It was a challenge and the \CNT\ masses were not blind to it. The next day the government dispatched a punitive expedition to Molins de Llobregat, seized the anarchist leaders there and brought them handcuffed back to Barcelona. That night and the next, \CNT\ and \PSUC\ Assault Guard groups were disarming each other in the streets. The first barricades were erected in the workers’ suburbs.

The Carabineros, reinforced and joined by the local \PSUC\ forces, attacked the worker-patrols in Puigcerda. Antonio Martin, mayor and \CNT\ leader, popular throughout Catalonia, was shot dead by the Stalinists.

May Day, oldest and dearest of proletarian holidays, dawned the government prohibited all meetings and demonstrations throughout Spain.

In those last days of April, the Barcelona workers learned for the first time, through the pages of \emph{Solidaridad Obrera}, what had happened to their comrades in Madrid and Murcia at the hands of the Stalinist \GPU.

\dinkus

The Telefonica, the main telephone building dominating Bar\-ce\-lo\-na’s busiest square, had been occupied by fascist troops on July 19, 1936, surrendered to them by the Assault Guards the government had sent there. The \CNT\ workers had lost many comrades in re-conquering it. So much the dearer was possession of it. Since July 19, the red and black flag of the \CNT\ had flown from its tower, visible to workers throughout the city. Since July 19, the exchange had been managed by a \CNT-\UGT\ committee, with a government delegation stationed in the building. The working staff was almost entirely \CNT\ in allegiance and \CNT\ armed guards defended it against fascist forays.

Control of the Telefonica was a concrete instance of the dual power. The \CNT\ was in a position to listen in on government calls. The bourgeois-Stalinist bloc would never be master in Catalonia so long as it was possible for the workers to cut off telephonic co-ordination of the government forces.

On Monday, May 3, at 3 \textsc{pm}, three lorry loads of Assault Guards arrived at the Telefonica under the personal command of the Commissioner of Public Order, Salas, a \PSUC\ member.\endnote{%
%
The knotty problem of justifying the armed seizure of the Telefonica was ``solved,'' in the Stalinist press, by giving at least four different explanations:

\begin{enumerate}
  \item ``Salas sent the armed republican police to disarm the employees there, most of them members of \CNT\ unions. For a considerable time the telephone service had been run in a way which was open to the gravest criticism, and it was imperative to the whole conduct of the war that the defects of the service should be remedied.'' (London \emph{Daily Worker}, May 11)
  
  \item The police ``occupied the central telephone exchange. In so doing the police by no means intended to interfere with the rights of the workers as guaranteed by law (as alleged subsequently by the Trotskyist provocateurs). What the police wanted was to put all telephone connections under the immediate supervision of the Government.'' (\emph{Inprecorr}, May 22) What was ``guaranteed by law,'' however, was the \emph{workers’ control} sanctioned by the collectivization decree of October 24, 1936!
  
  \item A week later, a new story: ``Comrade Salas went to the Telefonica which on the previous night had been occupied by 50 members of the \POUM\ and various uncontrollable elements. The guard forced its way into the building and turned the occupants out of it. The affair was soon settled. Surprised by the rapid move on the part of the Government the 50 people left the building and the Telephone Exchange was again (!!) in the hands of the Government.'' (\emph{Inprecorr}, May 29)
  
  \item In the final version, issued by the Catalan section of the Comintern as Salas’ own story: ``In the first place there was no occupation of the Telefonica, nor was there any question of occupation of the Telefonica. I received a signed order from Ayguade, Minister for Public Order, that a Government delegate was to be installed and that I was to be responsible for seeing that he was so installed. Accordingly, I, with Captain Menendez, and a personal escort of four men, entered the Telephone Building. I explained my business and said that I wished to speak with some responsible member of the Committee. We were told that there was no one in the building. However, we waited downstairs while they went to look. Two minutes later some individuals started shooting at us from the stairs. None of us was hit. Immediately I phoned for the guards to come, not to occupy the building in which we were already but to cordon off the building and prevent anyone from entering\dots\ I and Eroles (Anarchist police functionary) went up to the top of the building, where they had established themselves with a machinegun, hand grenades and rifles. We went up together without escort, and without arms. At the top I explained the purpose of my visit. They came down. The delegate was installed according to orders. The forces were withdrawn. There were no casualties and arrests.''
  
  The \CNT\ account brands this story a lie: Salas began by disarming the guards and forcing the telephone workers to put their hands up; the guards on the upper floors withdrew only the next day as part of a general agreement for both sides withdrawing---which the Government promptly violated. The four different Stalinist versions testify to the difficulty of covering up the simple truth: they wanted to end workers’ control of the Telefonica and they did.
\end{enumerate}

}

Surprised, the guards on the lower floors were disarmed. But halfway up, a machine gun barred further occupation. Salas sent for additional Guards. Anarchist leaders pleaded with him to withdraw from the building. He refused. The news spread like wildfire to the factories and workers’ suburbs.

Within two hours, at 5 \textsc{pm}, the workers were pouring into the local centres of the \CNT-\FAI\ and \POUM, arming and building barricades. From the dungeons of the Rivera dictatorship until today, the \CNT-\FAI\ have always had their local defence committees, with a tradition of local initiative. So far as there was leadership in the coming week, these defence committees provided it. There was almost no firing the first night, for the workers were overwhelmingly stronger than the government forces. In the workers’ suburbs, many of the government police, with no stomach for the struggle, peacefully surrendered their arms. Lois Orr, an eye-witness, wrote:

\begin{quotation}
  By the next morning (Tuesday, May 4), the armed workers dominated the greatest part of Barcelona. The entire port, and with it Montjuich fortress, which commands the port and city with its cannon, was held by the anarchists; all the suburbs of the city were in their hands; and the government forces, except for a few isolated barricades, were completely outnumbered and were concentrated in the centre of the city, the bourgeois area, where they could easily be closed in on from all sides as the rebels were on July 19, 1936.
\end{quotation}

\CNT, \POUM, and other accounts substantiate this fact.

In Lerida, the civil guards surrendered their arms to the workers Monday night, as also in Hostafranchs. \PSUC\ and Estat Catala headquarters in Tarragona and Gerona were seized by \POUM\ and \CNT\ militants as a ``preventive measure.'' These overt steps were but the beginning of what could be done, for the masses of Catalonia were ranged overwhelmingly under the banner of the \CNT. The formal seizure of Barcelona, the constitution of a revolutionary government, would have, overnight, led to working-class power. That this would have been the outcome is not seriously contested by the \CNT\ leaders nor by the \POUM.\endnote{Even the \textsc{ilp} leader, Fenner Brockway, always to the right of the \POUM, in this case concedes that ``for two days the workers were on top. Bold and united action by the CNT leadership could have overthrown the Government.''}

That is why the left wingers in the CNT and POUM ranks, sections of the Libertarian Youth, the Friends of Durruti and the Bolshevik-Leninists called for a seizure of power by the workers through the development of democratic organs of defence (soviets). On May 4, the Bolshevik-Leninists issued the following leaflet, distributed on the barricades:

\medskip

\begin{oframed}
\begin{quotation}
  \noindent
  {\bfseries\sffamily\normalsize\centering Long Live the Revolutionary Offensive \par}
  
  \bigskip
  
  \noindent
  No compromise. Disarmament of the National Republican Guard and the reactionary Assault Guards. This is the decisive moment. Next time it will be too late. General strike in all the industries excepting those connected with the prosecution of the war, until the resignation of the reactionary government. Only proletarian power can assure military victory.

  \begin{itemize}
  	\item Complete arming of the working class.
  	\item Long live unity of action of \CNT-\FAI-\POUM.
  	\item Long live the revolutionary front of the proletariat.
  	\item Committees of revolutionary defence in the shops, factories, districts.
  \end{itemize}
	
  \begin{flushright}
    Bolshevik-Leninist section of Spain \\
    (for the Fourth International)
  \end{flushright}
\end{quotation}
\end{oframed}

\medskip

The leaflets of the Friends of Durruti, calling for ``a revolutionary Junta, complete disarmament of the Assault Guards and the National Republican Guards,'' hailing the \POUM\ for joining the workers on the barricades, estimated the situation in conceptions identical with those of the Bolshevik-Leninists. Still adhering to the discipline of their organizations, and issuing no independent propaganda, the \POUM\ Left, the \CNT\ Left and the Libertarian Youth agreed on perspective with the Bolshevik-Leninists.

They were undoubtedly correct. No apologist for the \CNT\ and \POUM\ leaders has adduced any argument against the seizure of power which stands up under analysis. None of them dares deny that the workers could easily have seized power in Catalonia. They adduce three main arguments to defend the capitulation: that the revolution would have been isolated, limited to Catalonia, and defeated there from the outside; that the fascists would have been able at this juncture to break through and win; that England and France would have crushed the revolution by direct intervention. Let us closely examine these arguments.

\vspace{-0.5 \baselineskip}

\subsection*{Isolation of the Revolution}

\vspace{-0.25 \baselineskip}

The most plausible, most radical, form given to this argument is that based on an analogy with the ``armed demonstration'' of July 1917 in Petrograd. ``Even the Bolsheviks in July 1917 did not decide to seize power and limited themselves to the defensive, leading the masses out of the line of fire with as few victims as possible.'' Ironically enough, the \POUM, \textsc{ilp}, Pivertists, and other apologists who use this argument are precisely those who have been incessantly reminding ``sectarian Trotskyists'' that ``Spain is not Russia,'' and that, therefore, the Bolshevik policy is not applicable.

The Trotskyist, i.e., Bolshevik, analysis of the Spanish Revolution, however, has always based itself on the concrete conditions of Spain. In 1931, we warned that the rapid rhythm of the developments of Russia in 1917 would not be duplicated in Spain. On the contrary, we then used the analogy of the great French Revolution which, beginning in 1789, passed through a series of stages before attaining its culmination in 1793. Just because we Trotskyists do not schematize historic events, we cannot take seriously the analogy with July 1917.\endnote{Leon Trotsky, \emph{The Revolution in Spain}, April 1931; \emph{The Spanish Revolution in Danger}, 1931, Pioneer Publishers, New York.}

The armed demonstration broke out in Petrograd only four months after the February revolution, only three months after Lenin’s April Theses had given a revolutionary direction to the Bolshevik party:

\begin{quotation}
  The overwhelming mass of the population of the gigantic country was only just beginning to emerge from the illusions of February.
  
  At the front was an army of twelve million men who were only then being touched by the first rumours about the Bolsheviks. In these conditions the isolated insurrection of the Petrograd proletariat would have led inevitably to their being crushed. It was necessary to gain time. It was these circumstances that determined the tactic of the Bolsheviks.
\end{quotation}

In Spain, however, May 1937, came after six full years of revolution in which the masses of the whole country had amassed a gigantic experience. The democratic illusions of 1931 had been burned out. We can cite testimony from \CNT, \POUM, socialist leaders that the refurbished democratic illusions of the People’s Front never caught hold of the masses---they voted in February 1936 not for the People’s Front but against Gil Robles and for the release of the political prisoners. Again and again the masses had shown that they were ready to go through to the end: the numerous anarchist-led armed struggles, the land seizures during six years, the October 1934 revolt, the Asturian Commune, the seizure of the factories and land after July 19! The analogy with Petrograd of July 1917 is childish.

Twelve million Russian soldiers, scarcely touched by Bolshevik propaganda, were available to be used against Petersburg in 1917. But in Spain more than a third of the armed forces carried \CNT\ membership cards; nearly another third \UGT\ cards, most of them left socialists or under their influence. Even grant that the revolution would not immediately spread to Madrid and Valencia. But this is entirely different from asserting that the Valencia government would have found troops with which to smash the Workers’ Republic of Catalonia!

Immediately after the May events, the \UGT\ masses showed their determined hostility to repressive measures against the Catalonian proletariat. That was one reason why Caballero had to leave the government. All the more could they not have been used against a victorious workers’ republic. Not even the Stalinist ranks would have provided a mass army for that purpose: it is one thing to get backward workers and peasants to limit their struggle to one for a democratic republic; it is something entirely different to get them to crush a workers’ republic. Any attempt by the bourgeois-Stalinist bloc to gather a proletarian force would have simply precipitated the extension of the workers’ state to all Loyalist Spain.

We can assert more than this: that the example of Catalonia would have been followed elsewhere immediately. Proof? The Stalinist-bourgeois bloc, while seeking to consolidate the bourgeois republic nevertheless was compelled by the revolutionary atmosphere to raise the slogan: ``Let us finish Franco first and make the revolution afterward.''

It was a clever slogan, well designed to keep the masses in check. But the very fact that the counter-revolution needed this slogan demonstrates that it based its hopes for victory over the revolution, \emph{not} on the agreement of the masses but on the masses’ \emph{embittered toleration}. Gritting their teeth, the masses were saying, ``we must wait until we finish off Franco, then we shall finish off the bourgeoisie and their lackeys.'' This feeling, undoubtedly widespread, would have disappeared in the face of the example of Catalonia. That example would have ended the feeling---``we must wait.''

Nor would the example of Catalonia have affected only Loyalist Spain. For a workers’ Spain would have embarked on a revolutionary war against fascism which would have disintegrated the ranks of Franco, more by political weapons than by military ones. All the political weapons against fascism which the People’s Front had refused to permit to be used, which could only be used by a workers’ republic, would now confront Franco. Trotsky wrote a few days after July 19:

\begin{quotation}
  A civil war is waged, as everybody knows, not only with military but also with political weapons. From a purely military point of view, the Spanish Revolution is much weaker than its enemy. its strength lies in its ability to arouse the great masses to action. It can even take the army [of Franco] away from its reactionary officers. To accomplish this it is only necessary seriously and courageously to advance the programme of the socialist revolution.

  It is necessary to proclaim that, from now on, the land, the factories, and shops will pass from the capitalists into the hands of the people. It is necessary to move at once toward the realization of this programme in those provinces where the workers are in power. The fascist army could not resist the influence of such a programme: the soldiers would tie their officers hand and foot and hand them over to the nearest headquarters of the workers’ militia. But the bourgeois ministers cannot accept such a programme. Curbing the social revolution, they compel the workers and peasants to spill ten times as much of their own blood in the civil war.
\end{quotation}

Trotsky’s prediction proved all too true. Fearing the revolution more than Franco, the People’s Front government conducted no propaganda aimed at the peasants in Franco’s forces and behind his lines. The government absolutely refused to promise the land to these peasants, and that promise would have had no effect unless the government had actually decreed giving the land to the peasant committees in its own regions, from which, by a thousand roads, the news would have spread to the peasants in the rest of Spain.
\noclub

Fearing the revolution more than Franco, the government had rejected all proposals (including those of Abdel-Krim and other Moors) to incite revolution in Morocco under a declaration of independence for Morocco. Fearing the revolution more than Franco, the government appealed to the international proletariat to get ``their'' governments to help Spain---but never appealed to the international proletariat to help Spain in spite of and against their governments.

We are not doctrinaires. We do not declare the revolution every day. We judge from our concrete analysis of the conditions in Spain in May 1937: Had the workers’ republic been established in Catalonia, it would not have been isolated or crushed. It would have been quickly extended to the rest of Spain.

\subsection*{``The Fascists Would Have Broken Through''}

The second apology for not taking power in Catalonia overlaps the first to the extent that it implicitly denies the effect of the taking of power on Franco’s forces.\endnote{A well-known anarchist leader said to me: ``You Trotskyists are worse utopians than we ever were. Morocco is in Franco’s hands ruled by him with an iron hand. Our declaration of the independence of Morocco would have no effect.''

I reminded him that Lincoln’s Declaration of Emancipation of the slaves was issued while the Confederacy still held all the South. Marxists, at least, should recall, that Marx and Engels gave this political act enormous weight in the defeat of the South.

Another anarchist said: ``Our peasants have already seized much land, yet it has no effect on the peasants under Franco.''

Under questioning, however, he admitted that the peasants feared that the government would try to recover the land after the war. In Russia, too, by November 1917 the peasants seized much land. They tilled it, however, sullenly and fearfully. The Soviet decree nationalizing the land transformed the psychology of the peasants and made them overwhelmingly partisans of the Soviet regime.}

Admitting that a proletarian revolution in May would have extended itself throughout Loyalist Spain, the \CNT\ leaders argue:

\begin{quotation}
  It is obvious that, had we so desired, the defence movement could have been transformed into a purely libertarian movement. This is all very well but... the fascists would have, without doubt, taken advantage of these circumstances to break all lines of resistance.\endnote{Speech in Paris, \emph{Spain and the World} (Anarchist) July 2, 1937.} (Garcia Oliver)
\end{quotation}

Though ostensibly dealing with the immediate situation in May in Catalonia, this line of argument is, in actuality, much more fundamental: \emph{it is an argument against the working class taking power during the course of the civil war.}

That was also the line of the \POUM. The Central Committee held that, in the event the government refused to sign its own death warrant by convoking a Constituent Assembly (congress of soldiers’, peasants’ and union delegates), it would be wrong to wrest the power forcibly from the government.

\begin{quotation}
  It believed that the workers would in time protest against the counter-revolution which the government was carrying through and that the demand for such a Constituent Assembly would become so strong that the government would be compelled to submit. It held that an insurrection would be wrong and inadvisable until after the fascists were defeated, and there was a difference of opinion in its ranks whether even then an insurrection would be necessary.\endnote{Fenner Brockway, Secretary of the Independent Labour Party, \emph{The Truth About Barcelona}, London 1937.}
\end{quotation}

In other words, the \CNT\ and \POUM\ called for socialism through the government. But if the government would not yield, then we must wait until after the war at least. In practice, this came down to covert adaptation to the bourgeois-Stalinist slogan---``Let us finish Franco first and make the revolution afterward.''

The \POUM-\CNT\ tactic of waiting until Franco was finished off meant, concretely, the doom of the revolution. For, as we have already pointed out, the bourgeois-Stalinist slogan of ``wait'' was designed to check the masses until the bourgeois state was supreme. Precisely for this reason, the bourgeois-Stalinist bloc and its Anglo-French allies had no intention of finishing off Franco or (more likely) making an armistice with him, until the counter-revolution had securely consolidated its power in Loyalist Spain.

We have commented on the failure of the People’s Front and its government to conduct revolutionary propaganda to disintegrate Franco’s forces. But in the field of purely military struggle, too, the government failed to fight Franco conclusively. More accurately, there is no wall between political and military tasks in civil war. Fearing the revolution more than Franco, the government was massing huge forces of picked soldiers and police in the cities, thereby withdrawing men and arms needed at the front. Fearing the revolution more than Franco, the government was pursuing a dilatory war strategy which could provide no decisive conclusion, while the counter-revolution was carried through. Fearing the revolution more than Franco, the government was subordinating the Asturian and Basque workers to the command of the treacherous Basque bourgeoisie who were soon to surrender the Northern front. Fearing the revolution more than Franco, the government was directly sabotaging the Aragon and Levante fronts held by the \CNT. Fearing the revolution more than Franco, the government was giving fascist agents (Asensio, Villalba, etc., etc.) the opportunity to betray Loyalist fortresses to Franco (Badajoz, Irun, Malaga).\footnote{The military policy of the government is analysed in detail in Chapters {\AlegreyaLF 15 and 16.}}

The counter-revolution dealt terrible blows to the morale of the anti-fascist troops. ``Why should we die fighting Franco when our comrades are shot by the government?'' This mood, so dangerous to the struggle against fascism, was prevalent after the May Days and was hard to fight.

In all these ways, therefore, the government policy was making easier the military inroads of Franco. The establishment of a workers’ republic would have put an end to all this treachery, sabotage, disruption of morale. Wielding the instrument of state planning, the Workers’ Republic would utilize as no capitalist regime could the full material and moral resources of Loyalist Spain.

Far from enabling the fascists to break through, only workers’ power could lead to the victory over Franco.

\subsection*{The Menace of Intervention}

The \CNT\ darkly referred to English and French warships appearing in the harbour on May 3, to plans for landing Anglo-French troops. 

\begin{quotation}
  In the case of a triumph of libertarian communism, it would have been crushed some time later by the intervention of capitalistic and democratic powers. (Garcia Oliver)
  \noclub
\end{quotation}

\CNT\ references to specific warships, to a specific plot, deliberately obscured the fundamental character of the issue: \emph{every social revolution must face the danger of capitalist intervention.} The Russian Revolution had to survive both capitalist-financed civil war and direct imperialist intervention. The Hungarian Revolution was crushed by intervention as well as by its own mistakes. When, however, the German and Austrian Social Democrats justified stabilization of their bourgeois republics because the Allied Powers would intervene against socialist states, revolutionary socialists and communists the world over---anarchists denounced the Kautskys and Bauers as betrayers, and rightly so.

The Austrian and German proletariat, the revolutionists said then, must take account of the possibility of defeat at the hands of Anglo-French intervention because revolutions always face that danger, and to wait for that hypothetical moment at which the Allies would be too preoccupied to interfere, meant to lose the conjuncture favourable to revolution. But the social democrats prevailed\dots\ and ended up in the concentration camps of Hitler and Schuschnigg.

Neither \CNT\ nor \POUM\ circles dare argue that there was any specific conjunctural situation which made capitalist intervention in May 1937 more threatening than at another time. The apologists merely refer to the intervention danger without adducing specific analysis. We ask: was intervention more dangerous in May 1937 than, for example, it was at the time of the April 1931 revolution? The advantages, for the workers, were all with May 1937. In 1931 the European proletariat was prostrated at the bottom of the well of the world crisis. If the German workers were not yet betrayed to Hitler by their leaders---without a fight---the French proletariat was as dormant as if exhausted by a dictator. France, contiguous to Spain, is decisive for Spain. And in May 1937 the French proletariat was begining the second year of the upsurge which opened with the revolutionary strikes of June 1936.

It is inconceivable that the millions of socialist and communist workers of France, already chafing against neutrality, and kept in line by their leaders only with the greatest difficulty, would permit capitalist intervention in Spain, whether by the French or any other bourgeoisie. The transformation of the struggle in Spain, from one for the preservation of a bourgeois republic, to one for the social revolution, would fire the French, Belgian, and English proletariat even more than had the Russian Revolution---for this time the revolution would be at their own doors!

In the face of an alert proletariat, what would the bourgeoisie do? The French bourgeoisie would open its borders to Spain, not for intervention but for trade, enabling the new regime to secure supplies---or face immediately a revolution at home. The Spanish Workers’ Republic would not, like Caballero and Negrin, aid and abet ``non-intervention!''

England, irrevocably tied to the fate of France, would be held back from intervention both by the whole weight of France and by her own working class for whom the Iberian Revolution would open a new epoch. Portugal would face immediate revolution at home. Germany and Italy would, of course, seek to increase their aid to Franco. But Anglo-French policy must continue to be: neither a Socialist Spain nor a Hitler-Mussolini Spain. Hoping to whittle down both sides eventually, Anglo-French imperialism would be forced to keep Italo-German intervention within such bounds as to prevent the Rome-Berlin axis from dominating the Mediterranean.

We, least of all, need to be told that all capitalist powers have in common, and seek in common, to destroy any threat of social revolution. Nevertheless, it is clear that two factors which saved the Russian Revolution from destruction by intervention would operate in May 1937: In 1917 the world working class, inspired by the revolution, forced a halt to intervention, while the imperialists could not sink their differences sufficiently to unite on a single plan for crushing the workers’ republic. With the European proletariat on the rise again, the imperialists would seek to quench the Spanish fire at their peril.

Yes, above all, we invoke the aid of the workers of the world! You Stalinists for whom the masses are no longer anything but sacrificial carcasses which you offer at the altar of an alliance with the democratic imperialists; ydu bureaucrats whose contempt for the masses, on whose backs you stand, makes you forget that these same masses carried through the October Revolution and the victorious civil war, on the moral and material capital of which you are still living, and which shrinks under your incompetent mismanagement! We know you do not like to be reminded that in 1919–1922 the world working class saved the Soviet Union from the imperialists. The revolutionary capacities of the proletariat are a factor which you have come to hate and fear, for they threaten your privileges.

Not we, but the Stalinists, believe possible peaceful cohabitation of capitalist and workers’ states. Certainly, European capitalism could not indefinitely bear the existence of a Socialist Spain. But the specific conjuncture in May 1937 was sufficiently favourable to enable a workers’ Spain to establish its internal regime and to \emph{prepare to resist imperialism by spreading the revolution to France and Belgium and then wage revolutionary war against Germany and Italy, under conditions which would precipitate the revolution in the fascist countries.}

This is the \emph{only} perspective of the revolution in Europe in this period before the next war, whether the revolution begins with Spain or France. Whoever does not accept this perspective, rejects the socialist revolution.

Risks?

\begin{quotation}
  World history would indeed be very easy to make if the struggle were taken up only on condition of infallibly favourable chances,  
\end{quotation}
wrote Marx while the Paris Commune still lived.
Clear-eyed, he saw\dots
\nowidow

\begin{quotation}
  the unfavourable accident \dots\ in the presence of the Prussians in France and their position right before Paris. Of this the Parisian workers were well aware. But of this the bourgeois \emph{canaille} of Versailles were also well aware. Precisely for that reason they presented the Parisians with the alternative of taking up the fight or succumbing without a. struggle. In the latter case the demoralization of the working class would have been a far greater misfortune than the fall of any number of “leaders”. The struggle of the working class against the capitalist class and its state has entered upon a new phase, with the struggle in Paris. Whatever the immediate results may be, a new point of departure of world-historic importance has been gained. (``Letter to Kugelmann,'' April 17, 1871)
\end{quotation}

Berneri had been right. Crushed between the Franco-Prussians and Versailles-Valencia, the commune of Catalonia could have struck a flame to light up the world. And under conditions so incomparably more favourable than those of the Commune!

We have sought to analyse as seriously as possible the reasons given by the centrist leadership for not waging a struggle for power against the counter-revolution. Being centrists and not brazen reformists, they have sought to justify their capitulation by references to the ``special,'' ``specific'' situation in Spain in the month of May 1937 but without providing us with the precise details. Upon examination we have found that, as usual in all such alibis, the references to the specific are false and conceal a fundamental retreat from the revolutionary path. Not mistakes in fact but differences in principle, in world and class outlook, separate the revolutionists from both the reformist and centrist leaders.

On Tuesday morning, May 4, the armed workers on barricades throughout Barcelona felt again, as on July 19, masters of their world. As on July 19, the terrified bourgeois and petty-bourgeois elements hid in their homes. The \PSUC-led trade unionists remained passive. Only part of the police, the armed guards of the \PSUC, and the armed Estat Catala hooligans were on the government barricades. These barricades were limited to the centre of the city, surrounded by the armed workers. The state of affairs is indicated by Companys’ first radio address: a declaration that the Generalidad was not responsible for the provocation at the Telefonica. Every outer section of the city, directed by its local defence committees and aided by \POUM, \FAI, and Libertarian Youth groups, was solidly in the control of the workers. There was almost no firing Monday night, so complete was the workers’ control. All that remained to establish supremacy was co-ordination and joint action directed from the centre \dots\ At the centre, the Casa. \CNT, the leaders, forbade all action and ordered the workers to leave the barricades.\endnote{For critical accounts of the events of the next days, I am indebted to two American comrades, Lois and Charles Orr (the latter was editor of the \POUM’s English-language \emph{Spanish Revolution}) and to the long and documented report of the Spanish Bolshevik-Leninists, appearing in \emph{La Lutte Ouvrière}, June 10, 1937.}

It was not the organizing of the armed masses that interested the \CNT\ leaders. What occupied them was interminable negotiation with the government. This was a game which suited the government perfectly: to hold back the leaderless masses in the barricades by deluding them with hopes that a decent solution would be found. The meeting at the Generalidad Palace dragged on until six o’clock in the morning. The government forces thus got enough breathing space to fortify the government buildings and, as the fascists had done in July, occupy the cathedral towers.

At eleven Tuesday morning, the functionaries met, not to organize the defence, but to elect a new committee to negotiate with the government. Now Companys introduced a new wrinkle. Of course, we can come to an amicable settlement; we are all anti-fascists, etc., etc., said Companys and Premier Tarradellas---but we cannot carry on negotiations so long as the streets are not cleared of armed men. Whereupon the Regional Committee of the \CNT\ spent Tuesday before the microphones calling the workers away from the barricades:

\begin{quotation}
  We appeal to all of you to put down your arms. Think of our great goal, common to all \dots\ Above all else, unity! Put down your arms. Only one slogan: We must work to beat fascism!
\end{quotation}

\emph{Solidaridad Obrera} had the effrontery to appear with the story of Monday’s attack on the Telefonica on page~8---not to alarm the militiamen at the front to whom went hundreds of thousands of copies---with no mention of the barricades erected, and no directives except ``keep calm.'' At five o’clock delegations from the National Committees of the \UGT\ and \CNT\ arrived from Valencia and jointly issued an appeal to the ``people'' to lay down their arms. Vasquez, \CNT\ National Secretary, joined Companys in the radio appeal. The night was spent in new negotiations---the government was always ready to make agreements involving the workers leaving the barricades!---out of which came an agreement for a provisional cabinet of four: one each from the \CNT, \PSUC, Peasant Union and Esquerra. The negotiations were punctuated with calls for authoritative \CNT\ leaders to go to points where the workers were on the offensive, as at Collblanc the workers had to be persuaded from carrying out occupation of the barracks. Meanwhile other calls were coming in---from the Leather Workers’ Headquarters, the Medical Union, the local centre of the Libertarian Youth---asking the Regional Committee to send help, the police were attacking\dots

Wednesday: neither the numerous radio appeals, the joint appeal of the \UGT-\CNT, nor the establishment of a new cabinet, had budged the armed workers from the barricades. On the barricades, anarchist workers tore up \emph{Solidaridad Obrera} and shook their fists and guns at the radios as Montseny---when Vasquez and Garcia Oliver had failed, she had been hurriedly called from Valencia---exhorted the barricades to disperse. The local defence committees reported to Casa \CNT: the workers will not leave without conditions. Very well, we give them conditions. The \CNT\ radioed the proposals it was making to the government: hostilities to cease, every party to keep its positions, the police and civilians fighting on the side of the \CNT\ (i.e., non-members) to retire altogether, the responsible committees to be informed at once if the pact is broken anywhere, solitary shots not to be answered, the defenders of union quarters to remain passive and await further information.

The government soon announced its agreement with the \CNT\ proposals, and why not? The government’s sole objective was to end the fighting of the masses, the better to break their resistance for all time. Furthermore, the ``agreement'' pledged the government to nothing. The control of the Telefonica, disarming of the masses, were---not accidentally---unmentioned. The agreement was followed during the night by orders from the local \CNT\ and \UGT\ (the latter Stalinist-controlled, remember) to return to work.

\begin{quotation}
  The anti-fascist organizations and parties in session at the Palace of the Generalidad have solved the conflict that has created this abnormal situation,
\end{quotation}
said the joint manifesto.

\begin{quotation}
  \noindent
  These events have taught us that from now on we shall have to establish relations of cordiality and comradeship, the lack of which we all regretted deeply during the last few days.
\end{quotation}

Nevertheless, as Souchy admits, the barricades remained fully manned Wednesday night.

But on Thursday morning, the \POUM\ ordered its members to leave the barricades, many of them still under fire. On Tuesday, the manifesto of the Friends of Durruti, hitherto cool to the \POUM, had hailed its joining the barricades as a demonstration that it was a ``revolutionary force.'' Tuesday’s \emph{La Batalla} had remained within the limits of the theory that there should be no insurrectionary overthrow of the government during the civil war but had called for defence of the barricades, the dismissal of Salas and Ayguade, withdrawal of the decrees dissolving the worker-patrols. Limited as this programme was, it contrasted so with the \CNT\ Regional Committee’s appeal to desert the barricades that the prestige of the \POUM\ soared among the anarchist masses. The \POUM\ had an unparalleled opportunity to come to the head of the movement.

Instead, the \POUM\ leadership, once again, put its fate in the hands of the \CNT\ leadership. \emph{Not} public proposals to the \CNT\ for joint action made before the masses, proposals which would give the inchoate rebellion a focus of specific steps to demand of their leaders---in a whole year the \POUM\ had, fawningly deferential to the \CNT\ leaders, not made a single united front proposal of this specific character---but a behind-the-scenes conference with the \CNT\ Regional Committee. Whatever the \POUM\ proposals were, they were rejected. \emph{You don’t agree? Then we shall say nothing about them.}

And the next morning, May 5, \emph{La Batalla} had not a word to say about the \POUM’s proposals to the \CNT, about the cowardly behaviour of the \CNT\ leaders, their refusal to organize the defence, etc.\endnote{The English language bulletin of the \POUM, \emph{Spanish Revolution} (May 19, 1937), says: ``Caught up in the reins of the government (the \CNT) tried to straddle the fence with a `union' of the opposing forces \dots\ The attitude of the \CNT\ did not fail to bring forth resistance and protests. The `Friends of Durruti' group brought the unanimous desire of the \CNT\ masses to the surface but it was not able to take the lead\dots\ The workers who were deeply wounded by the capitulation of their trade union federation, are now looking for a new lead in other directions. The \POUM\ should provide it for them.''

These radical words were for export purposes only. Nothing like them appeared in the regular press of the \POUM. In general \emph{Spanish Revolution} has given English readers, who could not follow the \POUM’s Spanish press, a distorted picture of the \POUM’s conduct; it has been a ``left face.'' This is said without any intention of reflecting on the revolutionary integrity of Comrade Charles Orr, its editor, who can scarcely be held responsible for the disparity between the English bulletin and the voluminous Spanish press of the \POUM.} Instead, ``the Barcelona proletariat has won a partial battle against the counter-revolution.''

And, twenty-four hours later, ``the counter-revolutionary provocation having been repulsed, it is necessary to leave the streets. Workers, return to work.'' (\emph{La Batalla}, May 6).

The masses had demanded victory over the counter-revolution. The \CNT\ bureaucrats had refused to fight. The centrists of the \POUM\ thus bridged the gap between masses and bureaucrats---by assuring them that the victory had already been achieved!

The Friends of Durruti had forged to the front on Wednesday, calling upon the \CNT\ workers to repudiate the desertion orders of Casa \CNT\ and continue the struggle for workers’ power. It had warmly welcomed the collaboration of the \POUM. The masses were still on the barricades. The \POUM, numbering at least thirty thousand workers in Catalonia, could tip the quivering scales either way. Its leadership tipped the scales for capitulation.

One more terrible blow against the embattled workers: The Regional Committee of the \CNT\ gave to the entire press---Stalinist and bourgeois included---a denunciation of the Friends of Durruti as \emph{agents provocateurs}; it was, of course, prominently published everywhere on Thursday morning. The \POUM\ press did not defend the left-wing anarchists against this foul slander.

\dinkus

Thursday was replete with instances of the ``victory'' in the name of which the \POUM\ called the workers to leave the barricades.


  \chapter{The Dismissal of Largo Caballero}

\lettrineT{he defeat of} the Catalonian proletariat marked a new stage in the advance of the counter-revolution. Hitherto, the reaction had developed under cover of collaboration with the \CNT\ and \UGT\ leaders, and even from September to December in the Generalidad with the \POUM\ leaders. Thus, the gap between the openly bourgeois programme of the bourgeois-Stalinist bloc and the revolutionary aspirations of the masses had been obscured by the centrists.\footnote{This is the Marxian term employed to describe the variety of political formations which are not revolutionary but which also do not proclaim the class-collaboration doctrines of classical reformism.} Now the moment arrived for the bourgeois-Stalinist bloc to dispense with the centrists.

The process is a familiar one in recent history. When blows to the left have sufficiently strengthened the right, the latter are then enabled to turn against the centrists whose services, heretofore, had been indispensable in crushing the left. The result of the suppression of the revolutionary workers is a government far to the right of the régime that suppressed them. Such was the result of the bloody suppression of the Spartacists in 1919 by Noske and Scheidemann. Such was the aftermath of the ``stabilization'' of Austria by Renner and Bauer. It was now the turn of the Spanish centrists to pay the price for having abetted the crushing of the Catalonian proletariat.

The first item of the bill presented by the Stalinists to the Valencia cabinet was the complete suppression of the \POUM. Why the \POUM? Like all renegades, the Stalinists understand the dynamics of revolutionary development better than their allies who have always been reformists. In spite of its vacillating policies, the \POUM\ had in its ranks many revolutionary fighters for the interests of the proletariat. Even the \POUM\ leaders, unready for revolution, would be driven to resist the naked counter-revolution. Stalin has understood that even the capitulators, the Zinovievs and Kamenevs, will be a danger on the day the masses rebel. Stalin’s formula is: wipe out every possible focus, every capable figure, around whom the masses can rally. That bloody formula, already carried out in the August and January trials in Moscow, was now applied to Spain and the \POUM.

The left socialists recoiled. One of their organs, \emph{Adelante} (of Valencia) said editorially on May 11:

\begin{quotation}
  If the Caballero government were to apply the measures of repression which the Spanish section of the Comintern is trying to incite, it would approximate a Gil Robles or Lerroux government; it would destroy working-class unity and expose us to the danger of losing the war and wrecking the revolution~\dots
  
  A government composed in its majority of people from the labour movement cannot use methods reserved for reactionary and fascist-like governments.
\end{quotation}

The cabinet convened on May 15, and Uribe, the Stalinist Minister of Agriculture, bluntly put the question to Caballero: \emph{was he prepared to agree to the dissolution of the \POUM, confiscation of its broadcasting stations, presses, buildings, goods, etc., and imprisonment of the Central Committee and local committees which had supported the Barcelona rising?}

Federica Montseny awoke sufficiently to the occasion to present a dossier to prove that a plan had been prepared, both in Spain and abroad, to strangle the war and revolution. She accused Lluhi y Vallesca and Gassol (Esquerra), and Comorera (\PSUC), together with a Basque representative, of having participated in a meeting in Brussels at which it was agreed to annihilate the revolutionary organizations (\POUM\ and \CNT-\FAI) in order to prepare for ending the civil war by the intervention of ``friendly powers'' (France, England).

Caballero declared he could not preside over repression against other workers’ organizations, and that it was necessary to smash the false theory that there had been a movement against the government in Catalonia, much less a counter-revolutionary movement.\endnote{On May 4, the Valencia \emph{Adelante} (obviously speaking for Caballero) solved the problem of which side of the barricades to support by denying the real meaning of the struggle: ``We understand that this is not a movement against the legitimate power \dots\ And even if it were a revolt against the legitimate authority, and we do not admit that such was the case, instead of being merely an inopportune and poorly prepared collision between the organizations with different orientations and political and trade union interests opposed to each other within the general anti-fascist front in which the proletarian groups of Catalonia move, the responsibility for the consequences would have to be charged, naturally, to those who provoked the collisions.''}

As the Stalinists continued to press their demands, Montseny sent for a package containing hundreds of scarves adorned with the shield of the monarchy. Thousands of these had been found in the hands of the \PSUC\ provocateurs and Estat Catala members, who were to have planted them in \POUM\ and \CNT\ buildings. The two Stalinist ministers rose and rushed out of the meeting, and the ministerial crisis had begun.

Caballero looked at the others. He wanted them to state their positions. The bourgeois and Prieto ministers solidarized themselves with the Stalinists and went out. Such was the last meeting of the Caballero cabinet.

\dinkus

The outlawry of the \POUM\ was the first demand of the counter-revolution, but the Stalinists followed it up with other basic demands which Caballero and the left socialists would not accept responsibility for.
Friction between the Stalinists and left socialists had, indeed, been developing for some months.

A stealthy campaign against Caballero himself had been waged in the Stalinist press since March, when the flow of adulatory telegrams to the ``leader of the Spanish people'' from ``the workers of Magnitogorsk'' had been turned off like a faucet. The Stalinist campaign had been the subject of comment in the organs of the \CNT\ and \POUM, and of resentful polemics in the left socialist press. The befuddled anarchists interpreted the Stalinist campaign in terms of the original sin of politics: this was the way political parties acted toward each other.

The \POUM\ sought to make easy capital among socialist workers by berating the Stalinists for attempting to absorb the socialists. Juan Andrade, the \POUM\ commentator, saw more clearly, recognizing that Caballero was resisting the Anglo-French directives in their fullest implications. But the main \POUM\ line of shouting ``absorption'' lost it the opportunity to make use of the real conflicts between Caballero and the bourgeois-Stalinist bloc. For there were real conflicts. Not, of course, as basic as the conflict between reform and revolution; but important enough so that a bold revolutionary policy could have driven a wedge between the Stalinists and Caballero’s mass base, could have aroused the \UGT\ workers to the meaning of the road which Caballero had followed for eight months.

Stalinist inroads into Caballero’s ranks were a fact. It is a familiar enough phenomenon in the labour movement that when two organizations follow the same policy, the one with the stronger apparatus will proceed to absorb the other. By holding identical views with the Stalinists on the People’s Front, winning the war before making the revolution, conciliating foreign opinion, building a regular bourgeois army, etc., Caballero had ceased to differ from Stalinism in the eyes of the masses.

With the native Stalinist apparatus tremendously reinforced by Comintern functionaries and funds---the International Brigades came in with hundreds of such functionaries attached to them---the Stalinists were in a position to recruit at Caballero’s expense.

Particularly was this true in the youth. The socialist youth had been Caballero’s strongest support but its fusion with the Stalinist youth had left him the loser, although the latter had not had one-tenth the membership of the socialist youth. The usual Stalinist methods of corruption---trips to Moscow, adulatory relationships with the Russian and French \textsc{yci}, the offer of posts in the Central Committee of the party, etc.---had been successful.

Shortly after fusion, the socialist youth leadership had entered the Communist Party and the ``united'' youth organization came under rigidly Stalinist control. Dissenting branches were ``reorganized'' and left wingers expelled as Trotskyists. Caballero was scarcely in a position to protest at the outcome, having himself connived at the bureaucratic method of fusion, without a congress of the socialist youth having been held to pass on the decision.

Under the slogan of ``unifying the whole youth generation,'' the Stalinist leadership bulwarked itself by recruiting indiscriminately anyone who could be persuaded to accept a card. Santiago Carillo at a Central Committee Plenum of the Communist Party shamelessly advocated recruiting of ``fascist sympathizers'' among the youth. Leaning on backward elements, including many Catholics, the Stalinists were able for a time to muzzle the thousands of left wingers still in the youth organization.

Nevertheless, Caballero’s losses to the Stalinists had not led him to break with them. Absorption of his following only made him feel weaker and make further concessions.

Only when Caballero discovered that Stalinist inroads were less serious than he had supposed, and that he was more likely to lose his following to the left than to Stalinism, did he come into serious conflict with the Stalinists. The two biggest sections of the socialist youth, the Asturian and Valencian organizations, denounced the Stalinist top leadership and refused to accept seats in the ``united'' National Committee. In the delegates’ meeting of the Madrid \UGT, the Caballero ticket carried all eight seats on the Municipal Council allotted to the \UGT, against a Stalinist ticket. In the Asturian Congress of the \UGT, the Caballero group held 87,000 votes against 12,000 for the Stalinists. These indices, shortly before the government crisis, showed that Caballero could have the dominant following in the \UGT, and that he would have to pacify his following and not the Stalinists in the coming period.

There was one step, above all, which Caballero could not accept responsibility for: the final moves in smashing the workers’ control of the factories. Whatever else happened, the \UGT\ masses were firmly convinced; they would never give up the factories. The Madrid organ of the \UGT\ declared repeatedly: ``The ending of the war must signify also the ending of capitalism.''
\nowidow

\begin{quotation}
  That the exploiters of all life cease to be masters of all the means of production, it has sufficed that the people take up arms in the struggle for national independence. From the great financial establishments to the smallest shops, they are, in actual fact, in the hands and under the direction of the working class \dots
  
  What vestiges remain of the old economic system? The revolution has eliminated all the privileges of the bourgeoisie and the aristocracy. (\emph{Claridad}, May 12, 1937)
\end{quotation}

\emph{Claridad},\endnote{With the Negrin cabinet, \emph{Claridad} passed into Stalinist control still continuing to call itself ``organ of the \UGT,'' although twice repudiated by the National Executive Committee.} indeed, continually studded its pages with quotations from Lenin. That these quotations were often enough damning commentary on Caballero’s political conceptions hardly requires documentation. Quotations appeared from \emph{The State and Revolution}, while Caballero strengthened and rebuilt the bourgeois state apparatus which would inevitably attempt to wrest the factories from the workers. But, unless he was prepared to lose the support of the masses of the \UGT, Caballero could not himself participate in wresting the factories from the workers. Caballero was just enough of a labour politician to recognize that the state he had himself revived was alien to the workers and that the bourgeois-Stalinist slogan of ``state control of the factories'' meant smashing the power of the factory committees.

We may sum up the fundamental differences between Caballero---i.e., the bureaucracy of the \UGT---and the bourgeois-Stalinist bloc in this way: Caballero wanted a bourgeois democratic republic (with some form of workers’ control of production co-existing with private property), victorious over Franco. The bourgeois-Stalinist bloc was ready to accept whatever Anglo-French imperialism proposed, which, at the stage of the overthrow of Caballero, was a stabilized bourgeois regime based on participation in the regime of the capitalist-landlord forces behind Franco, parliamentary in form, but actually Bonapartist since unacceptable to the masses.
\nowidow

Caballero’s perspective was not so fundamentally different from that of the bourgeois-Stalinist bloc, that they could not go along together for a considerable distance. They had gone along together for eight months. Was May 15 the correct moment for the rightists to break with Caballero? Should not the bourgeois-Stalinist bloc have bided its time for a few more months while the army and police were still further strengthened as bourgeois institutions? Should they not have carried the \CNT\ ministers deeper and deeper into the swamp? Were they not risking a regroupment of forces by forcing the two mass labour organizations out of the cabinet? Were not the Stalinists too nakedly revealing their reactionary role by becoming the only labour group, apart from the long-hated Prieto group, to participate in the government?

The Stalinists probably overestimated their ability to secure expressions of support for the new cabinet from enough \UGT\ unions to obscure the fact that the labour unions as a whole were opposed to the new government. Even in the bureaucratically controlled \UGT\ of Catalonia, the Stalinists proved unable to prevent many of the most important unions from declaring support for Caballero. Elsewhere the Stalinists got only a handful of unions to sanction the dismissal of Caballero.

If, however, the Stalinists miscalculated their ability to provide a labour ``front'' for Negrin, they were undoubtedly correct in other calculations. For them, the Barcelona events revealed that the \CNT\ ministers were no longer of use in keeping the \CNT\ masses in line; the fighting of May 3--8 had revealed the chasm between the leaders and the masses of the \CNT. Further governmental participation of the \CNT\ would provide little brake to the resistance of the masses and, on the other hand, could only speed up a split between these leaders and the masses. For the next period, the Olivers and Montsenys were more useful as a ``loyal opposition'' outside the government. As oppositionists they could regain control over their following, yet their opposition would be of a kind that would not unduly embarrass the Negrin government.

As for the opposition from Caballero, its temper and quality had already been experienced: his ``revolutionary criticism'' of the People’s Front government of February–July 1936, and his even more radical declarations during the first war cabinet of July 19--September 4, 1936. In those periods Caballero had channelized discontent---and then had entered the government himself. If unforeseeable obstacles arose to endanger the government, the bourgeois-Stalinist bloc could always return to the status of May 15, for the centrists were demanding nothing more than that: ``One cannot govern without the \UGT\ and \CNT'' was the slogan of Caballero and the \CNT\ leaders. Meanwhile, it was safe to predict that Caballero’s opposition would not take the form of revival of the network of workers’ committees and the co-ordinating of them into soviets---and only along that road did the bourgeois-Stalinist bloc have anything serious to fear.

If dropping the \UGT\ and \CNT\ involved no serious dangers, it offered immediate and far-reaching advantages for the bourgeois-Stalinist bloc. Their immediate requirements were:

\begin{enumerate}
  \item Complete control of the army. The mobilization and army reorganization decrees had been carried out by Caballero, as Minister of War, to a considerable extent. The regiments formed of drafted soldiers were built entirely on the old bourgeois model, largely officered by old army officers or the hand-picked graduates of the government-controlled training schools. Any attempts among the conscripts at election of officers’ and soldiers’ committees had been stamped out. But the workers’ militias which had carried the brunt of the struggle during the first six months were not yet all ``reorganized;'' their masses resisted fiercely any systematic replacement of their officers, most of whom came from their own ranks. Even on the Madrid front the \CNT\ and \UGT\ militias, despite partial reorganization, retained most of their former officers and continued to print their own political papers at the front. On the Catalan fronts, the anarchist militias refused to honour the decrees which the \CNT\ ministers had signed. Equally important, Caballero became alarmed enough, after the loss of Malaga, to arrest General Asensio and the Malaga commander, Villalba, for treason, and cleaned out of the staffs many bourgeois friends of Prieto and the Stalinists. Caballero’s caution thereafter in army reorganiation was a serious obstacle to the Prieto-Stalinist programme. For a ruthless reorganization of the militias into bourgeois regiments, officered by bourgeois appointees in consonance with the old military code, and a purge of radical army leaders thrown up by the July days, it was necessary to wrest the army entirely from Caballero.
  
  \item The War Ministry offered the best vantage point from which to begin wresting control of the factories from the workers. In the name of the exigencies of the war, the ministry could step in and break the hold of the workers in the most strategic industries: railroad and other transportation, mining, metals, textiles, coal and oil. The Stalinists had already begun to prepare for this in April by a barrage against the war-supply factories. Unfortunately for the Stalinists, they had organized this campaign (a persistent weakness of campaigns carried out obediently under orders of Comintern representatives from Moscow) at a time when the atmosphere was not yet propitious for a pogrom. Their charges were refuted by joint statements of the \CNT\ and \UGT\ organizations in the Catalonian factories involved and, as we have seen, were disavowed even by Premier Tarradellas who, as Minister of Finance, disbursed to the factories the funds received from the Valencia treasury. It was clear, then, that this campaign could not be successfully consummated from outside but that the bourgeois-Stalinist bloc needed the Ministry of War to further their inroads into workers’ control of the factories.
  
  \item In Caballero’s cabinet the Ministry of the Interior, which controlled the two main police bodies (Assault and National Republican Guard) and the press, was presided over by Angel Galarza, a member of the Caballero group. The revolutionary workers had sufficient reason to denounce his policies. Above all, Caballero and Galarza had sanctioned the decree forbidding police to join political and trade union organizations; and quarantining the police against the labour movement could only mean, inevitably, pitting them against the labour movement.
  
  Nevertheless, the Caballero group recognized that repression of the \CNT\ would be a fatal blow to the Caballero base, the \UGT, and Caballero needed the \CNT\ as a counterweight to the bourgeois-Stalinist bloc. Galarza had sent five thousand police to Barcelona, but had refused to carry out the Prieto-Stalinist proposals for complete liquidation of the \POUM\ and reprisals against the \FAI-\CNT. Here again the Caballero group had built the instrument for hostilities against the workers, but drew back at carrying out its complete implications. Once Caballero and Galarza had induced the Generalidad, during the Barcelona fighting, to extend control of public order by the central government to Catalonia, the moment was ripe to oust Galarza, in order for the Stalinists to secure control of the police and press in Catalonia and elsewhere.
  
  \item The Prieto-Stalinist programme for conciliation with the Cath\-ol\-ic Church---halfway house to conciliation with Franco---was being resisted by Caballero. Backbone of the monarchy and of the \emph{bienio negro}, the two black years of Lerroux--Gil Robles, the churches had been the fortresses of the fascist uprising. To be a member of a labour organization has always, in Spain, had to mean to be against the church, for the official catechism has declared it to be mortal sin to ``vote liberal.'' The masses had spontaneously forced the closing of all Catholic churches in July. One could scarcely propose a more unpopular measure than to permit the church organization to operate freely again and in the midst of civil war! Furthermore, it was actually dangerous to the anti-fascist movement; for with the Vatican on the side of the Franco regime, it would inevitably use the church organization to help Franco. Yet this was the proposal of the Basque Government and its allies, Prieto and the Stalinists. Caballero had done many things to curry favour with the Anglo-French imperialists; but to permit the church organization to operate freely in the midst of the civil war was too much for him.
\end{enumerate}

\dinkus

These causes of conflict between Caballero and the reactionary bloc arc clearly revealed in the demands expressed by the various parties on May 16, during the customary visits to President Azaña, to acquaint him with the position of each group on the ministerial crisis.\endnote{The statements of the parties are published in the press.}

Manuel Cordero, spokesman for the Prieto Socialists, Piously declared his organization stood for a government including all fac\-tions---but ``I have insisted very particularly on the necessity of an absolute change in the policy of the Ministry of the Interior.''

Pedro Corominas, for the Catalan Esquerra, declared: ``Whatever be the solution that is adopted, it will be necessary to strengthen it and do away with difficulties of personal origin, by greater and more frequent contact with the Cortes of the Republic.'' In other words, the government policy should be dictated by the remnants of the Cortes elected in February 1936 under an electoral agreement which gave the overwhelming majority of the Cortes to the bourgeois parties!

Manuel Irujo, for the Basque capitalists, spoke fairly bluntly:

\begin{quotation}
  I have advised His Excellency for a government of national concentration presided over by a socialist minister who has the confidence of the [bourgeois] republicans. Since Caballero \dots\ has lost the political confidence of the groups of the Popular Front, it would be advisable to form a government, in our opinion, of Negrin, Prieto, or Besteiro, with the co-operation of all the political and trade union organizations which would accept the proposed bases.
  
  As specific demands, I feel obliged to make two, at present. The first is the necessity of proceeding, with such guarantees and restrictions as the war and public order dictate, to the re-establishment of the constitutional regime of liberty of conscience and religion.
  
  The second demand refers to Catalonia. The Catalan republicans would have preferred earlier and effective intervention by the Government of the Republic in assuming control of public order in support of the Generalidad. What is more, in now carrying out these duties, I feel that it is an unescapable duty of the Government that it liquidate to the bottom the problem which disturbs Catalonian life, firmly doing away with the causes of the disorder and insurrection, be they circumstantial or endemic \dots
\end{quotation}

It was to this Irujo that the Prieto-Stalinist bloc were soon to entrust \dots\ the Ministry of Justice.

Salvador Quemades, for the Left Republicans, Azaña’s own party, required that the next cabinet ``must have a decided policy in the matter of public order and of economic reconstruction, and that the commands of war, marine and the air force be placed in a single hand.'' Prieto was already Minister of Marine and Air. This meant adding to his posts that of control of the army (as was done).
  \chapter{``El Gobierno de la Victoria''}

\lettrineL{a Pasonaria} christened the new cabinet ``the government of victory.'' We have made up our minds,'' she said, ``to win the war quickly, though that victory cost us an argument with our dearest comrades.'' The Stalinists launched a worldwide campaign to prove that victory had been held up by Caballero and that it would now be forthcoming.

The annals of the Negr\'in government, however, proved to be not the record of military victory, nor even of serious attempts at military victory, but of ruthless repression of the workers and peasants. That reactionary course was dictated to the government by the Anglo--French rulers to whom it looked for succor. The spokesman for the Quai~d’Orsay, \emph{Le Temps}, indicated the real meaning of the ministerial crisis:

\begin{quotation}
  The republican government of Valencia has reached the point where it must decide. It can no longer remain in the state of ambiguity in which it has hitherto lived. It must choose between democracy and proletarian dictatorship, between order and anarchy. (May 17)
\end{quotation}

The next day the Negr\'in cabinet was formed. \emph{Le Temps} approved but peremptorily pointed the road the new regime must resolutely travel:

\begin{quotation}
  It would be too early to conclude that the orientation in Valencia is toward a more moderate government determined to free itself finally from the control of anarcho-syndicalists. But this is an attempt which, in the end, will have to be made no matter what the resistance of the extremists may be.
\end{quotation}

Clear directives, indeed!

\medskip

The government, wrote an ardent sympathizer with its reactionary course, the \emph{New York Times} correspondent, Matthews, ``intends to use an iron hand to maintain internal order.''

\begin{quotation}
  \noindent
  —\,By so doing, the government hopes to win the sympathy of the two democracies that mean most to Spain~-- Great Britain and France~-- and to retain the support of the nation that has been most helpful, Russia. The government’s main problem now is to pacify or squash the anarchist opposition. (May~19, 1937).
\end{quotation}

``In a word, the government unloosed a completely repressive machinery without any regard for the state of the war, or the requirements of keeping up the morale of the war,'' as a \FAI\ statement of July 6 put it.

``Anarchists are being eliminated as an active factor. The Caballero Socialists, if they persist in their present tactics, may be outlawed within three months,'' wrote the Stalinist, Louis Fischer. (\emph{The Nation}, July 17).

In Caballero’s cabinet, Garc\'ia Oliver, the ``one hundred percent anarchist,'' had labored mightily, fashioning democratic tribunals and judicial decrees, while the counterrevolution advanced behind him. The Generalidad had used Nin for the same purpose during the early months of the revolution. Now the government named as its Minister of Justice, the Basque capitalist and devout Catholic, Manuel Irujo. That such a man could hold this office meant: the time for pretending is over. Irujo in 1931 had voted against adoption of the republican constitution as a radical and atheistic document. Was he not then just the man for the Minister of Justice?

Irujo’s first step was to dismantle the popular tribunals which, each constituted by a presiding judge and fifteen members designated by the different antifascist organizations, had been set up after July~19, 1936. The \FAI\ members were now barred from the tribunals~-- by the device of, decreeing that only organizations legal on February~16, 1936, could participate. The \FAI, of course, had been outlawed by the \emph{bienio negro}!

Most of the presiding judges had been left-wing attorneys. Roca, former subsecretary of the Ministry, has since told how, in September 1936, the Ministry of Justice had called a meeting of the old judges and magistrates and had asked for volunteers to go to the provinces and set up the tribunals. Not one would volunteer. They knew the fascists would have to be convicted. Now the tribunals were cleansed of the left-wing attorneys who were replaced by the once-reluctant judges, for the tribunals no longer were to ferret out fascists but to prosecute workers. \emph{Daily bulletins} listing fascists and reactionaries put at liberty were issued by Irujo’s Ministry.

On June 23, the government decreed special courts to deal with sedition. Among ``seditious acts'' were included: ``giving military, diplomatic, sanitary, economic, industrial or commercial information to a foreign state, armed organization or private individual,'' and all offenses ``tending to depress public morale or military discipline.'' The judges were to be appointed by the Ministries of Justice and Defense, empowered to sit in secret session and to bar any third parties. The decree concludes:

\begin{quotation}
  \noindent
  —\,Attempted or frustrated offenses, conspiracies and plans, as well as complicity in sheltering of persons subject to this decree, may be punished in the same way as if the offenses had actually been committed. Whoever, being guilty of such offenses, denounces them to the authorities shall be free of all punishment. Death sentences may be imposed without formal knowledge of the cabinet.
\end{quotation}

The confession clause, the punishment for acts never committed, the secret trials, were translated directly from Stalin’s laws. The sweeping definition of sedition made treason of any opinion, spoken or written or indicated by circumstantial evidence, which was construable as critical of the government. Applicable to any worker who agitated for better conditions, to strikers, to any governmental criticism in a newspaper, to almost any statement, act or attitude other than adoration of the regime, this decree was not only unprecedented in a democracy, it was more brazen than Hitler or Mussolini’s juridical procedure.

On July 29, the Ministry of Justice announced that trials under this decree were being prepared for ten members of the Executive Committee of the \POUM\kn. These men had been arrested on June 16 and 17~-- before the new decree. That meant that the decree, to cap everything else, was an \emph{ex post facto} law, punishing crimes allegedly committed before the law was passed! Thus, the most unquestioned juridical principle of modern times was expressly repudiated.

\medskip

Irujo sponsored another decree, adopted and issued by the government on August 12, which declared:

\begin{quotation}
  Whoever censures as fascist, as traitor, as anti-rev\-o\-lu\-tion\-ary, a given person or group of persons, unreasonably or without sufficient foundation, or without the [court] authority having pronounced sentence on [the accused]\dots.
  
  Whoever denounces a citizen for being a priest or for administering the sacrament \dots\ causes an unnecessary and disruptive disturbance of public order when not committing an irreparable crime worthy of penal punishment.
\end{quotation}

This decree not only outlawed sharp ideological criticism of anybody in the governmental bloc, but also put an end to ferreting out of fascists by the workers. It also ended all forms of surveillance of the Catholic priesthood~-- just after the Vatican had openly thrown full support to Franco. Denunciations ``without the court authority having pronounced sentence'' in practice applied only against criticism from the left. The Stalinists continued, of course, to denounce the \POUM\ as fascists, though no sentence had been passed.

\begin{sloppypar}
Press censorship operated under a system which not only destroyed free criticism but required that the very acts of censorship be concealed from the people. Thus, on August 7, \emph{Solidaridad Obrera} was suspended for five days for disobeying censors’ orders, the specific act of disobedience being~-- according to G\'omez, General Delegate of Public Order in Barcelona, who had given the order that ``they should not publish white spaces.\kn\kn'' That is, deletions by the censor who worked from galleys must be hidden from the masses by inserting other material! As a silent protest, the \CNT\ press had been leaving censored spaces blank.
\end{sloppypar}

\medskip

On August 14, the government issued a decree outlawing all press criticism of the Soviet government:

\begin{quotation}
  With repetitions that permit divining a deliberate plan of offending an exceptionally friendly nation, thereby creating difficulties for the government, various newspapers have occupied themselves with the \USSR\ unsuitably\dots.
  This absolutely condemnable license should not be permitted by the council of censors\dots. The disobeying newspaper will be suspended indefinitely, even though it may have been passed by the censor; in that case the censor who reads the proofs is to be held for the Special Tribunal charged with dealing with crimes of sabotage.
\end{quotation}

The censorship decrees no longer referred to the radio. For, on June~18, police detachments had appeared at all the radio stations belonging to the trade unions and political parties and closed them down. Thenceforward, the government monopolized radio broadcasting.

One of the most extraordinary uses of the press censorship came when the Stalinist--Prieto bloc on October~1 split the \UGT\ by a rump meeting of some unions which declared the Caballero-led Executive Committee deposed. While the new ``Executive'' freely published a stream of abusive declarations, the statements of the Caballero Executive were cut to pieces, as were the headlines in the \CNT\ press referring to it as the rightful Executive. Formal protests of the \CNT\ press against the government, thus taking sides in the inner-union fight, were fruitless.

Despite terrible instances~-- in almost every city captured by the fascists~-- of Assault and Civil Guards in large numbers going over to the fascists during the siege, the Ministry of the Interior proceeded to cleanse the police, not of the old elements, but of the workers sent in by their organizations after July~19. Examinations were decreed for all who entered the services in the past year. The Security Councils, formed by the antifascists among the police to clean out fascist elements, were ordered dissolved. More, the director general of the police, the Stalinist, Gabriel Mor\'on, ordered the ranks not to make denunciations of fascist suspects in the police, on pain of dismissal. (\emph{\CNT,} September 1).

\index{Private property}
Held to a slower pace until the political preconditions had been more fully achieved, the economic counterrevolution was now sped up. In agriculture, the road to be followed had been mapped by the very first decree, October~7, 1936, which merely confiscated estates of fascists, leaving untouched the system of private property in land, including the right to own large properties and to exploit wage labor.

\index{Collectivization}
\begin{sloppypar}
Despite the decree, however, collectivized agriculture became wide\-spread during the first months of the revolution. The \UGT\ was at first unfriendly to the collectives, and changed its attitude only after the movement developed strong roots in its own ranks. Several factors explained the speedy development of collectivized farming.
\end{sloppypar}

\index{Peasantry}
Unlike the old Russian \emph{moujik,} the Spanish peasants and agricultural workers had built trade unions for decades, and had provided considerable sections of the membership of \CNT\ and \FAI, \UGT\kn, \POUM\ and the Socialist Party. This political phenomenon flowed in part from the economic fact that the division of the land was even more unequal in Spain than in Russia, and almost the entire Spanish peasantry was dependent partially or wholly on wage labor on the big estates. Hence, even those with a bit of land were weakened in the peasants’ traditional preoccupation with his own plot of ground. Collective labor also derived strength from the almost universal necessity for joint work in providing water for the dry land.

To these factors were added the enthusiastic aid given to the collectives by many factories, providing equipment and funds, the equitable purchasing of produce from the collectives by the workers’ supply committees and the cooperative markets, the friendly collaboration of the collectivized railroads and trucks in bringing the produce to town.

Another important factor was the peasant’s realization that he no longer stood alone. ``If, in some locality, a crop is lost or greatly reduced because of a long drought, etc.,\kn\kn''
wrote the head of the \CNT\ agrarian federation in Castille, speaking for 230 collectives, ``our peasants don’t have to worry, don’t have to fear hunger, for the collectives in the other localities or regions consider it their duty to help them out.\kn\kn'' Many factors thus joined to encourage the swift development of collective agriculture.

But with the Stalinist Uribe’s assumption of the Ministry of Agriculture, first in Caballero’s cabinet and then in Negr\'in’s, the weight of the government was thrown against the collectives.

``Our collectives did not receive any sort of official aid. On the contrary, if they received anything at all, it was obstruction and calumnies from the Minister of Agriculture and from the majority of institutions that depend upon this Minister,'' reported the \CNT’s Castilian agrarian federation. (\emph{Tierra y Libertad,} July~17).

\medskip

Ricardo Zabalza, national head of the \UGT’s Peasant and Land\-work\-ers’ Federation declared:

\begin{quotation}
  The reactionaries of yesterday, the erstwhile agents of the big landowners are given all sort of assistance by the government while we are deprived of the very minimum of it or are even evicted from our small holdings\dots.
  
  They want to take advantage of the fact that our best comrades are now fighting on the war fronts. Those comrades will weep with rage when they find, upon leave from the war fronts, that their efforts and sacrifices were of no avail, that they only led to the victory of their enemies of old, now flaunting membership cards of a proletarian organization [the Communist Party].
\end{quotation}

These agents of the big landowners, the hated \emph{caciques}~-- overseers and village bosses~-- had been the backbone of the political machine of Gil-Robles and the landowners. Now they were to be found in the ranks of the Communist Party. Even such an outstanding chieftain of Gil-Robles’ machine as the secretary of the \CEDA\ in Valencia had survived the revolution\dots\ and joined the Communist Party.

Uribe justified the assault on the collectives by claiming that unwilling peasants were being forced to join them. One need scarcely comment on the irony that a Stalinist should complain of forced collectivization, after the Draconian slaughters and exiles of the ``liquidation'' of the Russian \emph{kulaks}! Uribe would undoubtedly have produced evidence on this score, had he been able to find it, but none was forthcoming. \emph{Both} big peasant and landworkers’ federations, the affiliates of the \CNT\ and \UGT, opposed forced collectivization, favoured voluntary collectives, and denounced the Stalinists as supporters of the \emph{caciques} and reactionary rich peasants.

In June, the Socialist \emph{Adelante} sent a questionnaire to the various provincial sections of the \UGT\ peasants’ organization: almost unanimously they defended the collectives, and to a man reported that the main opposition to the collectives came from the Communist Party, which for this purpose recruited the \emph{caciques} and utilized the governmental institutions. All declared the October~7 decree was creating a new bourgeoisie. In a letter of protest to Uribe, Ricardo Zabalza described the simple but effective system of the Stalinists in attacking the collectives: old \emph{caciques,} \emph{kulaks,} landowners, were recruited and organized by the Stalinists and thereupon demanded the dissolution of the local collective, making claims on its land, equipment, stores of grain.

Every such controversy brought in its wake the ``mediation'' of Uribe’s representatives who invariably decided in favor of the reactionaries, imposing ``settlements'' whereby the collectives were gradually deprived of their equipment and land. When asked to explain this strange behavior, said Zabalza, the government agents declared they were acting under specific orders from their superior: Uribe. It was not surprising then that the \UGT\ Peasants’ Federation of Levante Province denounced Uribe as ``Public Enemy Number One.'' Irujo’s wards, the ex-fascists recently released, became, by the very fact of release, eligible to demand return of their lands. When one of these returned as landlord, peasants fiercely resisted~-- and the Assault Guards were sent against them.

In the cities and industrial towns, too, the government proceeded to destroy all elements of socialization.

\medskip

``It is unquestionably true that had the workers not taken over control of industry the morning after the insurrection, there would have been complete economic paralysis,''
wrote the Stalinist, Joseph Lash,
``but the improved schemes of workers’ control of industry have not worked out very well.'' (\emph{New Masses,} October 19).

\medskip

There was a half-truth in this but the whole truth leads not back to the old proprietors but forward to the workers’ state. Planning on a national scale is obviously impossible through factory and union apparatus alone. What is needed is a centralized apparatus, i.e., a state. Had the \CNT\ understood this, and initiated the election of militia and peasant and factory committees, joined in a national council which would have constituted the government~-- that would have been a \emph{workers’ state,} which would have given, full scope to the workers’ committees and yet achieved the necessary centralization.

Instead, the anarchist leaders fought a losing battle, arguing as to just how much authority the state should have. Peir\'o, ex-minister of industry, for example:

\begin{quotation}
  I was prepared to nationalize the electric industry in the only way compatible with my principles~-- leave its administration and direction in the hands of the trade unions and not in the hands of the state. The state has only the right to act as accountant and inspector.
\end{quotation}

Formally correct: Lenin said that socialism was simply bookkeeping. But only a workers’ state would faithfully accept the functions of accountant and inspector, while the existent Spanish state, a bourgeois state, must fight socialization. Here again the anarchists, continuing to make no distinction between a workers’ and bourgeois state, yielded to the bourgeois state, instead of fighting for the workers’ state.

Through the Ministry of Defense, factories were taken over one by one. On August~28, a decree gave the government the right to intervene in or take over any mining or metallurgical plant. Quite explicitly, the government stated that workers’ control was to be limited to protection of working conditions and stimulation of production. Resisting factories found themselves without credits or, having made deliveries to the government, payment was not forthcoming until the government’s will was accepted. In many foreign-owned plants, the workers had already been stripped of any form of authority. The Department of Purchases of the Ministry of Defense announced that on a given date it would make contracts for purchases only with enterprises functioning ``on the basis of their old owners'' or ``under the corresponding intervention controlled by the Ministry of Finance and Economy.'' (\emph{Solidaridad Obrera,} October~7).

The next step, for which the Stalinists had been campaigning for months, was militarization of all industries necessary to war: transportation, mines, metal plants, munitions, etc.\@ This barracks regime is reminiscent of Gil-Robles under whom munitions workers were also militarized~-- strikes and trade union membership forbidden. The militarization decree is sugar-coated by being titled ``militarization and nationalization decree.'' But to militarize factories already in workers’ hands, coupled with government recognition of full indemnification of former owners, simply ends workers’ control and prepares for returning the factories to their former owners.

Above all, the government was symbolized by new-found friends~-- reactionary deputies~-- who now appeared for the first time in Spain since July 1936.

Miguel Maura was there! Chief of the extreme right republicans, Minister of the Interior in the first republican government, an implacable enemy of the trade unions, the first minister of the republic to reinstitute the dreaded ``law of flight'' to shoot political prisoners~-- Maura had fled the country in July. His brother, Honorio, a monarchist, had been shot by the workers; the rest of his family had gone over to Franco. In exile, Maura had made no contact with the Spanish Embassies.

Portela Valladares was there! Governor General of Catalonia for Lerroux after the crushing of Catalan autonomy in October 1934, he had been the last premier of the \emph{bienio negro,} just before the February 1936 elections. He had fled Spain in July. What he had done in the interim was unknown. Now he rose in the Cortes:

\begin{quotation}
  This parliament is the \emph{raison d’etre} of the Republic; it is the title to life of the Republic. My first duty before you, before Spain, before the world, is to assure the legitimacy of your power\dots.
  
  Today is for me one of intimate and great satisfaction, in having contributed with you to seeing our Spain in transition to a serious and profound reconstruction.
\end{quotation}

At the end of the session, he and Negrin embraced. To the press, Valladares praised ``the prevailing atmosphere as he had observed it in Spain.\kn\kn'' He went back to Paris, while the Stalinist press proved statistically that the presence of Valladares and Maura, signifying the centre’s support of the regime, gave a statistical majority of the electorate to the government.\endnote{This anti-Marxist criterion enabled the fascists, by the same criterion, to argue that the rightist vote, plus that of those centre deputies now with them, constituted a majority of the people. The claims of both, of course, were based on the February~1936 election figures. The Marxist criterion is that a revolution derives its justification from the revolutionary vanguard’s representing a majority of the working class, supported by the peasantry. By the present Stalinist criterion, one could condemn the Russian Revolution!}

The ardour of the Stalinist press was cut abruptly by the reproduction in the fascist \emph{Diario~Vasco,} October~8, 1937, of a letter of Va\-lla\-da\-res to Franco, dated October~8, 1936, offering his services to the ``national~cause.''

The Stalinist welcome to Valladares and Maura was ``offset'' by a passing reference of La~Pasionaria to the unwelcome presence in the Cortes of another reactionary a minor figure, a member of Lerroux’s ruling party of the \emph{bienio negro.} The deputy, Guerra del~Rio, was given the floor to answer, in effect, that if the government rested on the Cortes, here he was. La~Pasionaria subsided. \CNT\ attacks on Maura and Valladares were deleted by the censor.

\medskip

Was it for this then that the masses had shed their blood?

\smallskip

But we have still to tell the story of the government’s conquest of Catalonia and Aragon.
  \chapter{The Conquest of Catalonia}

\lettrineO{n May 5,} Catalan autonomy\index{Catalonia!Autonomy} had ceased to exist. The central government had taken over the Catalan Ministries of Public Order and Defense.\index{Ministries!Public Safety, Catalonia} Caballero’s{\indexLCaballero} delegate in Barcelona had broadcast:

\begin{quotation}
  From this moment, all the forces are at the orders of the central government\dots. These do not consider any union or antifascist organization as an enemy. There is no other enemy than the fascists.
\end{quotation}

But a week later, the Ministries of Defense and Public Order were surrendered by Caballero’s delegate to the representatives of Negr\'in--Stalin, and the pogrom began in earnest. The \POUM\indexPOUM\ went down with hardly a ripple. The \PSUC\ opened a monstrous campaign against it identical in language, slogans, etc., with the witch hunts of the Soviet bureaucracy before the Moscow trials. \emph{The Trotskyists in the \POUM\ have organized the latest insurrection on orders from the German and Italian secret police.} The \POUM’s answer to the \PSUC\ was \dots\ to institute a libel\index{Libel} suit against the Stalinist editors in a court filled with bourgeois and Stalinist judges and officials!

On May 28, \emph{La Batalla} was suppressed\index{Censorship} permanently and the \POUM\ radio seized. The Friends of Durruti\index{Friends of Durruti} headquarters were occupied and the organization outlawed.\index{Political repression} Simultaneously, the official anarchist press was put under iron political censorship. Yet the \POUM\ and \CNT\ did not join in a mass protest. ``We formulate no protest. We only make public the facts,\kn\kn'' wrote \emph{Solidaridad Obrera,} May 29.

\medskip

The \POUM\ youth organ, \emph{Juventud Comunista,} grandly remarked:

\begin{quotation}
  These are cries of panic and of impotence against a firmly revolutionary party\dots. The [libel] trial goes forward. The \PSUC\ organ must appear before the Popular Tribunals and they shall be revealed before national and international labor for what they are: vulgar calumniators. (June 3).
\end{quotation}
  
Naturally, on a technicality the trial was soon dismissed.

\medskip

\index{Political repression}
On the night of June 3, the Assault Guards\index{Assault Guards} attempted to disarm one of the remaining workers’ patrols. Shots were exchanged. There were dead and wounded on both sides. Here was the government’s opportunity to finish off the patrols. But here, also, was the \POUM’s opportunity to force the \CNT\ leaders to defend the workers’ elementary rights by demanding a united front for simple, concrete proposals~-- defense of free assembly, the press, the patrols, common defense of workers’ districts against the Stalinist hooligans, freedom of political prisoners, etc.\@ The anarchist leaders could have hardly rejected these proposals without compromising themselves irreparably before their membership. Even against the \CNT\ leaders’ will, united front committees could have been created in the localities to fight for such simple, concrete demands.

For the \POUM\ leaders, however, to raise such simple demands meant: we have been wrong in estimating the May Days as a defeat of the counterrevolution; it was a defeat of the workers and now we must fight for the most elementary democratic rights. Secondly, it meant: we have been wrong in leaning on the \CNT\ leaders, limiting ourselves to the general, abstract proposal of a ``revolutionary front''\index{Revolutionary Front} of \CNT-\FAI--\POUM\ which implies that the \CNT\ is a revolutionary organization with which we can have a common platform on fundamental policies.\kn\kn%
\endnote{Juan Andrade\index{Andrade, Juan} had justified the ``revolutionary front'' absurdity by the following argument:
	
\begin{quotation}
  The disillusioned worker, turning away from the democratic tendencies of the socialists and communists inclines to join a powerful organization, such as, \CNT-\FAI\indexCNT, which holds radical positions \emph{even if they are not applied in fact,} rather than join a minority party bothered by material difficulties. The workers already in the \CNT\ see no need of leaving it to join a revolutionary Marxist party because in contrasting the \emph{surface} revolutionary positions of the \CNT-\FAI\ with the simply democratic ones of the socialists and Stalinists, they believe the tactics of their organization still hold the guarantee for continued development of the revolution toward the building of a socialist economy. In this sense, all those who hold a strictly sectarian schematic concept of how a minority with a correct political line can rapidly become a decisive force, can learn a valuable lesson from the events in Spain\dots

  The difficulties in the way of the rapid development of a great mass party which would assume the effective leadership of the struggle can be largely resolved by the establishment of the Revolutionary Front between these two organizations.
\end{quotation}

In other words, it is impossible to build the party of the revolution; the Revolutionary Front is a substitute. But the \emph{main} obstacle to building the revolutionary party, beside the \POUM’s own false program, was that the \CNT's \emph{surface} radicalism was \emph{not} systematically criticized before the masses by the \POUM\kn. The \POUM\ had thus cut itself off from growth~-- and used its failure to justify continuing that failure.}
We must say openly\kn, that a united front for the most elementary workers’ rights is the most than can be expected of the anarchist leadership, if even that.

\emph{Not once} in the year had the \POUM\ called for a united front with the \CNT\ for concrete tasks of struggle! The whole policy of the \POUM\ leadership essentially consisted of nothing but trying to curry favor with the \CNT\ leadership. \emph{Not once} did they characterize the capitulatory policy of the \CNT\ leadership, not even when they expelled the Friends of Durruti and left them to the mercies of the Assault Guards!

In its darkest hour the \POUM\ was completely isolated. On June~16, Nin{\indexANin} was arrested in his office. The same night, widespread raids caught almost all the forty members of the Executive Committee. A few who escaped were forced to give themselves up because their wives were seized as hostages. The next morning the \POUM\ was outlawed.\index{Political repression}

The Regional Committee of the \CNT\ did not come to the defense of the \POUM\kn. \emph{La Noche} (\CNT) of June 22, published in bold face:

\begin{quotation}
  About the espionage service discovered in the last days. The principal ones implicated were found in the leading circles of the \POUM. Andr\'es Nin and other known persons arrested.
\end{quotation}

There followed some general reflections on slander, with copious references to Shakespeare, Gorki, Dostoevsky, and Freud\dots. If censorship was to be blamed, then where were the \CNT’s illegal leaflets? In Madrid, `\CNT' did come to the defense of the \POUM\kn, and was followed by \emph{Castilla Libre} and \emph{Frente Libertario,} militia organ. On June~28, the National Committee of the \CNT\ addressed a letter to the ministers and their organizations reminding them that Nin, Andrade, David Rey, Gorkin, etc., ``had acquired their prestige among the masses by long years of sacrifice.\kn\kn''

\begin{quotation}
  Let them solve their problem in the \USSR\ as they can or circumstances advise them. It is not possible to transplant to Spain the same struggle, prosecuted with blood and fire, internationally by means of the press and here by means of the law utilized as a weapon.
\end{quotation}

The letter indicated an entire lack of understanding of the significance of the persecutions.

\begin{quotation}
  Before all it imports us to declare that the \CNT\kn, by its intact and powerful strength, today perfectly organized and disciplined, is above all fear that tomorrow this method of elimination can overcome us. Placed above this semi-internal struggle\dots.
\end{quotation}

This pompous chest-beating meant that the \CNT\ masses would not be aroused by their leaders to the counterrevolutionary meaning of the persecutions.

Above all, the great masses had not been prepared to understand the Stalinist system of frameup and slander. Currying favor with Stalin,\index{Stalinism} the anarchist leaders had been guilty of such statements as that of Montseny: ``Lenin was not the true builder of Russia but rather Stalin with his practical realism.\kn\kn'' The anarchist press had preserved a dead silence about the Moscow trials and purges, publishing only the official news reports. The \CNT\ leaders even ceased to defend their anarchist comrades in Russia. When the anarchist Erich M\"uhsam was murdered by Hitler\kn, and his wife sought refuge in the Soviet Union\indexUSSR, only to be imprisoned shortly after her arrival, the \CNT\ leadership stifled the protest movement in the \CNT\ ranks. Even when the Red Generals were shot, the \CNT\ organs published only the official bulletins.

By mid-July\kn, the \POUM’s leaders and active cadres were all in jail!\index{Political prisoners} Over its buildings flew the violet-yellow-red flags of the bourgeoisie. The Lenin barracks were occupied by the Republican ``People’s Army.\kn\kn'' Its presses had been destroyed or given to the \PSUC. On the bulletin board of \emph{Batalla} was a copy of \emph{Julio,} \PSUC\ youth organ, headlining: ``Trotskyism is synonymous with counterrevolution.'' The \POUM’s dormitories, ex--Hotel Falcon, had become a prison for \POUM\ members and the headquarters of the Spanish \GPU\indexGPU\index{Political repression}. Its members were dispersed, disoriented, living in fear of nightly raids by the Assault Guards. ``Small groups work on their own hook,\kn\kn'' wrote an authoritative eyewitness early in July. ``It reminds one very much of the crumbling of the Communist Party of Germany in January 1933. The working class remains passive and permits anything to happen. The \CNT\ press prints only official notices. No protest! Nowhere even a word of protest! The \POUM\ has been swept away like a speck of dust. “Like under Hitler{\indexAHitler},\kn\kn” say the German comrades. The Russian Bolshevik--Leninists would add: “Almost like under Stalin.\kn\kn”\index{Stalinism}

In July\kn, the local \FAI\ committees began illegal propaganda.\index{Illegal propaganda} Unfortunately\kn, it did not center around rallying the workers to the concrete tasks of freeing the political prisoners. One typical leaflet recalled the German Social Democracy’s propaganda on the eve of Hitler\kn, demanding the help of the state~-- \emph{Staat, greif zu!}\index{Staat, greif zu@\emph{Staat, greif zu}}~-- against its own bands. Protesting Stalinist assaults on anarchist youth buildings, ``How long? It is time for the Government Council to speak, or lacking that, the Delegate General of Public Order and the Chief of Police,\kn\kn'' read one pathetic leaflet.\index{Leaflet}

Nor were the illegal \POUM\ leaflets, which now began to appear\kn, much better. They, who had always reproached the Bolshevik--Len\-in\-ists for seeing only Stalinism as the enemy, became themselves anti-Stalinist and nothing more. One typical leaflet, for example, addressed itself to everybody on the left and on the right: to the anarchists as well as to the ``young separatists'' of the Estat Catala.\index{Estat Catala} ``The men of the Left cannot betray their postulates. The Separatists cannot sell Catalonia by their silence.\kn\kn'' And the final slogan! ``Prevent the establishment of the dictatorship of a party behind the lines.\kn\kn'' What of the Estat Catala and Esquerra, Prieto and Azaña, accomplices of the Stalinists, indeed the main beneficiaries?

Thus, false policies facilitated the deadly advance of the counterrevolution. Only the small forces of the Bolshevik--Leninists,\index{Bolshevik-Leninists} who had been expelled as ``Trotskyists''\index{Trotskyism} from the \POUM\kn, and had formed their organization in the spring of 1937~-- only this small band, working under the three-fold illegality of the state, the Stalinists and the \CNT--\POUM\ leadership, clearly pointed the road for the workers. Not only the ultimate road of the workers’ state but the immediate task of defending the democratic rights of the workers. That the \CNT\ masses could be aroused was shown by the protection they accorded Bolshevik--Leninists distributing illegal leaflets.\index{Illegal propaganda} At one meeting (of the woodworkers’ union), lorries of Assault Guards arrived and attempted to arrest the distributors. The meeting declared that the leaflet distributors were under their protection and would repulse with arms any attempt to break in. The police were forced to leave without our comrades.

A Bolshevik--Leninist leaflet of July 19 points the road: the united front\index{United front} of struggle of the \CNT-\FAI, \POUM\kn, the Bolshevik--Leninists and the dissident anarchists (p.~\pageref{fig:bolshlenleaflet2}).\index{Leaflet}\index{Anarchism!Dissident anarchists}

\begin{figure}
\index{Leaflet}
\begin{oframed}
  \begin{quote}
  	\normalsize
  	\raggedright
  	
  	\medskip
  	
  	Workers---
  
    \smallskip
  	
  	Demand of your organization and your leaders a united front pact which must contain:
  
    \smallskip
  
    \begin{enumerate}[leftmargin=2em, labelindent=0.5em, itemindent=-1em]
      \item Struggle for the freedom of the workers’ press!
      
      Down with political censorship!\index{Censorship}
      
      \item Liberation of all revolutionary prisoners.\index{Political prisoners}
      
      For the liberation of Comrade~Nin,{\indexANin} transported to Valencia!
      
      \item Joint protection of all centers and enterprises in the possession of our organizations.
      
      \item Reconstitution of strengthened Workers’ Patrols.\index{Workers' militias}
      
      Cessation of disarming the working class.
      
      \item Equal pay for officers and soldiers.
      
      The return to the front of all the armed forces sent from Valencia.
      
      \index{General offensive}
      General offensive on all fronts.
      
      \item Control of prices and distribution through committees of working men and working women.\index{Price control}
      
      \item Arrest of the provocateurs of May~3: Rodr\'iguez Salas, Ayguad\'e, etc.
    \end{enumerate}
  
    \smallskip
  
    To achieve this, all workers form the united front!
    
    \smallskip
    
    \index{Workers' committees}
    Organize Committees of Workers, Peasants and Combatants in all enterprises, barracks and districts on the land and at the front!
    
    \medskip
  \end{quote}
\end{oframed}

\vspace{-0.5\baselineskip}

\caption{Leaflet distributed by Bolshevik--Leninists on July~19, 1937}
\label{fig:bolshlenleaflet2}
\end{figure}

But not in a day, or a month, does a new organization win the leadership of the masses. The road is long and hard~-- and yet the only road.

\dinkus

By July, according to the official \CNT\ figures, eight hundred of their members in Barcelona alone were imprisoned, and sixty had ``disappeared''~-- euphemism for assassination. The left socialist press reported scores of its leading militants everywhere seized and jailed.\index{Political prisoners}\index{Assassination}

One of the most repulsive phases of the counterrevolution was its merciless persecution of the foreign revolutionists\index{Foreign revolutionaries} who had come to Spain to fight in the ranks of the militias. A single report to the \CNT\ on July 24, counted 150 foreign revolutionists in a Valencia prison~-- arrested on the charge of ``illegally entering Spain.\kn\kn'' Hundreds were expelled from the country and the \CNT\ cabled the workers’ organizations in Paris, appealing to them to prevent the German, Italian, Polish exiles from being delivered into the hands of their consulates.
\nowidow

But the foreigners arrested and expelled did not meet the worst fate. Others among them were selected to complete the amalgam between the \POUM\ and the fascists. Maur\'in was in fascist hands, in danger of death. Nin, Andrade, Gorkin were too well known to the Spanish masses. The \POUM\ had too many thousands of its best men at the front. Too many of its leaders had died fighting fascism: Germinal Vidal, the Youth Secretary, at the taking of the Atarazañas barracks on July 19; his successor, Miquel Pedrola, commandant on the Huesca front; Etcheb\'eh\`ere, commandant at Sig\"uenza; Cahué and Adriano Nathan, commandants on the Aragon front; Jes\'us Blanco, commandant on the Pozuelo front, etc. Among the \POUM’s military figures were men like Rovira and Josh Alcantarilla, famed throughout Spain. A few unknown foreigners, fighting in the \POUM\ battalions, would serve to add to the credibility of the fantastic charges.

Georges Kopp, a Belgian ex-officer\kn, serving in the \POUM’s Lenin Division, had just returned to Barcelona from Valencia, where he had been given a major’s commission, the highest commission awarded to foreigners, when the Stalinists arrested him.
Then the Stalinist propaganda factory went to work.

Robert Minor, American Stalinist leader, announced that the pau\-ci\-ty of arms on the Aragon front (this was the first time the Stalinists admitted this charge of the \CNT) was now explained: ``The Trotskyist General Kopp had been carting enormous supplies of arms and ammunition across no man’s land to the fascists!'' (\emph{Daily Worker,} August~31 and October~5).\index{Framing}\index{Stalinism}

The choice of Kopp, however\kn, was a \GPU\ blunder of major proportions, comparable to the story of Romm’s meeting with Trotsky in Paris or Piatakov’s flight to Norway. For Georges Kopp, forty five years old, was a militant of long standing in the Belgian revolutionary movement. When the Spanish war broke out, he was chief engineer in a large firm in Belgium. It had been usual for him to experiment at night. He circulated the story that he was trying out a new machine, perfecting it by the actual process of manufacture. What he manufactured, however, were the ingredients for millions of rounds of cartridges. Left socialists organized illegal transport to Barcelona.

When Kopp discovered that he was under suspicion, he took leave of his four children and headed for the frontier. The very day he fled, the police raided his laboratory. \emph{In absentia,} Kopp was sentenced by the Belgian courts to fifteen years at hard labor: five for making explosives for a foreign power, five for leaving the country without permission while a reserve officer in the Belgian Army\kn, and five for joining a foreign army. Twice wounded on the Aragon front, he soon won the rank of commandant.\kn\endnote{The British \emph{New Leader}, August 13, 1937, published two detailed articles on Kopp’s record. [Kopp in fact survived and lived till 1951. See his biography by Don Bateman in \emph{Revolutionary History}, Vol.~4 Nos.~1/2, pp.~242--252. ---\ERC]}

Kopp cannot answer the Stalinist slanderers, for they have killed him. He was in a Barcelona jail with our American comrade, Harry Milton. In the middle of the night, Kopp was dragged out. That was in July\kn, and the last time he was ever seen.\index{Assassination}

On July~17\kn, a group of \kern -0.125pt\POUM\ members were released from prison in Valencia. It is a fact that most of them were extreme right-wingers, such as, Luis~Portela, editor of \emph{El Comunista}; Jorge~Arquer\kn, etc. Consequently their subsequent testimony was particularly cogent. Upon release, they went to Zugazagoitia, Minister of Interior\kn,\index{Ministries!Interior} who told them that Nin{\indexANin} had been taken from Barcelona to one of the private prisons of the Stalinists in Madrid. Arquer thereupon requested a safe-conduct\index{Safe conduct} to search for Nin. The minister, a Prieto man, told him: ``I guarantee you nothing; what is more, I advise you [not] to go to Madrid because, with my safe-conduct or without, you would endanger your life. These communists don’t respect me and they do as they please. And there would be nothing strange if you were seized and shot immediately by them.\kn\kn''

Publicly, however, Zugazagoitia was still saying that Nin was in a government prison. On July~19, however, Montseny, for the \CNT\kn, publicly charged that Nin had been murdered. Embarrassed by the numerous inquiries from abroad about specific prisoners’ whereabouts which the government was unable to answer for the simple reason that most of the prominent ones were in private Stalinist ``preventoriums,\kn\kn'' it was arranged that the prisoners be transferred from the Stalinist jails in Madrid and Valencia to the formal keeping of the Ministry of Justice. Nin was not among them. Irujo issued a statement that Nin was ``missing.\kn\kn'' Fled, the Stalinists said, toward the fascist lines. But the truth finally got out. On August~8, the \emph{New York Times} reported that ``nearly a month ago, a band of armed men kidnapped Nin from a Madrid prison. Although every effort has been made to hush up the affair\kn, it is now a matter of common knowledge that he was found dead on the outskirts of Madrid, a victim of assassination.\kn\kn''\index{Assassination}

As a personal friend of Nin and Andrade, the great Italian novelist, Ignazio Silone, had tried to save them. ``But,\kn\kn'' he warned, ``unless the revolutionary proletariat of other countries is watchful, the Stalinists are capable of every crime.\kn\kn'' \'Alvarez del Vayo, former Minister of Foreign Affairs\index{Ministries!Foreign Affairs} in Caballero’s cabinet, notoriously Stalin’s agent in the Caballero group, had the effrontery to tell the wife of Andrade that Nin had been murdered by his own comrades.\footnote{It is only just to add that del Vayo has since been excluded by the Socialist organization (Caballero-led) of Madrid.} Premier Prieto shrived his soul for this and other crimes by dismissing the Stalinist Police Chief, Ortega~-- and replaced him with the Stalinist, Mor\'on.

To cap the suppression of revolutionists with slander is scarcely a new invention.\index{Political repression}

When, in Paris, the June~1848 insurrection was drowned in blood, the National Assembly was assured by the left democrat, Flocon, that the insurrectionists had been bribed by monarchists and foreign governments. When the Spartacists were shot down, Ludendorff charged that they~-- and indeed, the social democrats who shot them too!~-- were agents of England. When the counter revolution got the upper hand in Petrograd, after the July Days, Lenin and Trotsky were indicted as agents of the Kaiser\kn. The destruction of the generation of 1917 is now carried out by Stalin under the charge that they have sold themselves to the Gestapo.

The parallel goes further. While Kerensky\index{Kerensky, Alexander} was shouting that Lenin and Trotsky were German agents, Tseretelli and Lieber in the soviets were, under questioning, dissociating themselves from the charge, limiting themselves to demanding the outlawry of the Bolsheviks for planning an insurrection. But, profiting from Kerensky’s charge, the Mensheviks did not ascend the housetops to proclaim the Bolsheviks’~innocence.

So, too, in Spain. The Stalinists were not even as successful as Kerensky: the indictment handed down against the \POUM\ leaders made no mention of collaboration with Franco or the Gestapo. The charge was based on the May Days and similar subversive and oppositional deeds. Prieto and other collaborators of the Stalinists told the \textsc{ilp} delegation they did not believe the Stalinist linking of \POUM\ to the fascists. They ``merely'' did not come to the defense of the \POUM\kn. Companys not only disavowed belief in the charges but made the fact public. Thus there was a division of labor: if you don’t believe the slanders, then you must believe the \POUM\ was organizing an insurrection, i.e., they were either counterrevolutionists or revolutionists, whichever you preferred. A narrower division of labor was that between the world Stalinist press, which repeated the ``Trotskyist-fascist'' slanders, and the anti--\POUM-\CNT\ propaganda of Louis Fischer,{\indexLFischer\index{Bates, Ralph}} Ralph Bates, Ernest Hemingway, Herbert Matthews, etc., who ``merely'' repeated such myths as that the \POUM\ militias played football in no man’s land with the fascists.

\dinkus

Already by the end of June, Catalan autonomy, though guaranteed by statute, was completely suppressed.\index{Catalonia!Autonomy} The authorities distrusted anybody who had any tie with the Catalan masses, no matter how tenuous. With the exception of the most reactionary sector, the old Civil Guards,\index{Civil Guards} all the police in Catalonia were transferred to other parts of the country.\index{Police} Even the firemen were transferred to Madrid. Parades were forbidden, and union meetings could be held only by permission of the delegate of public order on three days’ notice~-- as under the monarchy!\index{Political repression}
\nowidow

The workers’ patrols\index{Workers' militias} had been wiped out, their most active members imprisoned, their chiefs ``disappeared.''

Having done all this with the aid of the screen provided by \CNT\ ministers still sitting in the Generalidad, the bourgeois--Stalinist bloc now dispensed with their services.
A June~7 bulletin of the \FAI\ published a Stalinist communication which had been intercepted:\indexCNT

\begin{quotation}
  Based on the provisional composition of the government, our party will demand the presidency. The new government will have the same characteristics as that of Valencia; a strong government of the People’s Front \emph{whose chief task} will be to calm the spirits and demand punishment for the authors of the last counterrevolutionary movement. The anarchists will be offered posts in this government but in such a manner that they will be compelled to refuse to collaborate, and in this manner we shall be able to present ourselves to the public as the only ones willing to collaborate with all sections.
\end{quotation}

The anarchists challenged the \PSUC\ to deny the authenticity of this document, but there were no challengers.

At the end of June came the ministerial crisis. The \CNT\ agreed to whatever demands were made, and the new ministry was formed. Publication of the ministerial list on June~29, however\kn, revealed to the \CNT\ that, without their knowledge, a minister without portfolio had been added~-- an ``independent'' named Dr.~Pedro Gimpera, a notorious reactionary and anarchist-baiter. Companys blandly refused to withdraw him. The \CNT\ at last withdrew, leaving a government of the Stalinists and the bourgeoisie. {\indexCNT}

The only difference between the Stalinist bulletin exposed by the \FAI\ and the actual course of the ministerial crisis was that the Stalinists had not asked for the presidency. But six weeks later\kn, without any previous hint, the Stalinists clashed with President Companys.

In November~1936, when the \CNT\ intelligence service had seized Reberter, the Chief of Police, and had him tried and shot for organizing a \emph{coup d’\'etat,} the investigation had implicated Casanovas,\index{Casanovas i Maristany, Joan} President of the Catalan Parliament.\index{Catalonia!Parliament} But the Stalinists had supported Companys in prevailing upon the \CNT\ to let Casanovas leave the country and Casanovas had fled to Paris. After the May Days, he had returned to Barcelona with impunity. He spent the next three months pleasantly reestablishing himself in political life. During all these nine months, he had not been subject to a word of condemnation from the Stalinists.\kn\kn\footnote{Stalin has employed this method systematically in Russia: a bureaucrat is involved in a misdeed; he is permitted to go on because all the more servile for knowing that his crime is detected, then~-- sometimes years later~-- Stalin needs a scapegoat and the wretch is pilloried.} On August~18, the Catalan parliament opened. Without a previous word of warning to their allies~-- it could obviously have been settled behind closed doors~-- the \PSUC\ delegation of four publicly attacked Casanovas as a traitor. The Esquerra had been tricked into a position where it had to reject Casanovas’ offer to resign. With this excellent little whip, the Stalinists began to drive the Esquerra as they pleased, ending with the announcement of Companys’ early resignation from the presidency, after the Stalinists had boycotted the October~1 session of the Catalan parliament.

Why did the Stalinists break with Companys? He had done their bidding in so much! Why\kn, then, was Companys now slated to go?\kp\kp\footnote{The speed with which the fascists broke through the Aragon front disrupted the Stalinists’ plans, and Companys was not dropped.}

He had made one unforgivable break with the Stalinists. Companys had publicly declared that he had known nothing of the plans for outlawing the \POUM\kn; had protested against the transfer of the prisoners from Barcelona; and had sent to Madrid the chief of the Catalan press bureau, Jaime Miravittlles, to see the Stalinist Police Chief, Ortega, on behalf of Nin.

When Ortega showed him the ``crushing proofs''~-- a document ``found'' in a fascist center\kn, linking one ‘N’ to a spy ring~-- Miravittlles, by his own account, had burst out laughing, and declared the document was such an obvious forgery no one would dream of taking it seriously. Companys had then written to the Valencia government that Catalan public opinion could not believe Nin was a fascist spy.

\index{Framing}
Not that Companys was going to fight for the \POUM\ prisoners. Having salved his conscience~-- and made the record for any future overturn!~-- Companys relapsed into silence. That his silence did not save him from attack indicated that the Stalinists could not forgive any ally who exposed their frameups: the frameup is the very foundation stone of Stalinism today.

But there was a more profound reason for the break with the Esquerra. The Nin incident merely indicated that Companys was not hardened enough for the future moves of the Stalinists. He was, after all, a nationalist, desiring a return to Catalan autonomy. And for Stalinism,\index{Stalinism} Spain and Catalonia were merely pawns which they were ready to sacrifice, with which they were ready to do anything that Anglo--French imperialism dictated, in return for a military alliance for Stalin in the coming war. That is why there had to be a selection even from among the Prieto socialists and the Azaña republicans: only the most brutalized, most corrupt, most cynical could weather the coming storms created by the Stalinists, and remain in collaboration with them.

The economic counterrevolution in Catalonia advanced against the collectives.\index{Workers' militias} To the honor of the local sections of the libertarian movement, they stood their ground. For example, the strong anarchist \index{Anarchism!Dissident anarchists} movement in Bajo Llobregat (heart of armed struggles against the monarchy and the republic) declared in its weekly\kn, \emph{Ideas,} on May~20:

\begin{quotation}
  Here is what we must do, workers! You have the opportunity to be free. For the first time in our social history the arms are in our hands: don’t drop them. Workers and peasants! When you hear that the government, or anybody else, tells you that the arms should be at the front, answer them that that is certainly so, that the thousands of rifles, machine guns, mortars, etc., that are kept in the barracks, that are used by the Carabineros, Assault and National Guards,\index{Assault Guards} etc., should be sent to the front because to defend your fields and factories, nobody can do that better than you.\index{Battle}
  
  Remember always that airplanes, cannons and tanks are what are needed at the front quickly to crush fascism \dots\ what the politicians are looking for is to disarm the workers, to have them at their mercy, and to take away from them what has cost so much proletarian blood and lives. Let nobody permit the disarming of anybody; let no village allow another to be disarmed; let us all disarm those who try to disarm us. This should be, must be, the revolutionary slogan of the hour.
\end{quotation}

The gap between the pusillanimity of the central organs of the \CNT\ and the fighting spirit of the local papers, close to the masses, was as wide as that between craven cowards and revolutionary workers.

But tens of thousands of Assault Guards, concentrated behind the lines, struck systematically at the collectives. Without centralized direction, the villages were overpowered, one by one. \emph{Libertad,} one of the dissident illegal anarchist papers of Barcelona (incidentally\kn, it paid its contemptuous respects to \emph{Solidaridad Obrera,} which had denounced the illegal organs), described the situation in the countryside in its issue of August~1:

\begin{quotation}
  \index{Censorship}
  It is useless that the censorship, in the power of one party, prevents a word being said about the thousands of blows inflicted on the workers’ organizations, the peasant collectives. In vain that they prohibit mention of that terrible word, counterrevolution. The working masses know perfectly that the thing exists, that the counterrevolution advances under the protection of the government, and that the black beasts of reaction, the disguised fascists, the old \emph{caciques,} are again raising their heads.
  
  \index{Reprivatization}
  And how should they not know it, if there is not a village in Catalonia where the punitive expeditions of the Assault Guards have not been, where they have not assaulted the \CNT\ workers, destroying their branch organizations or what is worse, destroyed those portentous works of the revolution, the collectives of the peasants, in order to return the land to the old proprietors, almost always known as fascists, ex-\emph{caciques} of the black epoch of Gil-Robles, Lerroux or Primo de~Rivera?
  
  \index{Collectivization}
  The peasants took the goods of the bosses~-- which in justice did not belong to them~-- to place them at the service of collective labor, permitting the old bosses to dignify themselves, if they wished, by work. They believed, the peasants, that so noble a work was guaranteed by its own efficiency, if fascism were not triumphant, and it could not triumph. Scarcely did they suspect that in the midst of war against the terrible enemy, with the government being men of the left, the public forces [police] would come to destroy that which had been created with such fatigue and joy. For this inconceivable thing to happen, there had to come to power\kn, by dirty means, those called communists. And the workers, ready always to make the greatest sacrifices to defeat fascism do not end wondering how it is possible that they be attacked from behind, that they be humiliated and betrayed, when there still is so much lacking for conquering the common enemy\dots.
  
  \index{Political repression}
  The technique of repression is always the same. Lorries of Assault Guards that enter the village like conquerors. Sinister registrations in the branch organizations of the \CNT. Annulment of municipal councils where the \CNT\ is represented. Plundering searches and arrests. Seizure of the food of the collectives. Return of the land to their old proprietors.
\end{quotation}

\index{Assassination}
This movingly simple description was followed by a long list of villages, the dates on which they were assaulted, the names of those arrested or killed~-- and in the ensuing months the list grew longer and longer\kn.

In industry and commerce, the juridical basis of the collectivized establishments rested on the insecure foundation of the collectivization decree of October~24, 1936. But immediately after the May Days, the Generalidad repudiated the decree! The occasion was the attempt of the \CNT\ to release the factories from the stranglehold of the customs officials, without whose certification of ownership of export goods arriving abroad were being sequestered under claims of emigrated former owners. The anarchist-led Council of Economy (of the Ministry of Industry)\index{Ministries!Industry} adopted on May~15 a proposed decree to record the collectivized establishments as the official owners in the Mercantile Register. But the bourgeois--Stalinist majority in the Generalidad rejected the proposal on the ground that the October~24 collectivization decree was ``dictated without competency by the Generalidad,\kn\kn'' because ``there was not, nor is there yet, legislation of the [Spanish] State to apply,\kn\kn'' and ``Article~44 of the [Spanish] Constitution\index{Constitution of Spain} declares expropriation\index{Collectivization} and socialization are functions of the [Spanish] State,\kn\kn'' i.e., the Catalan autonomy\index{Catalonia!Autonomy} statute had been exceeded! The Generalidad would now await action by Valencia. But Companys had signed the October Decree! That was during the revolution\dots.

\dinkus

The chief agency of economic counterrevolution was the \GEPCI, the long-established businessman’s organization taken bodily into the Catalan section of the \UGT\ by the Stalinists but repudiated by the \UGT\ nationally. With union cards in their pockets, these men did with impunity what they would never have dared before July~19 against the organized workers. Many of them were now no longer petty manufacturers but great entrepreneurs. They received ``pref\-er\-en\-tial consideration in securing financial credits, raw materials, export services, etc.,\kn\kn'' as against the factory collectives. One little item will destroy the Stalinist myth that these were petty little storekeepers, one-man establishments. In June~1937\kn, the \UGT\ clothing workers drafted a scale of wages, identical with those in the clothing collectives, and sought to negotiate with the capitalist-owned clothing factories. The employers rejected the demands. But who were the employers? Members, to a man, of \GEPCI\indexGEPCI, that is, like the employees whom they were refusing wage rises, they were members of the \UGT\ in Catalonia! (\emph{Solidaridad Obrera,} June 10). Would the most reactionary trade union bureaucrat, of the stamp of Bill~Green or Ernest~Bevin, propose that bosses and workers be in one ``union?\kp'' No, that vast step backward could come only from the Stalinists, aping Fascist Italy and Nazi Germany.

In June, under the slogan of ``municipalization,\kn\kn''\index{Municipalization} the \PSUC\ launch\-ed a campaign to wrest the transportation,\index{Ministries!Utilities} electric, gas, and other key industries from workers’ control. On June 3, the \PSUC\ delegation formally proposed, in the Barcelona Municipal Council,\index{Barcelona} that it municipalize public services. On the morrow, of course, the \CNT\ councillors would be thrown out, and the Stalinists would have the public services in their hands for the next step in returning them to their former owners. But this time they were confronted, not merely by the temporizing \CNT\ leaders~-- who proposed that municipalization was ``premature'' in this field, one ought to begin with housing~-- but with the mass response of the workers involved. The Transport Workers Union plastered every block of the city with huge~posters:

\medskip

\begin{oframed}
  \centering
  \setlength{\parskip}{0.5\baselineskip}
  
  \medskip
  The revolutionary conquests belong to the workers.
  
  The workers’ collectives are the product of these conquests.
  
  We must defend them\dots.
  
  To municipalize the urban public services, yes~--
  
  but only when the municipalities belong to the workers
  
  and not to the politicians.
  \medskip
\end{oframed}

\medskip

The posters demonstrated that since the workers had taken control,\index{Collectivization} there had been a thirty percent increase in plant facilities, lowering of fares, additional workers employed, big donations to the agricultural collectives, subventions to the harbor workers, social insurance to families of deceased or wounded workers, etc. For the moment, the Stalinist advance was beaten in this field.

But the Stalinists continued toward their goal of destroying the worker-controlled factories. The Catalan Generalidad\index{Catalonia!Generalidad} set September~15 as the deadline for proving the legality of collectivized factories. Since much of the collectivization was done overnight to speed the civil war against the fascists, few factories had established any juridical procedure. What, indeed, were the legalities involved in the expropriations? The original decree of October~24, 1936, we have discussed in our chapter on the first Generalidad cabinet. It was designed precisely to provide entering wedges for the future. And the Generalidad had now repudiated it! At leisure and at will, the Generalidad would now examine the legal title of the social revolution and find it undoubtedly full of legal flaws. What a preposterous business! But a tragic one.

It was in the food industries, distribution, markets, etc., that the Stalinists had got their first grip, holding the Ministry of Supplies\index{Ministries!Supplies} in the Generalidad since December\kn, when they had promptly dissolved the workers’ supply committees, which then had been provisioning the cities under controlled prices. Even through the temporizing of the \CNT\ press and the opacity of the censorship, the accounts now reflected what was happening here:

\begin{quotation}
  Collectives, socialized undertakings and cooperatives, embracing both members of the \UGT\ and \CNT\kn, have been made the target of attack on the part of those who hid in desertion on the 19th of duly\dots.
  \index{Reprivatization}
  The dairymen of both unions are being arrested right and left. The cows and the dairy farms, organized legally on a cooperative basis, are being confiscated, although their statutes have been officially approved by the Generalidad for several months. These cows and dairy establishments are being hand\-ed over to their former owners\dots.
  The same thing is happening, although still on a small scale, in the bread industry\dots.
  Our markets, the central fish market, etc., although collectivized legally\kn, are also suffering from these vicious attacks by the former bourgeoisie. They are being encouraged by the poisonous campaigns conducted daily in the press of the party that has constituted itself the Champion for the Defense of the \GEPCI\ (Corporations and Units of Petty Merchants and Manufacturers). It is no longer merely a fight against the \CNT\ collectives, but against all the revolutionary conquests of the \UGT--\CNT\dots.
  A hard fist against the fascists and counterrevolutionaries hiding behind a trade union card! (\emph{Solidaridad Obrera,} June~29).\indexSolidaridadObrera
\end{quotation}

``Is the Ministry of Supplies at the service of the people, or has it been transformed into a bigger merchant?\kp'' asked the \CNT\ press. ``The basic articles of food are: rice, string beans, sugar\kn, milk, etc. Why are these not included among those items that the Committee of Distribution,\index{Ministries!Distribution} recently formed by the \UGT--\CNT, distributed equally among all the stores of Barcelona, regardless of the organization to which they belong?\kp'' Instead, these basic articles were uncontrolled, left to the mercy of the \GEPCI.

\emph{La Noche} (June 26) responding to the bitterness of the masses: ``The death penalty for thieves! Scandalous abuses of the merchants at the expense of the people.\kn\kn'' And, after showing, from the official statistics, the precipitous rise of food prices between June~1936 and February~1937\kn, \emph{La Noche} said: ``Nor would it have been so bad if the prices had remained at that level! One can speak to the housekeepers about the increase in the cost of living since February. It is reaching inaccessible figures\dots. We must create some form of protection for the interests of the people against the egoism of the merchants who are carrying on with full impunity.\kn\kn''

Yes, it was in food supplies that the Stalinists had their grip long\-est. And the result: hunger\kn, yes, actual hunger\index{Hunger} stalked Catalonia. The bitterness of the masses breaks through in \emph{Solidaridad Obrera\indexSolidaridadObrera} (September~19):

\begin{quotation}
  Proletarian mothers with sons at the front here suffer stoically of hunger together with their innocent little ones\dots.
  We say that sacrifices ought to be by all and it is an inconceivable situation that in actuality there are places where, by paying prices outside the reach of any worker, it is possible to obtain all kinds of food. These luxurious restaurants are veritable foci of provocation and should disappear, as ought to disappear all privileges of any sector. Flagrant inequality, privilege, is in such cases a terrible dissolvent of popular cohesion. It must be eliminated at all costs\dots. Protected \dots\ there has entered into action a repugnant caste of speculators and profiteers who traffic in the hunger of the people\dots.
  \index{Inequality}
  We repeat that our people do not fear sacrifices but do not tolerate monstrous inequality\dots. Respect the proletariat that fights and suffers!
\end{quotation}

Yes, the masses do not fear sacrifices. The workers of Petrograd suffered the most extreme privations~-- not even running water in the city during the civil war. But what there was belonged to all equally. It is not the bare pangs of hunger that contort the faces of the Barcelona workers and their women and children. It is that while they hunger, the bourgeoisie eat luxuriously~-- and this in the midst of civil war against fascism! But that is the inevitable consequence of not finishing with bourgeois ``democracy.\kn\kn''

To those who have been impressed by Stalinist ``common sense'' in modestly fighting for democracy: Do you begin to understand what it meant in the concrete, in the seared souls of the Spanish people?

  \chapter{The Conquest of Aragon}

\index{Battle}\index{Aragon}
\lettrineT{he fertile} province of Aragon was the living embodiment of victorious struggle against fascism. It was the only province actually invested by the fascists and then conquered from them by force of arms. It was especially the pride of the Catalan masses, for they had saved Aragon. Within three days of the victory in Barcelona, the \CNT\ and \POUM\ militias were off for Aragon. The \PSUC\ then was small and contributed little or nothing. Imperishable names of battles there---Montearagón, Estrecho Quinto, etc.---were associated solely with the \CNT\ and \POUM\ heroes who had won them. It was in the victorious conquest of Aragon that Durruti acquired his legendary fame as a military leader, and the forces he brought to the defense of Madrid in November were the picked troops whose victorious morale had been welded in Aragon victories.

\index{Fortress of the revolution@``Fortress of the revolution''}
Not the least of the reasons for the successes in Aragon had been that, under Durruti’s leadership, the militias marched as an army of social liberation. Every village wrested from the fascists was transformed into a fortress of the revolution. The militias sponsored elections of village committees, to which were turned over all the large estates and their equipment. Property titles, mortgages, etc., went into bonfires. Having thus transformed the world of the village, the \CNT--\POUM\ columns could go forward, secure in the knowledge that every village behind them would fight to the death for the land that was now theirs.

Backed by their success in freeing Aragon, the anarchists met with little resistance there from the bourgeois--Stalinist bloc in the first months. Aragon’s municipal councils were elected directly by the communities. The Council of Aragon was at first largely anarchist. When Caballero’s cabinet was formed, the anarchists agreed to give representation to the other anti-fascist groups in the Council, but up to the last days of its existence the masses of Aragon were grouped around the libertarian organizations. The Stalinists were a tiny and uninfluential group.

\index{Collectivization}\index{Cooperatives}
At least three-fourths of the land was tilled by collectives. Of four hundred collectives, only ten adhered to the \UGT. Peasants desiring to work the land individually were permitted to do so, provided they employed no hired labor. For family consumption, cattle were owned individually. Schools were subsidized by the community. Agricultural production increased in the region from thirty to fifty per cent over the previous year, as a result of collective labor. Enormous surpluses were voluntarily turned over to the government, free of charge, for use at the front.

\index{Anarchism}\index{Accounting}\index{Family wage}
Libertarian principles were attempted in the field of money and wages. Wages were paid by a system of coupons exchangeable for goods in the cooperatives. But this was merely pious genuflection to anarchist tradition, since the committees, carrying on sale of produce and purchase of goods with the rest of Spain, perforce used money in all transactions, so that the coupons were merely an internal accounting system based on the money held by the committees. Wages were based on the family unit: a single producer was paid the equivalent of 25~pesetas; a married couple with only one working, 35~pesetas, and four pesetas weekly additional for each child.

This system had a serious weakness, particularly while the rest of Spain operated on a system of great disparity in wages between manual and professional workers, since that prompted trained technicians to migrate from Aragon.

For the time being, however, ideological conviction, inspiring the many technicians and professionals in the libertarian organizations, more than made up for this weakness. Granted that, with stabilization of the revolution, a transitional period of higher wages for skilled and professional workers would have had to be instituted. But the Stalinists who had the effrontery to contrast the Aragon situation with the monstrous disparity of wages in the Soviet Union, appeared to have forgotten completely that the family wage---which is the essence of Marx’s ``to each according to his needs''---was a goal toward which to strive, from which the Soviet Union is infinitely further away under Stalin than under Lenin and Trotsky.

\index{Aragon!Council of Aragon}\index{Stalinism}
The anarchist majority in the Council of Aragon led in practice to the abandonment of the anarchist theory of the autonomy of economic administration. The Council acted as a centralizing agency. The opposition was in such a hopeless minority within Aragon, and the masses were so wedded to the new order, that there was no record of a single Stalinist mass meeting in Aragon in direct opposition to the Council. Many joint meetings were celebrated, with Stalinist participation, including one as late as July 7, 1937. Neither at these meetings nor elsewhere in Aragon did the Stalinists repeat the calumnies which the Stalinist press elsewhere was spreading, in order to prepare the ground for an invasion.

\index{Internationalism}
Many workers’ leaders from abroad saw Aragon and praised it: among them Carlo Rosselli, the Italian anti-fascist leader, serving as a commandant on the Aragon front (on leave in Paris when he and his brother were assassinated by the Italian fascists). The prominent French socialist, Juin, wrote strong praise of Aragon in \emph{Le Peuple.} \emph{Giustizia~e~Liberta,} the leading Italian anti-fascist organ, said of the Aragon collectives:

\begin{quotation}
  The manifest benefits of the new social system strengthened the spirit of solidarity among the peasants, arousing them to greater efforts and activity.
\end{quotation}

\begin{sloppypar}
The manifest benefits of social revolution, however, scarcely weighed in the balance against the grim necessities of the bourgeois--Stalinist programme for stabilizing a bourgeois regime and winning the favor of Anglo--French imperialism. The preconditions of such favor was destruction of every vestige of social revolution. But the masses of Aragon were united. The destruction must, therefore, come from outside. Once the Negrin government came to power, a terrific barrage of propaganda against Aragon was laid down in the bourgeois and Stalinist press. And, after three months of this preparation, the invasion was launched.
\end{sloppypar}

\index{Political repression}\index{Reprivatization}
On August~11, the government decreed the dissolution of the Council of Aragon. Instead was appointed a governor-general ``with the faculties which the prevailing legislation attributes to civil governors'' ---legislation from the days of reaction. The governor-general, Mantec\'on, proved only a figurehead, however. The real job was done by military forces under the leadership of the Stalinist, Enrique Lister.

\index{Censorship}\index{Ascaso, Joaquin@Ascaso, Joaqu\'in}\index{Political prisoners}
One of the manufactured heroes of the Stalinists,\footnote{\emph{\CNT} published his picture with the title, ``Hero of many battles. We know it because the Communist Party has told us so''---irony was the only way of getting past the censor.} Lister marched his troops into the rear of Aragon. The municipal councils elected directly by the population were forcibly dissolved. Collectives were broken up and their leaders jailed. As with the \POUM\ prisoners in Catalonia, not even the Governor-General knew the whereabouts of the members of the \CNT\ Regional Committee arrested by Lister’s bands. They had, indeed, carried safe-conducts\index{Safe conduct} from the Governor-General but that did not save them.

Joaqu\'in Ascaso, President of the Council of Aragon, was jailed on the charge \dots\ of stealing jewels! The government censorship forbade the \CNT\ press to publish the news of Ascaso’s imprisonment, refused to divulge the place of his incarceration, and from their foully reactionary viewpoint, they were right. For Ascaso was flesh and blood of the masses, as the dead Durruti had been, and they would have torn the jail down with their bare hands.

Suffice it to say that the official \CNT\ press---none too anxious to arouse the masses---compared the assault on Aragon with the subjection of Asturias by L\'opez~Ochoa in October 1934.

\index{Libel}\index{Stalinism}\index{Aragon!Council of Aragon}
To justify the rape of Aragon, the Stalinist press published fantastic tales. \emph{Frente Rojo} wrote:
\begin{quotation}
  Under the regime of the extinguished Council of Aragon, neither the citizens, nor property, could count on the least guarantee\dots
  
  The government will find in Aragon gigantic arsenals of arms and thousands of bombs, hundreds of the latest model machine-guns, cannons and tanks, reserved there, not to fight fascism on the battle fronts, but the private property of those who wished to make of Aragon a bastion from which to fight the government of the republic\dots
  
  Not a peasant but had been forced to enter the collectives. He who resisted suffered on his body and his little property the sanctions of terror. Thousands of peasants have emigrated from the region, preferring to leave the land than to endure the thousand methods of torture of the Council\dots
  
  The land was confiscated, and rings, lockets, and even the earthen cooking pots were confiscated. Animals were confiscated, grain and even the cooked food and wine for home consumption\dots
  
  In the Municipal Councils there were installed known fascists and Falangist chiefs. Holding union cards they officiated as mayors and councillors, as agents of the public order of Aragon, bandits by origin making a profession and a government regime of banditry.
\end{quotation}

Was anybody expected seriously to believe this nonsense? The police mentality of the Stalinists was evident in the alibi that an insurrection was preparing. Unfortunately, it was not true. The arms? The Aragon front had come under complete government control on May~6, with a Stalinist party member, General Pozas,\index{Pozas, Sebastian} in supreme command. Prior to that the \CNT, \POUM, \FAI\ press from October~1936 on had abounded in long and precise complaints that the Aragon front was being deprived of arms, and that the armed guard of the Aragon collectives---actually, with the front irregular and shifting, part of the frontline defenses---were dangerously stripped of arms. For eight months these charges had been made, from press, platform, and radio and with it the charge that Russian aid was being conditioned by Stalinist control of the disposition of incoming arms. The Stalinists had met these specific charges with dead silence. Now, in the pogrom atmosphere of August~1937, their answer was that arms were there! No one was, nor could be expected to believe this poppycock, not even the party members.

But the charges do not require serious rebuttal. For on September~18, the man who presumably had been the chief culprit, who had terrorized, installed, fascists, etc., etc., etc., Joaqu\'in Ascaso, was released from prison. If the Stalinists were ready to prove their charges against Ascaso even in their corrupted courts, why did they not do so? The answer is: the charges were balderdash. What was terribly real, however, was the destruction of the Aragon collectives.

\index{Bates, Ralph}
After the bourgeois--Stalinist bloc conquered Aragon and the story of their invasion began to seep out to the world labor movement where the Stalinists did not dare to attempt to repeat their fantastic charges, they adopted a new tack which sought to move away from these charges to the assertion that the dissolution of the Council was required in order to re-organize the Aragon front. Thus Ralph Bates wrote,

\begin{quotation}
  There have been exaggerated charges against the Council~of~Aragon, but I think the following can be substantiated by detailed evidence: the wholesale application of extreme measures in land and social reform had confused and even antagonized non-anarchist peasantry and workers; anarchist control of village military committees had undoubtedly hampered efficient conduct of operations\dots
  
  The problem, therefore, was to bring this strip of Aragon under the control of the Valencia government, as part of a campaign to reform the Aragon military forces. (\emph{New Republic,} October 27, 1937).
\end{quotation}

This latest alibi had two functions: first, to get away from the preposterous charges on which the dissolution had first been justified; second, to cover up the fact that, although the central government had been in complete control of the Aragon front since May, its so-called offensives had been fiascos. The infinite infamy of all this will become apparent if we now turn to the military question itself and examine the Aragon front as part of the whole program of military strategy.
  \chapter{Military Struggle Under Giral and Caballero}

\index{Dual power}\index{Battle}
\lettrineM{ilitary warfare is} merely the continuation of politics by forcible means. A proclamation dropped over the enemy lines, expressing the aspirations of the landless peasants, is also an instrument of warfare. A successfully incited revolt behind enemy lines may be infinitely more efficacious than a frontal attack. Maintenance of the morale of the troops is as important as equipping them. To guard against treacherous officers is as important as training efficient officers. In sum, the creation of a workers’ and peasants’ government for which the masses will work and die like heroes is the best political adjunct of military struggle against the fascist enemy in civil war.

\index{Russia!Civil War}
By these methods, the workers and peasants of Russia defeated imperialist intervention and White Guard armies on twenty-two fronts, despite the most rigid economic blockade ever imposed on any nation. In the organization and direction of the Red Armies under these adverse conditions, Trotsky seemed to perform miracles, but these were miracles compounded of revolutionary politics, of the capacities for sacrifice, labor and heroism of a class defending its newly-won freedom.

That reactionary political policies determined the false military policies of the Loyalist government can be demonstrated by surveying the course of the military struggle.

\indexIPrieto
From July~19 until September~4, 1936---seven decisive weeks---the Giral cabinet of the People’s Front was at the helm, with the unconditional political support of the Stalinists and the Prieto Socialists (Prieto, indeed, was unofficially part of the ministry, establishing an office in the government on July 20).

\indexJGiral\indexGRobles\index{Workers' militias}
The Giral government had about \$600,000,000 in gold at its disposal. Recall that the real embargo on the sale of munitions to Spain was not established until August~19, when the British Board of Trade revoked all licences for export of arms and planes to Spain. Thus, the Giral regime had at least a month in which to purchase stores of arms---but the damning fact is that it bought almost nothing! The story of the treacherous attempt of Azaña--Giral to reach a compromise with the fascists has already been told. One further fact: Franco and his friends waited six days before forming their own government. Gil~Robles later revealed that they waited in expectation of a satisfactory arrangement with the Madrid government. By that time the militias had emerged from the ranks of the workers and Giral no longer had the power to meet Franco’s demands.

The most important gains of the first seven weeks were the successful march on Aragon by the Catalan militias, using socialization of the land as much as they used their rifles; and the attack of the loyalist warships on Franco’s transportation of troops from Morocco to the mainland.

Wrote two Stalinists at the time,

\begin{quotation}
  The loyalty of a large part of the navy decisively prevented Franco from transporting large numbers of Moroccan troops to the mainland in the first two weeks of the war. The naval patrol in the south made transport by sea extremely hazardous. Franco was forced to resort to airplane passage, but this was slow work. In this respect, again, the government was given a chance to organize defense and take stock. (\emph{Spain in Revolt,} Gannes and Repard, p.~119).
\end{quotation}

\index{Sailors}\index{Ministries!Navy}
What they failed to add was that the warships were under the command of elected sailors’ committees, which, like the militias, had no faith in the Giral government and carried on operations, despite the passivity of the government. The significance of this fact will become apparent when we come to the naval policy of the Caballero--Stalinist--Prieto cabinet.

The terrible defeats of Badajoz and Irun finished the Giral cabinet. Why Irun fell was told in a moving dispatch by Pierre~van~Paassen:

\index{Battle}\index{Irun}
\begin{quotation}
  They fought to the last cartridge, the men of Irun. When they had no more ammunition, they hurled packets of dynamite. When dynamite was gone, they rushed forward barehanded and tackled each their man, while the sixty times stronger enemy butchered them with bayonets. A girl held two armored cars at bay for half an hour by hurling glycerin bombs. Then the Moroccans stormed the barricade of which she was the last living defender and tore her to pieces. The men of Fort~Martial held three hundred foreign legionnaires at a distance for half a day by rolling rocks down the hill on which the old fort is perched.
\end{quotation}

\index{Central Committee of Anti-Fascist Militias of Catalonia}
Irun fell because the Giral government had made no attempt to provide its defenders with ammunition. The Central Committee of Anti-Fascist Militias of Catalonia, having already transformed the available factories into munitions works, had sent several carloads of ammunition to Irun by way of the regular railroad from Catalonia to Irun. But that railroad runs partway through French territory. And the government of ``Comrade'' Blum, the ally of Stalin, had held up the cars at Behobia, just across the frontier, for days \dots\ they rumbled across the bridge into Irun after the fascists had won.

\indexLCaballero\indexIPrieto\index{Popular Front}
The Giral cabinet gave way to the ``real, complete'' People’s Front government of Caballero--Prieto--Stalin. Undoubtedly, it had the confidence of a large part of the masses. The militias and the sailors’ committees obeyed its orders from the first. There were three major military campaigns that the new government had to undertake. There were, of course, other tasks, but these were the most important, the most pressing, and essentially, the simplest.

\section{Morocco and Algeciras}

\indexFFranco\index{Spanish Legion}\index{Morocco}
Franco’s military base during the first six months was Spanish Morocco. From here he had to bring his Moors and Legionnaires and military stores.

\index{Sailors}\index{Workers' militias}\index{Battle!Naval}\index{Foreign intervention!Italy}
The first successes of the Loyalist navy under the sailors’ committees in harassing Franco’s communication lines with Morocco were followed by others. On August~4, the Loyalist cruiser \emph{Libertad} effectively shelled the fascist fortress at Tarrifa in Morocco. It was a deadly blow at Franco. So deadly that it was answered by the first open Italian act of intervention: an Italian plane proceeded to bomb the \emph{Libertad.} When Loyalist warships steamed into position for a large-scale shelling of Ceuta in Morocco while fascist transports were loading, the German battleship \emph{Deutschland} brazenly steamed back and forth between the Loyalist warships and Ceuta to prevent the bombardment. A week later a Spanish cruiser stopped the German freighter \emph{Kamerun,} found her loaded to the decks with arms for Franco, and prevented her from landing at Cadiz. Whereupon Portugal openly went over to the fascists, permitting the \emph{Kamerun} to unload at a Portuguese port and forwarding the munitions by rail to Franco. German naval commanders received orders to fire on any Spanish vessel attempting to stop German munition shipments. The Loyalist naval operations, if continued, were fatal to Franco, and his allies had to unmask completely to save him.

\indexLCaballero\indexIPrieto\index{Ministries!Navy}
At this point the Caballero cabinet was formed, and Prieto, now in the closest collaboration with the Stalinists, and always ``France’s man,'' became head of the Naval Ministry. He put an end to naval operations off Morocco and the straits of Gibraltar, and recalled the Loyalist forces which had held Majorca.

\index{Spanish Legion}\index{Andalusia!Algeciras}
The task of the hour was to prevent the Moors and Legionnaires from landing at Algeciras and constituting that army which was soon to make that fearful march from Badajoz straight through to Toledo, through Toledo and Talavera~de~la~Reina to the gates of Madrid. The first line in that task belonged to the navy.

\smallskip

\emph{It was not used for this purpose.}

\smallskip

\index{Basque Country!Biscay}\index{Battle!Naval}\index{Battleships}
Instead, in mid-September, almost the whole fleet---including the battleship \emph{Jaime I,} the cruisers \emph{Cervantes} and \emph{Libertad,} and three destroyers---were ordered to leave Malaga and go all the way around the peninsula to the Biscayan coast! They left behind the destroyer \emph{Ferrandiz} and the cruiser \emph{Gravina.} On September~29, two fascist cruisers sank the Ferrandiz, after having shelled and driven away the Gravina. What considerations determined that the naval forces go off to the Biscayan coast, while news dispatches reported---to quote but one instance---an armed trawler conveying Moroccan troops from Ceuta and escorted by the \emph{Canarias,} the \emph{Cervera} and a destroyer and torpedo boat crossed the straits this evening. The convoy landed the troops at Algeciras without hindrance. It transported from Morocco a supply of field and other guns and abundant supplies of ammunition. (\emph{New~York~Times,} September~29).

What considerations? Certainly not military ones, for the forces sent to Biscay were more than sufficient to hold their own with the fascists’ armed convoy; and certainly barring communications from Morocco was the chief task of the navy.

The American military expert, Hanson~W.~Baldwin, writing (in the \emph{New~York~Times} of November~21) on the naval question in Spain said:

\indexNYT\index{Ministries!Navy}
\begin{quotation}
  The Spanish navy has been to a great extent neglected, particularly in the recent troubled years of the republic’s history, and it has never been properly led or manned. But with efficient, well-drilled crews, Spain’s handful of cruisers and destroyers could be a force to be reckoned with, particularly in the narrow basin of the Mediterranean, \emph{where well-handled ships long ago could have cut General Franco’s line of communication to his reservoir of manpower in Africa\dots}
  
  Judging from the somewhat obscure reports, most of the ships---despite the officers’ efforts---continued to fly the red, yellow and mauve of [Loyalist] Spain or else hoisted the Red flags at their gaffs\dots
  
  \dots\ but altogether the role of the navy in the civil war has to date been a minor one. The occasional engagements in which the ships have participated have had in most instances an \emph{op\'era bouffe} quality and have attested to the poor marksmanship and seamanship of the crews.
\end{quotation}

\index{Basque Country}\index{Foreign revolutionaries}
The Loyalist operations of September~27, at Zumaia near Bilbao, however, demonstrated accurate firing. The real point, however, was that it would have been a simple matter to equip the Loyalist warships with able crews. Toulon, Brest, and Marseilles were filled with thousands of socialist and communist sailors, veterans of the navy, including skilled gunners and officers. They could more than have manned the fleet, and other ships that could have been built at the main construction docks, in Cartagena in Loyalist hands.

\index{Andalusia!Algeciras}\index{Murcia!Cartagena}\index{Battle!Naval}
Returning eventually from the northern coast, the fleet was anchored far from the strait, at Cartagena---and there it stayed, except for a few pointless trips down the coast. That it existed at all, one learned on November~22, when foreign submarines entered the port of Cartagena and loosed torpedoes, one damaging the Cervantes. The same day the Ministry of Marine announced reorganization of the fleet to combat attempted blockades---and that was the last heard of that project. Franco’s transports moved at will from Ceuta to Algeciras, bringing tens of thousands of troops and the armaments they required.

\index{Anarchism}
In a letter to Montseny, demanding that the anarchist ministers publicly fight against the false governmental policies, Camillo~Ber\-neri said of the navy:

\indexIPrieto\index{Morocco}\index{Berneri, Camillo}
\begin{quotation}
  The concentration of the forces coming from Morocco, the piracy in the Canaries and the Balearics, the taking of Malaga, are the consequences of this inactivity. If Prieto is incapable and inactive, why tolerate him? If Prieto is bound by a policy which paralyses the fleet, why not denounce this policy?
\end{quotation}

\index{Foreign influence!England and France}\index{Morocco}\index{Counterfactual}\index{Ministries!Navy}
Why did Prieto and the governmental bloc follow this suicidal policy? It was simply one factor of the whole policy which rested on securing the goodwill of England and France. What these were seeking is clear. An aggressive Loyalist naval policy, as the August incidents off Morocco had shown, would have precipitated the decisive stage of the civil war. It would have threatened to crush Franco immediately. Germany and Italy, their prestige involved in supporting Franco, would perhaps have been driven to desperate steps in his defense, such as, the open resort to use of the Italian and German navies in sweeping the Loyalists out of the straits. But England and France could not have tolerated Italo--German control of the straits (it might be retained thenceforward). That open war would thus begin was, of course, not a certainty. Especially prior to November~9, 1936, when Germany and Italy formally recognized the Burgos regime, Germany and Italy might have retreated before precipitating war. Had revolutionists been at the helm and boldly moved in August and September to a systematic naval campaign, and succeeded in cutting off Morocco from Spain, the probability was that Italy and Germany would have retreated as gracefully as they could.

Anglo-French imperialism, however, was not interested in a Loyalist victory but in staving off a war crisis while resisting encroachments on their imperialist interests in the Mediterranean. And they had their way due to the Anglo--French orientation of the Loyalist government. Each month that passed thereafter, involving Germany and Italy more deeply, rendered more and more likely an international explosion if the Loyalist navy were activized. \emph{It simply ceased to exist as a Loyalist weapon.}

Here is the first terrible instance of how anti-revolutionary politics hamstrung the military struggle.

\index{Andalusia!Algeciras}\index{Andalusia!Malaga@Málaga}\index{Battle}\index{Nationalist army}\indexCNT\index{Ministries!Navy}
The same Anglo--French orientation explains the failure to strike by land at Algeciras, the Spanish port at which the fascist forces were landing from Morocco. Malaga was strategically located to be the spearhead for this drive. Instead, Malaga itself was left defenseless. Chiefly defended by \CNT\ forces, who pleaded in vain from August until February for the necessary equipment, Malaga was invaded by an Italian landing force, while the fleet which could have stopped them rode at anchor in Cartagena. Malaga fell on February~8. For two days before that, the militias had received no instructions from the military headquarters and then, the day before Malaga fell, they discovered that the headquarters had already been abandoned without a word to the defending militiamen. It was not a military defeat, but a betrayal. The base treachery was not the last-minute desertion by the general staff but the political policy which dictated the inactivity of the navy and disuse of Malaga as a base against Algeciras.%
\endnote{%
  On February~21, Undersecretary~of~War, José~Asenio, was dismissed, and soon arrested together with Colonel~Villalba, for the betrayal of Malaga. War~Commissar~Bolivar, a Stalinist, who had joined Villalba in abandoning headquarters, was not arrested. Nor was a word breathed---until the National Committee of the \CNT\ got really desperate for the moment---at the Stalinists’ assaults---that Antonio Guerra, Stalinist representative in the Military Command of Malaga, stayed behind and went over to the fascists. (\emph{\CNT\ Boletin Valencia,} August 26, 1937).

  The day Gijon fell---eight months later---the government announced it would try the Malaga traitors---Asensio, his chief of staff, Cabrera, and another general. Why try these and not those guilty at Bilbao, Santander, etc., etc.? Because Malaga fell under Caballero, while the far more brazen betrayals in the north took place under Negrin\dots

  \index{Stalinism}\indexCNT\index{Andalusia!Malaga@Málaga}
}

If not by sea or by land, there was still another way of striking at Franco’s Moroccan base. We quote Camillo~Berneri:

\index{Berneri, Camillo}\index{Morocco!Autonomy}\index{Foreign influence!Arab world}
\begin{quotation}
  The fascist army’s base of operations is in Morocco. We should intensify the propaganda in favor of Moroccan autonomy on every sector of Pan-Islamic influence. Madrid should make unequivocal declarations, announcing the abandonment of Morocco and the protection of Moroccan autonomy. France views with concern the possibility of repercussions and insurrections in North Africa and Syria; England sees the agitation for Egyptian autonomy being reinforced as well as that of the Arabs in Palestine. It is necessary to make the most of such fears by adopting a policy which threatens to unloose revolt in the Islamic world.
  
  For such a policy money is needed and speed to send agitators and organizers to all centres of Arab emigration, to all the frontier zones of~French Morocco. (\emph{Guerra di Classe,} October 24, 1936).
\end{quotation}

\index{Morocco}\index{Foreign influence!England and France}\index{Ministries!Foreign Affairs}
But the Loyalist government, far from arousing French and English fears by inciting insurrection in Spanish Morocco, proceeded to offer them concessions in Morocco! On February~9, 1937, Foreign Minister del~Vayo delivered to France and England a note, the exact text of which was never revealed but was stated later, without denial by the cabinet, to include the following points:

\index{Foreign intervention!Italy}\index{Foreign intervention!Germany}
\begin{enumerate}
  \item Proposing to base its European policy on active collaboration with Great~Britain and France, the Spanish government proposes the modification of the African situation.
  
  \item Desiring a rapid end to the civil war, now susceptible of being prolonged by German and Italian aid, the government is disposed to make certain sacrifices in the Spanish zone of Morocco, if the British and French governments should take steps to prevent Italo--German intervention in Spanish affairs.
\end{enumerate}

The first inkling of the existence of this shameful note came a month after its dispatch, in the French and English press on March~19, when Eden made a passing reference to it. The \CNT\ ministers swore they had not been consulted in its dispatch. Berneri addressed them bitingly:

\index{Berneri, Camillo}\indexCNT\index{Reconciliation}
\begin{quotation}
  You are in a government which has offered France and England advantages in Morocco, while from July, 1936, it should have been obligatory on us to proclaim officially the political autonomy of Morocco.\ \dots
  
  The hour has come to make known that you, Montseny, and the other anarchist ministers are not in agreement with the nature and purport of such proposals.\ \dots
  
  It goes without saying that one cannot guarantee English and French interests in Morocco and at the same time agitate for an insurrection there.\ \dots
  
  But this policy must change. And to change it, a clear and strong statement of our own intentions must be made -- because at Valencia, some influences are at work to make peace with Franco. (\emph{Guerra~di~Classe,} April~14, 1937). 
\end{quotation}

But the anarchist leaders remained silent, and Morocco remained undisturbed in Franco’s power.%
\endnote{%
  The only official \CNT\ pamphlet dealing with Morocco that I have been able to find is: ``What Spain Could Have Done in Morocco and What It Has Done,'' a speech by Gonzalo~de~Reparaz on January~17, 1936, telling how he tried to get the monarchy and the republic to organize things efficiently in Morocco, and how they did not! Not a hint that the only advice a revolutionist can give on the colonial question is: get out of Morocco.
}

\section{The Aragon Offensive against Saragossa and Huesca}

\index{Aragon!Aragon Offensive}\index{Aragon!Huesca}\index{Aragon!Zaragoza}\index{Battle}
Thumb through the Spanish or the French or American press of August--November~1936 and note the sharp contrast between the Loyalist defeats on the central and western fronts and the victories on the Aragon front. The \CNT, \FAI, and \POUM\ troops predominated in Aragon. They obeyed the military orders of the bourgeois officers sent by the government but kept them under surveillance. By the end of October, having captured the surrounding heights of Montearagón and Estrecho~Quinto, the Aragon militias were in position to take Huesca, the gateway to Saragossa.
  \chapter{Military Struggle Under Negr\'in and Prieto}

\indexIPrieto\index{Ministries!Defense}\index{Bourgeois--Stalinist bloc}\indexJNegrin
\lettrineT{hat the} ``government of victory'' would inevitably continue the disastrous military policy of its predecessor was apparent the day it was constituted. Prieto would continue his inactive naval policy and his political discrimination in the assignment of aircraft to the fronts. He was now also head of the army, with all services in a single Ministry~of Defense, but the Supreme War Council, established in December, had already then been dominated by the bourgeois--Stalinist bloc through their majority of the ministries.\footnote{The Stalinist ``demand'' that the council function normally, raised on May~16, was simply a prop for the myth to make Caballero the scapegoat for the conduct of the war.}

The political course which had dictated the previous military stra\-te\-gy---hostility to lighting the flame of revolt in North Africa, support of the Basque bourgeoisie against the workers, persecution of Catalonia and Aragon---all this continued, intensified. In addition, the Negr\'in cabinet added new obstacles to prosecution of the war.

\index{National question}\index{Catalonia!Autonomy}\index{Political repression}
On the national question---relations with minority peoples---the Negr\'in regime moved not only to the right of Caballero, but also to the right of the republic of 1931--33. The bureaucratic centralization for which the monarchists and fascists stood had been an important factor in alienating from them the peoples of Catalonia, Euzkadi (Basques) and Galicia. Once the civil war began, the limited autonomy of the Catalans and Basques had broadened de facto. A declaration of autonomy for Galicia would have immeasurably facilitated the guerilla warfare there. It was not forthcoming because it would have provided a precedent for Catalonia to seize on. The Negr\'in regime proceeded, as we have seen, to wipe out Catalan autonomy. Where the Bolsheviks had gained strength for the prosecution of the civil war from the intensified loyalty of the autonomous minority nations, the Loyalist government quenched the fires of national aspirations.

\index{Inequality}
The pay of militiamen was reduced from ten pesetas a day to seven, while the ascending scale for officers was: 25~pesetas for second lieutenants, 39~for first lieutenants, 50~for captains, 100~for lieutenant colonels. Economic distinctions thus sharply reinforced military regulations. One need scarcely emphasize the deleterious effect on the soldiers’ morale of this and their increasing subordination to the~officers.

\index{Bourgeoisie}\index{Fifth column}\index{Assault Guards}\index{Civil Guards}\index{Berneri, Camillo}\index{Political repression}
The whole northern front was soon to be betrayed by the Basque bourgeoisie and officers, and by the ``Fifth Column'' of fascist sympathizers in the Assault and Civil Guards and among the civilian population. The struggle against the ``Fifth Column'' was indissolubly part of the military struggle. But, as Camillo Berneri had written even before the intensification of repression under Negr\'in,

\begin{quotation}
  It is self-evident that during the months when an attempt is being made to annihilate the [\POUM--\CNT] “uncontrollables,” the problem of eliminating the Fifth Column cannot be solved. The suppression of the Fifth Column is primarily to be achiev\-ed by an investigatory and repressive activity which can only be accomplished by experienced revolutionists. An internal policy of collaboration between the classes and of consideration toward the middle classes, leads inevitably to tolerance toward elements that are politically doubtful. The Fifth Column is made up not only of fascist elements but also of all the malcontents who hope for a moderate republic.
\end{quotation}

While the northern front was left to the Basque bourgeoisie, the Aragon front was subjected to a frightful purge. General Pozas initiated what was ostensibly a general offensive in June. After several days of artillery and aerial conflict, orders to advance were given to the 29th~Division (formerly the \POUM’s Lenin Division) and other formations. But on the day for the advance, neither artillery nor aviation was provided to protect it\dots\

\index{Aragon}\index{Basque Country!Biscay!Bilbao}\indexSPozas\index{Battle}\index{Battle!Aerial}\indexPOUM\index{Military withdrawal}
Pozas later claimed this was because the air forces were defending Bilbao---but the day of advance was three days \emph{after} Franco had taken~Bilbao. The \POUM\ soldiers fully realized that they were being exposed deliberately. But not to go into fire would have given the bourgeois--Stalinist bloc a case against the Aragon front. They went into the line of fire. One flank was ostensibly assigned to an International Brigade (Stalinist)---but shortly after the advance began, it received orders to withdraw to the rear. The lieutenant colonel in charge of a formation of Assault Guards on the other flank later congratulated the \POUM\ troops: 

\begin{quotation}
  At Sarinena, I was warned against you on the ground that you might shoot us in the back. Not only did it not happen but thanks to your bravery and your discipline, we have avoided a catastrophe. I am prepared to go to Sarinena to protest against those who sow the seeds of demoralization, to effect the triumph of their partisan political aims.
\end{quotation}

\index{Political repression}\index{Assassination}\indexPOUM\index{Trotskyism}
During this offensive, Cahué and Adriano Nathan, \POUM\ commanders, were killed in action. Police were at that moment coming for Cahué, to arrest him as a ``Trotskyist--fascist.''

\index{Workers' militias!Liquidation of militias}\indexSPozas\index{Aragon!Huesca}\index{Political prisoners}\indexPOUM\index{Military supplies!Rifles}\index{Battle}\index{Military withdrawal}
When the attack was over, the 29th was sent to the rear. That, customarily, had meant to give up rifles---there were still not enough for front-line and reserves simultaneously on this front! But the suspicious \POUM\ troops refused to yield their arms. They declared themselves ready to return to the front. A few days later two battalions of the division were ordered to march on Fiscal (on the Jaca front) to repulse a fascist attack. Not only did they crush the attack but they reconquered positions and material previously lost. Then they were retired to await new orders---but not sent back to their division. Why? To disarm them. Pozas ordered it. They were concentrated in the village of Rodeano and surrounded by a Stalinist brigade. They were relieved of all their valuables---watches, chains, even good underwear and new shoes. Their leaders were arrested, the rest permitted to go---on foot. Hiking home, many were arrested in the towns on the way. The only reason the same methods were not employed against the rest of the division was that the news leaked out quickly and Pozas feared the \CNT\ divisions would come to its defense. But a few weeks later the 29th was officially dissolved, the remaining men being distributed far and wide in small groups.\endnote{This account is that of the front correspondent of \emph{Avanti,} émigré (Paris) organ of the Italian Maximalist-Socialists, scarcely a \POUM\ or Trotskyist source.}

\index{Workers' militias!Liquidation of militias}\indexCNT
The Ascaso (\CNT) division was also cut to pieces. \emph{Acracia,} \CNT\ organ of Lerida, wrote:

\vspace{-0.33\baselineskip}

\index{Aragon!Huesca}\index{Ministries!Air force}\index{Battle}\index{Military supplies!Planes}
\begin{quotation}
  Now we know exactly why Huesca was not taken. The last operation at Santa Quiteria furnishes a good proof of it. Hues\-ca was surrounded on all sides and only the betrayal of the air forces (controlled by \PSUC) was responsible for the disaster with which this operation ended. Our militiamen were not backed up by the air forces and were thus left defenseless in face of an intensive machine-gunning by the fascist air forces. This is only one of the numerous operations which ended in the same manner on account of the same betrayal by the air forces.
\end{quotation}

\vspace{-0.33\baselineskip}

\indexPSUC\indexSPozas
Soon after there was a plenary session of the \PSUC\ Central Committee in Barcelona. Among those prominently participating were ``Comrades'' General Pozas, chief of the Aragon front, Virgilio Llanos, political commissioner of the front, and Lieutenant Colonel Gordon, chief of staff\dots

Acceptance of control by the central government had been dangled before the Aragon front troops as the end of all their worries. Instead, it was used to break them down still further. The front correspondent of the anarchist (Paris) \emph{Libertaire} wrote on July~29:

\index{Political repression!Boycott}\index{Friends of Durruti}\index{Aragon!Zaragoza!Bujaraloz}\index{Exhaustion}
\begin{quotation}
  Ever since the central government took over control, the financial boycott became accentuated. Most of the militiamen have not received their pay for a long time. In Bujaraloz, where the general staff of the Durruti column is located, both officers and soldiers have not seen a cent for the last three months. They cannot wash their clothes for lack of soap. In many a place visited after several months of absence, I found comrades whom I knew well: now they looked pale, thin and visibly weakened. The physical state of the troops is such that they cannot keep up any prolonged exercises; they cannot march for more than fifteen kilometers a day. In the region of Farlete the troops live by hunting, without which they would starve to death.
\end{quotation}

\vspace{-0.33\baselineskip}

\indexLibertad\index{Loss of territory}\index{Casualties}
Systematic persecution of the chief forces of the Aragon front scarcely laid the basis for military victories, although at Belchite and Quinto the 25th division (\CNT) gave a good account of itself. But the alleged success of the July offensive on the Aragon front was so much newspaper talk. ``Results?'' wrote the illegal anarchist organ, \emph{Libertad} (August~1): ``Two villages lost in the Pirineo sector and three thousand men lost. This is what they call a success. Disastrous, calamitous, shameful success!''

After the fall of Santander (August 26), the persecution of the \CNT\ troops abated somewhat. But now came a terrible lesson in the consequences of creating counter-revolutionary forces of repression, such as the Stalinist-controlled Karl Marx division. In the midst of an offensive in the Zuera sector,

\vspace{-0.33\baselineskip}

\indexCNT\indexPSUC\index{Court martial@Court-martial}\index{Desertion}\index{Karl Marx Division}\index{Friends of Durruti}
\begin{quotation}
  Fifty officers of that division and six hundred soldiers pass\-ed over to the fascists. As a result of these desertions a battalion was destroyed. Despite the mettle of the \CNT\ forces, the operation could not be terminated well. The enemy had the necessary time to recover and it was impossible to continue the attack. After a summary court-martial that was immediately convened, thirty officers of the Karl Marx Division have been shot. In addition, the political commissar of the division, Trueba, a \PSUC\ member, has been dismissed. (\emph{Amigo del Pueblo,} illegal organ of Friends of Durruti, September~21).
\end{quotation}

\vspace{-0.33\baselineskip}

\index{Censorship}
Needless to say, the \CNT\ press was forbidden to publish the facts.

\section{The Northern Front}





  \chapter{Only Two Roads}

\lettrineS{ixteen} months of civil war have conclusively demonstrated that all the roads pointed out to the Spanish people reduce themselves to but two choices. One is the road we point: revolutionary war against fascism. All other paths lead into the road marked out by Anglo--French imperialism.

Anglo--French imperialism has demonstrated that it has no intention whatsoever of aiding the Loyalists to victory. Even the Stalinized \emph{New Republic} (October~27, 1937) was at long last compelled to admit: ``It is clear that by now the concern of France and England over a fascist victory in Spain has become—if it was not from the start—a~completely secondary consideration.''

The Spanish question is but one factor in the conflict of interests among the imperialist powers, and will be finally ``settled''—if the imperialists of both camps have their way—only when they come to the point of a general settlement of all questions, i.e., the imperialist~war.

Having most to lose, the Anglo--French bloc holds back from the war, although it must eventually fight to hold its own. Until that moment it avoids decisive showdowns, in Spain as elsewhere. It permitted a trickle of aid to the Loyalists from the Soviet Union because it did not want a victory for Franco while his Italo--German allies dominated his regime. British interests have employed the interim to arrange with Burgos for joint exploitation of the British-owned Bilbao region. The first week of November, Chamberlain announced establishment of formal relations with Franco (as small coin to anti-fascist sentiment, the diplomatic and consular officials were designated merely as ``agents''), while Eden assured Parliament that a Franco victory would not mean a regime hostile to Britain. Thus, the masters of the Anglo--French bloc prepared themselves for a Franco victory.

Whatever fears the Anglo--French bloc may have had about a Fran\-co victory, it never wanted a Loyalist victory. An early victory would have been followed by social revolution. Even now, after six months of repressions by the Negr\'in government, the Anglo--French rulers still doubt that a Loyalist victory would not be followed by social revolution. They are correct. For the millions of \CNT\ and \UGT\ workers, held in leash by the civil war, at its victorious completion would shatter the bourgeois limits of the People’s Front. Moreover, an imminent Loyalist victory would be such a blow to Italo--German prestige that it would be countered by invasion of Spain on an imperialist scale of warfare and an attempt to bottle up the Mediterranean. The danger to the ``lifeline of empire'' of the Anglo--French bloc would bring war on the immediate order of the day. Anglo--French desire to postpone the war thus led directly to opposition to Loyalist victory.

The only reason the Anglo--French bloc did not openly court Franco was that it dared not abandon its chief advantage in the coming war: the myth of democratic war against fascism, by which the proletariat is being mobilized to support the imperialist war.

The main pre-occupation of Anglo--French imperialism from the first was: how postpone the war, maintain the democratic myth, and yet begin to edge Hitler and Mussolini out of Spain? The answer was also obvious: a compromise between the Loyalist and fascist camps. As early as December~17, 1936, ``Augur'' semi-officially stated that English agents were working for a local armistice in the North, while French agents were doing likewise in Catalonia. Even the social-patriot, Zyromski, stated in \emph{Populaire} (March~3, 1937):

\begin{quotation}
  Moves can be seen that are aiming at concluding a peace which would signify not only the end of the Spanish revolution, but also the total loss of the social victories already achieved.  
\end{quotation}

The Caballero Socialist, Luis Araquist\'ain, Ambassador to France from September 1936 to May 1937, afterward declared:

\begin{quotation}
  We have counted too much, in illusion and hope, on the London committee, that is to say, on the aid of the European democracies. Now is the hour to realize that we can expect nothing decisive from them in our favor, and from one of them much against us, at the least. (\emph{Adelante,} July~18, 1937).
\end{quotation}

The Negr\'in government put itself entirely in the hands of the Anglo--French bloc; and Negr\'in’s speeches, notably that to the Cortes on October~1, in dwelling on the need for preparing for peace, and his speech after the fall of Gij\'on, revealed that his government was ready to carry out the Anglo--French proposals for compromise.

The face of Negr\'in was turned not to the battle fronts but to London and Paris. The government’s orientation was summed up succinctly by the pro-Loyalist Matthews, after the fall of Gij\'on: ``On the whole, there is more discouragement here over the London discussion than what has happened in the north.'' Matthews continued:

\begin{quotation}
  There was a passage in Premier Negr\'in’s broadcast last ni\-ght which so perfectly expressed the government’s opinion that it deserves to be put on record: ``Once more our foreign enemies are trying to take advantage of the \emph{ingenuous candor of European democracies} by fine subtleties. \dots\ I am now warning the free countries of the world, for our cause is their cause. \emph{Spain will accept any means of reducing the anguish of this country,} but let the democracies not be seduced by the Machiavellianism of their worst enemies, and let them not again be the victim of a base decision.'' (\emph{New York Times,} October~24, 1937).
\end{quotation}

Truly that passage perfectly expressed the government’s opinion. Were not the consequences of this policy so tragic for the masses, one would roar with laughter at the picture of the ``ingenuous candor'' of perfidious Albion\footnote{The oldest name of Great Britain.} and the Quai d’Orsay.\footnote{The location of the French Ministry of Foreign Affairs.} Fearing that he was to be abandoned altogether, Negr\'in was thus begging his imperialist mentors to remember that he ``will accept any means of reducing the anguish of this country.'' Had he not already proved that by his repressions of the workers?%
\endnote{%
\indexLFischer
``Chautemps reflects the bourgeois and fascist dislike of Valencia. He, therefore, constantly urges Valencia to moderate its action and emphasize the democratic character of the regime.'' This testimony is Louis Fischer’s! (\emph{The Nation,} October~16, 1937).
}

That the Loyalist government had already agreed to support a compromise with the fascists is attested to, not only by authoritative revolutionary and bourgeois sources, but also by a Stalinist source:

\indexLFischer
\begin{quotation}
  A Spanish government representative who attended King George~\textsc{v\kern -1.5pt i}’s coronation outlined to Foreign Minister, Eden, Valencia’s plan for ending the civil war. A truce was to be declared. All foreign troops and volunteers serving on both sides would then be immediately withdrawn from Spain. During the truce no battle lines would be shifted. Non-Spaniards having been eliminated, Great Britain, France, Germany, Italy and the Soviet Union were to devise a scheme, which the Spanish government \emph{pledged itself in advance to accept,} whereby the will of the Spanish nation regarding its political and social future might be authoritatively ascertained. (Louis Fischer, \emph{The~Nation,} September~4, 1937).
\end{quotation}

At the best, such an arrangement would mean a plebiscite under the supervision of the European powers. With Franco in possession of territory including more than half of the Spanish people, and with the Italo--German and Anglo--French blocs competing for Franco’s friendship, one can imagine the outcome of the plebiscite: unity of the bourgeois elements in both Spanish camps in a Bonapartist regime, decked out at the beginning with formal democratic rights, but ruling the masses primarily through the armed might of Franco’s~armies.

Such was the end of the road pointed out by the Anglo--French imperialists and already accepted by the Negr\'in government. There were still objective difficulties in the way: Franco hoped to win everything and was encouraged to fight on by Italy and Germany. But this much was clear. If not a complete Franco victory, to which England and France were already reconciled, then the best that can come from Anglo--French ``aid'' was a joint regime with the fascists.


Stalin might find this a bitter pill to swallow. However a compromise with the fascists would be dressed up, it would, nevertheless, be a terrible blow to Stalinist prestige throughout the world. But rather than break with the main objective of Soviet policy—the winning of an alliance with Anglo--French imperialism—Stalin was ready to submit to a settlement dictated by them. He would ``find a formula.'' The same arguments which were used to justify Soviet entry into the non-intervention committee, if accepted, would justify the final act of treachery against the Spanish people.

\smallskip

Let us recall those shabby arguments.

\begin{quotation}
  The Soviet Union emphatically was not in favor of the non-intervention agreement. With sufficient support from the socialist parties, the labor and anti-fascist movements of the world, besides the support of the communist parties, the Soviet Union would have been able to stop the non-interven\-tion move in its track.\endnote{Harry Gannes, \emph{How the Soviet Union Helps Spain,} November~1936. This was the official Stalinist apology for supporting the London committee.}\label{en:GannesUSSR}
\end{quotation}

Need we remind anyone that Stalin never tried to rally the world labor movement before he endorsed the non-intervention scheme? If the Stalin regime was powerless to stop the bandits, did it have to join them? The Stalinists understood quite well the role of England: 

\begin{quotation}
  The Baldwin cabinet gauged its international action to retain the goodwill of the prospective fascist dictators of Spain [and] \dots\ to prevent a victory by the People’s Front. \dots\ Sufficient has appeared \dots\ to make positive the assertion that Britain has come to its own agreement with General Franco.\endnotemark[\ref{en:GannesUSSR}]
\end{quotation}

But what mattered the fate of Spain, the future of the European revolution? All that weighed nothing in Stalin’s scale as against the tenuous friendship of imperialist France:

\begin{quotation}
  The Soviet Union could not come to an open clash with Blum on the non-intervention pact because that would have played into the hands of Hitler and the pro-Nazi faction in the London Tory cabinet which was trying to provoke just such a state of affairs.\endnotemark[\ref{en:GannesUSSR}]  
\end{quotation}

Therefore?

\smallskip
  
Pretend that the non-intervention committee has its uses:

\begin{quotation}
  Rather than to allow collusion between the Nazis and the Tory ministers to confront Spain, the Soviet Union strove to do all it could \emph{within the non-intervention committee} to stop fascist arms from being shipped to Spain!{\kern 0.5pt}\endnotemark[\ref{en:GannesUSSR}]  
\end{quotation}

Likewise, we have no doubt, Stalin will strive to do all he can \emph{within the committee of compromise} to get an equitable arrangement for the participation of the Loyalists in the joint regime with the fascists.

Precisely in these last months, when the Anglo--French scheme was taking final shape, Stalin found a new alibi to supplement those provided by the Franco--Soviet pact and ``collective security,'' with which to push the Loyalists into still greater dependence on the Anglo--French bloc. Louis Fischer gave it crudely enough:

\begin{quotation}
  The Spanish war has assumed such large dimensions and is lasting so long that Russia alone, especially if it must help China also, cannot bear the burden. Some other nation or nations must contribute. \dots\ If England would save Spain from Franco, Russia would perhaps be ready and able to save China from Japan. (\emph{The Nation,} October~16, 1937)
\end{quotation}

Thus, China becomes an alibi for not decisively aiding Spain, while Spain remains an alibi for not saving China! ``If England would save Spain from Franco\dots''

The Spanish people were also directed down the road of Anglo-French imperialism by the Communist International, of course, and by the Labor and Socialist International. Apart from pious gestures of organizing fundraising, the two internationals have called only for the workers to get ``their'' democratic governments to come to Spain’s aid. The ``international proletariat'' is called upon to ``compel the fulfillment of its main demands on behalf of the Spanish people, which are the immediate withdrawal of the interventionist armed forces of Italy and Germany; the lifting of the blockade; the recognition of all the international rights of the lawful Spanish government; the application of the statutes of the League of Nations against the fascist aggressors.'' (\emph{Daily Worker,} July~19, 1937).

All these ``demands'' are calls for \emph{governmental actions.} Since the French Socialists and British Labourites knew that serious governmental action could only take place in the event of war, and since their capitalist masters made plain they were not yet ready for war, they objected to too-precipitate nudges from the Comintern. Their accusation of war-mongering, Dimitrov could only answer by terming ``unworthy speculation on the anti-war sentiments of the masses at large!'' But the Socialists and Labourites were at one with the Stalinists in putting the fate of the Spanish people in the hands of ``their'' governments. For both had already pledged to support their capitalists in the coming war.

\dinkus

\begin{sloppypar}
  Whence would come the leadership to organize the Spanish masses in implacable struggle against the betrayal of Spain?
\end{sloppypar}

\medskip

That leadership could scarcely come from the ruling group in the \CNT\kn, not the least of whose crimes was its failure to steel the workers against illusions about Anglo--French aid. The very manifesto of July~17, 1937, addressed to the world proletariat, declaring, ``there is only one salvation: your aid,'' launched a slogan perfectly acceptable to the bourgeois--Stalinist bloc: ``Put pressure on your governments to adopt decisions favorable to our struggle.'' Roosevelt’s Chicago speech was acclaimed by the \CNT\ press. According to \emph{Solidaridad~Obrera} (October~7), it proved that ``democratic unity in Europe will be achieved only through energetic action against fascism.''

The \CNT\ leaders clung to their old course, merely asking the face-saving formula of ``anti-fascist front'' to be substituted for Popular Front, in their return to the government. Many of the local anarchist papers, close to the masses, reflected their outrage at the conduct of the leadership. One wrote:

\begin{quotation}
  To read a great part of the \CNT\ and anarchist press of Spain makes one indignant or give vent to tears of rage. Hundreds of our comrades were massacred in the streets of Barcelona during the May fighting, through the treason of our allies in the anti-fascist struggle; in Castille alone, almost a hundred comrades have been cowardly assassinated by the communists; other comrades have been assassinated by the same party in other regions; public and covert campaigns of de\-fam\-a\-tion and lies of all kinds are conducted against anarchism and the \CNT\kn, in order to poison and twist the spirit of the masses against our movement. And in the face of these crimes our press continues speaking of unity, of political decency; asking loyalty among, all, calm, serenity, sincerity, spirit of sacrifice, and all those sentiments we are the only ones to believe and feel and which only serve for the other political sectors to cover their ambitions and treason. \dots\ Not to tell the truth from now on would be to betray ourselves and the proletariat. (\emph{Ideas,} Bajo~Llo\-bre\-gat, September~30, 1937).
\end{quotation}

But the conduct of the \CNT\ leadership grew even more shameful. The rage of the masses after the fall of Santander forced the Stalinists to utter some mollifying words, calling for a cessation of the campaign against the \CNT. Whereupon even the most left of the big \CNT\ papers immediately hailed ``the rectification that undoubtedly has begun to be produced in the politics of the Communist Party.'' (\emph{\CNT,} October~6). The fall of Gij\'on, isolating the government still more from the masses, led to negotiations for \CNT\ support. All complaints forgotten, the \CNT\ leaders hastened to declare their readiness to enter the government!

Of the \UGT\ leaders, even less need be said. They had uttered not a word in defense of the \POUM\kn. Caballero made not a single public speech for five months, while the Stalinists prepared to split the \UGT\kn. The pact for united action, signed by the \CNT\ and \UGT\ on July~9, which could have organized the defense of the elementary rights of the workers, remained stillborn. Though obviously representing a majority of the provincial federations of the socialist party, the Caballero group went no further than to protest against the actions of the unrepresentative Prieto National Committee. Rather than an ally, the \UGT\ leaders simply further weakened the already impotent \CNT\ leaders.

Of the \POUM\kn, one can no longer speak of it as an entity. It was riv\-en irrevocably. All the blows of the leadership had been directed against the left wing, while the right wing had been courted and flattered. \emph{El~Comunista} of Valencia had openly flouted the party decisions, holding to a flagrantly People’s Front line, moving steadily toward Stalinism. Finally, a week before the outlawing of the party, the Central Committee was driven to publish a resolution (\emph{Juventud~Comunista,} June~10) declaring: ``The enlarged Central Committee \dots\ has agreed to propose to the Congress the summary expulsion of the fractional group that in Valencia has worked against the revolutionary policy of our dear party.''

The party congress was never held. Scheduled for June~19, it was preceded by the June~16 raids. The \POUM\ was entirely unprepared for illegal work, as the sweeping success of the raids indicated. Had the congress been held, it would have found the chief centers of the party, Barcelona and Madrid, aligned to the left against the leadership. One left-wing group called for condemnation of the London Bureau and for the creation of a new, fourth International. The other declared: ``It has become clear that there does not exist in our revolution a real Marxist party of the vanguard.''

It was not, then, to the existing organizations as such that one could look for new leadership to prevent a compromise with the fascists. Fortunately, events had only passed the leaders by. Among the masses of the \CNT\ and \UGT\ arose new cadres seeking a way out.
\nowidow

Of special significance were the Friends of Durruti, for they represented a conscious break with the anti-statism of traditional anarchism. They explicitly declared the need for democratic organs of power, juntas or soviets, in the overthrow of capitalism, and the necessary state measures of repression against the counter-revolution. Outlawed on May 26, they had soon reestablished their press. Despite the threefold illegality of government, Stalinists and \CNT\ leadership, the \emph{Amigo del Pueblo} voiced the aspirations of the masses. \emph{Libertad,} also published illegally, was another dissident anarchist organ. Numerous local anarchist papers, as well as the voice of the Libertarian Youth and many local \FAI\ groups, were raised against the capitulation of the \CNT\ leaders. Some still took the hopeless road of ``No More Governments.'' But the development of the Friends of Durruti was a harbinger of the future of all revolutionary workers in the \CNT-\FAI.

\begin{sloppypar}
  The masses of the \UGT\ and left socialists had long indicated their impatience with the pusillanimity of their leadership. But the first overt sign of revolutionary crystallization came only in October, when over five hundred youth withdrew from the ``United Youth'' to rebuild a revolutionary socialist youth organization.
\end{sloppypar}

Simultaneously, the split in the \UGT\kn, forced by the Stalinists, effectively awoke many left-wing workers to the problem of saving their unions from the Stalinist destroyers. In this struggle were inescapably posed all the fundamental problems of the Spanish revolution, the nature of class-struggle unionism, the role of the revolutionary party among the masses. From it would crystallize forces for the new party of the revolution.

Here, then, was the Herculean task of the Bolshevik-Leninists. These Fourth Internationalists, condemned to illegality by the \POUM\ leadership even at the height of the revolution, organized by those expelled from the \POUM\ only in the spring of 1937, seeking a way to the masses, must help to fuse the left wing of the \POUM\kn, the revolutionary socialist youth and the politically awakened \CNT\ and \UGT\ workers, to create the cadres of the revolutionary party in Spain. Could that party, if based on revolutionary foundations, be any but a party standing on the platform of the Fourth International?

Where else, indeed, would it seek for international comradeship and collaboration? The Second and Third Internationals were the organs of the betrayers of the Spanish people. Nor was it an arbitrary act, when the left wing of the \POUM\ called for repudiation of the London Bureau, the so-called ``International Bureau for Revolutionary Socialist Unity.'' For this center to which the \POUM\ had been affiliated, had sabotaged the struggle against Stalin’s frame-up system to which the \POUM\ had fallen victim.

\begin{sloppypar}
  While the \POUM\ itself had from the first denounced the Moscow trials and propagated a ``Trotskyist analysis,'' the London Bureau had worked in the opposite direction. It had refused to collaborate in a commission of inquiry into the Moscow trials. Why? Brockway—then launching a joint \textsc{ilp–cp} ``Unity Campaign''—blurted out the reason: it would ``cause prejudice in Soviet circles.''
  Brockway therefore proposed \hspace{1pt}\textls[-40]{\dots} a commission to investigate Trotskyism! When taxed with this course, Brockway defended himself~by impugning the character of the Commission of Inquiry headed by~John~Dewey.
\end{sloppypar}

Meanwhile, the London Bureau was blowing up. The \textsc{sap} (Socialist Workers Party of Germany) had first attacked the Moscow trials, but soon abandoned criticism of Stalinism, signing a joint pact for a People’s Front in Germany. \emph{Juventud Comunista} (June~3) reported the split of the London Youth Bureau: ``The \textsc{sap} Youth had placed itself in a Stalinist and reactionary position \dots\ The Youth of the \textsc{sap} had signed one of the most shameful documents which the history of the German workers’ movement has known.'' On the very day the \POUM\ leadership was arrested as Gestapo agents, \emph{Julio,} \PSUC\ youth organ (June~19) under the headline, ``Trotskyism is Synonymous with Counter-Revolution,'' had hailed the policies of the \textsc{ilp} and \textsc{sap} youth groups and proudly pointed out that the Swedish affiliate of the London Bureau was steadily approaching a Stalinist People’s Front policy.

Even fouler was the position of those other ``allies'' of the \POUM\kn, the Brandler–Lovestone groups. For a decade they had defended the Stalinist bureaucracy’s every crime, on the basis of a false distinction between Stalin’s policies in the Soviet Union and the erroneous policies of the Comintern elsewhere. When Zinoviev and Kamenev were done to death, these lawyers for Stalinism had hailed the dread deed as a vindication of Soviet justice. They had likewise defended the second Moscow trial in February 1937. I was myself present at a public meeting in the Lovestone headquarters when Bertram Wolfe apologized because a \POUM\ representative had just called the trials frame-ups! Only after the execution of the Red Generals had the Lovestone group—with no explanation—begun to reverse its course. In ten years they had done their utmost to aid Stalin in pinning the appellation counter-revolutionary on the Trotskyists, and even when they had been driven to accepting Trotsky’s analysis of the Stalin purge, these foreheads of brass remained as ever the implacable enemies of the resurgence of the revocation in Russia as, indeed, elsewhere. As the \textsc{sap,} the Swedish affiliate, etc., move out of one end of the London Bureau, they are being replaced by the Brandler–Lovestone movement. The change has scarcely brought improvement!

How did this International Bureau for Revolutionary Socialist Unity prepare for the defense of the \POUM? Its meeting of June~6, 1937, adopted two resolutions.

\begin{quote}
  \textsf{\textsc{Resolution~1}} \quad
  The \POUM\ alone has recognized and proclaimed the necessity of transforming the anti-fascist struggle into a fight against capitalism under the hegemony of the proletariat. This is the real reason for the ferocious attacks and calumnies of the Communist party allied with the capitalist forces in the Popular Front against the \POUM\kn.

  \textsf{\textsc{Resolution~2}} \quad
  Every measure directed against the revolutionary working class of Spain is at the same time a measure in the interests of French and British imperialism and a step towards compromise with the fascists.
  \noclub

  In this hour of danger we appeal to all working class organizations of the whole world, and \emph{particularly} to the Second \emph{and Third International} \kp\dots\ let us at last take up a unified stand against all these treacherous manoeuvres of the world bourgeoisie. [My italics.]
\end{quote}

One resolution for the left, one for the semi-Stalinist right—that is, the London Bureau.%
\endnote{%
%
In the June~4 issue of the \emph{New Leader,} the \textsc{ilp} leader, Fenner Brockway, gave the \POUM\ some advice at this critical juncture. Some revealing extracts:

\begin{quotation}
  It is important that \POUM\kn, together with other workers’ forces, should concentrate on the fight against Franco. \dots\ The Spanish Communist Party had justifiably criticized the absence of coordination at the front and the bad organization of the armed forces. \POUM\ must be careful not to appear to resist proposals which will facilitate efficiency in the fight against Franco but that does not mean that it must accept without protest a return to the reactionary structure of the old army.
\end{quotation}

This kind of advice a week before the \POUM\ was outlawed! That the task of the \POUM\ was relentless, implacable struggle against the government, placing no confidence whatsoever in the \CNT\ and \UGT\ leaders, to issue united front proposals for concrete, daily defence of elementary workers’ rights, and to immediately combine legal with illegal work—this was naturally beyond Brockway. The same issue carries a letter from the \textsc{ilp} representative in Spain, McNair, to the Stalinist leader, Dutt, beginning:

\begin{quotation}
  It is painful for me to be compelled to enter into controversy with a comrade of the CP in view of the desire I have to see unity among the working class parties. \dots\ I still hold the point of view \dots\ “the important thing to have in mind is that the Unity Campaign in Britain should engender unity in Spain rather than allow Spanish disunity to break up the Unity Campaign in Britain.\ \dots”
\end{quotation}

}

``But are not the principles you propose for the regroupment of the Spanish masses, are they not intellectual constructions to which the masses will feel alien? And is it not too late?''

No! We revolutionists are the only practical people in the world. For we merely articulate the fundamental aspirations of the masses, indeed, what they are already saying in their own way. We merely clarify the nature of the instrumentalities, above all, the nature of the revolutionary party and the workers’ state, which the masses need to achieve what they want. It is never too late for the masses to begin to hew their road to freedom. Pessimism and skepticism are luxuries for the few. The masses have no other choice except to fight for their lives and the future of their children.

If our analysis has not illuminated the inner forces of the Spanish revolution, let us recall a few words of Durruti on the battlefield of Aragon, when he was leading the ill-armed militias in the only substantial advance of the whole civil war. He was no theoretician, but an activist leader of masses. All the more significantly do his words express the revolutionary outlook of the class-conscious workers. The \CNT\ leaders have buried these words deeper than they buried Durruti! But let us remember them:

\begin{quotation}
  For us it is a question of crushing fascism once and for all. Yes, and in spite of the government.
  
  No government in the world fights fascism to the death. When the bourgeoisie sees power slipping from its grasp, it has recourse to fascism to maintain itself. The liberal government of Spain could have rendered the fascist elements powerless long ago. Instead it temporized and compromised and dallied. Even now at this moment, there are men in this government who want to go easy with the rebels. You never can tell, you know—he laughed—the present government might yet need these rebellious forces to crush the workers’ movement\dots
  
  We know what we want. To us it means nothing that there is a Soviet Union somewhere in the world, for the sake of whose peace and tranquillity the workers of Germany and China were sacrificed to fascist barbarism by Stalin. We want the revolution here in Spain, right now, not maybe after the next European war. We are giving Hitler and Mussolini far more worry today with our revolution than the whole Red Army of Russia. We are setting an example to the German and Italian working class how to deal with fascism.
  
  I do not expect any help for a libertarian revolution from any government in the world. Maybe the conflicting interests in the various imperialisms might have some influence on our struggle. That is quite possible. Franco is doing his best to drag Europe into the conflict. He will not hesitate to pitch Germany against us. But we expect no help, not even from our own government in the last analysis.
\end{quotation}

``You will be sitting on top of a pile of ruins if you are victorious,'' said Van~Paassen.

\medskip

Durruti answered:

\begin{quotation}
  We have always lived in slums and holes in the wall. We will know how to accommodate ourselves for a time, For, you must not forget, we can also build. It is we who built theses palaces and cities, here in Spain and in America and everywhere. We, the workers, we can build others to take their place. And better ones. We are not in the least afraid of ruins. We are going to inherit the earth. There is not the slightest doubt about that. The bourgeoisie might blast and ruin its own world before it leaves the stage of history. We carry a new world, here, in our hearts. That world is growing this minute.%
  \endnote{Interview of Durruti with Pierre Van~Paassen, \emph{Toronto Star,} September~1936.}
\end{quotation}

\begin{flushright}
  November 10, 1937
\end{flushright}

\backmatter

  \chapter{Postscript}

\lettrineT{he jailing of} workers and peasants and the opening of the front lines by ``republican'' officers to the fascists: that is the story of Loyalist Spain from November 1937 to May 1938. There is time and space to add only a few words as this book goes belatedly to press.

General Sebastian Pozas\index{Pozas, Sebastian} adequately symbolizes the period: An officer under the monarchy; an officer under the republican--socialist coalition of 1931--1933; an officer under the Lerroux--Gil Robles \emph{bienio negro} of 1933--1935. Minister of War before the fascist revolt broke out. He moved heaven and earth to get away from Madrid in the dark days of the siege in November 1936. When Catalan autonomy was done away with and the \CNT\ troops were at last subordinated entirely to the bourgeois regime, Pozas was appointed chief of all the armed forces of Catalonia and the Aragon front. He effectively purged the armies of  \CNT\ and \POUM\ ``uncontrollables,'' arranging for whole divisions to be wiped out when they were sent under fire without artillery or aerial protection.

``Comrade'' Pozas, who graced the plenum of the Central Committee of the \PSUC, was ``obviously'' the man to hold the Aragon front against Franco \dots\ Now he is in a Barcelona prison, charged---and the military story is only too clear---with betraying the Aragon front to Franco.

The consequences of the alliance with the ``republican'' bourgeoisie, of the People’s Front programme, are now apparent. The fascists have reached the Mediterranean. They have split the remaining anti-fascist forces in twain. For the time being the race between Franco and the regroupment of the proletariat has been won by Franco. The Stalinists, the Prieto and Caballero socialists, the anarchist leaders, have proven insurmountable obstacles on the road to regroupment, immeasurably facilitating Franco’s victory.

These criminals will soon fall out among themselves. They will attempt to shift the blame on each other. In that attempt, much more will be revealed concerning the machinations whereby they bound the workers and peasants hand and foot, and made impossible a successful war against Franco. But already we know enough to say that no alibis will enable them to clear themselves. All---Stalinists, socialists and anarchists---are equally guilty of having betrayed their followers. All have betrayed the interests of the workers and peasants---the interests of humanity---to the bestial regime of fascism.

Many will escape from Franco, as did the Stalinist and social-democratic functionaries from Hitler. But the millions of workers and peasants cannot escape. For them, today, tomorrow, the next day, as long as life continues, the task of smashing fascism remains to be carried out. Fight or be crushed---they have no other alternative.

The Spanish proletariat---crushed, as Berneri said, between the Stalino--Prussians and the Franco--Versaillaise---may yet touch off a flame that will again light up the world. Passing beyond the Pyrenees, where the period of the People’s Front, as in Spain, is closing, that flame can unite with the hopes of the French proletariat, now faced with a choice between naked bourgeois dictatorship and the road of revolution.

But if the revolutionary conflagration does not break out, or is smothered, what then?

The tragic lessons of Spain are, in any case, of profound concern to the American\index{U.S.} working class and have an immediate bearing upon ``purely American'' problems.

Here the issue will soon enough be posed as inexorably as in Spain or France. The simple truth is that American capitalism has arrived at such an impasse that it can no longer feed its slaves. An army of the unemployed as large as that of 1932 now receives at Roosevelt’s hand a fraction of the inadequate hand-out offered in 1933. The production index drops at four and five and six times the rate of the 1929--1932 decline. The government prepares cold-bloodedly for imperialist war as a ``way out.'' Crisis, unemployment, war---these have become the ``normal'' characteristics of the declining capitalist order. Since 1929 America has been ``Europeanized.'' We face here the problems which have been faced since the war by the European proletariat.

Pessimism, defeatism, these are the reactions of the few, who thereby deduce from the reformist betrayals in Europe a justification for abandoning the American masses to a like doom. But to the workers and the oppressed toiling masses of city and country, pessimism and defeatism are alien. They must fight or be smashed---they have no other alternative. The vast, inexhaustible vitality of the American working class is the richest capital of the international labour movement. It has yet to be used, yet to be thrown into the breach. In the last four years, the American proletariat has given such evidence of its resources and its power as many of us did not dream of in 1933. It has organized itself within the very citadel of American capitalism---steel, rubber, autos. It can overthrow that citadel---if it has the will to do so and a leadership capable of assimilating the lessons of these catastrophes.

The task of the book is to provide the class-conscious worker and his allies in America with materials for understanding why the Spanish proletariat has been defeated, and by whom betrayed.

The heroism of the Spanish workers and peasants must not be in vain. From their failing hands the banner of struggle to the death against capitalism can be taken up by the American workers. Let them grasp it with the aid of a vanguard which has assimilated all the terrible lessons of Russia, Spain and France, with a strength and assurance such as the world has not yet seen, and carry it through to victory, not only for themselves, but for the whole world of toiling humanity!

\begin{flushright}
  May 5, 1938 \\
  Minneapolis, MN
\end{flushright}

  \pagestyle{myheadings}

  \markboth{Notes}{}
  \toggletrue{endnotes}
    \printendnotes
  \togglefalse{endnotes}
  
  \markboth{Index}{}
  \addcontentsline{toc}{chapter}{Index}
  \printindex

\end{document}