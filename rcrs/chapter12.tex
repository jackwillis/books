\chapter{``El Gobierno de la Victoria''}

\lettrineL{a Pasonaria} christened the new cabinet ``the government of victory.'' We have made up our minds,'' she said, ``to win the war quickly, though that victory cost us an argument with our dearest comrades.'' The Stalinists launched a worldwide campaign to prove that victory had been held up by Caballero and that it would now be forthcoming.

The annals of the Negr\'in government, however, proved to be not the record of military victory, nor even of serious attempts at military victory, but of ruthless repression of the workers and peasants. That reactionary course was dictated to the government by the Anglo--French rulers to whom it looked for succor. The spokesman for the Quai~d’Orsay, \emph{Le Temps}, indicated the real meaning of the ministerial crisis:

\begin{quotation}
  The republican government of Valencia has reached the point where it must decide. It can no longer remain in the state of ambiguity in which it has hitherto lived. It must choose between democracy and proletarian dictatorship, between order and anarchy. (May 17)
\end{quotation}

The next day the Negr\'in cabinet was formed. \emph{Le Temps} approved but peremptorily pointed the road the new regime must resolutely travel:

\begin{quotation}
  It would be too early to conclude that the orientation in Valencia is toward a more moderate government determined to free itself finally from the control of anarcho-syndicalists. But this is an attempt which, in the end, will have to be made no matter what the resistance of the extremists may be.
\end{quotation}

Clear directives, indeed!

\medskip

The government, wrote an ardent sympathizer with its reactionary course, the \emph{New York Times} correspondent, Matthews, ``intends to use an iron hand to maintain internal order.''

\begin{quotation}
  \noindent
  —\,By so doing, the government hopes to win the sympathy of the two democracies that mean most to Spain~-- Great Britain and France~-- and to retain the support of the nation that has been most helpful, Russia. The government’s main problem now is to pacify or squash the anarchist opposition. (May~19, 1937).
\end{quotation}

``In a word, the government unloosed a completely repressive machinery without any regard for the state of the war, or the requirements of keeping up the morale of the war,'' as a \FAI\ statement of July 6 put it.

``Anarchists are being eliminated as an active factor. The Caballero Socialists, if they persist in their present tactics, may be outlawed within three months,'' wrote the Stalinist, Louis Fischer. (\emph{The Nation}, July 17).

In Caballero’s cabinet, Garc\'ia Oliver, the ``one hundred percent anarchist,'' had labored mightily, fashioning democratic tribunals and judicial decrees, while the counterrevolution advanced behind him. The Generalidad had used Nin for the same purpose during the early months of the revolution. Now the government named as its Minister of Justice, the Basque capitalist and devout Catholic, Manuel Irujo. That such a man could hold this office meant: the time for pretending is over. Irujo in 1931 had voted against adoption of the republican constitution as a radical and atheistic document. Was he not then just the man for the Minister of Justice?

Irujo’s first step was to dismantle the popular tribunals which, each constituted by a presiding judge and fifteen members designated by the different antifascist organizations, had been set up after July~19, 1936. The \FAI\ members were now barred from the tribunals~-- by the device of, decreeing that only organizations legal on February~16, 1936, could participate. The \FAI, of course, had been outlawed by the \emph{bienio negro}!

Most of the presiding judges had been left-wing attorneys. Roca, former subsecretary of the Ministry, has since told how, in September 1936, the Ministry of Justice had called a meeting of the old judges and magistrates and had asked for volunteers to go to the provinces and set up the tribunals. Not one would volunteer. They knew the fascists would have to be convicted. Now the tribunals were cleansed of the left-wing attorneys who were replaced by the once-reluctant judges, for the tribunals no longer were to ferret out fascists but to prosecute workers. \emph{Daily bulletins} listing fascists and reactionaries put at liberty were issued by Irujo’s Ministry.

On June 23, the government decreed special courts to deal with sedition. Among ``seditious acts'' were included: ``giving military, diplomatic, sanitary, economic, industrial or commercial information to a foreign state, armed organization or private individual,'' and all offenses ``tending to depress public morale or military discipline.'' The judges were to be appointed by the Ministries of Justice and Defense, empowered to sit in secret session and to bar any third parties. The decree concludes:

\begin{quotation}
  \noindent
  —\,Attempted or frustrated offenses, conspiracies and plans, as well as complicity in sheltering of persons subject to this decree, may be punished in the same way as if the offenses had actually been committed. Whoever, being guilty of such offenses, denounces them to the authorities shall be free of all punishment. Death sentences may be imposed without formal knowledge of the cabinet.
\end{quotation}

The confession clause, the punishment for acts never committed, the secret trials, were translated directly from Stalin’s laws. The sweeping definition of sedition made treason of any opinion, spoken or written or indicated by circumstantial evidence, which was construable as critical of the government. Applicable to any worker who agitated for better conditions, to strikers, to any governmental criticism in a newspaper, to almost any statement, act or attitude other than adoration of the regime, this decree was not only unprecedented in a democracy, it was more brazen than Hitler or Mussolini’s juridical procedure.

On July 29, the Ministry of Justice announced that trials under this decree were being prepared for ten members of the Executive Committee of the \POUM\kn. These men had been arrested on June 16 and 17~-- before the new decree. That meant that the decree, to cap everything else, was an \emph{ex post facto} law, punishing crimes allegedly committed before the law was passed! Thus, the most unquestioned juridical principle of modern times was expressly repudiated.

\medskip

Irujo sponsored another decree, adopted and issued by the government on August 12, which declared:

\begin{quotation}
  Whoever censures as fascist, as traitor, as anti-rev\-o\-lu\-tion\-ary, a given person or group of persons, unreasonably or without sufficient foundation, or without the [court] authority having pronounced sentence on [the accused]\dots.
  
  Whoever denounces a citizen for being a priest or for administering the sacrament \dots\ causes an unnecessary and disruptive disturbance of public order when not committing an irreparable crime worthy of penal punishment.
\end{quotation}

This decree not only outlawed sharp ideological criticism of anybody in the governmental bloc, but also put an end to ferreting out of fascists by the workers. It also ended all forms of surveillance of the Catholic priesthood~-- just after the Vatican had openly thrown full support to Franco. Denunciations ``without the court authority having pronounced sentence'' in practice applied only against criticism from the left. The Stalinists continued, of course, to denounce the \POUM\ as fascists, though no sentence had been passed.

\begin{sloppypar}
Press censorship operated under a system which not only destroyed free criticism but required that the very acts of censorship be concealed from the people. Thus, on August 7, \emph{Solidaridad Obrera} was suspended for five days for disobeying censors’ orders, the specific act of disobedience being~-- according to G\'omez, General Delegate of Public Order in Barcelona, who had given the order that ``they should not publish white spaces.\kn\kn'' That is, deletions by the censor who worked from galleys must be hidden from the masses by inserting other material! As a silent protest, the \CNT\ press had been leaving censored spaces blank.
\end{sloppypar}

\medskip

On August 14, the government issued a decree outlawing all press criticism of the Soviet government:

\begin{quotation}
  With repetitions that permit divining a deliberate plan of offending an exceptionally friendly nation, thereby creating difficulties for the government, various newspapers have occupied themselves with the \USSR\ unsuitably\dots.
  This absolutely condemnable license should not be permitted by the council of censors\dots. The disobeying newspaper will be suspended indefinitely, even though it may have been passed by the censor; in that case the censor who reads the proofs is to be held for the Special Tribunal charged with dealing with crimes of sabotage.
\end{quotation}

The censorship decrees no longer referred to the radio. For, on June~18, police detachments had appeared at all the radio stations belonging to the trade unions and political parties and closed them down. Thenceforward, the government monopolized radio broadcasting.

One of the most extraordinary uses of the press censorship came when the Stalinist--Prieto bloc on October~1 split the \UGT\ by a rump meeting of some unions which declared the Caballero-led Executive Committee deposed. While the new ``Executive'' freely published a stream of abusive declarations, the statements of the Caballero Executive were cut to pieces, as were the headlines in the \CNT\ press referring to it as the rightful Executive. Formal protests of the \CNT\ press against the government, thus taking sides in the inner-union fight, were fruitless.

Despite terrible instances~-- in almost every city captured by the fascists~-- of Assault and Civil Guards in large numbers going over to the fascists during the siege, the Ministry of the Interior proceeded to cleanse the police, not of the old elements, but of the workers sent in by their organizations after July~19. Examinations were decreed for all who entered the services in the past year. The Security Councils, formed by the antifascists among the police to clean out fascist elements, were ordered dissolved. More, the director general of the police, the Stalinist, Gabriel Mor\'on, ordered the ranks not to make denunciations of fascist suspects in the police, on pain of dismissal. (\emph{\CNT,} September 1).

\index{Private property}
Held to a slower pace until the political preconditions had been more fully achieved, the economic counterrevolution was now sped up. In agriculture, the road to be followed had been mapped by the very first decree, October~7, 1936, which merely confiscated estates of fascists, leaving untouched the system of private property in land, including the right to own large properties and to exploit wage labor.

\index{Collectivization}
\begin{sloppypar}
Despite the decree, however, collectivized agriculture became wide\-spread during the first months of the revolution. The \UGT\ was at first unfriendly to the collectives, and changed its attitude only after the movement developed strong roots in its own ranks. Several factors explained the speedy development of collectivized farming.
\end{sloppypar}

\index{Peasantry}
Unlike the old Russian \emph{moujik,} the Spanish peasants and agricultural workers had built trade unions for decades, and had provided considerable sections of the membership of \CNT\ and \FAI, \UGT\kn, \POUM\ and the Socialist Party. This political phenomenon flowed in part from the economic fact that the division of the land was even more unequal in Spain than in Russia, and almost the entire Spanish peasantry was dependent partially or wholly on wage labor on the big estates. Hence, even those with a bit of land were weakened in the peasants’ traditional preoccupation with his own plot of ground. Collective labor also derived strength from the almost universal necessity for joint work in providing water for the dry land.

To these factors were added the enthusiastic aid given to the collectives by many factories, providing equipment and funds, the equitable purchasing of produce from the collectives by the workers’ supply committees and the cooperative markets, the friendly collaboration of the collectivized railroads and trucks in bringing the produce to town.

Another important factor was the peasant’s realization that he no longer stood alone. ``If, in some locality, a crop is lost or greatly reduced because of a long drought, etc.,\kn\kn''
wrote the head of the \CNT\ agrarian federation in Castille, speaking for 230 collectives, ``our peasants don’t have to worry, don’t have to fear hunger, for the collectives in the other localities or regions consider it their duty to help them out.\kn\kn'' Many factors thus joined to encourage the swift development of collective agriculture.

But with the Stalinist Uribe’s assumption of the Ministry of Agriculture, first in Caballero’s cabinet and then in Negr\'in’s, the weight of the government was thrown against the collectives.

``Our collectives did not receive any sort of official aid. On the contrary, if they received anything at all, it was obstruction and calumnies from the Minister of Agriculture and from the majority of institutions that depend upon this Minister,'' reported the \CNT’s Castilian agrarian federation. (\emph{Tierra y Libertad,} July~17).

\medskip

Ricardo Zabalza, national head of the \UGT’s Peasant and Land\-work\-ers’ Federation declared:

\begin{quotation}
  The reactionaries of yesterday, the erstwhile agents of the big landowners are given all sort of assistance by the government while we are deprived of the very minimum of it or are even evicted from our small holdings\dots.
  
  They want to take advantage of the fact that our best comrades are now fighting on the war fronts. Those comrades will weep with rage when they find, upon leave from the war fronts, that their efforts and sacrifices were of no avail, that they only led to the victory of their enemies of old, now flaunting membership cards of a proletarian organization [the Communist Party].
\end{quotation}

These agents of the big landowners, the hated \emph{caciques}~-- overseers and village bosses~-- had been the backbone of the political machine of Gil-Robles and the landowners. Now they were to be found in the ranks of the Communist Party. Even such an outstanding chieftain of Gil-Robles’ machine as the secretary of the \CEDA\ in Valencia had survived the revolution\dots\ and joined the Communist Party.

Uribe justified the assault on the collectives by claiming that unwilling peasants were being forced to join them. One need scarcely comment on the irony that a Stalinist should complain of forced collectivization, after the Draconian slaughters and exiles of the ``liquidation'' of the Russian \emph{kulaks}! Uribe would undoubtedly have produced evidence on this score, had he been able to find it, but none was forthcoming. \emph{Both} big peasant and landworkers’ federations, the affiliates of the \CNT\ and \UGT, opposed forced collectivization, favoured voluntary collectives, and denounced the Stalinists as supporters of the \emph{caciques} and reactionary rich peasants.

In June, the Socialist \emph{Adelante} sent a questionnaire to the various provincial sections of the \UGT\ peasants’ organization: almost unanimously they defended the collectives, and to a man reported that the main opposition to the collectives came from the Communist Party, which for this purpose recruited the \emph{caciques} and utilized the governmental institutions. All declared the October~7 decree was creating a new bourgeoisie. In a letter of protest to Uribe, Ricardo Zabalza described the simple but effective system of the Stalinists in attacking the collectives: old \emph{caciques,} \emph{kulaks,} landowners, were recruited and organized by the Stalinists and thereupon demanded the dissolution of the local collective, making claims on its land, equipment, stores of grain.

Every such controversy brought in its wake the ``mediation'' of Uribe’s representatives who invariably decided in favor of the reactionaries, imposing ``settlements'' whereby the collectives were gradually deprived of their equipment and land. When asked to explain this strange behavior, said Zabalza, the government agents declared they were acting under specific orders from their superior: Uribe. It was not surprising then that the \UGT\ Peasants’ Federation of Levante Province denounced Uribe as ``Public Enemy Number One.'' Irujo’s wards, the ex-fascists recently released, became, by the very fact of release, eligible to demand return of their lands. When one of these returned as landlord, peasants fiercely resisted~-- and the Assault Guards were sent against them.

In the cities and industrial towns, too, the government proceeded to destroy all elements of socialization.

\medskip

``It is unquestionably true that had the workers not taken over control of industry the morning after the insurrection, there would have been complete economic paralysis,''
wrote the Stalinist, Joseph Lash,
``but the improved schemes of workers’ control of industry have not worked out very well.'' (\emph{New Masses,} October 19).

\medskip

There was a half-truth in this but the whole truth leads not back to the old proprietors but forward to the workers’ state. Planning on a national scale is obviously impossible through factory and union apparatus alone. What is needed is a centralized apparatus, i.e., a state. Had the \CNT\ understood this, and initiated the election of militia and peasant and factory committees, joined in a national council which would have constituted the government~-- that would have been a \emph{workers’ state,} which would have given, full scope to the workers’ committees and yet achieved the necessary centralization.

Instead, the anarchist leaders fought a losing battle, arguing as to just how much authority the state should have. Peir\'o, ex-minister of industry, for example:

\begin{quotation}
  I was prepared to nationalize the electric industry in the only way compatible with my principles~-- leave its administration and direction in the hands of the trade unions and not in the hands of the state. The state has only the right to act as accountant and inspector.
\end{quotation}

Formally correct: Lenin said that socialism was simply bookkeeping. But only a workers’ state would faithfully accept the functions of accountant and inspector, while the existent Spanish state, a bourgeois state, must fight socialization. Here again the anarchists, continuing to make no distinction between a workers’ and bourgeois state, yielded to the bourgeois state, instead of fighting for the workers’ state.

Through the Ministry of Defense, factories were taken over one by one. On August~28, a decree gave the government the right to intervene in or take over any mining or metallurgical plant. Quite explicitly, the government stated that workers’ control was to be limited to protection of working conditions and stimulation of production. Resisting factories found themselves without credits or, having made deliveries to the government, payment was not forthcoming until the government’s will was accepted. In many foreign-owned plants, the workers had already been stripped of any form of authority. The Department of Purchases of the Ministry of Defense announced that on a given date it would make contracts for purchases only with enterprises functioning ``on the basis of their old owners'' or ``under the corresponding intervention controlled by the Ministry of Finance and Economy.'' (\emph{Solidaridad Obrera,} October~7).

The next step, for which the Stalinists had been campaigning for months, was militarization of all industries necessary to war: transportation, mines, metal plants, munitions, etc.\@ This barracks regime is reminiscent of Gil-Robles under whom munitions workers were also militarized~-- strikes and trade union membership forbidden. The militarization decree is sugar-coated by being titled ``militarization and nationalization decree.'' But to militarize factories already in workers’ hands, coupled with government recognition of full indemnification of former owners, simply ends workers’ control and prepares for returning the factories to their former owners.

Above all, the government was symbolized by new-found friends~-- reactionary deputies~-- who now appeared for the first time in Spain since July 1936.

Miguel Maura was there! Chief of the extreme right republicans, Minister of the Interior in the first republican government, an implacable enemy of the trade unions, the first minister of the republic to reinstitute the dreaded ``law of flight'' to shoot political prisoners~-- Maura had fled the country in July. His brother, Honorio, a monarchist, had been shot by the workers; the rest of his family had gone over to Franco. In exile, Maura had made no contact with the Spanish Embassies.

Portela Valladares was there! Governor General of Catalonia for Lerroux after the crushing of Catalan autonomy in October 1934, he had been the last premier of the \emph{bienio negro,} just before the February 1936 elections. He had fled Spain in July. What he had done in the interim was unknown. Now he rose in the Cortes:

\begin{quotation}
  This parliament is the \emph{raison d’etre} of the Republic; it is the title to life of the Republic. My first duty before you, before Spain, before the world, is to assure the legitimacy of your power\dots.
  
  Today is for me one of intimate and great satisfaction, in having contributed with you to seeing our Spain in transition to a serious and profound reconstruction.
\end{quotation}

At the end of the session, he and Negrin embraced. To the press, Valladares praised ``the prevailing atmosphere as he had observed it in Spain.\kn\kn'' He went back to Paris, while the Stalinist press proved statistically that the presence of Valladares and Maura, signifying the centre’s support of the regime, gave a statistical majority of the electorate to the government.\endnote{This anti-Marxist criterion enabled the fascists, by the same criterion, to argue that the rightist vote, plus that of those centre deputies now with them, constituted a majority of the people. The claims of both, of course, were based on the February~1936 election figures. The Marxist criterion is that a revolution derives its justification from the revolutionary vanguard’s representing a majority of the working class, supported by the peasantry. By the present Stalinist criterion, one could condemn the Russian Revolution!}

The ardour of the Stalinist press was cut abruptly by the reproduction in the fascist \emph{Diario~Vasco,} October~8, 1937, of a letter of Va\-lla\-da\-res to Franco, dated October~8, 1936, offering his services to the ``national~cause.''

The Stalinist welcome to Valladares and Maura was ``offset'' by a passing reference of La~Pasionaria to the unwelcome presence in the Cortes of another reactionary a minor figure, a member of Lerroux’s ruling party of the \emph{bienio negro.} The deputy, Guerra del~Rio, was given the floor to answer, in effect, that if the government rested on the Cortes, here he was. La~Pasionaria subsided. \CNT\ attacks on Maura and Valladares were deleted by the censor.

\medskip
Was it for this then that the masses had shed their blood?
\medskip

But we have still to tell the story of the government’s conquest of Catalonia and Aragon.