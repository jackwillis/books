\chapter[The Revival of the Bourgeois State]{The Revival of the Bourgeois State: September 1936--April 1937}

\section*{The Economic Counter-Revolution}

\lettrineT{he eight months} after the workers’ representatives entered the Madrid and Barcelona cabinets saw the proletarian conquests in the economic field slowly whittled down. Controlling the treasury and the banks, the government was able to force its will on the workers by the threat of withdrawing credits.

In Catalonia, the chief industrial centre, the process moved more slowly but to the same end. Some fifty-eight financial decrees of the Generalidad in January sharply restricted the scope of activity of the collectivized factories. On February~3, for the first time, the Generalidad dared decree illegal the collectivization of an industry-dairy products. During the April ministerial crisis, the Generalidad annulled workers’ control over the customs by refusing to certify workers’ ownership of material that had been exported and was being tied up in foreign courts by suits of former owners; henceforth the factories and agricultural collectives exporting goods were at the mercy of the government.

Comorera, \PSUC\ chieftain, had taken over the Ministry of Supplies on December 15, when the \POUM\ was ousted from the cabinet. On January 7, he decreed dissolution of the workers’ supply committees which had been purchasing food from the peasants. Into this breach poured the speculators and traders of the \textsc{gepci} (Corporation and Units of Petty Merchants and Manufacturers)---holding \UGT\ cards!---and the resultant hoarding and rise of food prices led to widespread malnutrition. Each family received ration cards but supplies were not rationed according to the number of persons served by each depot. In the workers’ suburbs of Barcelona long queues stood throughout the day, supplies often exhausted before the end of the queues was reached, while in the bourgeois districts there was plenty. The privately owned restaurants had ample supplies for those who could pay the price. Milk was unobtainable for workers’ children but purchasable in the restaurants. Though bread (at a fixed price) was often not to be had, cake (price uncontrolled) was always to be bought.

On the sixth anniversary of the republic (April 14, boycotted by \FAI, \CNT, and \POUM), the Esquerra and Stalinist demonstrations were overshadowed by women’s demonstrations against the food prices. Yet the Stalinists put to political use even their crimes. The masses were given to understand that \PSUC\ and \UGT\ membership would get them better rations. Anonymous stickers blamed the collectivized farms and transportation for the price rises.

Vicente Uribe, Stalinist Minister of Agriculture, played the same role here as a Stalinist minister of agriculture had played in the Wang Ching-wei regime of 1927, in Wuhan, in fighting the peasants. Uribe’s department dismantled collectives, organized former landowners to whom their lands were returned as state ``co-administrators,'' prevented the collectives from selling their produce without the use of middlemen.

A national campaign for ``state control'' and ``municipalization'' of industry laid the basis for wresting all control from the factory committees. The economic counter-revolution proceeded, however, comparatively slowly. For the bourgeois-Stalinist bloc understood, as the anarchists did not, that the necessary pre-condition for destroying the workers’ economic conquests was the crushing of the workers’ militias and police, and the disarming of the workers in the rearguard. But force alone was insufficient to achieve this end. Force had to be combined with propaganda.

\section*{Censorship}

To facilitate the success of their own propaganda, the bourgeois-reformist bloc resorted, through the government, to systematic curtailment of the \CNT-\FAI-\POUM\ press and radio. The \POUM\ was the chief victim. While it was still in the Generalidad, the Catalan \emph{Hoja Official} boycotted all mention of \POUM\ meetings and radio broadcasts. On February 26, the Generalidad forbade a \CNT-\POUM\ mass meeting in Tarragona. On March 5, \emph{La Batalla} was fined 5,000 pesetas and refused a bill of particulars on the charge of disobeying the military censor. On March 14, \emph{La Batalla} was suspended for four days, this time openly for a political editorial. At the same time the Generalidad refused to the \POUM\ use of the official radio station for broadcasts. The \POUM\ dailies in Lerida, Gerona, etc., were constantly harassed.

The deadliest blows to the \POUM\ in this period, however, were delivered outside Catalonia. The Stalinist-controlled Madrid Defence Junta in January permanently suspended \emph{\POUM}, a weekly. The same authority suspended and confiscated the presses of \emph{El Combatiente Rojo}, \POUM\ militia daily, on February 10, and shortly thereafter suspended the \POUM\ radio station, closing it permanently in April.

The Junta also refused to permit the \POUM\ Youth (Juventud Comunista Iberica) to publish \emph{La Antorcha}, the official prohibition cynically stating that ``the \textsc{jci} needs no press.'' \emph{Juventud Roja}, Valencian \POUM\ Youth organ, was submitted to severe political censorship in March. The only \POUM\ organ untouched was \emph{El Comunista} of Valencia, weekly organ of the ferociously anti-Trotskyist, half-Stalinist right-wing.

Another important field of work among the masses was closed to the \POUM\ when the \POUM\ Red Aid was excluded at the demand of the \PSUC\ from the Permanent Committee of Aid for Madrid. The \CNT, in the name of unity, agreed to this criminal act, which became national in scope in April, when the \POUM\ Red Aid was excluded from participating in Madrid Week.

This sketchy outline of governmental outlawry of \POUM\ activities \emph{before May} conclusively refutes the Stalinist claim that the \POUM\ was persecuted for its participation in the May events.

The censorship against the \POUM\ was carried out by cabinets in which sat \CNT\ ministers. Only the Anarchist Youth, Juventud Libertaria, publicly protested. But the \CNT\ press also was systematically harassed. Does history record another instance of cabinet ministers submitting to repression of their own press?

The \FAI\ daily, \emph{Nosotros} of Valencia, was suspended indefinitely on February 27 for an article attacking Caballero’s war policy. On March 26, the Basque Government suspended \emph{\CNT\ del Norte}, arrested the editorial staff and the \CNT\ Regional Committee---and gave the presses to the Basque Communist party. Various issues of \emph{\CNT} and \emph{Castilla Libre}, both of Madrid, were suppressed April 11--18. \emph{Nosotros} was again suspended on April 16.

Censorship and suspension were formal measures. At least equal\-ly efficacious were the ``informal'' measures whereby the \CNT-\FAI-\POUM\ newspaper packets ``failed'' to arrive at the front or arrived weeks late. Meanwhile enormous editions of the Stalinist and bourgeois press, untouched by the censor and always delivered, were distributed free to the \CNT, \UGT\ and \POUM\ militias. The government radio stations were always at the service of the Nelkens and Pasionarias. Almost all the so-called political commissioners at the front were Stalinists and bourgeois. Thus deceit supplemented naked force.

\section*{The Police}

In the first months after July 19, police duties were almost entirely in the hands of the workers’ patrols in Catalonia and the ``militias of the rearguard'' in Madrid and Valencia. But the opportunity permanently to dissolve the bourgeois police slipped by.
\nowidow

Under Caballero, the Civil Guard was re-christened the National Republican Guard. The remnants of this and the Assault Guards were gradually withdrawn from the front. Those who had gone over to Franco were more than replaced by new men.

The most extraordinary step in reviving the bourgeois police was the mushroom growth of the hitherto small customs force, the Carabineros, under Finance Minister Negrin, into a heavily armed praetorian guard of 40,000.\endnote{``A reliable police force is being built up quietly but surely. The Valencia government discovered an ideal instrument for this purpose in the Carabineros. These were formerly customs officers and guards, and always had a good reputation for loyalty. It is reported on good authority that 40,000 have been recruited for this force, and that 20,000 have already been armed and equipped\dots\ The anarchists have already noticed and complained about the increased strength of this force at a time when we all know there’s little enough traffic coming over the frontiers, land or sea. They realize that it will be used against them.'' (James Minifie, \emph{New York Herald Tribune}, April 28, 1937.)}

On February 28, the Carabineros were forbidden to belong to a political party or a trade union or to attend their mass meetings. The same decree was extended to the Civil and Assault Guards thereafter. That meant quarantining the police against the working class. The hopelessly disoriented anarchist ministers voted for this measure on the ground that it would stop Stalinist proselytizing!

By April the militias were finally pushed out of all police duties in Madrid and Valencia.

In the proletarian stronghold of Catalonia, this process ran into the determined opposition of the \CNT\ masses. There was also an ``unfortunate incident'' which slowed up the bourgeois scheme. The first Chief of Police for all Catalonia, appointed by the cabinet---Andre Reberter---proved to be one of the ringleaders in a plot to assassinate the \CNT\ leaders, establish an independent Catalonia and make a separate peace with Franco.\endnote{CNT’s intelligence service had discovered the plot and \emph{Solidaridad Obrera} published the facts on November 27 and 28. At first it was scoffed at by the Stalinists and the Esquerra; but they were forced to order an investigation. As a result, it was found that the chief forces in the plot were those of the separatist Estat Catala, a khaki-shirt organization, a split-off from the Esquerra, and its secretary-general and over a hundred leading members were arrested. Chief of Police Reberter, Estat Catala member, was executed upon conviction. Casanovas, President of the Catalan Parliament, ``at first toyed with the plot, then rejected it,'' said an official explanation. Casanovas was permitted to go to France---and to return to political life in Barcelona after the May Days!} His exposure strengthened the position of the workers’ patrols, largely manned by the \CNT.

But then the patrols were attacked from within. The \PSUC\ ordered its members to withdraw (most of them did not, and were expelled from the \PSUC). The Esquerra also withdrew from the patrols. Thereafter all the usual Stalinist methods of defamation were directed at the patrols, loudest when the patrols arrested \PSUC\ and \textsc{gepci} businessmen for hoarding and profiteering on food.

On March 1, a Generalidad decree unified all police into a single state-controlled corps, its members prohibited from association with trade unions and parties and to be chosen by seniority. This meant abolition of the workers’ patrols and the barring of their members from the unified police. Apparently the \CNT\ ministers voted for the decree. But the resultant outcry of the Catalan masses led the \CNT\ to join the \POUM\ in declaring they would refuse to submit to it. On March 15, nevertheless, the Minister of Public Order, Jaime Ayguade, attempted unsuccessfully to suppress forcibly workers’ patrols in the outlying districts of Barcelona. This question was one of those leading to the dissolution of the Catalan cabinet on March 27. But there was no change when the new cabinet, again with \CNT\ ministers, convened on April 16. Ayguade continued his attempts to disarm the patrols, while the \CNT\ ministers sat in the cabinet, their papers contenting themselves with warning the workers against provocation.

\section*{Liquidation of the Militias}

There could, of course, be no hope of reviving a stable bourgeois regime so long as the organization and administrative responsibility for the armed forces was in the hands of the unions and workers’ parties, which presented payrolls, requisitions, etc., to the Madrid and Catalan governments, and stood between the militias and the governments.

The Stalinists early sought to set an ``example'' by handing their militias over to government control, helping to institute the salute, supremacy of officers behind the lines, etc. ``No discussion, no politics in the army,'' cried the Stalinist press, meaning of course no working-class discussion or politics.

The example was wasted on the \CNT\ masses. At least a third of the armed forces were \CNT\ members, suspicious of the officers sent by the government, relegating them to the status of ‘technicians’ and barring them from interfering in the social and political life of the militias. The \POUM\ had 10,000 militiamen who acted likewise. The \POUM\ reprinted for distribution in the militias the original \emph{Red Army Manual} of Trotsky, providing for a democratic internal regime and political life in the army. The Stalinist campaign for wiping out the internal democratic life of the militias, under the slogan of ``unified command,'' was countered by the simple and unanswerable question: why does a unified command necessitate re-establishing the old barracks regime and the supremacy of a bourgeois officer caste?

But the government eventually had its way. The militarization and mobilization decrees passed in September and October with \CNT\ and \POUM\ consent provided conscription of regular regiments ruled by the old military code. Systematic selection of candidates for the officers’ schools gave preponderance to the bourgeoisie and Stalinists, and these manned the new regiments.

When the first drafts of the new army were ready and sent to the front, the government pitted them against the militias, demanding reorganization of the militias in a similar mould. By March the government had largely succeeded on the Stalinist-controlled Madrid front. On the Aragon and Levante fronts, manned chiefly by \CNT-\FAI\ and \POUM\ militias, the government prepared the liquidation of the militias by a ruthless, systematic policy of withholding arms. Only after reorganization, the militias were informed, would they be given adequate arms for an offensive on these fronts. Yet the sheer mass of the \CNT\ militias prevented the government from attaining its objectives until after the May days, when Azaña’s ex-Minister of War, General Pozas, took over the Aragon front.

In the last analysis, however, the government’s final success came not from its own efforts so much as from the politically false character of the \CNT-\POUM\ demand for a ``unified command under the control of the workers’ organizations.''

The Stalinists and their ``non-party'' publicists of the stripe of Louis Fischer and Ralph Bates have deliberately perverted the facts of the controversy between the \POUM-\CNT\ and the government on army reorganization. The Stalinists make it appear that the  \POUM-\CNT\ wanted to retain the loosely organized militias as against an efficiently centralized army. This is a lie made out of whole cloth, as may be demonstrated by a thousand articles in the \POUM-\CNT\ press of the time calling for a disciplined army under unified command. The real issue was: who will control the army, bourgeoisie or working class? Nor did the \POUM-\CNT\ alone raise this question. In opposing Giral’s original scheme for a special army, the \UGT\ organ, \emph{Claridad}, had declared:

\begin{quotation}
  We must take care that the masses and the leadership of the armed forces, which should be above all the people in arms, should not escape from our hands. (August 20, 1936.)
\end{quotation}

That was the real issue. The bourgeoisie won out because the \UGT, \POUM, and \CNT-\FAI\ made the hopeless error of seeking a proletarian-controlled army within a bourgeois state. So much were they for centralization and unified command that they voted for the governmental decrees which in the ensuing months served to wipe out all workers’ control of the army. \UGT, \POUM, and \CNT's consent to these decrees was not the least of their crimes against the working class.

Their slogan for a unified command under control of the workers’ organizations was false because it provided no method of achieving that goal. The demand which should have been raised, from the first day of the war, was for amalgamation of all the militias and the few existent regiments into a single force, with democratic election of soldiers’ committees in each unit, centralized in a national election of soldiers’ delegates to a national council. As new regiments were conscripted, their soldiers’ committees would have entered the local and national councils. Thus, in drawing the armed masses into daily political life, bourgeois control of the armed forces could have been effectively prevented.

The \POUM\ had a wonderful opportunity to demonstrate the efficacy of this method. On the Aragon front it had for eight months direct organizational control over some 9,000 militiamen. It had an unparalleled opportunity to educate them politically, to elect soldiers’ committees among them as an example to the rest of the militias, then to demand amalgamation in which its trained forces would have been a powerful leaven. Nothing was done. The \POUM\ press carried stories of representatives of the Aragon front meeting in congress. These meetings were nothing but gatherings of appointees of the national office. In fact, the \POUM\ \emph{forbade} election of soldiers’ committees. Why? Among other reasons was the fact that opposition to the \POUM’s opportunist politics was rife in the ranks and the bureaucracy feared that creation of the committees would provide the necessary arena in which the Left Opposition might conquer.

The simple, concrete slogan of elected soldiers’ committees was the only road for securing proletarian control of the army. This slogan, moreover, could only be a transitional step. For a worker-controlled army could not exist indefinitely side by side with the bourgeois state. If the bourgeois state continued to exist, it would inevitably destroy workers’ control of the army.

The \POUM-\CNT-\UGT\ proponents of workers’ control raised neither the concrete slogan nor had they any programme for displacing the bourgeois state. Their basic orientation, therefore, doomed to impotence their opposition to bourgeois domination of the army.

\section*{Disarming of the Workers in the Rear}

In the revolutionary days following July 19, the Madrid and Catalan governments had perforce sanctioned the arming of the workers who had already armed themselves. Workers’ organizations were empowered to issue arms permits to their members. For the workers it was not only a question of guarding against the counter-revolutionary attempts of the government, but the daily necessity of protecting the peasants’ committees against reactionaries, guarding the factories, railroads, bridges, etc., against fascist bands, protecting the coast from raids, ferreting out hidden fascist nests.

In October came the first disarming decree providing for delivery of all rifles and machine guns to the government. In practice, it was interpreted to allow the workers’ organizations to continue issuing permits for long arms to industrial guards and peasant committees. But it set the fatal precedent.

On February 15, the central government ordered the collection of \emph{all} long arms as well as all small arms not held by permission. On March 12, the cabinet ordered the workers’ organizations to collect large and small arms from their members and to surrender them within forty-eight hours. This order was applied directly to Catalonia on April 17. National Republican Guards began officially to disarm workers on sight in the streets of Barcelona. Three hundred workers---\CNT\ members holding organization permits---were thus disarmed by police during the last week in April.

The pretext that the arms were needed at the front was a bare-faced lie, as any worker could see with his own eyes. For while the workers were being deprived of rifles and revolvers, some of them in the possession of the \CNT\ since the days of the monarchy, the cities were being filled with the rebuilt police forces, armed to the teeth with new Russian rifles, machine guns, artillery, and armoured cars.

\section*{Extra-Legal Methods of Repression: The Spanish \textsc{gpu}}

On December 17, 1936, \emph{Pravda}, Stalin’s personal organ, declared:

\begin{quotation}
  As for Catalonia, the purging of the Trotskyists and the Anarcho-Syndicalists has begun; it will be conducted with the same energy with which it was conducted in the \USSR.
\end{quotation}
The ``legal methods,'' however, moved too slowly. They were supplemented by organized terrorist bands, equipped with private prisons and torture chambers, termed ``preventoriums.'' The worthies recruited for this work beggar description: ex-members of the Fascist \textsc{ceda}, Cuban gangsters, brothel-racketeers, passport forgers, sadists.\endnote{\emph{Cultura Proletaria}, NY anti-fascist paper, published a report from Cuba: ``The CT \dots\ sent 27 ex-officers of the old army who have nothing in common with workers and are mercenaries formerly in Machado’s service \dots\ On its last trip the Mexique took an expedition of these fake militia (with a few exceptions), among them went the three Alvarez brothers, former Machado gunmen active in breaking the Bahia strike. On the 29th of this month \dots\ `Sargento del Toro' sails, too, as a communist militiaman. He is a full-fledged assassin of the Machado days, bodyguard of the President of the Senate in that period. He was one of those who helped massacre workers in a demonstration here on August 27.''

The former Valencia Secretary of the \textsc{ceda} is now in the CP. Even Louis Fischer admits that ``bourgeois generals and politicians, and many peasants who approve the CP’s policy of protecting small property holders have joined \dots\ essentially their new political affiliation reflects a despair of the old social system as well as a hope to salvage one or two of its remnants.''

An apt description, as Anita Brenner points out, of the social group which swelled Hitler’s ranks. For further details on the Spanish \textsc{gpu} and the repressions, see Anita Brenner’s excellent article and ``Dossier of Counterrevolution,'' \emph{Modern Monthly}, September 1937.} Spawned by the petty-bourgeois composition of the Communist party, nurtured by its counterrevolutionary programme, these organized bands of the Spanish \textsc{gpu} exhibited toward the workers the ferocity of Hitler’s bloodhounds, for like them, they were trained to exterminate revolution.

Rodriguez, \CNT\ member and Special Commissioner of Prisons, in April formally charged José Cazorla, Stalinist Central Committee member and Chief of Police under the Madrid Junta, and Santiago Carillo, another Central Committee member, of illegally seizing workers arrested by Cazorla but acquitted by the popular tribunals, and ``taking said acquitted parties to secret jails or sending them into communist militia battalions in advanced positions to be used as `fortifications.''' The \CNT\ in vain demanded a formal investigation of its charges. Only when it was established that Cazorla’s gang, as a side-line, was working with racketeers who were releasing important fascists from prison without official sanction, was Cazorla removed. He was simply replaced by Carrillo, another Stalinist, and the extralegal \textsc{gpu} and its private prisons continued as before.

\begin{quotation}
  It is becoming clear that the Chekist\endnote{The anarchists refer to the \GPU. In general, they blind themselves to the vast gulf between the ``Cheka,'' which ruthlessly suppressed the White Guards and their associates in the early period of the Russian revolution, and the Stalinist \GPU, which ruthlessly suppresses and assassinates proletarian revolutionists.} organizations recently discovered in Madrid \dots\ are directly linked with similar centres operating under a unified leadership and on a preconceived plan of national scope,
\end{quotation}
wrote \emph{Solidaridad Obrera} on April 25, 1937. On April 8, the \CNT, armed with proofs, had finally forced the arrest of a gang of Stalinists in Murcia, and the removal of the civil governor for maintaining private prisons and torture chambers. On March 15, sixteen \CNT\ members had been murdered by Stalinists in Villanueva de Alcardete in Toledo Province. The \CNT\ demand for punishment was countered by \emph{Mundo Obrero}’s defence of the murderers as revolutionary anti-fascists. The subsequent judicial investigation established that an all-Stalinist gang, including the Communist party’s mayors of Villanueva and Villamayor, operating as a ``Defence Committee,'' had murdered political enemies, looted, levied tribute, and forcibly raped the defenceless women of the area. Five of the Stalinists were condemned to death, eight others sentenced to prison.

The organized gangsterism of the Spanish \GPU\ has been established in the Spanish Government’s own courts of law. We limit ourselves here to juridically established instances. But the \CNT\ press is filled with hundreds of instances in which the ``legal'' counter-revolution was supplemented by the \GPU\ in Spain.