\chapter{The Dismissal of Largo Caballero}

\lettrineT{he} defeat of the Catalonian proletariat marked a new stage in the advance of the counterrevolution. Hitherto, the reaction had developed under cover of collaboration with the \CNT\ and \UGT\ leaders, and even from September to December in the Generalidad with the \POUM\ leaders. Thus, the gap between the openly bourgeois program of the bourgeois--Stalinist bloc and the revolutionary aspirations of the masses had been obscured by the centrists.\kn\kn\footnote{\kp\kp This is the Marxian term employed to describe the variety of political formations which are not revolutionary but which also do not proclaim the class-collaboration doctrines of classical reformism.} Now the moment arrived for the bourgeois--Stalinist bloc to dispense with the centrists.

The process is a familiar one in recent history. When blows to the left have sufficiently strengthened the right, the latter are then enabled to turn against the centrists whose services, heretofore, had been indispensable in crushing the left. The result of the suppression of the revolutionary workers is a government far to the right of the regime that suppressed them. Such was the result of the bloody suppression of the Spartacists in 1919 by Noske and Scheidemann. Such was the aftermath of the ``stabilization'' of Austria by Renner and Bauer. It was now the turn of the Spanish centrists to pay the price for having abetted the crushing of the Catalonian proletariat.

The first item of the bill presented by the Stalinists to the Valencia cabinet was the complete suppression of the \POUM. Why the \POUM? Like all renegades, the Stalinists understand the dynamics of revolutionary development better than their allies who have always been reformists. In spite of its vacillating policies, the \POUM\ had in its ranks many revolutionary fighters for the interests of the proletariat. Even the \POUM\ leaders, unready for revolution, would be driven to resist the naked counterrevolution. Stalin has understood that even the capitulators, the Zinovievs and Kamenevs, will be a danger on the day the masses rebel. Stalin’s formula is: wipe out every possible focus, every capable figure, around whom the masses can rally. That bloody formula, already carried out in the August and January trials in Moscow, was now applied to Spain and the \POUM\kn.

\medskip

The left socialists recoiled. One of their organs, \emph{Adelante} (of Valencia) said editorially on May 11:

\begin{quotation}
  If the Caballero government were to apply the measures of repression which the Spanish section of the Comintern is trying to incite, it would approximate a Gil-Robles or Lerroux government; it would destroy working-class unity and expose us to the danger of losing the war and wrecking the revolution\dots.
  A government composed in its majority of people from the labor movement cannot use methods reserved for reactionary and fascist-like governments.
\end{quotation}

The cabinet convened on May 15, and Uribe, the Stalinist Minister of Agriculture, bluntly put the question to Caballero: was he prepared to agree to the dissolution of the \POUM\kn, confiscation of its broadcasting stations, presses, buildings, goods, etc., and imprisonment of the Central Committee and local committees which had supported the Barcelona rising?

Federica Montseny awoke sufficiently to the occasion to present a dossier to prove that a plan had been prepared, both in Spain and abroad, to strangle the war and revolution. She accused Lluhi y Vallesca and Gassol (Esquerra), and Comorera (\PSUC), together with a Basque representative, of having participated in a meeting in Brussels at which it was agreed to annihilate the revolutionary organizations (\POUM\ and \CNT-\FAI) in order to prepare for ending the civil war by the intervention of ``friendly powers'' (France, England).

Caballero declared he could not preside over repression against other workers’ organizations, and that it was necessary to smash the false theory that there had been a movement against the government in Catalonia, much less a counterrevolutionary movement.%
\endnote{On May 4, the Valencia \emph{Adelante} (obviously speaking for Caballero) solved the problem of which side of the barricades to support by denying the real meaning of the struggle:

\begin{quotation}
  We understand that this is not a movement against the legitimate power\dots. And even if it were a revolt against the legitimate authority, and we do not admit that such was the case, instead of being merely an inopportune and poorly prepared collision between the organizations with different orientations and political and trade union interests opposed to each other within the general antifascist front in which the proletarian groups of Catalonia move, the responsibility for the consequences would have to be charged, naturally, to those who provoked the collisions.
\end{quotation}}

As the Stalinists continued to press their demands, Montseny sent for a package containing hundreds of scarves adorned with the shield of the monarchy. Thousands of these had been found in the hands of the \PSUC\ provocateurs and Estat Catala members, who were to have planted them in \POUM\ and \CNT\ buildings. The two Stalinist ministers rose and rushed out of the meeting, and the ministerial crisis had begun.

Caballero looked at the others. He wanted them to state their positions. The bourgeois and Prieto ministers solidarized themselves with the Stalinists and went out. Such was the last meeting of the Caballero cabinet.

\dinkus

Outlawry of the \POUM\ was the first demand of the counterrevolution, but the Stalinists followed it up with other basic demands which Caballero and the left socialists would not accept responsibility for.
Friction between the Stalinists and left socialists had, indeed, been developing for some months.

A stealthy campaign against Caballero himself had been waged in the Stalinist press since March, when the flow of adulatory telegrams to the ``leader of the Spanish people'' from ``the workers of Magnitogorsk'' had been turned off like a faucet. The Stalinist campaign had been the subject of comment in the organs of the \CNT\ and \POUM\kn, and of resentful polemics in the left socialist press. The befuddled anarchists interpreted the Stalinist campaign in terms of the original sin of politics: this was the way political parties acted toward each other.

The \POUM\ sought to make easy capital among socialist workers by berating the Stalinists for attempting to absorb the socialists. Juan Andrade, the \POUM\ commentator, saw more clearly, recognizing that Caballero was resisting the Anglo--French directives in their fullest implications. But the main \POUM\ line of shouting ``ab\-sorp\-tion'' lost it the opportunity to make use of the real conflicts between Caballero and the bourgeois--Stalinist bloc. For there were real conflicts. Not, of course, as basic as the conflict between reform and revolution; but important enough so that a bold revolutionary policy could have driven a wedge between the Stalinists and Caballero’s mass base, could have aroused the \UGT\ workers to the meaning of the road which Caballero had followed for eight months.

Stalinist inroads into Caballero’s ranks were a fact. It is a familiar enough phenomenon in the labor movement that when two organizations follow the same policy, the one with the stronger apparatus will proceed to absorb the other. By holding identical views with the Stalinists on the People’s Front, winning the war before making the revolution, conciliating foreign opinion, building a regular bourgeois army, etc., Caballero had ceased to differ from Stalinism in the eyes of the masses.

With the native Stalinist apparatus tremendously reinforced by Comintern functionaries and funds~-- the International Brigades came in with hundreds of such functionaries attached to them~-- the Stalinists were in a position to recruit at Caballero’s expense.

Particularly was this true in the youth. The socialist youth had been Caballero’s strongest support but its fusion with the Stalinist youth had left him the loser, although the latter had not had one-tenth the membership of the socialist youth. The usual Stalinist methods of corruption~-- trips to Moscow, adulatory relationships with the Russian and French \textsc{yci,} the offer of posts in the Central Committee of the party, etc.~-- had been successful.

Shortly after fusion, the socialist youth leadership had entered the Communist Party and the ``united'' youth organization came under rigidly Stalinist control. Dissenting branches were ``reorganized'' and left wingers expelled as Trotskyists. Caballero was scarcely in a position to protest at the outcome, having himself connived at the bureaucratic method of fusion, without a congress of the socialist youth having been held to pass on the decision.

Under the slogan of ``unifying the whole youth generation,'' the Stalinist leadership bulwarked itself by recruiting indiscriminately anyone who could be persuaded to accept a card. Santiago Carillo at a Central Committee Plenum of the Communist Party shamelessly advocated recruiting of ``fascist sympathizers'' among the youth. Leaning on backward elements, including many Catholics, the Stalinists were able for a time to muzzle the thousands of left wingers still in the youth organization.

Nevertheless, Caballero’s losses to the Stalinists had not led him to break with them. Absorption of his following only made him feel weaker and make further concessions.

Only when Caballero discovered that Stalinist inroads were less serious than he had supposed, and that he was more likely to lose his following to the left than to Stalinism, did he come into serious conflict with the Stalinists. The two biggest sections of the socialist youth, the Asturian and Valencian organizations, denounced the Stalinist top leadership and refused to accept seats in the ``united'' National Committee. In the delegates’ meeting of the Madrid \UGT\kn, the Caballero ticket carried all eight seats on the Municipal Council allotted to the \UGT\kn, against a Stalinist ticket. In the Asturian Congress of the \UGT\kn, the Caballero group held \textls[40]{87,000} votes against \textls[40]{12,000} for the Stalinists. These indices, shortly before the government crisis, showed that Caballero could have the dominant following in the \UGT\kn, and that he would have to pacify his following and not the Stalinists in the coming period.

There was one step, above all, which Caballero could not accept responsibility for: the final moves in smashing the workers’ control of the factories. Whatever else happened, the \UGT\ masses were firmly convinced; they would never give up the factories. The Madrid organ of the \UGT\ declared repeatedly: ``\kp The ending of the war must signify also the ending of capitalism.\kn\kn''
\nowidow

\begin{quotation}
  \noindent
  —\,That the exploiters of all life cease to be masters of all the means of production, it has sufficed that the people take up arms in the struggle for national independence. From the great financial establishments to the smallest shops, they are, in actual fact, in the hands and under the direction of the working class\dots. What vestiges remain of the old economic system? The revolution has eliminated all the privileges of the bourgeoisie and the aristocracy. (\emph{Claridad,} May 12, 1937).
\end{quotation}

\emph{Claridad}, indeed, continually studded its pages with quotations from Lenin.\endnote{With the Negr\'in cabinet, \emph{Claridad} passed into Stalinist control still continuing to call itself ``organ of the \UGT\kn,'' although twice repudiated by the National Executive Committee.} That these quotations were often enough damning commentary on Caballero’s political conceptions hardly requires documentation. Quotations appeared from \emph{The State and Revolution,} while Caballero strengthened and rebuilt the bourgeois state apparatus which would inevitably attempt to wrest the factories from the workers. But, unless he was prepared to lose the support of the masses of the \UGT\kn, Caballero could not himself participate in wresting the factories from the workers. Caballero was just enough of a labor politician to recognize that the state he had himself revived was alien to the workers and that the bourgeois--Stalinist slogan of ``state control of the factories'' meant smashing the power of the factory committees.

We may sum up the fundamental differences between Caballero~-- i.e., the bureaucracy of the \UGT~-- and the bourgeois--Stalinist bloc in this way: Caballero wanted a bourgeois democratic republic (with some form of workers’ control of production coexisting with private property), victorious over Franco. The bourgeois--Stalinist bloc was ready to accept whatever Anglo--French imperialism proposed, which, at the stage of the overthrow of Caballero, was a stabilized bourgeois regime based on participation in the regime of the cap\-i\-tal\-ist--landlord forces behind Franco, parliamentary in form, but actually Bonapartist since unacceptable to the masses.

Caballero’s perspective was not so fundamentally different from that of the bourgeois--Stalinist bloc, that they could not go along together for a considerable distance. They had gone along together for eight months. Was May 15 the correct moment for the rightists to break with Caballero? Should not the bourgeois--Stalinist bloc have bided its time for a few more months while the army and police were still further strengthened as bourgeois institutions? Should they not have carried the \CNT\ ministers deeper and deeper into the swamp? Were they not risking a regroupment of forces by forcing the two mass labour organizations out of the cabinet? Were not the Stalinists too nakedly revealing their reactionary role by becoming the only labour group, apart from the long-hated Prieto group, to participate in the government?

The Stalinists probably overestimated their ability to secure expressions of support for the new cabinet from enough \UGT\ unions to obscure the fact that the labor unions as a whole were opposed to the new government. Even in the bureaucratically controlled \UGT\ of Catalonia, the Stalinists proved unable to prevent many of the most important unions from declaring support for Caballero. Elsewhere the Stalinists got only a handful of unions to sanction the dismissal of Caballero.

If, however, the Stalinists miscalculated their ability to provide a labor ``front'' for Negr\'in, they were undoubtedly correct in other calculations. For them, the Barcelona events revealed that the \CNT\ ministers were no longer of use in keeping the \CNT\ masses in line; the fighting of May 3 to 8 had revealed the chasm between the leaders and the masses of the \CNT\kn. Further governmental participation of the \CNT\ would provide little brake to the resistance of the masses and, on the other hand, could only speed up a split between these leaders and the masses. For the next period, the Olivers and Montsenys were more useful as a ``loyal opposition'' outside the government. As oppositionists they could regain control over their following, yet their opposition would be of a kind that would not unduly embarrass the Negr\'in government.

As for the opposition from Caballero, its temper and quality had already been experienced: his ``revolutionary criticism'' of the People’s Front government of February to July 1936, and his even more radical declarations during the first war cabinet of July 19 to September 4, 1936. In those periods Caballero had channelized discontent~-- and then had entered the government himself. If unforeseeable obstacles arose to endanger the government, the bourgeois--Stalinist bloc could always return to the status of May 15, for the centrists were demanding nothing more than that: ``One cannot govern without the \UGT\ and \CNT'' was the slogan of Caballero and the \CNT\ leaders. Meanwhile, it was safe to predict that Caballero’s opposition would not take the form of revival of the network of workers’ committees and the coordinating of them into soviets~-- and only along that road did the bourgeois--Stalinist bloc have anything serious to fear.

If dropping the \UGT\ and \CNT\ involved no serious dangers, it offered immediate and far-reaching advantages for the bourgeois--Stalinist bloc. Their immediate requirements were:

\begin{enumerate}
  \item Complete control of the army. The mobilization and army reorganization decrees had been carried out by Caballero, as Minister of War, to a considerable extent. The regiments formed of drafted soldiers were built entirely on the old bourgeois model, largely officered by old army officers or the hand-picked graduates of the government-controlled training schools. Any attempts among the conscripts at election of officers’ and soldiers’ committees had been stamped out. But the workers’ militias which had carried the brunt of the struggle during the first six months were not yet all ``reorganized;'' their masses resisted fiercely any systematic replacement of their officers, most of whom came from their own ranks. Even on the Madrid front the \CNT\ and \UGT\ militias, despite partial reorganization, retained most of their former officers and continued to print their own political papers at the front. On the Catalan fronts, the anarchist militias refused to honor the decrees which the \CNT\ ministers had signed. Equally important, Caballero became alarmed enough, after the loss of M\'alaga, to arrest General Asensio and the Malaga commander, Villalba, for treason, and cleaned out of the staffs many bourgeois friends of Prieto and the Stalinists. Caballero’s caution thereafter in army reorganiation was a serious obstacle to the Prieto--Stalinist program. For a ruthless reorganization of the militias into bourgeois regiments, officered by bourgeois appointees in consonance with the old military code, and a purge of radical army leaders thrown up by the July days, it was necessary to wrest the army entirely from Caballero.
  
  \item The War Ministry offered the best vantage point from which to begin wresting control of the factories from the workers. In the name of the exigencies of the war, the ministry could step in and break the hold of the workers in the most strategic industries: railroad and other transportation, mining, metals, textiles, coal and oil. The Stalinists had already begun to prepare for this in April by a barrage against the war-supply factories. Unfortunately for the Stalinists, they had organized this campaign (a persistent weakness of campaigns carried out obediently under orders of Comintern representatives from Moscow) at a time when the atmosphere was not yet propitious for a pogrom. Their charges were refuted by joint statements of the \CNT\ and \UGT\ organizations in the Catalonian factories involved and, as we have seen, were disavowed even by Premier Tarradellas who, as Minister of Finance, disbursed to the factories the funds received from the Valencia treasury. It was clear, then, that this campaign could not be successfully consummated from outside but that the bourgeois--Stalinist bloc needed the Ministry of War to further their inroads into workers’ control of the factories.
  
  \item In Caballero’s cabinet the Ministry of the Interior, which controlled the two main police bodies (Assault and National Republican Guard) and the press, was presided over by \'Angel Galarza, a member of the Caballero group. The revolutionary workers had sufficient reason to denounce his policies. Above all, Caballero and Galarza had sanctioned the decree forbidding police to join political and trade union organizations; and quarantining the police against the labor movement could only mean, inevitably, pitting them against the labor movement.
  
  Nevertheless, the Caballero group recognized that repression of the \CNT\ would be a fatal blow to the Caballero base, the \UGT, and Caballero needed the \CNT\ as a counterweight to the bourgeois--Stalinist bloc. Galarza had sent five thousand police to Barcelona, but had refused to carry out the Prieto-Stalinist proposals for complete liquidation of the \POUM\ and reprisals against the \FAI-\CNT. Here again the Caballero group had built the instrument for hostilities against the workers, but drew back at carrying out its complete implications. Once Caballero and Galarza had induced the Generalidad, during the Barcelona fighting, to extend control of public order by the central government to Catalonia, the moment was ripe to oust Galarza, in order for the Stalinists to secure control of the police and press in Catalonia and elsewhere.
  
  \item The Prieto--Stalinist program for conciliation with the Catholic Church~-- halfway house to conciliation with Franco~-- was being resisted by Caballero. Backbone of the monarchy and of the \emph{bienio negro,} the two black years of Lerroux--Gil-Robles, the churches had been the fortresses of the fascist uprising. To be a member of a labor organization has always, in Spain, had to mean to be against the church, for the official catechism has declared it to be mortal sin to ``vote liberal.\kn\kn'' The masses had spontaneously forced the closing of all Catholic churches in July. One could scarcely propose a more unpopular measure than to permit the church organization to operate freely again and in the midst of civil war! Furthermore, it was actually dangerous to the antifascist movement; for with the Vatican on the side of the Franco regime, it would inevitably use the church organization to help Franco. Yet this was the proposal of the Basque Government and its allies, Prieto and the Stalinists. Caballero had done many things to curry favor with the Anglo--French imperialists; but to permit the church organization to operate freely in the midst of the civil war was too much for him.
\end{enumerate}

\dinkus

These causes of conflict between Caballero and the reactionary bloc arc clearly revealed in the demands expressed by the various parties on May 16, during the customary visits to President Azaña, to acquaint him with the position of each group on the ministerial crisis.\endnote{The statements of the parties are published in the press.}

Manuel Cordero, spokesman for the Prieto Socialists, piously declared his organization stood for a government including all factions~-- but ``I have insisted very particularly on the necessity of an absolute change in the policy of the Ministry of the Interior.\kn''

Pedro Corominas, for the Catalan Esquerra, declared: ``\kp Whatever be the solution that is adopted, it will be necessary to strengthen it and do away with difficulties of personal origin, by greater and more frequent contact with the Cortes of the Republic.\kn\kn'' In other words, the government policy should be dictated by the remnants of the Cortes elected in February 1936 under an electoral agreement which gave the overwhelming majority of the Cortes to the bourgeois parties!

\medskip

Manuel de~Irujo, for the Basque capitalists, spoke fairly bluntly:

\begin{quotation}
  I have advised His Excellency for a government of national concentration presided over by a socialist minister who has the confidence of the [bourgeois] republicans. Since Cab\-a\-llero \dots\ has lost the political confidence of the groups of the Popular Front, it would be advisable to form a government, in our opinion, of Negr\'in, Prieto, or Besteiro, with the cooperation of all the political and trade union organizations which would accept the proposed bases.
  
  As specific demands, I feel obliged to make two, at present. The first is the necessity of proceeding, with such guarantees and restrictions as the war and public order dictate, to the reestablishment of the constitutional regime of liberty of conscience and religion.
  
  The second demand refers to Catalonia. The Catalan republicans would have preferred earlier and effective intervention by the Government of the Republic in assuming control of public order in support of the Generalidad. What is more, in now carrying out these duties, I feel that it is an unescapable duty of the Government that it liquidate to the bottom the problem which disturbs Catalonian life, firmly doing away with the causes of the disorder and insurrection, be they circumstantial or endemic\dots.
\end{quotation}

It was to this Irujo that the Prieto--Stalinist bloc were soon to entrust \dots\ the Ministry of Justice.

Salvador Quemades, for the Left Republicans, Azaña’s own party, required that the next cabinet ``must have a decided policy in the matter of public order and of economic reconstruction, and that the commands of war, marine and the air force be placed in a single hand.\kn\kn'' Prieto was already Minister of Marine and Air. This meant adding to his posts that of control of the army (as was done).

\medskip

The Stalinists demanded:

\begin{enumerate}
	\item The President of the Council (Premier) occupy himself exclusively with affairs of the Presidency. The War Ministry to be separately conducted by another minister.
	
	\item Elimination of Galarza from the new cabinet because of ``his lenity in the problems of Public Order.\kn''
	
	\item The Ministers of War and of the Interior ``should be persons who enjoy the support of all the parties and organizations which form the Government.'' Which meant that these key posts, essential to the further schemes of the Basque--Prieto--Stalinist bloc, should pass to them.
\end{enumerate}

The \CNT\ declared it would support no government not headed by Caballero as Premier and Minister of War. The \UGT\ issued a similar declaration. President Azaña, knowing the cards were already stacked, delegated Caballero to form a new cabinet with all groups represented. Caballero, in true centrist fashion, proceeded to cut the ground from beneath himself. He had already weakened his chief ally, the \CNT\kn, by his conduct in the Barcelona events. He now offered to cut the representation of the \CNT\ from four ministries to two, Justice and Sanitation. To the Prieto group he offered two ministries, but these were to combine Finance and Agriculture, Industry and Commerce. Education and Labor were the two ministries for the Stalinists. The bourgeoisie, which in the previous ministry had enjoyed no posts except ministries without portfolios, were to have the Ministries of Public Works and Propaganda (Left Republicans), the Ministry of Communications and Merchant Marine (Union Republican), and ministries without portfolios for the Esquerra and Basque Nationalists. Caballero’s proposed government was thus decidedly to the right of its predecesssor. Caballero’s conciliationism to the right could only impress the masses as meaning that the intransigence of the Right denoted superior strength, and paved the way for the Right assuming all power with impunity.

The Stalinists rejected Caballero’s compromise, and refused to participate in his cabinet except on the terms they had laid down. The Prieto group promptly declared it would not participate if the Stalinists refrained. The bourgeois parties followed suit. Caballero could now either form a government of the \UGT--\CNT\ or surrender the government to the bourgeois--Stalinist bloc.

Caballero conducted himself during the ministerial crisis according to the traditional rules of bourgeois politics, which is to say, that he kept the masses completely in the dark about developments and made no attempt to rally the workers against the right. So, too, the \CNT\kn. Later it became known that the day the cabinet had collapsed, Caballero had assured the \CNT\ he was ready, if necessary, to have the \UGT\ and \CNT\ assume power. However, he begged off within a few hours, on the grounds of opposition within the \UGT\kn.

\medskip

``During the government crisis, the \UGT\ played a double game,'' said a \FAI\ manifesto afterward.

\begin{quotation}
  \noindent
  —\,The bourgeois and communist influences are so strong within this organization that its revolutionary sector, that is, the one which is inclined to work with us, was paralyzed. That meant a victory not only for the bourgeois--communist bloc but also for France, England, and Russia, who had obtained what they wanted.
\end{quotation}

In other words, the anarchists leaned on Caballero, he pointed to the opposition, and in the general paralysis of the masses induced by their leaders, the rightist government came to power.

Perhaps, indeed, in his numerous sessions with Azaña during the days of the crisis, Caballero had broached the subject of a \UGT--\CNT\ government~-- and been refused. For Azaña constitutionally had the power to reject cabinets which did not suit him. The 1931 constitution endows the president with truly Bonapartist powers. Azaña himself had experienced this as premier, when in 1933 his cabinet, though still wielding a majority in the Cortes, was dismissed by President Zamora to make way for the semi-fascist government of Lerroux. These Bonapartist powers had not been wiped out on July 19.

Azaña had quietly retired to a country retreat in Catalonia, had remained quiescent for most of the period of Caballero’s rule. When members of the Caballero group were reproached for not having done away with the presidency during these months, they had patronizingly explained that the constitution and the presidency no longer existed, it was purely formalism to say that they did and, on the other hand, it was very useful for securing aid from abroad to continue the pretense of constitutionalism~-- and now, here was a very lively President Azaña, condescendingly receiving the spokesmen for the various parties, receiving reports from Caballero on his progress in getting together a cabinet, while Azaña’s party, the Left Republicans, were in the bourgeois--Stalinist bloc\dots.

In any event, Caballero saved this bloc the unpleasantness of a public controversy over the presidential prerogatives. He informed Azaña that he had failed to form a cabinet and Azaña promptly designated Negr\'in to form a government of the bourgeoisie, the Prieto group, and the Stalinists.