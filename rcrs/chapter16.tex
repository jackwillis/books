\chapter{Military Struggle Under Negr\'in and Prieto}

\indexIPrieto\index{Ministries!Defense}\index{Bourgeois--Stalinist bloc}\indexJNegrin
\lettrineT{hat the} ``government of victory'' would inevitably continue the disastrous military policy of its predecessor was apparent the day it was constituted. Prieto would continue his inactive naval policy and his political discrimination in the assignment of aircraft to the fronts. He was now also head of the army, with all services in a single Ministry~of Defense, but the Supreme War Council, established in December, had already then been dominated by the bourgeois--Stalinist bloc through their majority of the ministries.\footnote{The Stalinist ``demand'' that the council function normally, raised on May~16, was simply a prop for the myth to make Caballero the scapegoat for the conduct of the war.}

The political course which had dictated the previous military stra\-te\-gy---hostility to lighting the flame of revolt in North Africa, support of the Basque bourgeoisie against the workers, persecution of Catalonia and Aragon---all this continued, intensified. In addition, the Negr\'in cabinet added new obstacles to prosecution of the war.

\index{National question}\index{Catalonia!Autonomy}\index{Political repression}
On the national question---relations with minority peoples---the Negr\'in regime moved not only to the right of Caballero, but also to the right of the republic of 1931--33. The bureaucratic centralization for which the monarchists and fascists stood had been an important factor in alienating from them the peoples of Catalonia, Euzkadi (Basques) and Galicia. Once the civil war began, the limited autonomy of the Catalans and Basques had broadened de facto. A declaration of autonomy for Galicia would have immeasurably facilitated the guerilla warfare there. It was not forthcoming because it would have provided a precedent for Catalonia to seize on. The Negr\'in regime proceeded, as we have seen, to wipe out Catalan autonomy. Where the Bolsheviks had gained strength for the prosecution of the civil war from the intensified loyalty of the autonomous minority nations, the Loyalist government quenched the fires of national aspirations.

\index{Inequality}
The pay of militiamen was reduced from ten pesetas a day to seven, while the ascending scale for officers was: 25~pesetas for second lieutenants, 39~for first lieutenants, 50~for captains, 100~for lieutenant colonels. Economic distinctions thus sharply reinforced military regulations. One need scarcely emphasize the deleterious effect on the soldiers’ morale of this and their increasing subordination to the~officers.

\index{Bourgeoisie}\index{Fifth column}\index{Assault Guards}\index{Civil Guards}\index{Berneri, Camillo}\index{Political repression}
The whole northern front was soon to be betrayed by the Basque bourgeoisie and officers, and by the ``Fifth Column'' of fascist sympathizers in the Assault and Civil Guards and among the civilian population. The struggle against the ``Fifth Column'' was indissolubly part of the military struggle. But, as Camillo Berneri had written even before the intensification of repression under Negr\'in,

\begin{quotation}
  It is self-evident that during the months when an attempt is being made to annihilate the [\POUM--\CNT] “uncontrollables,” the problem of eliminating the Fifth Column cannot be solved. The suppression of the Fifth Column is primarily to be achiev\-ed by an investigatory and repressive activity which can only be accomplished by experienced revolutionists. An internal policy of collaboration between the classes and of consideration toward the middle classes, leads inevitably to tolerance toward elements that are politically doubtful. The Fifth Column is made up not only of fascist elements but also of all the malcontents who hope for a moderate republic.
\end{quotation}

While the northern front was left to the Basque bourgeoisie, the Aragon front was subjected to a frightful purge. General Pozas initiated what was ostensibly a general offensive in June. After several days of artillery and aerial conflict, orders to advance were given to the 29th~Division (formerly the \POUM’s Lenin Division) and other formations. But on the day for the advance, neither artillery nor aviation was provided to protect it\dots\

\index{Aragon}\index{Basque Country!Biscay!Bilbao}\indexSPozas\index{Battle}\index{Battle!Aerial}\indexPOUM\index{Military withdrawal}
Pozas later claimed this was because the air forces were defending Bilbao---but the day of advance was three days \emph{after} Franco had taken~Bilbao. The \POUM\ soldiers fully realized that they were being exposed deliberately. But not to go into fire would have given the bourgeois--Stalinist bloc a case against the Aragon front. They went into the line of fire. One flank was ostensibly assigned to an International Brigade (Stalinist)---but shortly after the advance began, it received orders to withdraw to the rear. The lieutenant colonel in charge of a formation of Assault Guards on the other flank later congratulated the \POUM\ troops: 

\begin{quotation}
  At Sarinena, I was warned against you on the ground that you might shoot us in the back. Not only did it not happen but thanks to your bravery and your discipline, we have avoided a catastrophe. I am prepared to go to Sarinena to protest against those who sow the seeds of demoralization, to effect the triumph of their partisan political aims.
\end{quotation}

\index{Political repression}\index{Assassination}\indexPOUM\index{Trotskyism}
During this offensive, Cahué and Adriano Nathan, \POUM\ commanders, were killed in action. Police were at that moment coming for Cahué, to arrest him as a ``Trotskyist--fascist.''

\index{Workers' militias!Liquidation of militias}\indexSPozas\index{Aragon!Huesca}\index{Political prisoners}\indexPOUM\index{Military supplies!Rifles}\index{Battle}\index{Military withdrawal}
When the attack was over, the 29th was sent to the rear. That, customarily, had meant to give up rifles---there were still not enough for front-line and reserves simultaneously on this front! But the suspicious \POUM\ troops refused to yield their arms. They declared themselves ready to return to the front. A few days later two battalions of the division were ordered to march on Fiscal (on the Jaca front) to repulse a fascist attack. Not only did they crush the attack but they reconquered positions and material previously lost. Then they were retired to await new orders---but not sent back to their division. Why? To disarm them. Pozas ordered it. They were concentrated in the village of Rodeano and surrounded by a Stalinist brigade. They were relieved of all their valuables---watches, chains, even good underwear and new shoes. Their leaders were arrested, the rest permitted to go---on foot. Hiking home, many were arrested in the towns on the way. The only reason the same methods were not employed against the rest of the division was that the news leaked out quickly and Pozas feared the \CNT\ divisions would come to its defense. But a few weeks later the 29th was officially dissolved, the remaining men being distributed far and wide in small groups.\endnote{This account is that of the front correspondent of \emph{Avanti,} émigré (Paris) organ of the Italian Maximalist-Socialists, scarcely a \POUM\ or Trotskyist source.}

\index{Workers' militias!Liquidation of militias}\indexCNT
The Ascaso (\CNT) division was also cut to pieces. \emph{Acracia,} \CNT\ organ of Lerida, wrote:

\vspace{-0.33\baselineskip}

\index{Aragon!Huesca}\index{Ministries!Air force}\index{Battle}\index{Military supplies!Planes}
\begin{quotation}
  Now we know exactly why Huesca was not taken. The last operation at Santa Quiteria furnishes a good proof of it. Hues\-ca was surrounded on all sides and only the betrayal of the air forces (controlled by \PSUC) was responsible for the disaster with which this operation ended. Our militiamen were not backed up by the air forces and were thus left defenseless in face of an intensive machine-gunning by the fascist air forces. This is only one of the numerous operations which ended in the same manner on account of the same betrayal by the air forces.
\end{quotation}

\vspace{-0.33\baselineskip}

\indexPSUC\indexSPozas
Soon after there was a plenary session of the \PSUC\ Central Committee in Barcelona. Among those prominently participating were ``Comrades'' General Pozas, chief of the Aragon front, Virgilio Llanos, political commissioner of the front, and Lieutenant Colonel Gordon, chief of staff\dots

Acceptance of control by the central government had been dangled before the Aragon front troops as the end of all their worries. Instead, it was used to break them down still further. The front correspondent of the anarchist (Paris) \emph{Libertaire} wrote on July~29:

\index{Political repression!Boycott}\index{Friends of Durruti}\index{Aragon!Zaragoza!Bujaraloz}\index{Exhaustion}
\begin{quotation}
  Ever since the central government took over control, the financial boycott became accentuated. Most of the militiamen have not received their pay for a long time. In Bujaraloz, where the general staff of the Durruti column is located, both officers and soldiers have not seen a cent for the last three months. They cannot wash their clothes for lack of soap. In many a place visited after several months of absence, I found comrades whom I knew well: now they looked pale, thin and visibly weakened. The physical state of the troops is such that they cannot keep up any prolonged exercises; they cannot march for more than fifteen kilometers a day. In the region of Farlete the troops live by hunting, without which they would starve to death.
\end{quotation}

\vspace{-0.33\baselineskip}

\indexLibertad\index{Loss of territory}\index{Casualties}
Systematic persecution of the chief forces of the Aragon front scarcely laid the basis for military victories, although at Belchite and Quinto the 25th division (\CNT) gave a good account of itself. But the alleged success of the July offensive on the Aragon front was so much newspaper talk. ``Results?'' wrote the illegal anarchist organ, \emph{Libertad} (August~1): ``Two villages lost in the Pirineo sector and three thousand men lost. This is what they call a success. Disastrous, calamitous, shameful success!''

After the fall of Santander (August 26), the persecution of the \CNT\ troops abated somewhat. But now came a terrible lesson in the consequences of creating counter-revolutionary forces of repression, such as the Stalinist-controlled Karl Marx division. In the midst of an offensive in the Zuera sector,

\vspace{-0.33\baselineskip}

\indexCNT\indexPSUC\index{Court martial@Court-martial}\index{Desertion}\index{Karl Marx Division}\index{Friends of Durruti}
\begin{quotation}
  Fifty officers of that division and six hundred soldiers pass\-ed over to the fascists. As a result of these desertions a battalion was destroyed. Despite the mettle of the \CNT\ forces, the operation could not be terminated well. The enemy had the necessary time to recover and it was impossible to continue the attack. After a summary court-martial that was immediately convened, thirty officers of the Karl Marx Division have been shot. In addition, the political commissar of the division, Trueba, a \PSUC\ member, has been dismissed. (\emph{Amigo del Pueblo,} illegal organ of Friends of Durruti, September~21).
\end{quotation}

\vspace{-0.33\baselineskip}

\index{Censorship}
Needless to say, the \CNT\ press was forbidden to publish the facts.

\section{The Northern Front}

\begin{sloppypar}
  As a government pledged to class collaborations even more completely than Caballero’s, the Negr\'in government did nothing to counter the more and more brazen sabotage of the Basque bourgeoisie. This front was almost inactive throughout the period from November 1936 to May 1937 when the fascists moved to wipe it out altogether. Nor had those six months been utilized in economic-military, preparations. One must emphasize again that the Euzkadi (the Basque country) was second only to Catalonia as an industrial region and superior to it in being a region of heavy industry, with iron and steel plants in the midst of iron and coal mines. \emph{Nothing} was done to develop out of this a great war industry. For this crime, the Stalinists bore equal responsibility, for two party representatives were ministers in the autonomous government. The coup against the \CNT\ in March, when its regional committee had been imprisoned and its press confiscated, was now followed by systematic repression of the workers, with public meetings prohibited. Thus, the one force which might have prevented betrayal was crushed by the Stalinist--bourgeois bloc.
\end{sloppypar}

In the Caballero cabinet, as we have said, there were constant fears about the loyalty of the Basques. Irujo’s constant threats of giving up altogether merely reflected the fact that the bourgeoisie had no serious stake in the struggle against fascism and would not fight under conditions destructive of their property. Consequently, when Franco began to move in the north, Caballero planned a large-scale offensive on the southern Madrid front to draw the fire of the fascist forces. According to his friends, 75,000 fully equipped troops were to have gone into action but two or three days before the offensive was to begin, his resignation was forced. Negr\'in’s first act was to withdraw these troops. Be that as it may, the fact was that no offensive was launched to relieve Bilbao, either on the Madrid or Aragon front, until mid-June—too late.

But the decisive factor in the loss of Bilbao was open treachery. ``Not even the insurgent heavy guns,\kn'' wrote the \emph{New York Times} correspondent,

\begin{quotation}
  \noindent
  \dots\ could have destroyed some of these underground fortifications with their three armored concrete tiers and blockhouses spaced about three miles apart all the way to the Biscay coast. The Insurgents themselves say that the “iron ring” of fortifications would never have been taken had not the Basques been outmaneuvered.
\end{quotation}

``Outmaneuvered,'' however, was a fascist euphemism for betrayal. After the city fell, this fact was admitted by the Basque delegation in Paris which put the blame on an engineer in charge of building the fortifications, who fled to Franco with the plans. Analysis of the delegation’s story revealed that the engineer in question had fled \emph{months before.} Why was not the intervening period utilized to redesign the fortifications? But the alibi was a subterfuge. For, as any tyro in military science knew, mere possession of the plans could not have solved the problem for the fascists of breaking through the fortification. \emph{They were let through the iron ring.}

Suppose we were to accept the Basque alibi. Why, then, was Bilbao not defended in a siege such as Madrid—less advantageously situated—had withstood? It is an elementary axiom of military science that no large city can be captured until its massive buildings—veritable fortifications—have been razed to the point they offer no further protection to beleaguered troops. The process of razing buildings by shelling and bombing required enormous equipment such as the fascists did not have less than an eighth of Madrid had been so razed after a year’s shelling and bombing.

But the bourgeoisie did not wait for the shelling of Bilbao at all! On June~19, they surrendered the city, as they had San Sebasti\'an the previous September. The uniform Basque policy of giving up cities intact has no parallel in any modern war, not to speak of civil war.

The pro-Loyalist correspondent of the \emph{New York Times} (June~21, 1937) wrote:

\begin{quotation}
  Details learned today of the last hours of Basque rule at Bilbao show that some 1,200 militiamen, who had before the civil war been soldiers of the regular army, decided in the early hours of the morning, after the bridges had been blown up, that chaos had gone far enough, and took control of the city in the capacity of police. The Asturian and Santander militiamen were driven out of the city.
  
  Assisted by some regular police and Civil Guards, this battalion accepted the surrender of their fellow-militiamen in the city and took their arms away, and afterwards hoisted a white flag on the telephone building. During the night they went around houses assuring the people that there was no cause for panic, placed guards on public buildings, and in the evening formed a cordon in the main street which prevented the excited crowds from pressing too closely on the National troops when they marched into the city.
\end{quotation}

Leisola, Minister of Justice in the Basque government, remained behind to superintend the betrayal. With the exception of seventeen (of whom we shall soon hear again), all fascist hostages were released and sent toward the fascist lines as a goodwill offering before the troops had reached the city. Simply put: the regular army of the Basques, directed by the bourgeois leaders, joined hands with the ``republican police'' in attacking the Asturians and militia from the rear, disarmed as many as they could, and dismantled the houses and street barricades which the workers had prepared for street fighting. Shortly after the occupation, the same police donned Carlist berets and became Franco’s regular police.

\CNT\ and \UGT\ press attempts to sound the alarm after the fall of Bilbao were cut to ribbons by the censorship. The Basque general staff was permitted to remain in command of the retreating troops. When, within a few weeks, the fascists began a second offensive, the industrial town of Reinosa, key to Santander’s defenses, collapsed, and once again the Basques made no attempt to defend the city itself.

Two days before the fall of Santander, the Basque general staff and remaining members of the government fled to France on a British warship. Under what terms, was revealed by the \emph{New York Times} dispatch of August~25:

\begin{quotation}
  At the time of the fall of Bilbao the Basques freed all their hostages except seventeen. Now these are considered to be in the gravest peril as the Basques admit that it is no longer possible to protect them from extremist elements in Santander.
  
  When the British Embassy agreed to take off the hostages, it arranged also to evacuate the Basques who have been guarding them as well as any remaining members of the Basque government\dots.
  
  It is hoped that the whole maneuver will be carried out before the more violent elements in Santander are aware of what is happening.
\end{quotation}

The next day (August~25), the British warship, \emph{Keith,} with Basque and fascist representatives aboard, arrived at Santander and ``res\-cued'' the Basque officials and the seventeen fascists.

President Aguirre was not in Santander. He banqueted the way across Spain, saying nothing, then joined his colleagues at Bayonne, France, where they issued the following statement:

\begin{quotation}
  The delegation of the Basque government, in refuge in Bayonne, take the responsibility of subscribing to the following: Franco’s offensive against Reinosa ended in terrible consequences. In a terrain composed of great mountains and deep gorges, Franco’s troops advanced with incomprehensible velocity. The military technicians were amazed at the rapidity of this advance, not only of the infantry, but of the heavy and mountain artillery, as well as the cumbersome services belonging to the different regiments and arms.
  
  This was an impossible or very difficult achievement and proves that the accidents of terrain were not utilized for resisting Franco’s army.
  
  In the face of this advance the troops of the Santander army offered no resistance to the enemy. Not only did they not come into contact with the enemy, but they did not lend themselves to retreating in a way that could be organized for defence.
  
  The Santander army’s organization was undone from the moment the offensive began. Neither communications, nor the sanitary services, nor means for avoiding surprise attacks, functioned. No line of resistance could be established, for the battalions which did not surrender at the first encounter were on the run through the countryside in the most complete disorder.
  
  Neither the general staff of Santander nor that of the Army of the North controlled the offensive at any moment. Once past Reinosa they could find neither the positions nor the situation of their troops, nor any unity on which they could count.
  
  Reinosa was surrendered to the enemy with no time for evacuating the population. The artillery factory fell into the hands of the rebels, with its shops for naval construction almost intact, and all the material in construction, including 38 batteries of artillery.
  
  The only resistance that the enemy encountered in its advance is that which the Basque battalions offered, rushing to the front. The incomprehensible conduct [of the others] ended by making the corps of the Basque army realize that it had been the victim of treachery, and that the advance of the Franco troops was being facilitated in such a way that the whole Basque army should fall into his power.
  
  The Basques, having resisted nearly ninety days against a brutal offensive [against Bilbao] incomparably more terrible than that of Reinosa, without having the means at its disposal that the Santander army had, cannot explain in any reasonable way the fact that in such a manner a terrain of eighty kilometers was lost in eight days. It is necessary to add to these data that the offensive against Euzkadi was a surprise while that of Reinosa had been announced and was anticipated.
  
  When the real situation was confirmed to it, the high command of the Basque army preoccupied itself in saving its troops and in preventing its effectives from falling into the hands of the enemy. To this mission it has consecrated all its efforts with the aid of the Basque government, which in this grave and difficult moment continued to give proof of its capacity and serenity.%
  \endnote{From the original Spanish text (\emph{\CNT\ Boletin,} Valencia, September~11).}
\end{quotation}

Somebody committed treachery, but not us, was the sum and substance of this amazing document, apart from its slander of the Asturian and Santander militiamen, fifteen thousand of whom had been executed by machine guns after being surrounded in San\-tander.

A Paris press dispatch of August~26 named some of the traitors, reporting that the commandant of the Assault Guards, Pedro Vega, the commandant of the Basque troops, Angel Botella, and Captain Luis Terez, presented themselves to the nearest outpost of the fascist troops and offered the surrender of Santander, but warned that a battalion of the \FAI\ militia had decided to fight to death.

\begin{sloppypar}
  Who, knowing anything of the \CNT\ and Asturian militiamen, would imagine that they had not stood at their posts ready to fight to the death? A thousand instances of their last-ditch heroism can be told. Why should they surrender or not fight, above all, the Asturian militiamen, who had learned in October~1934 that agreements ruling out reprisals were not kept by the reactionaries? On the other hand the Basques could not name a battle in which they stood up to the last. The Aguirre document’s alibi was threadbare. There was no striking contrast between what they did at Bilbao and the events at Santander. On the contrary, they merely followed the same pattern.
\end{sloppypar}

We repeat: the bourgeoisie had no serious stake in the struggle against fascism. Surrender of their property intact to Franco, with a prospect of eventual reconciliation, was infinitely preferable to them than destruction of their property in a death struggle. That they had not gone over to Franco to begin with was primarily due to their British connections. But during the offensive against Bilbao, that problem was ``solved'': the British had come to an understanding with Franco concerning the Basque provinces. As revealed by the authoritative Frederick Birchall in the \emph{New York Times,} British banks had extended to Franco, via Dutch connections, vast credits which were to be secured by products from the Bilbao region. Then came the gap in the \emph{iron ring.} But even without a final settlement with Britain, the fascists would have received Bilbao and Santander untouched, as San Sebasti\'an had been delivered to them the previous September.

\begin{sloppypar}
  That others were also treacherous, we are ready to concede to Aguirre. Once again, before the fascist troops entered Santander, yesterday’s ``loyal'' Assault and Civil Guards were patrolling the streets, disarming Asturian militiamen and preventing street fighting. These police were under the Ministry of the Interior (a Prieto man), and directly under a Stalinist director-general of police, who had dissolved the councils of anti-fascist guards for cleansing the police of dubious elements.

  What of that Supreme War Council, the ``real functioning'' of which had been one of the Stalinist demands which was not satisfied by Caballero, which could only be satisfied by Negr\'in?
\end{sloppypar}

What of the two Stalinist ministers in the Basque government, who had fled from Bilbao -- we may be sure they knew their colleagues better than we! -- even before Aguirre? What eyewitness testimony could they offer? It is a fact that their having ever existed cannot be discovered from the Stalinist press!%
\endnote{Except that, six months after the fall of Bilbao, one minister was expelled from the Communist Party -- a crude move to provide a scapegoat for Stalin’s crimes.}

The Basques had shifted the blame from their own shoulders with vague accusations. That treachery had occurred, they had authoritatively testified to. It is a fact that the government instituted no investigation, no hearings, and made no statement on this question!

The \UGT\ and \CNT’s comments on the loss of Santander were cut to ribbons by the censor, for they sought to draw some lessons. Nevertheless, a wave of bitterness shook the masses. Was it for this that they had fought? Verbal concessions, at least, had to be made. Even Prieto’s organ, \emph{El Socialista,} had declared: 

\begin{quotation}
  Without revealing any secret we can make this affirmation: treachery was present in M\'alaga; it was present in Bilbao; it was present in Santander\dots. The general staff abandoned Malaga without a fight; military leaders made their way to France when Bilbao was in danger; others were in agreement with the enemy to facilitate his entrance into Santander. (August~31).
\end{quotation}

The Stalinists attempted to unload all blame onto the Basque bourgeoisie, in a statement of its Political Bureau in mid-September. Its critical paragraphs corroborate our analysis:

\begin{quotation}
  The long inactivity of these [Bilbao and Santander fronts] was not made use of in order to organize the Army or seriously to fortify our positions. The cadres which were undermined by treachery were not purged; the promotion of new elements to commanding positions was not encouraged\dots.
  
  In the Basque provinces, in Santander, the policy which would have satisfied the desires of the workers and peasants was not carried out. The big landlords and the owners of big undertakings which maintained connections with the fascists retained their privileges, and this cooled the enthusiasm of the fighters.
\end{quotation}

A rotten liberalism secured impunity for the Fifth Column -- the prohibition of public meetings isolated the government and even the People’s Front from the active strata of the people, and prevented the utilization of the courage and enthusiasm of the citizens for defending the towns.

\begin{quotation}
  The questionable behavior and the dishonesty of means employed as well by certain elements (besides other causes which cannot be examined now) helped to undermine the enthusiasm of the population, to weaken the strength of the soldiers\dots. (\emph{Daily Worker,} October~25, 1937).
\end{quotation}

Note that the statement did not – and could not – refer to previous agitation by the Communist party for curtailing the privileges of the bourgeoisie, for the very good reason that, precisely in the name of anti-fascist unity, the party led the struggle against interference with the big bourgeoisie. Let us recall the party leader, Diaz’s declaration, at the previous plenary session of its Central Committee:

\begin{quotation}
  If in the beginning the various premature attempts at ``so\-cial\-i\-za\-tion'' and ``collectivization,'' which were the result of an unclear understanding of the character of the present struggle, might have been justified by the fact that the big landlords and manufacturers had deserted their estates and factories and that it was necessary at all costs to continue production, now on the contrary they cannot be justified at all. At the present time, when there is a government of the People’s Front, in which all the forces engaged in the fight against fascism are represented such things are not only not desirable, but absolutely impermissible. (\emph{Communist International,} May~1937).
\end{quotation}

After this, what utter hypocrisy to complain that ``the big landlords and the owners of big undertakings which maintained connections with the fascists retained their privileges.''

Even more important, the Stalinist statement ended, not with criticism of the bourgeoisie, but with the usual denunciation of the Trotskyists and the attribution of the reverses in the north ``to the lack of unity of firmness of the anti-fascist front.'' A pseudo-Marxist critique was thus put at the service of a program of intensified class collaboration!

At the October first session of the Cortes, the Basque delegation appeared, most of them coming from Paris and returning there afterward. La Pasionaria spoke for the Stalinists: not a word about the treachery of the Basque bourgeoisie. Instead:

\begin{quotation}
  We know that the salaries which the workers earn are not sufficient to take care of their homes\dots. In this sense, we have the example of what can occur, when the workers are not satisfied; we have the example of Euzkadi, where the workers continued with the same salaries because the same capitalist establishments continued.
\end{quotation}

How can one characterize these base words? No other conclusion could be drawn from them, except that the dissatisfied workers had lost the military struggle. The only blame of the bourgeoisie was that they hadn’t given the workers better salaries! If the pseudo-radical reference to ``the same capitalist establishments'' were anything but demagogy, why did not Pasionaria go on to demand that the other capitalist establishments in remaining Loyalist Spain be given to the workers? On the contrary, the cabinet was systematically taking factories and land away from the workers and giving them back to the old proprietors, as we have seen.

\section{The Fall of Asturias}

The Asturian and Santander militiamen – largely \CNT\ and left Socialists – bitterly contested every foot of the ground. The terrain here was even more favorable to the defense than the hilly Santander region. The Asturian dynamiters were still unshakeably holding their grip on the suburbs of Oviedo, immobilizing the garrison there since July~1936. The workers had a small arms and ammunition factory at Trubia in their hands and raw materials from the mining district, and this, with the considerable military stores brought from the Santander region, provided the material basis for holding the north indefinitely. All told nearly 140,000 armed troops were in the Loyalist area of the north. As long as the north held, Franco could launch no big offensive elsewhere. The striking contrast between the defense put up by the Asturians and the previous surrender of Bilbao and Santander was indicated by the fact that not a village was given up before the fascist artillery had razed it. And when encirclement did force retreat, nothing usable was left behind. ``The retreating Asturians seem determined to leave only smoking ruins and desolation behind them when they are finally forced to abandon a town or village\dots. The insurgents found them all dynamited and usually burned to the ground.'' (\emph{New York Times,} October~19, 1937). Every foot of ground cost the fascists huge expenditures of materials and men – until the fall of Cangas de On\'is.

Then something happened. Not in the Oviedo region, where the militia held firm. Not among the forces which, after retreating from Cangas de On\'is, had established new lines. But in the coastal region east of Gij\'on, where the Basque troops were, and which was under the direct command of the general staff stationed in Gij\'on. The fascist Navarrese advanced along the coast from Ribadesella, pushing twenty-eight miles through towns and villages in three days\dots. Even so the chief insurgent forces were fifteen miles east of Gijon when the city surrendered, on October~21.

Why was Gij\'on not defended? There were still sufficient military stores to conduct a struggle for a period. Again we must repeat: a city of buildings is a natural fortress which must be razed before being taken. The one alternative – to retreat elsewhere – was not present, for there was nowhere else the 140,000 soldiers or the civilians could go. There could be no illusions that Franco would not execute thousands upon thousands, especially of the Asturian militiamen. Yet the government left these men to the mercy of Franco. Already on the 16th, the Associated Press reported the arrival in France of the Governor of Asturias and other governmental officials, who, the customs officers reported, carried papers showing that the central government had authorized their flight. (The next day’s dispatch reported that the Spanish crew of the vessel which had brought them had refused to feed them!)

On the 20th, the United Press reported arrival at the Biarritz aerodrome of ``five Spanish Loyalist war planes and a French commercial plane, bringing \emph{fugitive officers from Gij\'on}.'' 

\begin{quotation}
  The aviators declared they left Gij\'on on orders from the chief of their squadron when fighting broke out among the defenders in the streets, and they were cut off from communication with other military units\dots. After questioning, the aviators were liberated and turned over to the Spanish consular authorities at Bayonne.
\end{quotation}

From the same source, the same day:

\begin{quotation}
  The Spanish government renewed pressure today on the French and British to speed up the evacuation of civilians from Gij\'on \emph{and insure the removal of officers} of the Loyalist army of 140,000 men forced to retreat to the sea.
\end{quotation}

Belarmino Tom\'as, Governor of Gij\'on, fled to France on the 20th. Thus, the government saved its functionaries, leaving the armed masses to their fate.

Nor did these masses have the opportunity to die fighting, instead of before execution squads. As a concession to the workers, a Socialist, Tom\'as, had been made Governor of Gij\'on. But this was merely a left facade. No measures had been taken in the two months intervening to purge the officialdom of the Basque Army, or the Santander staffs, or the other officers, or to create worker-patrols to cleanse the city of the Fifth Column. The Civil and Assault Guards of Gij\'on were likewise not sifted. As a result, the masses found themselves in a death trap.

\begin{quotation}
  The coastal column [of the fascists], one of four leading the advance, was nearest to Gij\'on~– fourteen miles away by road~– when the city revolted. The Gij\'on radio opened at 10~\textsc{am} with the sudden announcement: ``We are waiting with great impatience\dots. Viva Franco!''
  
  Shortly before 3:30~\textsc{pm} the red-bereted troops entered the city. Meantime, the Gij\'on radio had explained that the night before, when the government leaders left, secret organizations of Insurgents had gone into the streets in armed groups and had taken over the city. (\emph{New York Times,} October~22, 1937).
\end{quotation}

Three days later, one discovered the role of the ``loyal republican police.'' ``The same police force that has always maintained public order and regulated traffic was on duty there today.'' Once again the praetorian forces of the government and its bourgeois allies had gone over to Franco. It was linguistically appropriate that the formal offer of surrender to Franco came from Colonel Franco, a ``loyal republican.'' Nothing had been destroyed: the small ammunitions plant, the factories, etc., etc., fell intact to Franco. That fact illumined the relation of the governmental officers and functionaries who had fled. Either they had directly aided in the treachery, therefore, the city was intact or, more likely, they dared not inform the soldiers that the city was not to be defended, and, therefore, fled secretly, giving no warning to the armed masses to organize their own defense\dots.

\emph{El gobierno de la Victoria,} Pasionaria had christened it. Six months demonstrated the grotesque ludicrousness of that christening. The one conceivable ``justification'' for its repressions against the workers and peasants might have been its military victories. But precisely from its reactionary politics flowed its disastrous military policies. Whether Spain remained under this terrible yoke and went down to the depths, or freed herself from these organizers of defeat and went forward to victory – whatever happened, history had already stamped the government of Negr\'in--Stalin with its true title: \emph{the government of defeat.}