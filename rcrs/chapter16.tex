\chapter{Military Struggle Under Negr\'in and Prieto}

\indexIPrieto\index{Ministries!Defense}\index{Bourgeois--Stalinist bloc}
\lettrineT{hat the} ``government of victory'' would inevitably continue the disastrous military policy of its predecessor was apparent the day it was constituted. Prieto would continue his inactive naval policy and his political discrimination in the assignment of aircraft to the fronts. He was now also head of the army, with all services in a single Ministry~of Defense, but the Supreme War Council, established in December, had already then been dominated by the bourgeois--Stalinist bloc through their majority of the ministries.\footnote{The Stalinist ``demand'' that the council function normally, raised on May~16, was simply a prop for the myth to make Caballero the scapegoat for the conduct of the war.} The political course which had dictated the previous military strategy---hostility to lighting the flame of revolt in North Africa, support of the Basque bourgeoisie against the workers, persecution of Catalonia and Aragon---all this continued, intensified.

\indexJNegrin
In addition, the Negr\'in cabinet added new obstacles to prosecution of the war.

\index{National question}\index{Catalonia!Autonomy}\index{Political repression}
On the national question---relations with minority peoples---the Negr\'in regime moved not only to the right of Caballero, but also to the right of the republic of 1931--33. The bureaucratic centralization for which the monarchists and fascists stood had been an important factor in alienating from them the peoples of Catalonia, Euzkadi (Basques) and Galicia. Once the civil war began, the limited autonomy of the Catalans and Basques had broadened de facto. A declaration of autonomy for Galicia would have immeasurably facilitated the guerilla warfare there. It was not forthcoming because it would have provided a precedent for Catalonia to seize on. The Negr\'in regime proceeded, as we have seen, to wipe out Catalan autonomy. Where the Bolsheviks had gained strength for the prosecution of the civil war from the intensified loyalty of the autonomous minority nations, the Loyalist government quenched the fires of national aspirations.

\looseness=-1
\index{Inequality}
The pay of militiamen was reduced from ten pesetas a day to seven, while the ascending scale for officers was: 25~pesetas for second lieutenants, 39~for first lieutenants, 50~for captains, 100~for lieutenant colonels. Economic distinctions thus sharply reinforced military regulations. One need scarcely emphasize the deleterious effect on the soldiers’ morale of this and their increasing subordination to the officers.

\index{Bourgeoisie}\index{Fifth column}\index{Assault Guards}\index{Civil Guards}\index{Berneri, Camillo}\index{Political repression}\index{War in the North}
The whole northern front was soon to be betrayed by the Basque bourgeoisie and officers, and by the ``Fifth Column'' of fascist sympathizers in the Assault and Civil Guards and among the civilian population. The struggle against the ``Fifth Column'' was indissolubly part of the military struggle. But, as Camillo Berneri had written even before the intensification of repression under Negr\'in,

\begin{quotation}
  It is self-evident that during the months when an attempt is being made to annihilate the [\POUM--\CNT] “uncontrollables,” the problem of eliminating the Fifth Column cannot be solved. The suppression of the Fifth Column is primarily to be achiev\-ed by an investigatory and repressive activity which can only be accomplished by experienced revolutionists. An internal policy of collaboration between the classes and of consideration toward the middle classes, leads inevitably to tolerance toward elements that are politically doubtful. The Fifth Column is made up not only of fascist elements but also of all the malcontents who hope for a moderate republic.
\end{quotation}

While the northern front was left to the Basque bourgeoisie, the Aragon front was subjected to a frightful purge. General Pozas initiated what was ostensibly a general offensive in June. After several days of artillery and aerial conflict, orders to advance were given to the 29th~Division (formerly the \POUM’s Lenin Division) and other formations. But on the day for the advance, neither artillery nor aviation was provided to protect it\dots\

\index{Aragon Offensive}\indexSPozas\index{Battle}\index{Battle!Aerial}\indexPOUM
Pozas later claimed this was because the air forces were defending Bilbao---but the day of advance was three days after Franco had taken Bilbao. The \POUM\ soldiers fully realized that they were being exposed deliberately. But not to go into fire would have given the bourgeois--Stalinist bloc a case against the Aragon front. They went into the line of fire. One flank was ostensibly assigned to an International Brigade (Stalinist)---but shortly after the advance began, it received orders to withdraw to the rear. The lieutenant colonel in charge of a formation of Assault Guards on the other flank later congratulated the \POUM\ troops: 

\begin{quotation}
  At Sarinena I was warned against you on the ground that you might shoot us in the back. Not only did it not happen but thanks to your bravery and your discipline, we have avoided a catastrophe. I am prepared to go to Sarinena to protest against those who sow the seeds of demoralization, to effect the triumph of their partisan political aims.
\end{quotation}

\index{Political repression}
During this offensive, Cahué and Adriano Nathan, \POUM\ commanders, were killed in action. Police were at that moment coming for Cahué, to arrest him as a ``Trotskyist--fascist.''