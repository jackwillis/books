\chapter{Military Struggle Under Negr\'in and Prieto}

\indexIPrieto\index{Ministries!Defense}\index{Bourgeois--Stalinist bloc}\indexJNegrin
\lettrineT{hat the} ``government of victory'' would inevitably continue the disastrous military policy of its predecessor was apparent the day it was constituted. Prieto would continue his inactive naval policy and his political discrimination in the assignment of aircraft to the fronts. He was now also head of the army, with all services in a single Ministry~of Defense, but the Supreme War Council, established in December, had already then been dominated by the bourgeois--Stalinist bloc through their majority of the ministries.\footnote{The Stalinist ``demand'' that the council function normally, raised on May~16, was simply a prop for the myth to make Caballero the scapegoat for the conduct of the war.}

The political course which had dictated the previous military stra\-te\-gy---hostility to lighting the flame of revolt in North Africa, support of the Basque bourgeoisie against the workers, persecution of Catalonia and Aragon---all this continued, intensified. In addition, the Negr\'in cabinet added new obstacles to prosecution of the war.

\index{National question}\index{Catalonia!Autonomy}\index{Political repression}
On the national question---relations with minority peoples---the Negr\'in regime moved not only to the right of Caballero, but also to the right of the republic of 1931--33. The bureaucratic centralization for which the monarchists and fascists stood had been an important factor in alienating from them the peoples of Catalonia, Euzkadi (Basques) and Galicia. Once the civil war began, the limited autonomy of the Catalans and Basques had broadened de facto. A declaration of autonomy for Galicia would have immeasurably facilitated the guerilla warfare there. It was not forthcoming because it would have provided a precedent for Catalonia to seize on. The Negr\'in regime proceeded, as we have seen, to wipe out Catalan autonomy. Where the Bolsheviks had gained strength for the prosecution of the civil war from the intensified loyalty of the autonomous minority nations, the Loyalist government quenched the fires of national aspirations.

\index{Inequality}
The pay of militiamen was reduced from ten pesetas a day to seven, while the ascending scale for officers was: 25~pesetas for second lieutenants, 39~for first lieutenants, 50~for captains, 100~for lieutenant colonels. Economic distinctions thus sharply reinforced military regulations. One need scarcely emphasize the deleterious effect on the soldiers’ morale of this and their increasing subordination to the~officers.

\index{Bourgeoisie}\index{Fifth column}\index{Assault Guards}\index{Civil Guards}\index{Berneri, Camillo}\index{Political repression}
The whole northern front was soon to be betrayed by the Basque bourgeoisie and officers, and by the ``Fifth Column'' of fascist sympathizers in the Assault and Civil Guards and among the civilian population. The struggle against the ``Fifth Column'' was indissolubly part of the military struggle. But, as Camillo Berneri had written even before the intensification of repression under Negr\'in,

\begin{quotation}
  It is self-evident that during the months when an attempt is being made to annihilate the [\POUM--\CNT] “uncontrollables,” the problem of eliminating the Fifth Column cannot be solved. The suppression of the Fifth Column is primarily to be achiev\-ed by an investigatory and repressive activity which can only be accomplished by experienced revolutionists. An internal policy of collaboration between the classes and of consideration toward the middle classes, leads inevitably to tolerance toward elements that are politically doubtful. The Fifth Column is made up not only of fascist elements but also of all the malcontents who hope for a moderate republic.
\end{quotation}

While the northern front was left to the Basque bourgeoisie, the Aragon front was subjected to a frightful purge. General Pozas initiated what was ostensibly a general offensive in June. After several days of artillery and aerial conflict, orders to advance were given to the 29th~Division (formerly the \POUM’s Lenin Division) and other formations. But on the day for the advance, neither artillery nor aviation was provided to protect it\dots\

\index{Aragon}\index{Basque Country!Biscay!Bilbao}\indexSPozas\index{Battle}\index{Battle!Aerial}\indexPOUM\index{Military withdrawal}
Pozas later claimed this was because the air forces were defending Bilbao---but the day of advance was three days \emph{after} Franco had taken~Bilbao. The \POUM\ soldiers fully realized that they were being exposed deliberately. But not to go into fire would have given the bourgeois--Stalinist bloc a case against the Aragon front. They went into the line of fire. One flank was ostensibly assigned to an International Brigade (Stalinist)---but shortly after the advance began, it received orders to withdraw to the rear. The lieutenant colonel in charge of a formation of Assault Guards on the other flank later congratulated the \POUM\ troops: 

\begin{quotation}
  At Sarinena, I was warned against you on the ground that you might shoot us in the back. Not only did it not happen but thanks to your bravery and your discipline, we have avoided a catastrophe. I am prepared to go to Sarinena to protest against those who sow the seeds of demoralization, to effect the triumph of their partisan political aims.
\end{quotation}

\index{Political repression}\index{Assassination}\indexPOUM\index{Trotskyism}
During this offensive, Cahué and Adriano Nathan, \POUM\ commanders, were killed in action. Police were at that moment coming for Cahué, to arrest him as a ``Trotskyist--fascist.''

\index{Workers' militias!Liquidation of militias}\indexSPozas\index{Aragon!Huesca}\index{Political prisoners}\indexPOUM\index{Military supplies!Rifles}\index{Battle}\index{Military withdrawal}
When the attack was over, the 29th was sent to the rear. That, customarily, had meant to give up rifles---there were still not enough for front-line and reserves simultaneously on this front! But the suspicious \POUM\ troops refused to yield their arms. They declared themselves ready to return to the front. A few days later two battalions of the division were ordered to march on Fiscal (on the Jaca front) to repulse a fascist attack. Not only did they crush the attack but they reconquered positions and material previously lost. Then they were retired to await new orders---but not sent back to their division. Why? To disarm them. Pozas ordered it. They were concentrated in the village of Rodeano and surrounded by a Stalinist brigade. They were relieved of all their valuables---watches, chains, even good underwear and new shoes. Their leaders were arrested, the rest permitted to go---on foot. Hiking home, many were arrested in the towns on the way. The only reason the same methods were not employed against the rest of the division was that the news leaked out quickly and Pozas feared the \CNT\ divisions would come to its defense. But a few weeks later the 29th was officially dissolved, the remaining men being distributed far and wide in small groups.\endnote{This account is that of the front correspondent of \emph{Avanti,} émigré (Paris) organ of the Italian Maximalist-Socialists, scarcely a \POUM\ or Trotskyist source.}

\index{Workers' militias!Liquidation of militias}\indexCNT
The Ascaso (\CNT) division was also cut to pieces. \emph{Acracia,} \CNT\ organ of Lerida, wrote:

\vspace{-0.33\baselineskip}

\index{Aragon!Huesca}\index{Ministries!Air force}\index{Battle}\index{Military supplies!Planes}
\begin{quotation}
  Now we know exactly why Huesca was not taken. The last operation at Santa Quiteria furnishes a good proof of it. Hues\-ca was surrounded on all sides and only the betrayal of the air forces (controlled by \PSUC) was responsible for the disaster with which this operation ended. Our militiamen were not backed up by the air forces and were thus left defenseless in face of an intensive machine-gunning by the fascist air forces. This is only one of the numerous operations which ended in the same manner on account of the same betrayal by the air forces.
\end{quotation}

\vspace{-0.33\baselineskip}

\indexPSUC\indexSPozas
Soon after there was a plenary session of the \PSUC\ Central Committee in Barcelona. Among those prominently participating were ``Comrades'' General Pozas, chief of the Aragon front, Virgilio Llanos, political commissioner of the front, and Lieutenant Colonel Gordon, chief of staff\dots

Acceptance of control by the central government had been dangled before the Aragon front troops as the end of all their worries. Instead, it was used to break them down still further. The front correspondent of the anarchist (Paris) \emph{Libertaire} wrote on July~29:

\index{Political repression!Boycott}\index{Friends of Durruti}\index{Aragon!Zaragoza!Bujaraloz}\index{Exhaustion}
\begin{quotation}
  Ever since the central government took over control, the financial boycott became accentuated. Most of the militiamen have not received their pay for a long time. In Bujaraloz, where the general staff of the Durruti column is located, both officers and soldiers have not seen a cent for the last three months. They cannot wash their clothes for lack of soap. In many a place visited after several months of absence, I found comrades whom I knew well: now they looked pale, thin and visibly weakened. The physical state of the troops is such that they cannot keep up any prolonged exercises; they cannot march for more than fifteen kilometers a day. In the region of Farlete the troops live by hunting, without which they would starve to death.
\end{quotation}

\vspace{-0.33\baselineskip}

\indexLibertad\index{Loss of territory}\index{Casualties}
Systematic persecution of the chief forces of the Aragon front scarcely laid the basis for military victories, although at Belchite and Quinto the 25th division (\CNT) gave a good account of itself. But the alleged success of the July offensive on the Aragon front was so much newspaper talk. ``Results?'' wrote the illegal anarchist organ, \emph{Libertad} (August~1): ``Two villages lost in the Pirineo sector and three thousand men lost. This is what they call a success. Disastrous, calamitous, shameful success!''

After the fall of Santander (August 26), the persecution of the \CNT\ troops abated somewhat. But now came a terrible lesson in the consequences of creating counter-revolutionary forces of repression, such as the Stalinist-controlled Karl Marx division. In the midst of an offensive in the Zuera sector,

\vspace{-0.33\baselineskip}

\indexCNT\indexPSUC\index{Court martial@Court-martial}\index{Desertion}\index{Karl Marx Division}\index{Friends of Durruti}
\begin{quotation}
  Fifty officers of that division and six hundred soldiers pass\-ed over to the fascists. As a result of these desertions a battalion was destroyed. Despite the mettle of the \CNT\ forces, the operation could not be terminated well. The enemy had the necessary time to recover and it was impossible to continue the attack. After a summary court-martial that was immediately convened, thirty officers of the Karl Marx Division have been shot. In addition, the political commissar of the division, Trueba, a \PSUC\ member, has been dismissed. (\emph{Amigo del Pueblo,} illegal organ of Friends of Durruti, September~21).
\end{quotation}

\vspace{-0.33\baselineskip}

\index{Censorship}
Needless to say, the \CNT\ press was forbidden to publish the facts.

\section{The Northern Front}

\begin{sloppypar}
  As a government pledged to class collaborations even more completely than Caballero’s, the Negr\'in government did nothing to counter the more and more brazen sabotage of the Basque bourgeoisie. This front was almost inactive throughout the period from November 1936 to May 1937 when the fascists moved to wipe it out altogether. Nor had those six months been utilized in economic-military, preparations. One must emphasize again that the Euzkadi (the Basque country) was second only to Catalonia as an industrial region and superior to it in being a region of heavy industry, with iron and steel plants in the midst of iron and coal mines. \emph{Nothing} was done to develop out of this a great war industry. For this crime, the Stalinists bore equal responsibility, for two party representatives were ministers in the autonomous government. The coup against the \CNT\ in March, when its regional committee had been imprisoned and its press confiscated, was now followed by systematic repression of the workers, with public meetings prohibited. Thus, the one force which might have prevented betrayal was crushed by the Stalinist--bourgeois bloc.
\end{sloppypar}

In the Caballero cabinet, as we have said, there were constant fears about the loyalty of the Basques. Irujo’s constant threats of giving up altogether merely reflected the fact that the bourgeoisie had no serious stake in the struggle against fascism and would not fight under conditions destructive of their property. Consequently, when Franco began to move in the north, Caballero planned a large-scale offensive on the southern Madrid front to draw the fire of the fascist forces. According to his friends, 75,000 fully equipped troops were to have gone into action but two or three days before the offensive was to begin, his resignation was forced. Negr\'in’s first act was to withdraw these troops. Be that as it may, the fact was that no offensive was launched to relieve Bilbao, either on the Madrid or Aragon front, until mid-June—too late.

But the decisive factor in the loss of Bilbao was open treachery. ``Not even the insurgent heavy guns,\kn'' wrote the \emph{New York Times} correspondent,

\begin{quotation}
  \noindent
  \dots\ could have destroyed some of these underground fortifications with their three armored concrete tiers and blockhouses spaced about three miles apart all the way to the Biscay coast. The Insurgents themselves say that the “iron ring” of fortifications would never have been taken had not the Basques been outmaneuvered.
\end{quotation}

``Outmaneuvered,'' however, was a fascist euphemism for betrayal. After the city fell, this fact was admitted by the Basque delegation in Paris which put the blame on an engineer in charge of building the fortifications, who fled to Franco with the plans. Analysis of the delegation’s story revealed that the engineer in question had fled \emph{months before.} Why was not the intervening period utilized to redesign the fortifications? But the alibi was a subterfuge. For, as any tyro in military science knew, mere possession of the plans could not have solved the problem for the fascists of breaking through the fortification. \emph{They were let through the iron ring.}

Suppose we were to accept the Basque alibi. Why, then, was Bilbao not defended in a siege such as Madrid—less advantageously situated—had withstood? It is an elementary axiom of military science that no large city can be captured until its massive buildings—veritable fortifications—have been razed to the point they offer no further protection to beleaguered troops. The process of razing buildings by shelling and bombing required enormous equipment such as the fascists did not have less than an eighth of Madrid had been so razed after a year’s shelling and bombing.

But the bourgeoisie did not wait for the shelling of Bilbao at all! On June~19, they surrendered the city, as they had San Sebasti\'an the previous September. The uniform Basque policy of giving up cities intact has no parallel in any modern war, not to speak of civil war.

The pro-Loyalist correspondent of the \emph{New York Times} (June~21, 1937) wrote:

\begin{quotation}
  Details learned today of the last hours of Basque rule at Bilbao show that some 1,200 militiamen, who had before the civil war been soldiers of the regular army, decided in the early hours of the morning, after the bridges had been blown up, that chaos had gone far enough, and took control of the city in the capacity of police. The Asturian and Santander militiamen were driven out of the city.
  
  Assisted by some regular police and Civil Guards, this battalion accepted the surrender of their fellow-militiamen in the city and took their arms away, and afterwards hoisted a white flag on the telephone building. During the night they went around houses assuring the people that there was no cause for panic, placed guards on public buildings, and in the evening formed a cordon in the main street which prevented the excited crowds from pressing too closely on the National troops when they marched into the city.
\end{quotation}

Leisola, Minister of Justice in the Basque government, remained behind to superintend the betrayal. With the exception of seventeen (of whom we shall soon hear again), all fascist hostages were released and sent toward the fascist lines as a goodwill offering before the troops had reached the city. Simply put: the regular army of the Basques, directed by the bourgeois leaders, joined hands with the ``republican police'' in attacking the Asturians and militia from the rear, disarmed as many as they could, and dismantled the houses and street barricades which the workers had prepared for street fighting. Shortly after the occupation, the same police donned Carlist berets and became Franco’s regular police.

\CNT\ and \UGT\ press attempts to sound the alarm after the fall of Bilbao were cut to ribbons by the censorship. The Basque general staff was permitted to remain in command of the retreating troops. When, within a few weeks, the fascists began a second offensive, the industrial town of Reinosa, key to Santander’s defenses, collapsed, and once again the Basques made no attempt to defend the city itself.

Two days before the fall of Santander, the Basque general staff and remaining members of the government fled to France on a British warship. Under what terms, was revealed by the \emph{New York Times} dispatch of August~25:
