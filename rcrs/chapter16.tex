\chapter{Military Struggle Under Negr\'in and Prieto}

\indexIPrieto\index{Ministries!Defense}\index{Bourgeois--Stalinist bloc}\indexJNegrin
\lettrineT{hat the} ``government of victory'' would inevitably continue the disastrous military policy of its predecessor was apparent the day it was constituted. Prieto would continue his inactive naval policy and his political discrimination in the assignment of aircraft to the fronts. He was now also head of the army, with all services in a single Ministry~of Defense, but the Supreme War Council, established in December, had already then been dominated by the bourgeois--Stalinist bloc through their majority of the ministries.\footnote{The Stalinist ``demand'' that the council function normally, raised on May~16, was simply a prop for the myth to make Caballero the scapegoat for the conduct of the war.}

The political course which had dictated the previous military stra\-te\-gy---hostility to lighting the flame of revolt in North Africa, support of the Basque bourgeoisie against the workers, persecution of Catalonia and Aragon---all this continued, intensified. In addition, the Negr\'in cabinet added new obstacles to prosecution of the war.

\index{National question}\index{Catalonia!Autonomy}\index{Political repression}
On the national question---relations with minority peoples---the Negr\'in regime moved not only to the right of Caballero, but also to the right of the republic of 1931--33. The bureaucratic centralization for which the monarchists and fascists stood had been an important factor in alienating from them the peoples of Catalonia, Euzkadi (Basques) and Galicia. Once the civil war began, the limited autonomy of the Catalans and Basques had broadened de facto. A declaration of autonomy for Galicia would have immeasurably facilitated the guerilla warfare there. It was not forthcoming because it would have provided a precedent for Catalonia to seize on. The Negr\'in regime proceeded, as we have seen, to wipe out Catalan autonomy. Where the Bolsheviks had gained strength for the prosecution of the civil war from the intensified loyalty of the autonomous minority nations, the Loyalist government quenched the fires of national aspirations.

\index{Inequality}
The pay of militiamen was reduced from ten pesetas a day to seven, while the ascending scale for officers was: 25~pesetas for second lieutenants, 39~for first lieutenants, 50~for captains, 100~for lieutenant colonels. Economic distinctions thus sharply reinforced military regulations. One need scarcely emphasize the deleterious effect on the soldiers’ morale of this and their increasing subordination to the~officers.

\index{Bourgeoisie}\index{Fifth column}\index{Assault Guards}\index{Civil Guards}\index{Berneri, Camillo}\index{Political repression}
The whole northern front was soon to be betrayed by the Basque bourgeoisie and officers, and by the ``Fifth Column'' of fascist sympathizers in the Assault and Civil Guards and among the civilian population. The struggle against the ``Fifth Column'' was indissolubly part of the military struggle. But, as Camillo Berneri had written even before the intensification of repression under Negr\'in,

\begin{quotation}
  It is self-evident that during the months when an attempt is being made to annihilate the [\POUM--\CNT] “uncontrollables,” the problem of eliminating the Fifth Column cannot be solved. The suppression of the Fifth Column is primarily to be achiev\-ed by an investigatory and repressive activity which can only be accomplished by experienced revolutionists. An internal policy of collaboration between the classes and of consideration toward the middle classes, leads inevitably to tolerance toward elements that are politically doubtful. The Fifth Column is made up not only of fascist elements but also of all the malcontents who hope for a moderate republic.
\end{quotation}

While the northern front was left to the Basque bourgeoisie, the Aragon front was subjected to a frightful purge. General Pozas initiated what was ostensibly a general offensive in June. After several days of artillery and aerial conflict, orders to advance were given to the 29th~Division (formerly the \POUM’s Lenin Division) and other formations. But on the day for the advance, neither artillery nor aviation was provided to protect it\dots\

\index{Aragon}\index{Basque Country!Biscay!Bilbao}\indexSPozas\index{Battle}\index{Battle!Aerial}\indexPOUM\index{Military withdrawal}
Pozas later claimed this was because the air forces were defending Bilbao---but the day of advance was three days \emph{after} Franco had taken~Bilbao. The \POUM\ soldiers fully realized that they were being exposed deliberately. But not to go into fire would have given the bourgeois--Stalinist bloc a case against the Aragon front. They went into the line of fire. One flank was ostensibly assigned to an International Brigade (Stalinist)---but shortly after the advance began, it received orders to withdraw to the rear. The lieutenant colonel in charge of a formation of Assault Guards on the other flank later congratulated the \POUM\ troops: 

\begin{quotation}
  At Sarinena, I was warned against you on the ground that you might shoot us in the back. Not only did it not happen but thanks to your bravery and your discipline, we have avoided a catastrophe. I am prepared to go to Sarinena to protest against those who sow the seeds of demoralization, to effect the triumph of their partisan political aims.
\end{quotation}

\index{Political repression}\index{Assassination}\indexPOUM\index{Trotskyism}
During this offensive, Cahué and Adriano Nathan, \POUM\ commanders, were killed in action. Police were at that moment coming for Cahué, to arrest him as a ``Trotskyist--fascist.''

\index{Workers' militias!Liquidation of militias}\indexSPozas\index{Aragon!Huesca}\index{Political prisoners}\indexPOUM\index{Military supplies!Rifles}\index{Battle}\index{Military withdrawal}
When the attack was over, the 29th was sent to the rear. That, customarily, had meant to give up rifles---there were still not enough for front-line and reserves simultaneously on this front! But the suspicious \POUM\ troops refused to yield their arms. They declared themselves ready to return to the front. A few days later two battalions of the division were ordered to march on Fiscal (on the Jaca front) to repulse a fascist attack. Not only did they crush the attack but they reconquered positions and material previously lost. Then they were retired to await new orders---but not sent back to their division. Why? To disarm them. Pozas ordered it. They were concentrated in the village of Rodeano and surrounded by a Stalinist brigade. They were relieved of all their valuables---watches, chains, even good underwear and new shoes. Their leaders were arrested, the rest permitted to go---on foot. Hiking home, many were arrested in the towns on the way. The only reason the same methods were not employed against the rest of the division was that the news leaked out quickly and Pozas feared the \CNT\ divisions would come to its defense. But a few weeks later the 29th was officially dissolved, the remaining men being distributed far and wide in small groups.\endnote{This account is that of the front correspondent of \emph{Avanti,} émigré (Paris) organ of the Italian Maximalist-Socialists, scarcely a \POUM\ or Trotskyist source.}

\index{Workers' militias!Liquidation of militias}\indexCNT
The Ascaso (\CNT) division was also cut to pieces. \emph{Acracia,} \CNT\ organ of Lerida, wrote:

\index{Aragon!Huesca}\index{Ministries!Air force}\index{Battle}\index{Military supplies!Planes}
\begin{quotation}
  Now we know exactly why Huesca was not taken. The last operation at Santa Quiteria furnishes a good proof of it. Hues\-ca was surrounded on all sides and only the betrayal of the air forces (controlled by \PSUC) was responsible for the disaster with which this operation ended. Our militiamen were not backed up by the air forces and were thus left defenseless in face of an intensive machine-gunning by the fascist air forces. This is only one of the numerous operations which ended in the same manner on account of the same betrayal by the air forces.
\end{quotation}

\indexPSUC\indexSPozas
Soon after there was a plenary session of the \PSUC\ Central Committee in Barcelona. Among those prominently participating were ``Comrades'' General Pozas, chief of the Aragon front, Virgilio Llanos, political commissioner of the front, and Lieutenant Colonel Gordon, chief of staff\dots

Acceptance of control by the central government had been dangled before the Aragon front troops as the end of all their worries. Instead, it was used to break them down still further. The front correspondent of the anarchist (Paris) \emph{Libertaire} wrote on July~29:

\index{Political repression!Boycott}\index{Friends of Durruti}\index{Aragon!Zaragoza!Bujaraloz}\index{Exhaustion}
\begin{quotation}
  Ever since the central government took over control, the financial boycott became accentuated. Most of the militiamen have not received their pay for a long time. In Bujaraloz, where the general staff of the Durruti column is located, both officers and soldiers have not seen a cent for the last three months. They cannot wash their clothes for lack of soap. In many a place visited after several months of absence, I found comrades whom I knew well: now they looked pale, thin and visibly weakened. The physical state of the troops is such that they cannot keep up any prolonged exercises; they cannot march for more than fifteen kilometers a day. In the region of Farlete the troops live by hunting, without which they would starve to death.
\end{quotation}

\indexLibertad\index{Loss of territory}\index{Casualties}
Systematic persecution of the chief forces of the Aragon front scarcely laid the basis for military victories, although at Belchite and Quinto the 25th division (\CNT) gave a good account of itself. But the alleged success of the July offensive on the Aragon front was so much newspaper talk. ``Results?'' wrote the illegal anarchist organ, \emph{Libertad} (August~1): ``Two villages lost in the Pirineo sector and three thousand men lost. This is what they call a success. Disastrous, calamitous, shameful success!''

After the fall of Santander (August 26), the persecution of the \CNT\ troops abated somewhat. But now came a terrible lesson in the consequences of creating counter-revolutionary forces of repression, such as the Stalinist-controlled Karl Marx division. In the midst of an offensive in the Zuera sector,

\indexCNT\indexPSUC\index{Court martial@Court-martial}\index{Desertion}\index{Karl Marx Division}\index{Friends of Durruti}
\begin{quotation}
  Fifty officers of that division and six hundred soldiers pass\-ed over to the fascists. As a result of these desertions a battalion was destroyed. Despite the mettle of the \CNT\ forces, the operation could not be terminated well. The enemy had the necessary time to recover and it was impossible to continue the attack. After a summary court-martial that was immediately convened, thirty officers of the Karl Marx Division have been shot. In addition, the political commissar of the division, Trueba, a \PSUC\ member, has been dismissed.\footnote{\emph{Amigo del Pueblo,} illegal organ of Friends of Durruti, September~21.}
\end{quotation}

\index{Censorship}
Needless to say, the \CNT\ press was forbidden to publish the facts.

\section{The Northern Front}




