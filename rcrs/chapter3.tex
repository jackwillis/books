\chapter{The Revolution of July 19}

\lettrineT{he Barcelona proletariat} prevented the capitulation of the republic to the fascists. On July 19, almost barehanded, they stormed the first barracks successfully. By 2 \textsc{pm} the next day they were masters of Barcelona.\index{Barcelona}

It was not accidental that the honour of initiating the armed struggle against fascism belongs to the Barcelona proletariat. Chief seaport and industrial centre of Spain, concentrating in it and the surrounding industrial towns of Catalonia nearly half the industrial proletariat of Spain, Barcelona has always been the revolutionary vanguard. The parliamentary reformism of the socialist-led \textsc{ugt} had never found a foothold there. The united socialist and Stalinist parties (the \textsc{psuc}) had fewer members on July 19 than the \textsc{poum}. The workers were almost wholly organized in the \textsc{cnt}, whose suffering and persecution under both the monarchy and republic had imbued its masses with a militant anti-capitalist tradition, although its anarchist philosophy gave it no systematic direction. But, before this philosophy was to reveal its tragic inadequacy, the \textsc{cnt} reached historic heights in its successful struggle against the forces of General Goded.

As in Madrid, the Catalan government refused to arm the workers. \textsc{cnt} and \textsc{poum} emissaries, demanding arms, were smilingly informed they could pick up those dropped by wounded Assault Guards.

But \textsc{cnt} and \textsc{poum} workers during the afternoon of the 18th were raiding sporting goods stores for rifles, construction jobs for sticks of dynamite, fascist homes for concealed weapons. With the aid of a few friendly Assault Guards, they had seized a few racks of government rifles. (The revolutionary workers had painstakingly gathered a few guns and pistols since 1934.) That---and as many motor vehicles as they could find---was all the workers had when, at five o’clock on the morning of the 19th, the fascist officers began to lead detachments from the barracks.

Isolated engagements before paving-stone barricades led to a general engagement in the afternoon. And here political weapons more than made up for the superior armament of the fascists. Heroic workers stepped forward from the lines to call upon the soldiers to learn why they were shooting down their fellow toilers. They fell under rifle and machine-gun fire, but others took their place. Here and there a soldier began shooting wide. Soon, bolder ones turned on their officers. Some nameless military genius---perhaps he died then---seized the moment and the mass of workers abandoned their prone positions and surged forward. The first barracks were taken. General Goded was captured that afternoon. With arms from the arsenals the workers cleaned up Barcelona. Within a few days, all Catalonia was in their hands.

Simultaneously the Madrid proletariat was mobilizing. The left socialists distributed their scant store of arms, saved from October 1934. Barricades went up on key streets and around the Montaña barracks. Workers’ groups were looking for reactionary leaders. At dawn of the 19th the first militia patrols took their places. At midnight the first shots were exchanged with the barracks. But it was not until the next day, when the great news came from Barcelona, that the barracks were stormed.

Valencia, too, was soon saved from the fascists. Refused arms by the governor appointed by Azaña, the workers prepared to face the troops with barricades, cobblestones and kitchen knives---until their comrades within the garrison shot the officers and gave arms to the workers.

The Asturian miners, fighters of the Commune of October 1934, outfitted a column of five thousand dynamiters for a march on Ma\-drid. It arrived there on the 20th, just after the barracks had been taken, and took up guard duty in the streets.
\nowidow

In Malaga, strategic port opposite Morocco, the ingenious workers, unarmed at first, had surrounded the reactionary garrison with a wall of gasoline-fired houses and barricades.

In a word, without so much as by-your-leave to the government, the proletariat had begun a war to the death against the fascists. The initiative had passed out of the hands of the republican bourgeoisie.

Most of the army was with the fascists. It must be confronted by a new army. Every workers’ organization proceeded to organize militia regiments, equip them, and send them to the front. The government had no direct contact with the workers’ militia. The organizations presented their requisition’s and payrolls to the government, which handed over supplies and funds which the organizations distributed to the militias. Such officers as remained in the Loyalist camp were assigned as ``technicians'' to the militias. Their military proposals were transmitted to the militiamen through the worker-officers. Those Civil and Assault Guards still adhering to the government soon disappeared from the streets. In the prevailing atmosphere the government was compelled to send them to the front. Their police duties were taken over by worker-police and militiamen.

The sailors, traditionally more radical than soldiers, saved a good part of the fleet by shooting their officers. Elected sailors’ committees took over control of the Loyalist fleet, and established contact with the workers’ committees on shore.

Armed workers’ committees displaced the customs officers at the frontiers. A union book or red party card was better than a passport for entering the country. Few reactionaries managed to get out through the workers’ cordon.

The revolutionary-military measures were accompanied by re\-vo\-lu\-tion\-ary-economic measures against fascism. Why this happened, if the world-historical scheme called merely for ``defence of the republic,'' the Stalinist-democrats have yet to explain.

Especially was this true in Catalonia where, within a week from July 19, transport and industry was almost entirely in the hands of \textsc{cnt} workers’ committees, or where workers belonged to both, \textsc{cnt-ugt} joint committees. The union committees systematically took over, re-established order and speeded up production for wartime needs. Through national industries stemming from Barcelona, the same process spread to Madrid, Valencia, Alicante, Almeria and Malaga although never becoming as universal as in Catalonia. In the Basque provinces, however, where the big bourgeoisie had declared for the democratic republic, they remained masters in the factories. A \textsc{cnt-ugt} committee took charge of all transportation in Spain. Soon factory delegations were going abroad to arrange for exports and imports.

The peasants needed no urging to take the land. They had been trying to take it since 1931; but Casas Viejas, Castilblanco, Yeste, were honoured names of villages where the peasants had been massacred by Azaña’s troops because they had taken land. Now Azaña could not stop them. As the news came from the cities, the peasants spread over the land. Their scythes and axes took care of any government official or republican landowner injudicious enough to bar their way. In many places, permeated by anarchist and left-socialist teachings, the peasants organized directly into collectives. Peasant committees took charge of feeding the militias and the cities, giving or selling directly to the provisioning committees, militia columns and the trade unions.

Everywhere the existing governmental forms and workers’ organizations proved inadequate as methods of organizing the war and revolution. Every district, town and village created its militia committees, to arm the masses and drill them. The \textsc{cnt-ugt} factory committees, directing all the workers, including those never before organized, developed a broader scope than the existing trade union organizations. The old municipal administrations disappeared, to be replaced, generally, by agreed-upon committees giving representation to all the anti-fascist parties and unions. But in these the Esquerra and Republican Left politicians seldom appeared. They were replaced by workers and peasants who, though still adhering to the republican parties, followed the lead of the more advanced workers who sat with them.

The most important of these new organs of power was the Central Committee of Anti-Fascist Militias of Catalonia, organized July 21. Of its fifteen members, five were anarchists from the \textsc{cnt} and \textsc{fai}, and these dominated the Central Committee. The \textsc{ugt} had three members, despite its numerical weakness in Catalonia, but the anarchists hoped thereby to encourage similar committees elsewhere. The \textsc{poum} had one, the Peasant Union (Rabassaires) one, and the Stalinists (\textsc{psuc}) one. The bourgeois parties had four.

Unlike a coalition government which in actuality rests on the old state machine, the Central Committee, dominated by the anarchists, rested on the workers’ organizations and militias. The Esquerra and those closest to it---the Stalinists and the \textsc{ugt}---merely tagged along for the time being. The decrees of the Central Committee were the only law in Catalonia. Companys unquestioningly obeyed its requisitions and financial orders. Beginning presumably as the centre for organizing the militias, it inevitably had to take on more and more governmental functions. Soon it organized a department of worker-police; then a department of supplies, whose word was law in the factories and seaport.

In those months in which the Central Committee existed, its military campaigns were inextricably bound up with revolutionary acts. This is evident from its campaign in Aragon, on which the Catalan militias marched within five days. They conquered Aragon as an army of social liberation. Village anti-fascist committees were set up, to which were turned over all the large estates, crops, supplies, cattle, tools, etc., belonging to big land owners and reactionaries. Thereupon the village committee organized production on the new basis, usually collectives, and created a village militia to carry out socialization and fight reaction. Captured reactionaries were placed before the general assembly of the village for trial. All property titles, mortgages, debt documents in the official records, were consigned to the bonfire. Having thus transformed the world of the village, the Catalonian columns could go forward, secure in the knowledge that every village so dealt with was a fortress of the revolution!

Much malicious propaganda has been spread by the Stalinists concerning the alleged weakness of the military activity of the anarchists. The hasty creation of militias, the organization of war industry, were inevitably haphazard in all unaccustomed hands. But in those first months, the anarchists, seconded by the \textsc{poum}, made up for much of their military inexperience by their bold social policies. In civil war, politics is the determining weapon. By taking the initiative, by seizing the factories, by encouraging the peasantry to take the land, the \textsc{cnt} masses crushed the Catalonian garrisons. By marching into Aragon as social liberators, they roused the peasantry to paralyse the mobility of the fascist forces. In the plans of the generals, Saragossa, seat of the War College and perhaps the biggest of the army garrisons, was to have been for Eastern Spain what Burgos became in the west. Instead, Saragossa was immobilized from the first days.

Around the Central Committee of the militias rallied the multitudinous committees of the factories, villages, supplies, food, police, etc., inform joint committees of the various anti-fascist organizations, in actuality wielding an authority greater than that of its constituents. After the first tidal wave of revolution, of course, the committees revealed their basic weakness: they were based on mutual agreement of the organizations from which they drew their members, and after the first weeks, the Esquerra, backed by the Stalinists, recovered their courage and voiced their own programme. The \textsc{cnt} leaders began to make concessions detrimental to the revolution. From that point on, the committees could have only functioned progressively by abandoning the method of mutual agreement and adopting the method of majority decisions by democratically elected delegates from the militias and factories.

The Valencia and Madrid regions also developed a network of anti-fascist joint militia committees, worker-patrols, factory committees, and district committees to wipe out the reactionaries in the cities and send the militia to the front.
\nowidow

Thus, side by side with the official governments of Madrid and Catalonia there had arisen organs predominantly worker-controlled, through which the masses organized the struggle against fascism. In the main, the military, economic and political struggle was proceeding independently of the government and, indeed, in spite of it.

How are we to characterize such a regime? In essence, it was identical with the regime which existed in Russia from February to November 1917---a regime of dual power. One power, that of Azaña and Companys, without an army, police or other armed force of its own, was already too weak to challenge the existence of the other. The other, that of the armed proletariat, was not yet conscious enough of the necessity to dispense with the existence of the power of Azaña and Companys. This phenomenon of dual power has accompanied all proletarian revolutions. It signifies that the class struggle is about to reach the point where either one or the other must become undisputed master. It is a critical balancing of alternatives on a razor edge. A long period of equilibrium is out of the question; either one or the other must soon prevail! The ``Revolution of July 19'' was incomplete, but that it was a revolution is attested to by its having created a regime of dual power.