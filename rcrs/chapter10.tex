\chapter[The May Days]{The May Days: Barricades in Barcelona}

\lettrineE{ven more than} before the civil war, Catalonia was the chief economic centre of Spain; and these economic forces were now in the hands of the workers and peasants (\emph{so they thought}). The entire textile industry of Spain was located here. Its workers now provided clothing and blankets for the armies and the civilian population, and the vitally needed goods for export. With Bilbao’s iron and steel mills virtually cut off from the rest of Spain, the metal and chemical workers of Catalonia had, by the most heroic diligence, created a great war industry to equip the anti-fascist armies. The agricultural collectives, raising the greatest crops in Spanish history, were feeding the armies and cities and providing citrus fruit for export. The \CNT\ seamen were carrying away the exports which gave Spain credits abroad and were bringing home precious cargoes for use in the struggle against Franco. The masses of the \CNT\ were holding the Aragon and Teruel fronts; they had sent Durruti and the pick of their militias to save Madrid in the nick of time. The Catalonian proletariat, in a word, was the backbone of the anti-fascist forces, and knew it.

What is more, its power had been acknowledged, after July 19, even by Companys. The Catalan president, addressing the \CNT-\FAI\ in the July Days, had said:

\begin{quotation}
  You have always been severely persecuted and I, with much pain but forced by political realities, I, who was once with you, later saw myself obliged to oppose and persecute you. Today you are the masters of the city and of Catalonia, because you alone vanquished the Fascist soldiers. I hope you will not find it distasteful that I should now remind you that you did not lack the aid of the few or many men of my party and of the Guards\dots\ You have conquered, and all is in your power. If you do not need or want me as President tell me now, and I shall become another soldier in the anti-fascist fight. If, on the contrary, you believe me when I say that I would abandon this post to victorious fascism only as a corpse, perhaps, with my party comrades and my name and prestige, I can serve you.
\end{quotation}

Consequently, the alarm and rage of the Catalonian masses at the counter-revolutionary inroads were the emotions of freed men and masters of their fate in danger again of enslavement. Submission without a fight was out of the question!

On April 17---the day after the \CNT\ ministers rejoined the Ge\-ne\-ra\-li\-dad---a force of Carabineros arrived in Puigcerda and demanded that the \CNT\ worker-patrols there surrender control of the customs. While \CNT\ top-leaders hurried to Puigcerda to arrange a peaceful solution---i.e., to cajole the workers into surrendering control of the border---Assault and Civil Guards were sent to Figueras and other towns throughout the province to wrest police control from the workers’ organizations. Simultaneously, in Barcelona, the Assault Guards proceeded to disarm workers on sight, in the streets. During the last week of April they reported three hundred thus disarmed. Collisions between the workers and the Guards took place nightly. Truckloads of Guards would disarm solitary workers. The workers retaliated. Workers who refused to submit were shot. Guards were picked off in turn.

On April 25, a \PSUC\ trade union leader, Roldan Cortada, was assassinated in Molins de Llobregat. Who killed him is not known to this day. The \CNT\ denounced the murder and proposed an investigation. The \POUM\ pointed out that, significantly enough, Cortada had been a supporter of Caballero before the fusion and had been known to disapprove of the pogrom-spirit being generated by the Stalinists. But the \PSUC\ squeezed the opportunity dry, denouncing the ``uncontrollables,'' ``hidden fascist agents,'' etc.

On April 27, the \CNT\ and \POUM\ representatives appeared at Cortada’s funeral---and found it a demonstration of the forces of the counter-revolution. For three and a half hours the ``funeral''---\PSUC\ and government soldiers and police gathered from far and wide and armed to the teeth---marched through the workers’ districts of Barcelona. It was a challenge and the \CNT\ masses were not blind to it. The next day the government dispatched a punitive expedition to Molins de Llobregat, seized the anarchist leaders there and brought them handcuffed back to Barcelona. That night and the next, \CNT\ and \PSUC\ Assault Guard groups were disarming each other in the streets. The first barricades were erected in the workers’ suburbs.

The Carabineros, reinforced and joined by the local \PSUC\ forces, attacked the worker-patrols in Puigcerda. Antonio Martin, mayor and \CNT\ leader, popular throughout Catalonia, was shot dead by the Stalinists.

May Day, oldest and dearest of proletarian holidays, dawned the government prohibited all meetings and demonstrations throughout Spain.

In those last days of April, the Barcelona workers learned for the first time, through the pages of \emph{Solidaridad Obrera}, what had happened to their comrades in Madrid and Murcia at the hands of the Stalinist \GPU.

\dinkus

The Telefonica, the main telephone building dominating Bar\-ce\-lo\-na’s busiest square, had been occupied by fascist troops on July 19, 1936, surrendered to them by the Assault Guards the government had sent there. The \CNT\ workers had lost many comrades in re-conquering it. So much the dearer was possession of it. Since July 19, the red and black flag of the \CNT\ had flown from its tower, visible to workers throughout the city. Since July 19, the exchange had been managed by a \CNT-\UGT\ committee, with a government delegation stationed in the building. The working staff was almost entirely \CNT\ in allegiance and \CNT\ armed guards defended it against fascist forays.

Control of the Telefonica was a concrete instance of the dual power. The \CNT\ was in a position to listen in on government calls. The bourgeois-Stalinist bloc would never be master in Catalonia so long as it was possible for the workers to cut off telephonic co-ordination of the government forces.

On Monday, May 3, at 3 \textsc{pm}, three lorry loads of Assault Guards arrived at the Telefonica under the personal command of the Commissioner of Public Order, Salas, a \PSUC\ member.\endnote{%
%
The knotty problem of justifying the armed seizure of the Telefonica was ``solved,'' in the Stalinist press, by giving at least four different explanations:

\begin{enumerate}
  \item ``Salas sent the armed republican police to disarm the employees there, most of them members of \CNT\ unions. For a considerable time the telephone service had been run in a way which was open to the gravest criticism, and it was imperative to the whole conduct of the war that the defects of the service should be remedied.'' (London \emph{Daily Worker}, May 11)
  
  \item The police ``occupied the central telephone exchange. In so doing the police by no means intended to interfere with the rights of the workers as guaranteed by law (as alleged subsequently by the Trotskyist provocateurs). What the police wanted was to put all telephone connections under the immediate supervision of the Government.'' (\emph{Inprecorr}, May 22) What was ``guaranteed by law,'' however, was the \emph{workers’ control} sanctioned by the collectivization decree of October 24, 1936!
  
  \item A week later, a new story: ``Comrade Salas went to the Telefonica which on the previous night had been occupied by 50 members of the \POUM\ and various uncontrollable elements. The guard forced its way into the building and turned the occupants out of it. The affair was soon settled. Surprised by the rapid move on the part of the Government the 50 people left the building and the Telephone Exchange was again (!!) in the hands of the Government.'' (\emph{Inprecorr}, May 29)
  
  \item In the final version, issued by the Catalan section of the Comintern as Salas’ own story: ``In the first place there was no occupation of the Telefonica, nor was there any question of occupation of the Telefonica. I received a signed order from Ayguade, Minister for Public Order, that a Government delegate was to be installed and that I was to be responsible for seeing that he was so installed. Accordingly, I, with Captain Menendez, and a personal escort of four men, entered the Telephone Building. I explained my business and said that I wished to speak with some responsible member of the Committee. We were told that there was no one in the building. However, we waited downstairs while they went to look. Two minutes later some individuals started shooting at us from the stairs. None of us was hit. Immediately I phoned for the guards to come, not to occupy the building in which we were already but to cordon off the building and prevent anyone from entering\dots\ I and Eroles (Anarchist police functionary) went up to the top of the building, where they had established themselves with a machinegun, hand grenades and rifles. We went up together without escort, and without arms. At the top I explained the purpose of my visit. They came down. The delegate was installed according to orders. The forces were withdrawn. There were no casualties and arrests.''
  
  The \CNT\ account brands this story a lie: Salas began by disarming the guards and forcing the telephone workers to put their hands up; the guards on the upper floors withdrew only the next day as part of a general agreement for both sides withdrawing---which the Government promptly violated. The four different Stalinist versions testify to the difficulty of covering up the simple truth: they wanted to end workers’ control of the Telefonica and they did.
\end{enumerate}

}

Surprised, the guards on the lower floors were disarmed. But halfway up, a machine gun barred further occupation. Salas sent for additional Guards. Anarchist leaders pleaded with him to withdraw from the building. He refused. The news spread like wildfire to the factories and workers’ suburbs.

Within two hours, at 5 \textsc{pm}, the workers were pouring into the local centres of the \CNT-\FAI\ and \POUM, arming and building barricades. From the dungeons of the Rivera dictatorship until today, the \CNT-\FAI\ have always had their local defence committees, with a tradition of local initiative. So far as there was leadership in the coming week, these defence committees provided it. There was almost no firing the first night, for the workers were overwhelmingly stronger than the government forces. In the workers’ suburbs, many of the government police, with no stomach for the struggle, peacefully surrendered their arms. Lois Orr, an eye-witness, wrote:

\begin{quotation}
  By the next morning (Tuesday, May 4), the armed workers dominated the greatest part of Barcelona. The entire port, and with it Montjuich fortress, which commands the port and city with its cannon, was held by the anarchists; all the suburbs of the city were in their hands; and the government forces, except for a few isolated barricades, were completely outnumbered and were concentrated in the centre of the city, the bourgeois area, where they could easily be closed in on from all sides as the rebels were on July 19, 1936.
\end{quotation}

\CNT, \POUM, and other accounts substantiate this fact.

In Lerida, the civil guards surrendered their arms to the workers Monday night, as also in Hostafranchs. \PSUC\ and Estat Catala headquarters in Tarragona and Gerona were seized by \POUM\ and \CNT\ militants as a ``preventive measure.'' These overt steps were but the beginning of what could be done, for the masses of Catalonia were ranged overwhelmingly under the banner of the \CNT. The formal seizure of Barcelona, the constitution of a revolutionary government, would have, overnight, led to working-class power. That this would have been the outcome is not seriously contested by the \CNT\ leaders nor by the \POUM.\endnote{Even the \textsc{ilp} leader, Fenner Brockway, always to the right of the \POUM, in this case concedes that ``for two days the workers were on top. Bold and united action by the CNT leadership could have overthrown the Government.''}

That is why the left wingers in the \CNT\ and \POUM\ ranks, sections of the Libertarian Youth, the Friends of Durruti and the Bolshevik-Leninists called for a seizure of power by the workers through the development of democratic organs of defence (soviets). On May 4, the Bolshevik-Leninists issued the following leaflet, distributed on the barricades:

\medskip

\begin{oframed}
\begin{quotation}
  \noindent
  {\bfseries\sffamily\normalsize\centering Long Live the Revolutionary Offensive \par}
  
  \bigskip
  
  \noindent
  No compromise. Disarmament of the National Republican Guard and the reactionary Assault Guards. This is the decisive moment. Next time it will be too late. General strike in all the industries excepting those connected with the prosecution of the war, until the resignation of the reactionary government. Only proletarian power can assure military victory.

  \begin{itemize}
  	\item Complete arming of the working class.
  	\item Long live unity of action of \CNT-\FAI-\POUM.
  	\item Long live the revolutionary front of the proletariat.
  	\item Committees of revolutionary defence in the shops, factories, districts.
  \end{itemize}
	
  \begin{flushright}
    Bolshevik-Leninist section of Spain \\
    (for the Fourth International)
  \end{flushright}
\end{quotation}
\end{oframed}

\medskip

The leaflets of the Friends of Durruti, calling for ``a revolutionary Junta, complete disarmament of the Assault Guards and the National Republican Guards,'' hailing the \POUM\ for joining the workers on the barricades, estimated the situation in conceptions identical with those of the Bolshevik-Leninists. Still adhering to the discipline of their organizations, and issuing no independent propaganda, the \POUM\ Left, the \CNT\ Left and the Libertarian Youth agreed on perspective with the Bolshevik-Leninists.

They were undoubtedly correct. No apologist for the \CNT\ and \POUM\ leaders has adduced any argument against the seizure of power which stands up under analysis. None of them dares deny that the workers could easily have seized power in Catalonia. They adduce three main arguments to defend the capitulation: that the revolution would have been isolated, limited to Catalonia, and defeated there from the outside; that the fascists would have been able at this juncture to break through and win; that England and France would have crushed the revolution by direct intervention. Let us closely examine these arguments.

\vspace{-0.5 \baselineskip}

\subsection*{Isolation of the Revolution}

\vspace{-0.25 \baselineskip}

The most plausible, most radical, form given to this argument is that based on an analogy with the ``armed demonstration'' of July 1917 in Petrograd. ``Even the Bolsheviks in July 1917 did not decide to seize power and limited themselves to the defensive, leading the masses out of the line of fire with as few victims as possible.'' Ironically enough, the \POUM, \textsc{ilp}, Pivertists, and other apologists who use this argument are precisely those who have been incessantly reminding ``sectarian Trotskyists'' that ``Spain is not Russia,'' and that, therefore, the Bolshevik policy is not applicable.

The Trotskyist, i.e., Bolshevik, analysis of the Spanish Revolution, however, has always based itself on the concrete conditions of Spain. In 1931, we warned that the rapid rhythm of the developments of Russia in 1917 would not be duplicated in Spain. On the contrary, we then used the analogy of the great French Revolution which, beginning in 1789, passed through a series of stages before attaining its culmination in 1793. Just because we Trotskyists do not schematize historic events, we cannot take seriously the analogy with July 1917.\endnote{Leon Trotsky, \emph{The Revolution in Spain}, April 1931; \emph{The Spanish Revolution in Danger}, 1931, Pioneer Publishers, New York.}

The armed demonstration broke out in Petrograd only four months after the February revolution, only three months after Lenin’s April Theses had given a revolutionary direction to the Bolshevik party:

\begin{quotation}
  The overwhelming mass of the population of the gigantic country was only just beginning to emerge from the illusions of February.
  
  At the front was an army of twelve million men who were only then being touched by the first rumours about the Bolsheviks. In these conditions the isolated insurrection of the Petrograd proletariat would have led inevitably to their being crushed. It was necessary to gain time. It was these circumstances that determined the tactic of the Bolsheviks.
\end{quotation}

In Spain, however, May 1937, came after six full years of revolution in which the masses of the whole country had amassed a gigantic experience. The democratic illusions of 1931 had been burned out. We can cite testimony from \CNT, \POUM, socialist leaders that the refurbished democratic illusions of the People’s Front never caught hold of the masses---they voted in February 1936 not for the People’s Front but against Gil Robles and for the release of the political prisoners. Again and again the masses had shown that they were ready to go through to the end: the numerous anarchist-led armed struggles, the land seizures during six years, the October 1934 revolt, the Asturian Commune, the seizure of the factories and land after July 19! The analogy with Petrograd of July 1917 is childish.

Twelve million Russian soldiers, scarcely touched by Bolshevik propaganda, were available to be used against Petersburg in 1917. But in Spain more than a third of the armed forces carried \CNT\ membership cards; nearly another third \UGT\ cards, most of them left socialists or under their influence. Even grant that the revolution would not immediately spread to Madrid and Valencia. But this is entirely different from asserting that the Valencia government would have found troops with which to smash the Workers’ Republic of Catalonia!

Immediately after the May events, the \UGT\ masses showed their determined hostility to repressive measures against the Catalonian proletariat. That was one reason why Caballero had to leave the government. All the more could they not have been used against a victorious workers’ republic. Not even the Stalinist ranks would have provided a mass army for that purpose: it is one thing to get backward workers and peasants to limit their struggle to one for a democratic republic; it is something entirely different to get them to crush a workers’ republic. Any attempt by the bourgeois-Stalinist bloc to gather a proletarian force would have simply precipitated the extension of the workers’ state to all Loyalist Spain.

We can assert more than this: that the example of Catalonia would have been followed elsewhere immediately. Proof? The Stalinist-bourgeois bloc, while seeking to consolidate the bourgeois republic nevertheless was compelled by the revolutionary atmosphere to raise the slogan: ``Let us finish Franco first and make the revolution afterward.''

It was a clever slogan, well designed to keep the masses in check. But the very fact that the counter-revolution needed this slogan demonstrates that it based its hopes for victory over the revolution, \emph{not} on the agreement of the masses but on the masses’ \emph{embittered toleration}. Gritting their teeth, the masses were saying, ``we must wait until we finish off Franco, then we shall finish off the bourgeoisie and their lackeys.'' This feeling, undoubtedly widespread, would have disappeared in the face of the example of Catalonia. That example would have ended the feeling---``we must wait.''

Nor would the example of Catalonia have affected only Loyalist Spain. For a workers’ Spain would have embarked on a revolutionary war against fascism which would have disintegrated the ranks of Franco, more by political weapons than by military ones. All the political weapons against fascism which the People’s Front had refused to permit to be used, which could only be used by a workers’ republic, would now confront Franco. Trotsky wrote a few days after July 19:

\begin{quotation}
  A civil war is waged, as everybody knows, not only with military but also with political weapons. From a purely military point of view, the Spanish Revolution is much weaker than its enemy. its strength lies in its ability to arouse the great masses to action. It can even take the army [of Franco] away from its reactionary officers. To accomplish this it is only necessary seriously and courageously to advance the programme of the socialist revolution.

  It is necessary to proclaim that, from now on, the land, the factories, and shops will pass from the capitalists into the hands of the people. It is necessary to move at once toward the realization of this programme in those provinces where the workers are in power. The fascist army could not resist the influence of such a programme: the soldiers would tie their officers hand and foot and hand them over to the nearest headquarters of the workers’ militia. But the bourgeois ministers cannot accept such a programme. Curbing the social revolution, they compel the workers and peasants to spill ten times as much of their own blood in the civil war.
\end{quotation}

Trotsky’s prediction proved all too true. Fearing the revolution more than Franco, the People’s Front government conducted no propaganda aimed at the peasants in Franco’s forces and behind his lines. The government absolutely refused to promise the land to these peasants, and that promise would have had no effect unless the government had actually decreed giving the land to the peasant committees in its own regions, from which, by a thousand roads, the news would have spread to the peasants in the rest of Spain.
\noclub

Fearing the revolution more than Franco, the government had rejected all proposals (including those of Abdel-Krim and other Moors) to incite revolution in Morocco under a declaration of independence for Morocco. Fearing the revolution more than Franco, the government appealed to the international proletariat to get ``their'' governments to help Spain---but never appealed to the international proletariat to help Spain in spite of and against their governments.

We are not doctrinaires. We do not declare the revolution every day. We judge from our concrete analysis of the conditions in Spain in May 1937: Had the workers’ republic been established in Catalonia, it would not have been isolated or crushed. It would have been quickly extended to the rest of Spain.

\subsection*{``The Fascists Would Have Broken Through''}

The second apology for not taking power in Catalonia overlaps the first to the extent that it implicitly denies the effect of the taking of power on Franco’s forces.\endnote{A well-known anarchist leader said to me: ``You Trotskyists are worse utopians than we ever were. Morocco is in Franco’s hands ruled by him with an iron hand. Our declaration of the independence of Morocco would have no effect.''

I reminded him that Lincoln’s Declaration of Emancipation of the slaves was issued while the Confederacy still held all the South. Marxists, at least, should recall, that Marx and Engels gave this political act enormous weight in the defeat of the South.

Another anarchist said: ``Our peasants have already seized much land, yet it has no effect on the peasants under Franco.''

Under questioning, however, he admitted that the peasants feared that the government would try to recover the land after the war. In Russia, too, by November 1917 the peasants seized much land. They tilled it, however, sullenly and fearfully. The Soviet decree nationalizing the land transformed the psychology of the peasants and made them overwhelmingly partisans of the Soviet regime.}

Admitting that a proletarian revolution in May would have extended itself throughout Loyalist Spain, the \CNT\ leaders argue:

\begin{quotation}
  It is obvious that, had we so desired, the defence movement could have been transformed into a purely libertarian movement. This is all very well but... the fascists would have, without doubt, taken advantage of these circumstances to break all lines of resistance.\endnote{Speech in Paris, \emph{Spain and the World} (Anarchist) July 2, 1937.} (Garcia Oliver)
\end{quotation}

Though ostensibly dealing with the immediate situation in May in Catalonia, this line of argument is, in actuality, much more fundamental: \emph{it is an argument against the working class taking power during the course of the civil war.}

That was also the line of the \POUM. The Central Committee held that, in the event the government refused to sign its own death warrant by convoking a Constituent Assembly (congress of soldiers’, peasants’ and union delegates), it would be wrong to wrest the power forcibly from the government.

\begin{quotation}
  It believed that the workers would in time protest against the counter-revolution which the government was carrying through and that the demand for such a Constituent Assembly would become so strong that the government would be compelled to submit. It held that an insurrection would be wrong and inadvisable until after the fascists were defeated, and there was a difference of opinion in its ranks whether even then an insurrection would be necessary.\endnote{Fenner Brockway, Secretary of the Independent Labour Party, \emph{The Truth About Barcelona}, London 1937.}
\end{quotation}

In other words, the \CNT\ and \POUM\ called for socialism through the government. But if the government would not yield, then we must wait until after the war at least. In practice, this came down to covert adaptation to the bourgeois-Stalinist slogan---``Let us finish Franco first and make the revolution afterward.''

The \POUM-\CNT\ tactic of waiting until Franco was finished off meant, concretely, the doom of the revolution. For, as we have already pointed out, the bourgeois-Stalinist slogan of ``wait'' was designed to check the masses until the bourgeois state was supreme. Precisely for this reason, the bourgeois-Stalinist bloc and its Anglo-French allies had no intention of finishing off Franco or (more likely) making an armistice with him, until the counter-revolution had securely consolidated its power in Loyalist Spain.

We have commented on the failure of the People’s Front and its government to conduct revolutionary propaganda to disintegrate Franco’s forces. But in the field of purely military struggle, too, the government failed to fight Franco conclusively. More accurately, there is no wall between political and military tasks in civil war. Fearing the revolution more than Franco, the government was massing huge forces of picked soldiers and police in the cities, thereby withdrawing men and arms needed at the front. Fearing the revolution more than Franco, the government was pursuing a dilatory war strategy which could provide no decisive conclusion, while the counter-revolution was carried through. Fearing the revolution more than Franco, the government was subordinating the Asturian and Basque workers to the command of the treacherous Basque bourgeoisie who were soon to surrender the Northern front. Fearing the revolution more than Franco, the government was directly sabotaging the Aragon and Levante fronts held by the \CNT. Fearing the revolution more than Franco, the government was giving fascist agents (Asensio, Villalba, etc., etc.) the opportunity to betray Loyalist fortresses to Franco (Badajoz, Irun, Malaga).\footnote{The military policy of the government is analysed in detail in Chapters {\AlegreyaLF 15 and 16.}}

The counter-revolution dealt terrible blows to the morale of the anti-fascist troops. ``Why should we die fighting Franco when our comrades are shot by the government?'' This mood, so dangerous to the struggle against fascism, was prevalent after the May Days and was hard to fight.

In all these ways, therefore, the government policy was making easier the military inroads of Franco. The establishment of a workers’ republic would have put an end to all this treachery, sabotage, disruption of morale. Wielding the instrument of state planning, the Workers’ Republic would utilize as no capitalist regime could the full material and moral resources of Loyalist Spain.

Far from enabling the fascists to break through, only workers’ power could lead to the victory over Franco.

\subsection*{The Menace of Intervention}

The \CNT\ darkly referred to English and French warships appearing in the harbour on May 3, to plans for landing Anglo-French troops. 

\begin{quotation}
  In the case of a triumph of libertarian communism, it would have been crushed some time later by the intervention of capitalistic and democratic powers. (Garcia Oliver)
  \noclub
\end{quotation}

\CNT\ references to specific warships, to a specific plot, deliberately obscured the fundamental character of the issue: \emph{every social revolution must face the danger of capitalist intervention.} The Russian Revolution had to survive both capitalist-financed civil war and direct imperialist intervention. The Hungarian Revolution was crushed by intervention as well as by its own mistakes. When, however, the German and Austrian Social Democrats justified stabilization of their bourgeois republics because the Allied Powers would intervene against socialist states, revolutionary socialists and communists the world over---anarchists denounced the Kautskys and Bauers as betrayers, and rightly so.

The Austrian and German proletariat, the revolutionists said then, must take account of the possibility of defeat at the hands of Anglo-French intervention because revolutions always face that danger, and to wait for that hypothetical moment at which the Allies would be too preoccupied to interfere, meant to lose the conjuncture favourable to revolution. But the social democrats prevailed\dots\ and ended up in the concentration camps of Hitler and Schuschnigg.

Neither \CNT\ nor \POUM\ circles dare argue that there was any specific conjunctural situation which made capitalist intervention in May 1937 more threatening than at another time. The apologists merely refer to the intervention danger without adducing specific analysis. We ask: was intervention more dangerous in May 1937 than, for example, it was at the time of the April 1931 revolution? The advantages, for the workers, were all with May 1937. In 1931 the European proletariat was prostrated at the bottom of the well of the world crisis. If the German workers were not yet betrayed to Hitler by their leaders---without a fight---the French proletariat was as dormant as if exhausted by a dictator. France, contiguous to Spain, is decisive for Spain. And in May 1937 the French proletariat was begining the second year of the upsurge which opened with the revolutionary strikes of June 1936.

It is inconceivable that the millions of socialist and communist workers of France, already chafing against neutrality, and kept in line by their leaders only with the greatest difficulty, would permit capitalist intervention in Spain, whether by the French or any other bourgeoisie. The transformation of the struggle in Spain, from one for the preservation of a bourgeois republic, to one for the social revolution, would fire the French, Belgian, and English proletariat even more than had the Russian Revolution---for this time the revolution would be at their own doors!

In the face of an alert proletariat, what would the bourgeoisie do? The French bourgeoisie would open its borders to Spain, not for intervention but for trade, enabling the new regime to secure supplies---or face immediately a revolution at home. The Spanish Workers’ Republic would not, like Caballero and Negrin, aid and abet ``non-intervention!''

England, irrevocably tied to the fate of France, would be held back from intervention both by the whole weight of France and by her own working class for whom the Iberian Revolution would open a new epoch. Portugal would face immediate revolution at home. Germany and Italy would, of course, seek to increase their aid to Franco. But Anglo-French policy must continue to be: neither a Socialist Spain nor a Hitler-Mussolini Spain. Hoping to whittle down both sides eventually, Anglo-French imperialism would be forced to keep Italo-German intervention within such bounds as to prevent the Rome-Berlin axis from dominating the Mediterranean.

We, least of all, need to be told that all capitalist powers have in common, and seek in common, to destroy any threat of social revolution. Nevertheless, it is clear that two factors which saved the Russian Revolution from destruction by intervention would operate in May 1937: In 1917 the world working class, inspired by the revolution, forced a halt to intervention, while the imperialists could not sink their differences sufficiently to unite on a single plan for crushing the workers’ republic. With the European proletariat on the rise again, the imperialists would seek to quench the Spanish fire at their peril.

Yes, above all, we invoke the aid of the workers of the world! You Stalinists for whom the masses are no longer anything but sacrificial carcasses which you offer at the altar of an alliance with the democratic imperialists; ydu bureaucrats whose contempt for the masses, on whose backs you stand, makes you forget that these same masses carried through the October Revolution and the victorious civil war, on the moral and material capital of which you are still living, and which shrinks under your incompetent mismanagement! We know you do not like to be reminded that in 1919–1922 the world working class saved the Soviet Union from the imperialists. The revolutionary capacities of the proletariat are a factor which you have come to hate and fear, for they threaten your privileges.

Not we, but the Stalinists, believe possible peaceful cohabitation of capitalist and workers’ states. Certainly, European capitalism could not indefinitely bear the existence of a Socialist Spain. But the specific conjuncture in May 1937 was sufficiently favorable to enable a workers’ Spain to establish its internal regime and to \emph{prepare to resist imperialism by spreading the revolution to France and Belgium and then wage revolutionary war against Germany and Italy, under conditions which would precipitate the revolution in the fascist countries.}

This is the \emph{only} perspective of the revolution in Europe in this period before the next war, whether the revolution begins with Spain or France. Whoever does not accept this perspective, rejects the socialist revolution.

Risks?

\begin{quotation}
  World history would indeed be very easy to make if the struggle were taken up only on condition of infallibly favorable chances,  
\end{quotation}
wrote Marx while the Paris Commune still lived.
Clear-eyed, he saw\dots
\nowidow

\begin{quotation}
  the unfavourable accident \dots\ in the presence of the Prussians in France and their position right before Paris. Of this the Parisian workers were well aware. But of this the bourgeois \emph{canaille} of Versailles were also well aware. Precisely for that reason they presented the Parisians with the alternative of taking up the fight or succumbing without a. struggle. In the latter case the demoralization of the working class would have been a far greater misfortune than the fall of any number of “leaders”. The struggle of the working class against the capitalist class and its state has entered upon a new phase, with the struggle in Paris. Whatever the immediate results may be, a new point of departure of world-historic importance has been gained. (``Letter to Kugelmann,'' April 17, 1871)
\end{quotation}

Berneri had been right. Crushed between the Franco-Prussians and Versailles-Valencia, the commune of Catalonia could have struck a flame to light up the world. And under conditions so incomparably more favourable than those of the Commune!

We have sought to analyse as seriously as possible the reasons given by the centrist leadership for not waging a struggle for power against the counter-revolution. Being centrists and not brazen reformists, they have sought to justify their capitulation by references to the ``special,'' ``specific'' situation in Spain in the month of May 1937 but without providing us with the precise details. Upon examination we have found that, as usual in all such alibis, the references to the specific are false and conceal a fundamental retreat from the revolutionary path. Not mistakes in fact but differences in principle, in world and class outlook, separate the revolutionists from both the reformist and centrist leaders.

On Tuesday morning, May 4, the armed workers on barricades throughout Barcelona felt again, as on July 19, masters of their world. As on July 19, the terrified bourgeois and petty-bourgeois elements hid in their homes. The \PSUC-led trade unionists remained passive. Only part of the police, the armed guards of the \PSUC, and the armed Estat Catala hooligans were on the government barricades. These barricades were limited to the centre of the city, surrounded by the armed workers. The state of affairs is indicated by Companys’ first radio address: a declaration that the Generalidad was not responsible for the provocation at the Telefonica. Every outer section of the city, directed by its local defence committees and aided by \POUM, \FAI, and Libertarian Youth groups, was solidly in the control of the workers. There was almost no firing Monday night, so complete was the workers’ control. All that remained to establish supremacy was co-ordination and joint action directed from the centre \dots\ At the centre, the Casa. \CNT, the leaders, forbade all action and ordered the workers to leave the barricades.\endnote{For critical accounts of the events of the next days, I am indebted to two American comrades, Lois and Charles Orr (the latter was editor of the \POUM’s English-language \emph{Spanish Revolution}) and to the long and documented report of the Spanish Bolshevik-Leninists, appearing in \emph{La Lutte Ouvrière}, June 10, 1937.}

It was not the organizing of the armed masses that interested the \CNT\ leaders. What occupied them was interminable negotiation with the government. This was a game which suited the government perfectly: to hold back the leaderless masses in the barricades by deluding them with hopes that a decent solution would be found. The meeting at the Generalidad Palace dragged on until six o’clock in the morning. The government forces thus got enough breathing space to fortify the government buildings and, as the fascists had done in July, occupy the cathedral towers.

At eleven Tuesday morning, the functionaries met, not to organize the defence, but to elect a new committee to negotiate with the government. Now Companys introduced a new wrinkle. Of course, we can come to an amicable settlement; we are all anti-fascists, etc., etc., said Companys and Premier Tarradellas---but we cannot carry on negotiations so long as the streets are not cleared of armed men. Whereupon the Regional Committee of the \CNT\ spent Tuesday before the microphones calling the workers away from the barricades:

\begin{quotation}
  We appeal to all of you to put down your arms. Think of our great goal, common to all \dots\ Above all else, unity! Put down your arms. Only one slogan: We must work to beat fascism!
\end{quotation}

\emph{Solidaridad Obrera} had the effrontery to appear with the story of Monday’s attack on the Telefonica on page~8---not to alarm the militiamen at the front to whom went hundreds of thousands of copies---with no mention of the barricades erected, and no directives except ``keep calm.'' At five o’clock delegations from the National Committees of the \UGT\ and \CNT\ arrived from Valencia and jointly issued an appeal to the ``people'' to lay down their arms. Vasquez, \CNT\ National Secretary, joined Companys in the radio appeal. The night was spent in new negotiations---the government was always ready to make agreements involving the workers leaving the barricades!---out of which came an agreement for a provisional cabinet of four: one each from the \CNT, \PSUC, Peasant Union and Esquerra. The negotiations were punctuated with calls for authoritative \CNT\ leaders to go to points where the workers were on the offensive, as at Collblanc the workers had to be persuaded from carrying out occupation of the barracks. Meanwhile other calls were coming in---from the Leather Workers’ Headquarters, the Medical Union, the local centre of the Libertarian Youth---asking the Regional Committee to send help, the police were attacking\dots

Wednesday: neither the numerous radio appeals, the joint appeal of the \UGT-\CNT, nor the establishment of a new cabinet, had budged the armed workers from the barricades. On the barricades, anarchist workers tore up \emph{Solidaridad Obrera} and shook their fists and guns at the radios as Montseny---when Vasquez and Garcia Oliver had failed, she had been hurriedly called from Valencia---exhorted the barricades to disperse. The local defence committees reported to Casa \CNT: the workers will not leave without conditions. Very well, we give them conditions. The \CNT\ radioed the proposals it was making to the government: hostilities to cease, every party to keep its positions, the police and civilians fighting on the side of the \CNT\ (i.e., non-members) to retire altogether, the responsible committees to be informed at once if the pact is broken anywhere, solitary shots not to be answered, the defenders of union quarters to remain passive and await further information.

The government soon announced its agreement with the \CNT\ proposals, and why not? The government’s sole objective was to end the fighting of the masses, the better to break their resistance for all time. Furthermore, the ``agreement'' pledged the government to nothing. The control of the Telefonica, disarming of the masses, were---not accidentally---unmentioned. The agreement was followed during the night by orders from the local \CNT\ and \UGT\ (the latter Stalinist-controlled, remember) to return to work.

\begin{quotation}
  The anti-fascist organizations and parties in session at the Palace of the Generalidad have solved the conflict that has created this abnormal situation,
\end{quotation}
said the joint manifesto.

\begin{quotation}
  \noindent
  These events have taught us that from now on we shall have to establish relations of cordiality and comradeship, the lack of which we all regretted deeply during the last few days.
\end{quotation}

Nevertheless, as Souchy admits, the barricades remained fully manned Wednesday night.

But on Thursday morning, the \POUM\ ordered its members to leave the barricades, many of them still under fire. On Tuesday, the manifesto of the Friends of Durruti, hitherto cool to the \POUM, had hailed its joining the barricades as a demonstration that it was a ``revolutionary force.'' Tuesday’s \emph{La Batalla} had remained within the limits of the theory that there should be no insurrectionary overthrow of the government during the civil war but had called for defence of the barricades, the dismissal of Salas and Ayguade, withdrawal of the decrees dissolving the worker-patrols. Limited as this programme was, it contrasted so with the \CNT\ Regional Committee’s appeal to desert the barricades that the prestige of the \POUM\ soared among the anarchist masses. The \POUM\ had an unparalleled opportunity to come to the head of the movement.

Instead, the \POUM\ leadership, once again, put its fate in the hands of the \CNT\ leadership. \emph{Not} public proposals to the \CNT\ for joint action made before the masses, proposals which would give the inchoate rebellion a focus of specific steps to demand of their leaders---in a whole year the \POUM\ had, fawningly deferential to the \CNT\ leaders, not made a single united front proposal of this specific character---but a behind-the-scenes conference with the \CNT\ Regional Committee. Whatever the \POUM\ proposals were, they were rejected. \emph{You don’t agree? Then we shall say nothing about them.}

And the next morning, May 5, \emph{La Batalla} had not a word to say about the \POUM’s proposals to the \CNT, about the cowardly behaviour of the \CNT\ leaders, their refusal to organize the defence, etc.\endnote{The English language bulletin of the \POUM, \emph{Spanish Revolution} (May 19, 1937), says: ``Caught up in the reins of the government (the \CNT) tried to straddle the fence with a `union' of the opposing forces \dots\ The attitude of the \CNT\ did not fail to bring forth resistance and protests. The `Friends of Durruti' group brought the unanimous desire of the \CNT\ masses to the surface but it was not able to take the lead\dots\ The workers who were deeply wounded by the capitulation of their trade union federation, are now looking for a new lead in other directions. The \POUM\ should provide it for them.''

These radical words were for export purposes only. Nothing like them appeared in the regular press of the \POUM. In general \emph{Spanish Revolution} has given English readers, who could not follow the \POUM’s Spanish press, a distorted picture of the \POUM’s conduct; it has been a ``left face.'' This is said without any intention of reflecting on the revolutionary integrity of Comrade Charles Orr, its editor, who can scarcely be held responsible for the disparity between the English bulletin and the voluminous Spanish press of the \POUM.} Instead, ``the Barcelona proletariat has won a partial battle against the counter-revolution.''

And, twenty-four hours later, ``the counter-revolutionary provocation having been repulsed, it is necessary to leave the streets. Workers, return to work.'' (\emph{La Batalla}, May 6).

The masses had demanded victory over the counter-revolution. The \CNT\ bureaucrats had refused to fight. The centrists of the \POUM\ thus bridged the gap between masses and bureaucrats---by assuring them that the victory had already been achieved!

The Friends of Durruti had forged to the front on Wednesday, calling upon the \CNT\ workers to repudiate the desertion orders of Casa \CNT\ and continue the struggle for workers’ power. It had warmly welcomed the collaboration of the \POUM. The masses were still on the barricades. The \POUM, numbering at least thirty thousand workers in Catalonia, could tip the quivering scales either way. Its leadership tipped the scales for capitulation.

One more terrible blow against the embattled workers: The Regional Committee of the \CNT\ gave to the entire press---Stalinist and bourgeois included---a denunciation of the Friends of Durruti as \emph{agents provocateurs}; it was, of course, prominently published everywhere on Thursday morning. The \POUM\ press did not defend the left-wing anarchists against this foul slander.

\dinkus

Thursday was replete with instances of the ``victory'' in the name of which the \POUM\ called the workers to leave the barricades.

In the morning the shattered body of Camillo Berneri was found where it had been tossed by the \PSUC\ guards who had seized the frail man in his home the night before. Berneri, spiritual leader of Italian anarchism since the death of Malatesta, leader of the Ancona revolt of 1914, escaped from Mussolini’s clutches, had fought the reformists (including the \CNT\ leaders) in his influential organ, \emph{Guerra di Classe}. He had described the Stalinist policy in four words: ``It smells of Noske.'' In ringing words he had defied Moscow:

\begin{quotation}
  Crushed between the Prussians and Versailles, the Commune of Paris initiated a fire that lit up the world. Let the General Godeds of Moscow remember this.  
\end{quotation}
He had declared to the masses of the \CNT:

\begin{quotation}
  The dilemma “war or revolution” has no longer any meaning. The only dilemma is: either victory over Franco, thanks to the revolutionary war, or defeat.
\end{quotation}

How terribly true had been his identification of Noske and the Stalinists! As Noske, the Social Democrat, had Rosa Luxemburg and Karl Liebknecht kidnapped and murdered, so the Stalinist-democrats had assassinated Camillo Berneri.
 
Honour our comrade, Camillo Berneri. Let us remember him with the love we bear our Karl and Rosa. As I write, comrades, I cannot help weeping, weeping for Camillo Berneri. The list of our martyrs is as long as the life of the working class. Fortunate were those among them who fell fighting the open class enemy, fell in the midst of battle with their comrades beside them. Most terrible of all is it to die alone at the stiletto-point of those who call themselves socialists or communists, as Karl and Rosa died, as our comrades are dying in the execution chambers of Siberian exile. A special anguish was that of Camillo Berneri. He died at the hands of ``Marxists-Leninists-Stalinists,'' while his closest friends, Montseny, Garcia Oliver, Peiro, Vasquez, were handing over the Barcelona proletariat to his executioners.
 
Thursday, May 6, 1937.
 
Let us remember that day.

Government and anarchist leaders had gone to Lerida on Wednesday to stop a picked force of 500 \POUM\ and \CNT\ troops speeding from Huesca, with light artillery. Valencian and Generalidad representatives had promised that if the workers’ troops did not advance, the government would not try to bring any more of its troops into Barcelona. Upon this promise and the urging of the anarchist leaders, the workers’ troops had stopped.

On Thursday, however, came telephone calls from the \CNT\ militants in towns along the road from Valencia to Barcelona: 5,000 Assault Guards are on their way.

\emph{Shall we stop them?} ask the \CNT\ workers. The \CNT\ leaders ordered the Guards to be let through, sent no word to the workers’ troops waiting in Lerida, and suppressed the news that the Guards were on the way.

Thursday at 3 o’clock, Casa \CNT\ ordered its guards to vacate the Telefonica. The government and the \CNT\ had made an agreement: both sides should withdraw their armed forces. As soon as the \CNT\ guards had left, the police occupied the entire building and brought in government supporters to take over the technical work from the \CNT\ workers. You have broken your promise, the \CNT\ complained to the government. The Generalidad replied: the fait accompli cannot be recalled.

``Had the workers in the outlying districts been informed immediately of this development,'' admits the CNT spokesman, Souchy, ``they would surely have insisted upon taking firmer measures and turned to the attack.'' So the ultra-democratic, anarchist leaders of the CNT simply suppressed the news!
\noclub

Under the orders of Casa \CNT, the telephone workers had serviced all calls during the fighting: revolutionary and counter-revolutionary. Once the government took over, however, the \FAI\ and \CNT\ locals were cut off from the centre.

On the streets through which the workers had to come and go in returning to work as the \CNT-\UGT\ had instructed, police and \PSUC\ guards were searching the passersby, tearing up \CNT\ cards, arresting \CNT\ militants.

At four o’clock, the main railroad station of Barcelona, in the hands of the \CNT\ since July 19, was attacked by \PSUC\ and Assault Guards, with machine guns and hand-grenades. The small \CNT\ force guarding it tried to telephone for help\dots

At four o’clock, General Pozas presented himself to the Ministry of Defence of Catalonia (a \CNT\ ministry) and politely informed the comrades-ministers that the post of the Catalan Ministry of Defence had ceased to exist, that the Catalan armies were now the Fourth Brigade of the Spanish Army with Pozas as chief. The Valencian cabinet had made this decision by authority of the military decrees for a unified command signed by the \CNT\ ministers. The \CNT, of course, surrendered control to Pozas.

Terrible news from Tarragona. Wednesday morning a large police force had appeared and seized the telephone exchange. The \CNT\ had thereupon asked for the inevitable conference. While negotiations went on the Republicans and Stalinists were arming; the next day they assaulted the Libertarian Youth headquarters. Whereupon the \CNT\ asked for another conference at which they were informed that the Generalidad had sent explicit instructions to destroy the anarchist organizations if they did not surrender their arms. (Let us remember that these instructions came from a government in which sat anarchist ministers.) The \CNT\ representatives agreed to surrender their arms, if the government would set free all arrested, replace the police and \PSUC\ guards with regular army men, and guarantee immunity of attack of \CNT\ members and their headquarters.

Captain Barbeta, the government delegate, agreed, of course. The \CNT\ laid down its arms and during the night the Assault Guards occupied the \CNT\ buildings and killed a score of anarchists, among them Pedro Rua, the Uruguayan writer, come to fight against fascism and risen to commandant of the militias. Casa \CNT\ noted that this was ``breaking the word of honour given the evening before by the authorities.'' Not a word of this was meanwhile transmitted to the Barcelona masses, though Casa \CNT-\FAI\ knew hourly of the developments.\endnote{They only released the story on May 15–16, \emph{Solidaridad Obrera}.}

Thursday, 6 \textsc{pm}: Word arrived at Casa \CNT: the first detachment from Valencia, 1,500 Assault Guards, had arrived at Tortosa on the way to Barcelona. Casa \CNT\ had sent word ahead not to oppose them, everything was arranged, etc. The Assault Guards occupied all \CNT-\FAI-Libertarian Youth buildings of Tortosa, arresting all found, taking some, handcuffed, along to the Barcelona jails.

The masses knew nothing of the events at Tarragona, Tortosa, the Telefonica, Pozas, the coming of the Valencian Guards. But the attacks on workers in the streets, on the railroad station, the renewed firing at the barricades, spurred many who had left to return to the barricades.

In response to these cataclysmic events of Thursday, the Casa \CNT\ ``sent a new delegation to the government to find out what they intended doing'' (Souchy) but without waiting to learn, issued a new, calming manifesto. While the barricades still resounded, Casa \CNT\ declared,

\begin{quotation}
  Now that we have returned to normal, and those responsible for the outbreak have been dismissed from public office, when all the workers have returned to their jobs, and Barcelona is once more calm \dots
  
  The \CNT\ and \FAI\ continue to collaborate loyally as in the past with all political and trade union sectors of the anti-fascist front. The best proof of this is that the \CNT\ continues to collaborate with the central government, the government of the Generalidad and all the municipalities\dots
  
  The press of the \CNT\ appealed for calm and called upon the population to return to work. The news issued by radio to the unions and the defence committees was nothing but appeals for calm.
  
  A further proof that the \CNT\ did not want to break and did not break the anti-fascist front is that when the new government of the Generalidad was formed, on the 5th of May, the representatives of the \CNT\ of Catalonia offered it every facility, and the secretary of the \CNT\ formed part of the government\dots
  
  The members of the \CNT\ who controlled the Defence Council (Ministry) of the Generalidad, gave orders to all their forces not to intervene on either side in the conflict. And they also saw to it that their orders were carried out.
  
  The Defence Committee of the \CNT\ also gave orders to every district of Barcelona that no one should come from there to the centre to answer the provocations. These orders, too, were carried out because no one actually did come to the centre to answer the provocations\dots
  
  Many were the traps laid for the \CNT\ up to the very end but the \CNT\ remained firm in its position and did not allow itself to be provoked\dots
\end{quotation}

Thursday evening: The \PSUC\ and Assault Guards continued their raids, arrests, shootings. So\dots\ the Casa \CNT-\FAI\ sent a new delegation to the government with new proposals to cease hostilities: All groups to obligate themselves to remove their armed guards and patrols from the barricades; to release all prisoners; no reprisals.

News from Tarragona and Reus,

\begin{quotation}
  where members of the \PSUC\ and Estat Catala, taking advantage [!] of the presence of some Assault Guards passing through on their way to Barcelona, used their temporary advantage to disarm and kill the workers\dots
  
  The CNT tried to get a promise from the government in Valencia and Barcelona that the Assault Guards would not enter the city immediately [!], but should be detained outside the city limits until the situation had cleared up\dots
  
  They were somewhat sceptical regarding the assurances that the coming troops would be loyal to the workers. (Souchy)
\end{quotation}

But that scepticism (when did it arise?) had not been shared by the \CNT\ ministers in the Valencia and Catalonian cabinets who had voted for the central government to take over control of public order in Catalonia. The Ministry of Public Order of Catalonia had, therefore, ceased to exist on May 5.

The night of May 6–7: ``Again and again the anarchists offered to negotiate, eager to end the conflict.''

The government, of course, was always ready to negotiate while its forces broke the back of the working class under the cover provided by Casa \CNT.

The nearby anarchist workers had rallied to defend Tortosa and Tarragona. At four o’clock the Provincial Committee---the leadership of the \CNT\ in Catalonia outside Barcelona---informed Casa \CNT-\FAI\ that they were prepared to hold up the guards from Valencia.

\emph{No, you must not}, said Casa \CNT. At 5:15, the government and Casa CNT make another agreement: armistice, all to leave barricades, both parties to release their prisoners, the worker-patrols to resume their functions\dots

Again the Regional Committee radioed the workers:

\begin{quotation}
  Having reached an understanding\dots
  
  we wish to notify you\dots
  
  establishment of complete peace and calm\dots
  
  keep that calm and presence of mind\dots
\end{quotation}

Friday: Under orders from Casa \CNT-\FAI\ some workers began to tear down barricades. But the barricades of the Assault Guards, Estat Catala, \PSUC, remained intact. The Assault Guards systematically disarmed workers.

Again, as the workers saw the government forces continue the offensive, they returned to the barricades, against the will of both the \CNT\ and the \POUM. But disillusionment and discouragement set in: many of the anarchist workers had maintained faith in Casa \CNT-\FAI\ up to the last, others as their faith ebbed had looked for leadership to the \POUM\ workers until these were ordered off the barricades.

The Friends of Durruti and the Bolshevik-Leninists were able to bring the workers back to the barricades for Thursday and Friday night, but not strong enough, not sufficiently rooted in these masses, to organize them for a long struggle.

The promise to release prisoners was not kept; on the contrary mass arrests began. No reprisals was another promise; but the next weeks came brutal reprisals against towns and suburbs which had dared resist. The government, of course, retained control of the Telefonica---that was why they had begun the struggle. The control of the police was now in Valencia---soon to be turned over to the Stalinists!

The Ministry of Defence and the army of Catalonia had become the property of Valencia---soon to come under the control of Prieto. The worker-patrols were to be dissolved shortly, with the application of Ayguade’s public order decree. Catalan autonomy had ceased to exist as Valencia’s armed forces poured in.

Ayguade, ``dismissed'' said the \CNT, was in a week to go to Valencia to sit in the central government as the representative of the Generalidad\dots\ in which the \CNT\ continued to sit.

After the Assault Guards entered Barcelona, \emph{La Batalla} complained:

\begin{quotation}
  This is a provocation. By a demonstration of force they are attempting to convert our victory into a defeat.
\end{quotation}

And, whiningly:

\begin{quotation}
  It was the \POUM\ that counselled ceasing the struggle, abandoning the streets, returning to work; it was it---no one can doubt---that was one of those who most contributed to bring the situation back to normal.
\end{quotation}

The tameness of the \POUM ist lamb didn’t, however, save it from the wolf. Pitiful politicians, indeed, who cannot distinguish victory from defeat!

``We did not feel ourselves spiritually or physically strong enough to take the lead in organizing the masses for resistance,'' a member of the \POUM\ Central Executive had said to Charles Orr on Tuesday.

So\dots\ they had rationalized their impotence into a ``victory,'' to justify ending the struggle.

Suppose the \POUM\ had come to the fore and, in spite of the \CNT, had sought to lead the workers at least to a real armistice, i.e., with the workers remaining armed in the streets and factories ready to resist any further offensive. Suppose even this had not come about, that the \POUM\ and the workers would have been conquered by sheer force of arms.

\begin{quotation}
  In the worst case,
\end{quotation}
the \POUM\ opposition pointed out,

\begin{quotation}
  \noindent
  there could have been organized a central committee of defence, based on representation from the barricades. For this it would have been sufficient to hold first a meeting of delegates from each of the \POUM\ barricades and such others of the \CNT, to name a provisional central committee. During Tuesday afternoon the local \POUM\ committee was working along this line. But it met with no enthusiasm from the central leadership to carry this out.
\end{quotation}

At the least, such a central body directly rooted in the masses would have been able to organize resistance to the subsequent raids, arrests, suppression of the press, outlawry of the Friends of Durruti and the \POUM.

Certainly, the attempt to organize resistance would have resulted in no more victims than were produced by capitulation: 500 dead and 1,500 wounded, almost all after the \CNT\ began the retreat Tuesday afternoon; hundreds more killed and wounded during the ``mopping up'' of the following weeks; the ``cleansing'' of the \POUM\ and anarchist troops by sending them during the next weeks into the line of fire without protecting aviation and artillery; Nin, Mena, other \POUM\ leaders murdered, thousands and tens of thousands jailed in the ensuing period. Capitulation took at least as many victims as struggle and defeat would have taken.

The \POUM\ opposition---and it is not a Trotskyist opposition---were more than right when they said in their Bulletin of May 29:

\begin{quotation}
  This retreat, ordered without conditions, without obtaining the control of public order, without the guarantee of workers’ patrols, without practical organs of the workers’ [united] front, and without a satisfactory explanation to the working class, placing all the struggling elements---revolutionary and counter-revolutionary---in the same sack is one of the greatest capitulation and treason to the workers’ movement.
\end{quotation}

The iron logic of politics is inexorable. The wrong course carries its supporters to undreamed-of depths. Determined to continue the policy of collaboration with the bourgeois state, the anarchist leadership---it seems only yesterday that these men defied the monarchy to the death---were sacrificing the lives and future of their following in the most cowardly fashion.

Hanging on to the coat-tails of the \CNT, the \POUM\ leaders were chasing workers off barricades still under fire. They, least of all, would have believed themselves, a year ago, capable of falling so low\dots

Leaders who have betrayed the workers as these have are irrevocably lost to the revolutionary movement; they cannot turn back, admit their terrible complicity\dots But they are also pitiful, for on the morrow of their betrayal, the bourgeoisie, thus reinforced, will dispense with them too.

Let us remind the apologists for the \POUM\ of one other respect in which their analogy with Petersburg of July 1917 does not hold. The failure of the ``armed demonstration'' was followed by a savage hunt of the Bolsheviks: Trotsky was imprisoned, Lenin and Zinoviev went into hiding; the Bolshevik papers were suppressed. The cry went up: the Bolsheviks are German agents. Within four months, however, the Bolsheviks had carried through the October Revolution. I write six months after the May days, and the \POUM\ is still crushed, dead. 

The analogy does not hold on this point because the difference is: the Bolsheviks fearlessly placed themselves at the head of the July movement and thereby became flesh and blood of the masses, while the \POUM\ turned its back on the masses, and the masses, in turn, felt no urge to save the \POUM.