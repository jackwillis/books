\chapter{Biographical notes}

This book was first published in 1938 by Pioneer in the United States.
An expanded edition which contained Morrow’s much shorter \emph{Civil War in Spain} pamphlet was published in the United States by Path\-finder Press in 1974 with a new index and copies may still be available from them. Note the reference to \emph{Civil War in Spain} by Morrow in [Note~\ref{en:CivilWarReference}.]

The text of the following book is that of the second edition of 1963 published by New Park Publications together with the introduction written at that date by the late Tom Kemp. At the time, and for some years afterwards, the Socialist Labour League in Britain, who controlled New Park, was politically allied with the Socialist Workers Party in the United States, who controlled Pioneer/Pathfinder. The type of this edition was reset with English spellings etc. and the book was given its own index. Since it is now out of print, Index Books, the successor to New Park Publications, is consequently delighted that this important text is available, if only on the web, as they have no plans to reprint it.

It has been transcribed by Ted Crawford in 2003 [for the Marxists Internet Archive] and any errors and omissions are his responsibility. Notes have been placed at the end of the chapters and numbered rather than placed at the bottom of the page and the very occasional typographical errors removed. Any notes added are clearly stated to have been done by him and signed \ERC.

\begin{flushright}
	Ted Crawford \\
	2003
\end{flushright}

\chapter{Foreword}

\lettrineF{or the new} generations moving into political life, a study of the Spanish Civil War of 1936–39 contains valuable lessons. This book, hammered out in time with the dramatic events it describes, stands in direct line with the great writings of Marx and Engels on the Revolutions of 1848 and the Paris Commune and Trotsky’s \emph{History of the Russian Revolution}. No doubt later historical research has been able to uncover from the record a more detailed account of the facts, and the reports of participants have helped to throw light on the motives of many of those involved. The basic interpretation made by Morrow in the light of Marxism retains all its validity and offers an invaluable key to one of the decisive events of our epoch. In reprinting this book, which has become a rarity, a valuable service is being rendered to the working-class movement and to students of politics. Needless to say, while Morrow’s work has so well withstood the test of time there is not a single part of the voluminous literature produced by the Stalinists which could be reprinted today without courting ridicule.

The Spanish workers and peasants, in July 1936, made a revolution but they could not complete it or maintain the positions won. Their revolution was defeated and counter-revolution, in the shape of the bourgeois republic, by its triumph, prepared the way for the bloodier triumph of Franco. This defeat was one of a long series which issued from the betrayals of social democracy and Stalinism. It reflected, above all, the failure to build in time the necessary revolutionary leadership, a party of the Bolshevik type. This failure was not peculiarly Spanish but was part of an international crisis of proletarian leadership which called for the formation of the Fourth International\index{Fourth International} and is still with us today.
\nowidow

During these years the Spanish proletariat was in the vanguard of the international struggle of its class. It made a revolution in the teeth of the attempt by the fascist military junta to establish a dictatorship on behalf of the landed oligarchy, big business and the church, which saw in the crushing of the working-class movement the only way to secure their privileges. Before this determined threat the bourgeois state, wearing the trappings of the Popular Front, panicked and disintegrated. The army and police, upon which its existence depended, joined the rebellion or prepared to do so. The response of the workers, as well as of the peasants in many parts of the country, was to set up their own organs of power and to carry out a social revolution. In this way they sought, spontaneously, to give concrete form to the promise of the Popular Front government which they had elected in February 1936.

Morrow describes graphically, as others have since done in still greater detail, how this revolution was made. But he does more, he explains the significance of the dual power which thus came into being and defines the character of this revolution. According to the Communist Parties, applying the line of the Seventh World Congress of the Comintern\endnote{B.~Bolloten, \emph{The Grand Camouflage} (London 1961), p. 42.}\label{en:BollotenGrandCamouflageP42}\index{Communist International} it was a bourgeois democratic revolution from which should issue a national democratic state. True, the bourgeois democratic revolution in Spain had been delayed and was half completed, but it cannot be doubted that capitalism was firmly established and that the state was a bourgeois state. When the workers took over the factories and the peasants took over the land and established collectives, their armed militias conquered power by overcoming the rebel troops. ``Shorn of the repressive organs of the state,'' writes Bolloten{\indexBBolloten}, who is no Marxist, ``the government of Jos\'e Giral{\indexJGiral} \emph{possessed the nominal power, but not the power itself, for this was split into countless fragments.}''\endnotemark[\ref{en:BollotenGrandCamouflageP42}]

The failure of the workers and peasants to complete their victory can be explained by two main factors. First there was the influence of the anarchist leadership (\textsc{fai-cnt}) which responded to the revolutionary impulse of the masses with the pusillanimity of the petty-bourgeois and presented the world with the unique spectacle of anarchist ministers in a bourgeois government. These leaders were a powerful obstacle on the way to building the revolutionary party which the situation demanded; though hard experience pressed this lesson home on many anarchist workers and especially the youth, it was a lesson learned too late. The way was thus clear for counter-revolution. The spearhead of this counter-revolution was of the most insidious kind because it came decked out as Marxism and Communism. The Spanish Communist Party\indexPCE, at the beginning of 1936, was still a relatively small party. As a member of the Popular Front coalition it made itself, after July, the protagonist of the restoration of republican, i.e., bourgeois institutions, including a well-armed police and disciplined army to supersede the workers’ guards and militias. In this way it gained rapidly in strength; ``from the outset,'' writes Bolloten.\endnote{B. Bolloten, \emph{op. cit.}, p. 87, and whole of chapter 6.}

``The Communist Party appeared before the distraught middle classes not only as a defender of property, but as the champion of the Republic and the orderly processes of government''---which in the circumstances meant counter-revolution. It also became the most determined champion of the policy of win the war first which rallied to its side many former supporters of the other republican parties. As the Republic became increasingly dependent upon Russian military supplies so was the power of the Communist Party enhanced, or appeared to be.

In fact, it was not the Spanish Communists whose power grew but the direct power of the Soviet bureaucracy in Spain. As Jes\'us Hernandez, Communist Minister of Education\index{Ministries!Education} in the Negr\'in{\indexJNegrin} Government has subsequently revealed,\endnote{J.~Hernandez, \emph{La Grande Trahison} (Paris 1953), French translation of \emph{Yo fui un ministro de Stalin} (Mexico City 1953.) Hernandez{\indexJHernandez} was Communist Minister of Education in the government of Caballero, whose downfall he helped to bring about, then in that of Negrin.} the policy of the party was determined, often over the protests of its more honest leaders, by loyal agents of the policy of the Stalinised Comintern---Togliatti, Vidali{\indexVVidali} (alias Contreras), Orlov and others. This policy was not determined by the needs of the Spanish revolution but by the diplomatic requirements of the Kremlin which had no interest in revolution in the Iberian peninsula. Stalin settled for a limited committment in Spain, hoping that the ``democracies''---France and England---would intervene to stop the spread of fascism and ready to pull out as it became increasingly clear that this was not likely. To this end, then, not only was the Communist Party set on a counter-revolutionary course, but methods and organisation of the \textsc{nkvd} were transplanted to Spain. It was concerned primarily with tracking down and exterminating genuine revolutionaries, operating quite independently of the government, and even of the Spanish Communist Party. Its most spectacular success, following the provocation which led to the \emph{May Days} in Barcelona in 1937, was the hounding of the \textsc{poum} and the murder of Andreas Nin.{\indexANin} Hernandez has described how Nin was first tortured to the limits of human endurance and then, according to a scenario devised by Vidali, ``liberated'' by supposed agents of the Gestapo disguised as members of the International Brigade, leaving behind ``evidence'' indicating that he was a German spy.\endnote{Hernandez, \emph{op. cit.}, chapter 5.}

The energy with which this repression was carried on against left-wing militants indicated the deep concern of the Stalinists to destroy the Spanish revolution. But they could only play their counter-revolutionary role successfully by appearing themselves in revolutionary guise when occasion demanded; otherwise they could not have won mass support, including from many workers, and made an international impact. This was particularly so at the time of the battle for Madrid. As recent historians put it, ``The history of the defence of Madrid shows also that in certain circumstances the Communist Party{\indexPCE} is capable, not only of making an appeal to revolutionary traditions such as those of the Russian October Revolution or of the Red Army, but also of using really revolutionary methods, in a word, appearing, in the eyes of large masses of people, as an authentically revolutionary party. Many Spanish or foreign militants experienced in the defence of the capital a revolutionary epic in which the anti-fascist label was purely provisional. Against the mercenaries from Germany or Italy, they wished to be fighters in the international proletarian revolution. Many among them fought the revolution for the time being, with the conviction that it was nothing more than a tactical withdrawal of a provisional kind, and that at the end of the anti-fascist struggle the World Communist Revolution would be found.''\endnote{P. Broué and E. Témime, \emph{La révolution et la guerre d’Espagne} (Paris 1961), p. 213. This is the best of the recent books on the Spanish Revolution. Unfortunately it is not available in an English translation. [A translation was published in 1972 by Faber and Faber. ---\ERC]}

This cynical abuse of genuine revolutionary convictions was not least among the crimes of Stalinism in this period and it is necessary to recall that many of those who fought in Spain, or who backed the Republican Government, did so because they were deluded into believing that by so doing they supported the revolutionary cause.

The weight of this sentiment was so great that the task of those who, under the inspiration of Trotsky, stood against the tide and spoke up clearly against the Stalinist betrayal was virtually insuperable. All the more reason why, now that the counter-revolutionary policies of Stalin can be seen more clearly for what they were, works like Morrow’s should be studied.

It was from the ranks of the \textsc{poum}---so often wrongly described as Trotskyists, including by some who ought to know better\endnote{Including H.~Thomas, in his work \emph{The Spanish Civil War} which, although distorting the political significance of the struggle, is the fullest account so far available in English. Thus on p.~71 he produces the following curious piece of reasoning: ``Although not Trotskyist in the sense of being strict followers of Trotsky (they were not affiliated to the Fourth International), these men could justifiably be regarded as such since they were Marxist opponents of Stalin who shared Trotsky’s general views: permanent resolution abroad, working class collectives at home.'' The \textsc{poum} leaders did not regard themselves as Trotskyists, nor did the Trotskyists outside Spain regard them as such, \ldots therefore, Thomas accepts the Stalinist characterization of them as Trotskyists!}---that the chance of building a genuine revolutionary leadership was greatest. It is thus necessary to understand the nature of the \textsc{poum} and the reasons for its failures and capitulation both as displayed in this book and as explained in Trotsky’s pamphlet \emph{The Lessons of Spain}. The \textsc{poum} had a sufficient basis in the industrial working class in Catalonia to have given the necessary leadership. In order to do so, however, it was not sufficient to give lip-service to the Permanent Revolution or to make a literary exposure of the evils of Stalinism. It was also necessary to struggle against the reformist and anarchist leaders, however respected or powerful or personally honest, they may have been.

``Instead of mobilizing the masses against the reformist leaders, including the Anarchists,'' ran Trotsky’s indictment,

\begin{quotation}
  \noindent
  the \textsc{poum} tried to convince these gentlemen of the superiority of socialism over capitalism. This tuning fork gave the pitch to all the articles and speeches of the \textsc{poum} leaders. In order not to quarrel with the Anarchist leaders they did not form their own nuclei and in general did not conduct any kind of work inside the \textsc{cnt}.
  
  Evading sharp conflicts, they did not carry on revolutionary work in the republican army. They built instead ``their own'' trade unions and ``their own'' militia which guarded ``their own'' institutions or occupied ``their own'' section of the front.\index{Isolationism} By isolating the revolutionary vanguard from the class, the \textsc{poum} rendered the vanguard impotent and left the class without leadership.
  
  Politically the \textsc{poum} remained throughout far closer to the People’s Front, for whose left wing it provided the cover, than to Bolshevism. That the \textsc{poum} nevertheless fell victim to bloody and base repressions was due to the fact that the People’s Front could not fulfil its mission, namely, to stifle the socialist revolution---except by cutting off, piece by piece, its own left flank.
\end{quotation}

The ``salami tactic''\index{Salami tactic} was not invented by Rakosi; it was used in Spain and the \textsc{poum} was its chief victim. The \textsc{poum} provided an object lesson in showing how ``left'' parties can, through the vice of centrism, oppose a barrier to the solution of the crisis of leadership. As Trotsky continued,

\begin{quotation}
  Contrary to its own intentions, the \textsc{poum} proved to be, in the final analysis, the chief obstacle on the road to the creation of a revolutionary party.
\end{quotation}

The supporters of the Fourth International in Spain were but a handful, some were foreigners and they lacked roots in the working class. By the time that they found a hearing in the \textsc{poum}, or from workers under anarchist influence---especially the Friends of Durruti who, according to Morrow, ``explicitly declared the need for democratic organs of power, juntas or soviets, in the overthrow of capitalism, and the necessary state measures of repression against the counter-revolution''---the most militant sections of the working class had been defeated and divided. After May 1937, the victory of the counter-revolution was assured. From that time the last act was played out in the sombre tones of tragedy and against a backcloth of defeat and betrayal.

It is one of the merits of Morrow’s book that he shows in detail how the bourgeois nature of the Republican government made it incapable of waging a revolutionary war, the only type of war which stood any chance of overcoming the forces of international fascism. It could not promise liberation to the Moors, it could not offer agrarian reform to the peasants in Franco territory, it could not appear as the liberator of the working class nor could it appeal to the fascist troops with revolutionary propaganda. The deliberate cutting off of arms and supplies to sectors of the front held by \textsc{cnt} and \textsc{poum} militias had its counterpart in the superb armament and equipment of Assault Guards and army units kept in the rear for ``security'' purposes.\endnote{On this point the mordant record in G.~Orwell’s \emph{Homage to Catalonia}---the best personal account of the war---now available as a Penguin book.} Morrow recounts how on several sectors of the front the way was opened to the fascist forces by the betrayals of republican officers and police. He shows how the navy, an important weapon in republican hands, was reduced to impotence out of consideration for foreign powers.

Those who argue that a revolutionary seizure of power was out of the question and that the restoration of republican institutions was necessary to defeat Franco are answered by implication throughout Morrow’s work. They are dealt with more specifically on pages 82--91 which rank among the most important in the book and to which, therefore, the reader is especially referred. But the most convincing refutation of the argument for ``anti-fascism'' is its complete practical failure and the demoralisation which accompanied and resulted from its failure. The military defeat followed the counter-revolution with tragic inevitability once the ``democracies'' showed that they were not interested in getting involved in a war to stop Spain going fascist and Stalin necessarily pulled out his forces to cut his losses. The undignified scramble of the agents of the Comintern and the \textsc{nkvd} to find places on planes and boats leaving Spain in the final stages of the war was a fitting conclusion to their work.\endnote{J. Hernandez, \emph{op. cit.}, chapters 8 and 9.} 

Unfortunately for some of them, Stalin decided that they knew to much, or had been too shaken by the experiences; Rosenberg, Antonov-Ovseenko and Kolsov were among those who perished on their return to Russia. Nor can it have been accidental that the post-war purges in Eastern Europe included among their victims a remarkably high proportion of communists who had seen service in~Spain.

The Franco regime, erected on the ruins of the revolution of 1936, has endured to this day. In the Second World War it leant on the Axis powers, but without formally participating on their side. Its fate, in 1945, hung in the balance; but the betrayals of the Social Democrats and Communists in Western Europe decided the future of the Spanish people for a further lengthy period. The regime has, needless to say, solved none of the problems of backward Spain. Despite talk of ``agrarian reform'' the distribution of land ownership remains much as it was in 1936. Since the mid-1950s especially, the Spanish working class has provided, through continued resistance to the regime, including a number of great strikes, evidence of its vitality. The regime depends, as it has always done, upon the support of the other capitalist powers; by now that means mainly the United States. Even so, its future remains in question. There has been frequent talk of monarchist restoration. Reactionaries like Gil Robles{\indexGRobles} offer themselves as an alternative. Sections of the Church have taken their distance from the regime. No doubt, among many sections of the population, democratic illusions are rife. For many workers and peasants long years of poverty and repression have brought their toll of demoralisation from which only the younger elements are entirely free.

\begin{sloppypar}
The question of the forthcoming Spanish revolution remains posed. The crisis of the leadership still has to be solved. In the meantime, the Communist Party, still able to convince many that it is a revolutionary party, prepares new betrayals. In the policy of ``peaceful co-existence'' and the parliamentary road---as though there can be a peaceful path in keeping with totalitarian Spain!---it preaches ``national reconciliation.'' It is willing to collaborate with all and sundry, including former supporters of Franco, in a broad coalition to restore ``democracy'' in Spain. It therefore repeats, for the present period, precisely the same role that it played in 1936--39. A study of this book can forearm against such a policy, a policy which can only lead the workers and peasants of Spain to fresh defeats.
\end{sloppypar}

\begin{flushright}
  7th March, 1963 \\
  T. Kemp
\end{flushright}