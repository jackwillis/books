\chapter{Politics of the Spanish Working Class}

\section{The Right-Wing Socialists}

\lettrineP{rieto, Negrin, \&\ Besteiro} clung consistently to the theory that Spain had before it a considerable period of capitalist development. Besteiro and others had disgraced themselves by denouncing the workers’ resort to arms in October 1934. But Prieto, Negrin and their main associates had comported themselves at least as well as Caballero in the Asturian fighting and the general strike without, however, changing their political perspective. They had carried the party, in spite of left-wing opposition, into the electoral coalition of February 1936. The left socialists had, however, prevented Prieto’s entry into the cabinet. Prieto had clearly indicated that, were the left wing successful in gaining control of the party, he was prepared to fuse with Azaña’s Republican Left. He had gone so far, in the months before the civil war, as to join with Azaña in denouncing the strike epidemic. In their political conceptions the right-wing socialists were, indeed, simply petty-bourgeois republicans who, in the struggle against the monarchy, had correctly estimated that mass support could be had only by socialist colouration. In the republican arena opened in 1931, the first test had revealed them to be blood-brothers of Azaña.

Himself a Basque industrialist of considerable wealth, Prieto’s \emph{El Liberal} of Bilbao was one of the most influential organs among the bourgeoisie. Decades of class collaboration had given him the full confidence of the Basque bourgeoisie. More than any other figure, Prieto provided the link which united the Catholic, narrow-minded Basque capitalists, Azaña’s cosmopolitan and cynical intellectuals, and the Stalinist forces. Callous, ruthless, able, Prieto had none of the subjective fears of the Scandinavian and British Labour party leaders. He recognized the full significance of the policy embarked on by Stalin when the civil war began, and thereafter greeted the Stalinist spokesmen as ideological brothers.

\section{The Stalinists}

The political programme of Stalinism in 1936 appeared very different from its ultra-left denunciations of Azaña, Prieto, Caballero, the anarchists, as ``fascists'' and ``social fascists'' in 1931. Yet the basic policy remained the same. In 1936, as in 1931, the Stalinists wanted no proletarian revolution in Spain.

Walter Duranty, unofficial apologist for the Kremlin, in 1931 described its attitude:

\begin{quotation}
  The first Soviet comment on the events in Spain appears in the leading editorial in \emph{Pravda} today, but the organ of the Russian Communist Party seems none too jubilant over the prospects of the revolutionary struggle which it clearly expects will follow Alfonso’s downfall\ldots
  
  The unexpectedly pessimistic tone of \emph{Pravda}\ldots\ Perhaps is to be explained by Soviet anxiety lest the events in Spain disturb European peace during the present critical period of the five-year plan. Rightly or wrongly, it is believed here that the peace of Europe hangs literally on a thread, that the accumulation of armaments and national hatreds are much greater than before the war and make the present situation no less dangerous than the Spring of 1914, and that Spanish fireworks might easily provoke a general conflagration. (\emph{New York Times}, May 17, 1931.)
  
  Paradoxically enough, it appears that Moscow is not overdelighted by this circumstance---in fact it may almost be said that if the Spanish revolution ``swings left'' as Moscow now expects, Moscow will be more embarrassed than pleased\ldots
  
  \ldots For, first, the Soviet Union is excessively and perhaps unduly nervous about a war danger and ``views with alarm'' any event anywhere that may upset the European status quo\ldots\ Secondly, the Kremlin’s policy today stands much more on the success of socialist construction in Russia than upon world revolution\ldots\ (\emph{New York Times}, May 18, 1931.)
\end{quotation}

In 1931, the Kremlin had secured its goal by a policy of non-col\-lab\-or\-ation with the rest of the proletarian parties. The communists were thus isolated from the mass movement by union-splitting, no united front of organizations, attacks on other working-class meetings, etc. In 1931, the Kremlin had wanted nothing but maintenance of the status quo in Europe. In 1936, however, the Comintern adopted a new perspective, embodied in the Seventh Congress. The new course was to maintain the status quo as long as possible, this time not merely by preventing revolutions, but by active class collaboration with the bourgeoisie in the ``democratic'' countries. This collaboration was designed, in the event of war breaking out, to provide Russia with England. and France as its allies. The price Russia was offering to pay for an alliance with Anglo-French imperialism was the subordination of the proletariat to the bourgeosie. ``Socialism in a single country'' had revealed its full meaning as ``no socialism anywhere else.''

Lenin and the Bolsheviks were realists enough to allow the Soviet state to utilize conflicts between various capitalist powers even to the extent of using one against another in the event of war. Even more fundamental to their revolutionary politics, however, was the doctrine that, whatever the Soviet’s military alliances, the proletariat in every country has the unalterable duty to oppose its ``own'' bourgeoisie in war, to overthrow it in the course of the war, and to replace it with a workers’ revolutionary government which is the only possible real ally of the Soviet Union.

This fundamental tenet of Marxism was repudiated by the Seventh Congress of the Comintern. The French Communist party was already openly proclaiming its readiness to support its bourgeoisie in the coming war. Despite this, England’s coolness had largely negated the Franco-Soviet pact. Even under Blum the pact had not yet led to conferences between the two general staffs. The Spanish Civil War provided the Kremlin with an opportunity to prove once and for all to both French and English imperialists that, not only did the Kremlin intend to encourage no revolution, it was prepared to take the lead in crushing one that had nevertheless started.

Apparently not even all of the foreign Stalinist correspondents in Barcelona realized, in the first days of the civil war, that the Comintern had actually set itself the task of unravelling this all but completed revolution. On July 22, the London \emph{Daily Worker} carried a leading article:

\begin{quotation}
  In Spain, Socialists and Communists fought shoulder to shoulder in armed battle to defend their trade unions and political organizations, to guard the Spanish Republic and to defend democratic liberties so that they could advance towards a Spanish Soviet Republic.
\end{quotation}

And the same day, its Barcelona representative, Frank Pitcairn, wired: ``Red Militia Crushes Fascists, Triumph in Barcelona.''

\begin{quotation}
  The united working-class forces have already gained the upper hand. Streets here are being patrolled by cars filled with armed workers who are preserving order and discipline. Preparations are going forward for the organization of a permanent workers’ militia.
\end{quotation}

The Spanish Stalinists, however, joined Prieto and Azaña in appeals to the workers not to seize property. The Stalinists were the first to submit their press to the censorship. They were the first to demand liquidation of the workers’ militias, and the first to hand their militiamen over to Azaña’s officers. The civil war was not two months old when they began---what the government did not dare until nearly a year later---a murderous campaign against the \textsc{poum} and the Anarchist Youth. The Stalinists demanded subordination to the bourgeoisie, not merely for the period of the civil war, but afterward as well.

\begin{quotation}
  It is absolutely false,
\end{quotation}
declared Jesus Hernandez, editor of \emph{Mundo Obrero} (August 6, 1936),

\begin{quotation}
  \noindent
  that the present workers’ movement has for its object the establishment of a proletarian dictatorship after the war has terminated. It cannot be said we have a social motive for our participation in the war. We communists are the first to repudiate this supposition. We are motivated exclusively by a desire to defend the democratic republic.
\end{quotation}

\emph{L’Humanité}, organ of the French Communist Party, early in August published the following statement:

\begin{quotation}
  The Central Committee of the Communist Party of Spain requests us to inform the public, in reply to the fantastic and tendentious reports published by certain newspapers that the Spanish people are not striving for the establishment of the dictatorship of the proletariat, but know only one aim: the defence of the republican order, while respecting property.
\end{quotation}

As the months passed, the Stalinists adopted an even firmer stand against anything but a capitalist system. Josh Diaz, ``beloved leader'' of the Spanish party, at a plenary session of the Central Committee, March 5, 1937, declared:

\begin{quotation}
  If in the beginning the various premature attempts at ‘socialization’ and ‘collectivization,’ which were the result of an unclear understanding of the character of the present struggle, might have been justified by the fact that the big landlords and manufacturers had deserted their estates and factories and that it was necessary at all costs to continue production, now on the contrary they cannot be justified at all. At the present time, when there is a government of the Popular Front, in which all the forces engaged in the fight against fascism are represented, such things are not only not desirable, but absolutely impermissible. (\emph{Communist International}, May 1937)
\end{quotation}

Recognizing that the danger of a proletarian revolution came first of all from Catalonia, the Stalinists concentrated enormous resources in Barcelona. Having practically no organization of their own there, they recruited into their service the conservative labour leaders and petty-bourgeois politicians, by way of a fusion of the Communist Party of Catalonia with the Catalan section of the Socialist Party, the {Socialist Union} (a nationalist organization limited to Catalonia), and {Catala Proletari}, a split-off from the bourgeois Esquerra. The fusion, the {Unified Socialist Party of Catalonia} ({\textsc{psuc}}), affiliated to the Comintern. It had only a few thousand members at the beginning of the civil war but unlimited funds and hordes of Comintern functionaries. It took over the moribund Catalan section of the \textsc{ugt} and, when the Generalidad decreed compulsory syndicalization of all employees, recruited the most backward workers and clerks, who preferred this respectable institution to the radical \textsc{cnt}. But the biggest mass base of the Stalinists in Catalonia was a federation of traders, small businessmen, and manufacturers, the {Federaciones de Gremios y Entiadades de Pequefios Comerciantes e Industriales} ({\textsc{gepci}}), which in July was dubbed a union and affiliated to the Catalan \textsc{ugt}. The so-called Catalan section operated in complete independence from the Caballero-controlled National Executive of the \textsc{ugt}. Thereupon, as the chief and most vigorous defenders of the bourgeoisie, the \textsc{psuc} recruited heavily from the Catalan Esquerra.

The Stalinists followed a similar course in the rest of Spain. From the first, the {\textsc{cnt} Agricultural Union} and the {\textsc{ugt} Peasant and Land Workers’ Federation}---both supporting collectivization of the land---accused the Stalinists of organizing separate ``unions'' of the richer peasants who were opposed to the collectives. The Stalinist party grew more rapidly than any other organization, for it opened its doors wide. Dubious bourgeois elements flocked to it for protection. As early as August 19 and 20, 1936, Caballero’s organ, \emph{Claridad}, accused the Stalinist `Alliance of Anti-Fascist Writers' of harbouring reactionaries.\endnote{H.N. Brailsford, the British Socialist and People’s Frontist, says: The Communist Party ``is no longer primarily a party of the industrial workers or even a Marxist party'' and ``this development must be permanent. This prediction I base on the social composition of the C.P. both in Catalonia and in Spain.'' (\emph{New Republic}, June 9, 1937)}

When, after three long months of boycott, in the third week of October the first Soviet planes and guns finally arrived, the Communist party---which up to then had been on the defensive, unable to counter the sharp criticism of the \textsc{poum} on Stalin’s refusal to send arms---received a terrific impetus. Thenceforward its proposals were inextricably linked with the threat that Stalin would send no more planes and arms. Ambassador Rosenberg, in Madrid and Valencia, and Consul-General Antonov-Ovseyenko in Barcelona, made political speeches plainly indicating their preferences. When, at the November celebration of the anniversary of the Russian Revolution in Barcelona (at a parade participated in by all the bourgeois parties!), Ovseyenko ended his speech with ``Long live the Catalan people and its hero, President Companys,'' the workers were left in no doubt on which class the Kremlin was placing its approval.\endnote{One extraordinary incident deserves reporting. On November 27, 1936, \emph{La Batalla} was able to demonstrate that the \textsc{cnt}, \textsc{ugt}, Socialist Party, Republican Left---all favoured \textsc{poum} representation in the Madrid Defence Junta; yet the \textsc{poum} was not represented. How was it possible for Stalinist opposition alone to prevent the \textsc{poum}, with its militia columns on all fronts, from being represented? Could the Stalinists alone wield a veto? The answer was that the Soviet Embassy had intervened. ``It is intolerable that, on account of the aid they furnish us, they should attempt to impose upon us definite political norms, definite vetoes, to intervene and even to direct our politics,'' complained \emph{La Batalla}. The Madrid Defence Council incident, Ovseyenko’s November speech, Rosenberg’s addresses, were the public incidents which aroused the \textsc{poum}; through their cabinet post in the Generalidad they knew of ever more serious incidents to which they could not refer while in the government.

Consul-General Ovseyenko’s note to the press, answering the \textsc{poum}, probably has no parallel in all previous diplomatic history. It read like an editorial in \emph{Mundo Obrero}, denouncing the ``fascist manoeuvres'' of the \textsc{poum}, as an ``enemy of the Soviet Union.'' But before the year was up, Ovseyenko went further. On December 7, the \textsc{poum} called upon the Generalidad to offer a place of asylum to Leon Trotsky. Before the Generalidad could answer, the Soviet Consul-General declared to the press (\emph{La Prensa} reported it here) that if Trotsky were permitted to enter Catalonia, the Soviet government would cut off all aid to Spain. Truly, bureaucratic despotism could go no further!}

We have only sketched the Stalinist policy sufficiently to place it in the picture. We shall see it grow more openly, ruthlessly counter-revolutionary, in the ensuing year.

\section{Caballero, the Left Socialists, and the \textsc{ugt}}

Largo Caballero belonged to the same generation as Prieto. Both had reached middle age under the monarchy and modelled themselves upon the German Social Democrats of the right wing. As head of the \textsc{ugt}, Caballero had silently accepted the suppression of the Anarchist-led \textsc{cnt} by Primo de Rivera. More, he had sanctioned it by accepting a state councillorship from the dictator. He had joined the 1931–1933 coalition cabinet as Minister of Labour and had sponsored a law continuing Rivera’s mixed arbitration boards to settle strikes. ``We shall introduce compulsory arbitration. Those workers’ organizations which do not submit to it will be declared outside the law,'' he declared on July 23, 1931. Under his ministry, it was unlawful to strike for political demands or without ten days’ written notice to the employer. No trade union or any other labour meeting could be held without police witnesses present. Side by side with Prieto, Caballero had defended the repressions of the land-hungry peasants, the thousands of political arrests.

After the collapse of the 1931–33 coalition, a strong left wing developed, first in the Socialist Youth, demanding a re-orientation of the party. In 1934, Caballero unexpectedly declared for it. He had, said his friends, read Marx and Lenin for the first time after being ousted from the government. Nevertheless, Caballero’s group made no serious preparations for the October 1934 uprising. In Madrid, their chief stronghold, the rising never went beyond a general strike. On trial for inciting the insurrection---he was acquitted---Caballero denied the charge.

On record against coalitions, and for proletarian revolution, Caballero nevertheless agreed to the electoral coalition of February 1936 and supported Azaña’s cabinet in the Cortes on all basic questions. Caballero’s position, in effect, was that he would not repeat his rble as Minster of Labour in the 1931–33 coalition but that he would support Azaña from outside the cabinet, thereby being free to criticize. Ths was scarcely revolutionary irreconcilability. It was merely a form of critical loyalty, offering no threat to the bourgeois regime. During the February–July (1936) strike wave, Caballero incurred sharp criticism, both from the \textsc{cnt} and his own ranks, for discouraging strikes. An ardent advocate of fusion of the socialist and communist parties, he was mainly responsible for the fusion of the socialist and Stalinist youth. He had recouped his standing with the left wing of the party, however, by leading the fight to prevent Prieto from accepting the premiership. In the ensuing struggle, Prieto’s Executive had outlawed \emph{Claridad} (Caballero’s paper), reorganized pro-Caballero party districts, and indefinitely postponed the party convention. A split would have come, but the civil war intervened and, for the sake of presenting a picture of harmony, the Caballero forces had conceded to Prieto the national centre of the party.

At the height of the workers’ movement during the first weeks of the civil war, Caballero came into sharp collision with the Azaña-Prieto-Stalinist bloc. So long as discipline in barracks, management of feeding, lodging and payrolls, were in the hands of the workers’ organizations, and the militias freely organized discussions on political questions, the bourgeois-military caste could have no hope of securing real supremacy. Accordingly, the government, as a first feeler, called for enlistment of ten thousand reserve soldiers as a separate force under direct government control. The Stalinists defended the proposal. ``Some comrades have wished to see in the creation of the new voluntary army something like a menace to the role of the militias,'' said \emph{Mundo Obrero}, August 21. The Stalinists denied the very possibility and ended: ``Our slogan, today as yesterday, is the same for this. Everything for the People’s Front and everything through the People’s Front.''

This thoroughly reactionary position was effectively exposed by the \textsc{ugt} organ, \emph{Claridad}:

\begin{quotation}
  To think of another type of army to be substituted for those who are actually fighting and who in certain ways control their own revolutionary action, is to think in counter-revolutionary terms. That is what Lenin said (\emph{State and Revolution}): ``Every revolution, after the destruction of the state apparatus, shows us how the governing class tries to re-establish special bodies of armed men at `its' service, and how the oppressed class attempts to create a new organization of a type capable of serving not the exploiters but the exploited.''
\end{quotation}

Nevertheless, Caballero and the rest of the left-socialist leadership, in those critical early weeks, drew closer to Azaña, Prieto and the Stalinists. The dual power was proving a cumbersome and inadequate method of organizing the struggle against the fascist forces. Only two alternatives presented themselves inexorably: either join a coalition government, or replace the bourgeois power entirely by a workers’ regime.

Here, however, programmatic errors showed their terrible practical results. In April 1936 the leading group of the left socialists, the Madrid organization, had adopted a new programme, declaring for the dictatorship of the proletariat. What organizational form would it take? Luis Araquistain, Caballero’s theoretician, argued that Spain needed no soviets. The April programme had consequently embodied in it the conception that ``the organ of the proletarian dictatorship will be the Socialist party.'' But the left socialists had been prevented by Prieto’s postponement of the congress from assuming formal control of the party, and had desisted from further struggle for control when the civil war broke out. Furthermore, according to their programme, they would have to wait until the party included a majority of the proletariat. This programmatic failure to provide for united action through workers’ councils (soviets) in which socialists, communists, anarchists, \textsc{poum}ists, etc., would be gathered together with the deepest layers of the masses, this distorted notion of the lessons of the Russian Revolution, was a fatal error for the left socialists to make, and especially in Spain, with its anarchist traditions. They were saying precisely what the anarchist leaders had been accusing both communists and revolutionary socialists of meaning by the proletarian dictatorship.

The road to the proletarian dictatorship lay clearly before the proletariat. What was needed was to give the factory committees, militia committees, peasant committees, a democratic character, by having them elected by all of the workers in each unit; to bring together these elected delegates in village, city, regional councils, which in turn would send elected delegates to a national congress. True, the soviet form would not of itself solve the whole problem. A reformist majority in the executive committee would decline the assumption of state power. But the workers could still find in the soviets their natural organs of struggle until the genuinely revolutionary elements in the various parties banded together to win a revolutionary majority in the congress, and establish a workers’ state.

The road lay clearly before the proletariat but, not accidentally, the programme for that road was not the heritage of the left socialists. Caballero would criticize, grumble, excoriate, but he offered no alternative to the coalition with the bourgeoisie. Finally, he became its head.

\section[The \textsc{cnt-fai}]{\textsc{cnt-fai}: The National Confederation of Labour and the Anarchist Iberian Federation}

The followers of Bakunin had older roots in Spain than the Marxists. The \textsc{cnt} had been traditionally anarchist in leadership. The tide of the October Revolution had, for a short time, overtaken the \textsc{cnt}. It had sent a delegate to the Comintern Congress in 1921. The anarchists had then resorted to organized fraction work and recaptured it. Thenceforward, while continuing its traditional epithets against political parties, Spanish anarchism had in the \textsc{fai} a highly centralized party apparatus through which it maintained control of the \textsc{cnt}.

Ferociously persecuted by Alfonso and Primo de Rivera to the point where it actually dissolved for a time, the \textsc{cnt} from 1931 on, commanded an undisputed majority in the industrial centres of Catalonia and strong movements elsewhere. After the civil war began, it undoubtedly was larger than the \textsc{ugt} (some of whose biggest sections lay in fascist territory.)

Hitherto, in the history of the working class, anarchism had never been tested on a grand scale. Now, leading great masses, it was to have a definitive task.

Anarchism has consistently refused to recognize the distinction between a bourgeois and workers’ state. Even in the days of Lenin and Trotsky, anarchism denounced the Soviet Union as an exploiters’ regime. Precisely the failure to distinguish between a bourgeois and proletarian state had already led the \textsc{cnt}, in the honeymoon days of the revolution of 1931, to the same kind of opportunist errors as are always made by reformists---who also, in their way, make no distinction between bourgeois and workers’ states. Overcome by the ``fumes of the revolution,'' the \textsc{cnt} had benevolently greeted the bourgeois republic: ``Under a regime of liberty, the bloodless revolution is still more possible, still easier than under the monarchy.'' (\emph{Solidaridad Obrero}, April 23, 1931.) By October 1934 it swung to the equally false extreme of refusing to join with the republicans and socialists in the armed struggle against Gil Robles (with the honourable exception of the \textsc{cnt} regional organization in Asturias.)

Now, in the far more powerful fumes of the ``Revolution of July 19,'' when the accustomed boundary lines between bourgeoisie and proletariat were smeared over for the time being, the anarchists’ traditional refusal to distinguish between a bourgeois and workers’ state led them slowly, but decisively, into the ministry of a bourgeois state.

The false anarchist teachings on the nature of the state, it might seem, should logically have led them to refuse governmental participation in any event. Already running Catalan industry and the militias, however, the anarchists were in the intolerable position of objecting to the necessary administrative co-ordination and centralization of the work they had already begun. Their anti-statism \emph{as such} had to be thrown off. What \emph{did} remain, to wreak disaster in the end, was their failure to recognize the distinction between a workers’ and a bourgeois state.

Class collaboration, indeed, lies concealed in the heart of anarchist philosophy. It is hidden, during periods of reaction, by anarchist hatred of capitalist oppression. But, in a revolutionary period of dual power, it must come to the surface. For then the capitalist smilingly offers to share in building the new world. And the anarchist, being opposed to ``all dictatorships,'' including dictatorship of the proletariat, will require of the capitalist merely that he throw off the capitalist outlook, to which he agrees, naturally, the better to prepare the crushing of the workers.

There is a second fundamental tenet in anarchist teaching which led in the same direction. Since Bakunin, the anarchists had accused the Marxists of over-estimating the importance of state power, and had characterized this as merely the reflection of the petty-bourgeois intellectual’s pre-occupation with lucrative administrative posts. Anarchism calls upon the workers to turn their backs on the state and seek control of the factories as the real source of power. The ultimate sources of power (property relations) being secured, the state power will collapse, never to be replaced. The Spanish anarchists thus failed to understand that it was only the collapse of the state power, with the defection of the army to Franco, which had enabled them to seize the factories and that, if Companys and his allies were allowed the opportunity to reconstruct the bourgeois state, they would soon enough take the factories away from the workers. Intoxicated with their control of the factories and the militias, the anarchists assumed that capitalism had \emph{already} disappeared in Catalonia. They talked of the ``new social economy,'' and Companys was only too willing to talk as they did, for it blinded them and not him.

\section{The \textsc{poum}}

Here was a rare opportunity for even a small revolutionary party. Soviets cannot be built at will. They can be organized only in a period of dual power, of revolutionary upheaval. But in the period which calls for them, a revolutionary party can further their creation, in spite of the opposition of the most powerful reformist parties. In Russia, the Mensheviks and Social Revolutionaries, particularly after July, sought to siphon off the strength of the soviets into the government, sought to discourage their functioning or the creation of new ones, without success despite the fact that these reformists still wielded a majority in the soviets. In Germany, the social-democratic leadership sought, even more determinedly since they had the Russian lessons fresh before them, to prevent the creation of the workers’ and soldiers’ councils. In Spain, the direct hostility of the Stalinists and Prieto, the ``theoretical'' opposition of Caballero and the anarchists, would have been of no avail, for the basic units of the soviets were already there, in the factory, militia and peasant committees, and needed only democratization and bringing together in the localities. A single example, in \textsc{poum} controlled industrial towns like Lerida or Gerona, of delegates elected in every factory and shop, joining with delegates from the workers’ patrols and the militias to create a workers’ parliament which would function as the ruling body of the area, would have electrified Catalonia and initiated an identical process everywhere.

The \textsc{poum} was the one organization which seemed suited to undertake the task of creating the soviets. Its leaders had been the founders of the communist movement in Spain. It had, however, basic weaknesses. Its majority came from the {Workers’ and Peasants’ Bloc of Maurin}, whose cadre had collaborated with Stalin in the 1924–28 period in sending the Communist Party of China into the bourgeois Kuomintang \emph{bloc of four classes}; in creating farmer-labour and ``two-class'' parties ``of workers and farmers'' (a fancy name for a bloc with reformists and the liberal bourgeoisie)---and, in a word, in the whole opportunist course of those disastrous years. Maurin and his followers had broken with the Comintern not on these basic questions but on other issues---the Catalonian national question, etc.---when the Comintern had turned to dual unionism, ‘social fascism,’ etc., in 1929. Moreover, the fusion of the Maurinists with the former {Communist Left} (Trotskyists), led by Andres Nin and Juan Andrade---whose previous failure to sharply differentiate themselves from the Maurinist ideology had been the subject of years of controversy with the International Left Opposition---was an unprincipled amalgamation, in which the Communist Left elements had adopted a joint programme which was simply Maurin’s old conceptions, of which Trotsky had said as early as June 1931:

\begin{quotation}
  All that I have written in my latest work, \emph{The Spanish Revolution in Danger}, against the official policy of the Comintern in the Spanish question, applies entirely to the Catalonian Federation (Workers and Peasants bloc)\ldots\ it represents a pure ``Kuomintangism'' transported to Spanish soil. The ideas and the methods against which the Opposition fought implacably when it was a question of the Chinese policy of the Kuomintang, find their most disastrous expression in the Maurin programme\ldots\ A false point of departure during a revolution is inevitably translated in the course of events into the language of defeat. (\emph{The Militant}, August 1, 1931)
\end{quotation}

The first fruits of the fusion had scarcely been reassuring. After months of campaigning against a coalition with the bourgeoisie, the \textsc{poum} had overnight entered the electoral coalition of February 1936. It renounced the coalition after the elections, but on the very eve of the civil war (\emph{La Batalla}, July 17) called for ``an authentic Government of the Popular Front, with the direct [ministerial] participation of the Socialist and Communist Parties'' as a means to ``complete the democratic experience of the masses'' and hasten the revolution-an absolutely false slogan, having nothing in common with the Bolshevik method of demonstrating the necessity of the workers’ state and the impossibility of reforming the bourgeois state by forcing the reformists to assume governmental power \emph{without} the bourgeois ministers.

Nevertheless, many had the hope that the \textsc{poum} would take the lead in organizing the soviets. Nin now stood at the head of the party. He had been in Russia during the early years of the Russian Revolution, a leader of the {Red International of Labour Unions}. Would he not resist the provincialism of the Maurinist cadres? The \textsc{poum} workers, better trained politically than the anarchists, played a great role, entirely out of proportion to their numbers in the first revolutionary weeks, in seizing the land and factories. From a party of about 8,000 on the eve of civil war, the \textsc{poum} grew quickly, though remaining primarily a Catalonian organization. In the first months it quadrupled its numbers. Even more quickly its influence grew, as evidenced by the fact that it recruited over ten thousand militiamen under its banner.

The rising tide of coalitionism, however, engulfed the \textsc{poum}. The theoretical premises for it were already there, in the Maurinist ideology, to which Nin had signed his name in the fusion. The \textsc{poum} leadership clung to the \textsc{cnt}. Instead of boldly contending with the anarcho-reformists for the leadership of the masses, Nin sought illusory strength by identifying himself with them. The \textsc{poum} sent its militants into the smaller and heterogeneous Catalan \textsc{ugt} instead of contending for leadership of the millions in the \textsc{cnt}. It organized \textsc{poum} militia columns, circumscribing its influence, instead of sending its forces into the enormous \textsc{cnt} columns where the decisive sections of the proletariat were already gathered. \emph{La Batalla} recorded the tendency of \textsc{cnt} unions to treat collectivized property as their own. It never attacked the anarcho-syndicalist theories which created the tendency. In the ensuing year, it never once made a principled attack on the anarcho-reformist leadership, not even when the anarchists acquiesced in the expulsion of the \textsc{poum} from the Generalidad. Far from leading to united action with the \textsc{cnt}, this false course permitted the \textsc{cnt-fai} leadership, with perfect impunity, to turn its back on the \textsc{poum}.

More than once, in the days of Marx and Engels, and in the first revolutionary years of the Comintern, a weak national leadership had been corrected by its international collaborators. But the \textsc{poum}’s international connections stood to the right of the Spanish party. The {International Committee of Revolutionary Socialist Unity}---chiefly the \textsc{ilp} of England and the \textsc{sap} of Germany---issued a manifesto to the Spanish proletariat on August 17, 1936, which did not contain a single word of criticism of the Popular Front. The \textsc{sap} was shortly to go over to Popular Frontism itself, while the \textsc{ilp} embraced the Communist party in a ``Unity Campaign.'' Such were the ideological brethren for whom Nin and Andrade had renounced ``Trotskyism,'' the movement for the Fourth International. True enough, the Fourth Internationalists were small organizations compared to the reformist parties of Europe. But they offered the \textsc{poum} the rarest and most precious form of aid: a consistent Marxist analysis of the Spanish events and a revolutionary programme to defeat fascism. Nin was more ``practical,'' and abdicated the opportunity to lead the Spanish revolution.