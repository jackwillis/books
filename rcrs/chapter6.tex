\chapter{The Programme of the Caballero Coalition Government}

\textsc{Is it necessary}, at this late date, to explain that the cabinet of three Caballero men, three Prieto men, two Stalinists, and five bourgeois ministers, which was established on September 4, 1936, was a bourgeois government, a typical cabinet of class collaboration?

Apparently it is still necessary, for as late as May 9, 1937, a resolution of the National Executive Committee of the \textbf{Socialist Party, USA}, characterized this regime as ``a provisional revolutionary government.''

In turning over the premiership, Giral said: ``I remain as a cabinet minister in order to demonstrate that the new government is an amplification of the old from the moment in which the president of the resigning government continues forming part of the new.''

Caballero succinctly enough summarized his government’s programme to the Cortes:
\begin{quote}
  This government was constituted, all those forming it previously renouncing the defence of our principles and particular tendencies, in order to remain united on one sole aspiration: to defend Spain in her struggle against fascism. (\emph{Claridad}, October 1, 1936.)
\end{quote}
Certainly, Caballero had renounced his principles, but not the bourgeoisie and the Stalinists. For the common ground on which they joined with Caballero to form the government was the continuation of the old bourgeois order.

The programmatic declaration of the new cabinet had nothing in it which the previous government could not have signed. Point II is its essence:
\begin{quote}
  The ministerial programme signifies essentially, the firm decision to assure triumph over the rebellion, co-ordinating the forces of the people, through the required unity of action. To that is subordinated every other political interest, putting to a side the ideological differences, since at present there can be no other task than that of assuring the smashing of the insurrection. (\emph{Claridad}, September 5, 1936.)
\end{quote}
Not one word about the land! Not one word about the factory committees! And, as ``the representatives of the people,'' these ``democrats'' convened the former Cortes elected on February 16 by the electoral agreement which had given a majority to the bourgeoisie on the joint slate!

A few weeks before assuming the premiership, Caballero had been inveighing (through \emph{Claridad}) against separating the revolution from the war. He had protested against the displacement of the militias. Now he became the leader in reconstructing the bourgeois state. What had happened?

We need not speculate on what went on his mind. The observable change, reflected in \emph{Claridad}, was that instead of depending on the working class of Spain and on international working class aid, Caballero now put his hopes on winning the aid of the ``great democracies,'' Anglo-French imperialism.

On September 2, in an interview with the Havas Agency, Prieto had declared himself ``pleased that the French government had taken the initiative in the proposals for nonintervention,'' although ``it had not had the full value that France wishes to give it.''

``Each day is more urgent for France to work with great energy to avoid dangers for all.''

``Why does the \textsc{cnt} act as if we were finding ourselves before a completed revolution?'' complained \emph{El Socialista}:
\begin{quote}
  Our geographic law is not that of immense Russia, by no means. And we have to take into account the attitude of the states that surround us, in order to determine our own attitude. Let not everything rest on spiritual force nor on reason, but on knowing how to renounce four in order to gain a hundred. We still hope that the estimate of the Spanish events made by certain democracies will be changed, and it would be a pity, a tragedy, to compromise these possibilities by pushing the velocity of the revolution, which at present does not conduct us to any positive solution. (\emph{El Socialista}, October 5, 1936.)
\end{quote}

The classical social democrats of the Prieto school could thus say, quite plainly, what the ``Spanish Lenin,'' Caballero, and the ex-Le\-ni\-nists, the Stalintern, had to obscure: they were currying favour with Anglo-French imperialism by strangling the revolution. As late as August 24, Caballero had hoped that Hitler’s intransigence would block the formation of the non-intervention committee. But with Hitler’s embargo on arms shipments on that date, and the Soviet’s declaration of adherence, it was clear that the Spanish blockade would be of long duration. The question was sharply posed: either fight the non-intervention blockade and denounce Blum and the Soviet Union for backing it, or accept the Stalinist perspective of gradually winning away France and England from the blockade by demonstrating the bourgeois respectability and stability of the Spanish Government. In other words, either accept the perspective of proletarian revolution and the necessity of arousing the international proletariat to aid Spain and spread the revolution to France, or accept class collaboration in Spain and abroad. When the alternatives became inescapable, Caballero chose the latter. Within a few days, his comrade, Alvarez del Vayo, was off to grovel at the feet of the imperialists in the League of Nations.

Caballero understood quite well that to arouse the Spanish masses to supreme efforts, it was necessary to offer them a programme of social reconstruction. A circular order to the political commissioners at the front from Caballero’s War Ministry emphasizes:
\begin{quote}
  It is necessary to convince the fighters who are defending the republican regime with their lives that at the termination of the war the organization of the state will undergo a profound modification. From the present we shall go on to a structure, socially, economically, and juridically, all for the benefit of the working masses. We should try to imbue such conceptions in the spirit of the troops by means of simple and plain examples. (\emph{Gaceta de la Republica}, October 17, 1936.)
\end{quote}

But the masses, Caballero presumably hoped, could be inspired with words, while the hard-headed imperialists of England and France would be content only with deeds.

\emph{To rouse the peasantry to struggle}, to provide their best sons for the war, not as sullen, demoralized conscripts but as lionhearted soldiers, to raise the food and fibres necessary to feed and clothe the army and the rear---that could only be done by giving the land to the peasantry, land to those who till it, the land as a national possession given in usufruct to the working farmer. Propaganda for liberty, etc., is absurdly insufficient. These are not your American or French farmers, who have already some land, enough to live on without being hungry:
\begin{quote}
  Misery is still appalling in Estremadura, Albacete, Andalusia, Caceres and Ciudad Real. It is by no means a figure of speech when it is said that peasants are dying of hunger. There are villages in Hurdes, in La Mancha where the peasants, in absolute despair, revolt no longer. They eat roots and fruit. The events in Yeste [land seizures] are dramas of hunger. In Navas de Estena, about thirty miles from Madrid, forks and beds are unknown. The villagers’ chief diet consists of a soup made from bread, water, oil and vinegar.
\end{quote}
These words are not those of a Trotskyist agitator but the involuntary testimony of a Stalinist functionary (\emph{Inprecorr}, August 1, 1936). How can one seriously hope to rouse these depths, except by the one act which can convince them of a new era: give them the land. Can one expect them to ``defend the republic''---that republic of Azaña---that had shot them like dogs for seizing land or stored grain?

Now the peasants and agricultural workers had seized land---not everywhere yet---but still had no assurance that the government was not permitting it merely as a provisional measure for the war which it would attempt to annul afterward. What the peasants wanted was a general decree nationalizing the land throughout Spain, giving it in usufruct to those who till it, so that no usurer can ever take it away from them. Equally, the tillers of the soil wanted the power to assure their land tenure, and that could only be a government of their flesh and blood workers’ and peasants’ regime.

Does it require much insight to see what effect such a land decree would have on the fascist forces? Not only on the land-hungry peasants in the fascist areas but, above all, among the sons of the peasants who constituted the ranks of the fascist armies, deceived by their officers as to the causes of the conflict. A few airplane loads of leaflets spread on the fascist fronts, announcing the land decree, would be worth an army of a million men. No single other move of the Loyalist side could sow more demoralization and decomposition in the fascist forces.

But thirty years as a ``responsible leader'' had left its mark too deeply. The inner forces of the masses had been too long an object of concern and fear to Caballero, something which he had to curb and channelize into safe boundaries. The land decree of October 7, 1936, merely sanctioned division of estates belonging to known fascists; other wealthy landowners, peasant exploiters, etc., remained untouched. The aroused hopes of the peasantry were smothered.

The \textsc{ugt} workers in the factories, shops, and railroads were setting up their factory committees, taking over the plants. What would Caballero have to say to them? In Valencia and Madrid the government swiftly intervened, placing government directors in charge who confined the factory committees to routine activity. Not until February 23, 1937, was a comprehensive decree on the industries adopted (issued over the name of Juan Peiro, the anarchist Minister of Industry). It gave the workers no security for the future regime in industry; established strict intervention by the government. ``Workers’ control,'' by its terms, proved little more than a collective contract, such as, for example, operates in shops dealing with the \textbf{Amalgamated Clothing Workers’ Union} in America---that is, no real measure of workers’ control at all.

Caballero had denounced the Giral cabinet for building an army outside the workers’ militias and for re-building the old Civil Guard. (The great ``Caballero'' column on the Madrid front, in its uncensored paper, had called for direct resistance to Giral’s proposal.) Now Caballero put his prestige behind the Giral plans. The conscription decrees followed the traditional forms, gave no place to soldiers’ committees. That meant reviving the bourgeois army, with supreme power in the hands of a military caste.

\section{Freedom for Morocco?}

Delegations of Arabs and Moors came to the government, pleading for a decree. The government would not budge. The redoubtable Abd-el-Krim, exiled by France, sent a plea to Caballero to intervene with Blum. so that he might be permitted to return to Morocco to lead an insurrection against Franco. Caballero would not ask, and Blum would not grant. To rouse Spanish Morocco might endanger imperialist domination throughout Africa.

Thus Caballero and his Stalinist allies set their faces as flint against revolutionary methods of struggle against fascism. In due time, at the end of October, came their reward: a modicum of army supplies from Stalin. In the ensuing months, came more supplies, particularly after great defeats: after the encirclement of Madrid, after the fall of Malaga, after the fall of Bilbao, supplies enough to save the Loyalists for the moment, but never enough to permit them to carry through a really sustained offensive which might lead to the total collapse of Franco.

What was the political logic behind this careful turning-off-and-on of the spigot of supplies? If it were a question of Soviet Russia’s limited resources that still does not explain, for example, why all the planes that were to go to Spain could not have been massed for a decisive struggle in one period. The explanation of the spigot is not technical but political. Enough was given to prevent early defeat of the Loyalists and the consequent collapse of Soviet prestige in the international working class. And this fitted in, at bottom, with Anglo-French policy, which did not desire an immediate Franco victory. But not enough was given desire facilitate a victorious conclusion from which might issue-once the spectre of Franco was gone---a Soviet Spain.

Such was the programme of the ``provisional revolutionary government'' of Caballero. Nothing was added or subtracted from it with the entry of the \textsc{cnt} ministers on November 4, 1936. By then the ``great democracies'' had had an opportunity, observing the \textsc{cnt} in the Catalonian government formed on September 26, to be reassured about the ``responsibility'' of these anarchists.

There was one troublesome point: the anarchist-controlled \textbf{Council of Defence of Aragon}, comprising the territory wrested from the fascists by the Catalonian militias on the Aragon front, had a fearful reputation as an arch-revolutionary body. The price of four cabinet seats for the \textsc{cnt} was some reassurance on Aragon. Accordingly, on October 31, the Aragon Council met with Caballero. ``The object of our visit,'' declared the Council’s president, Joaquim Ascaso, ``has been to pay our respects to the head of the government and to assure him of our attachment to the government of the people. We are disposed to accept all the laws it passes and we, in our turn, ask the Minister for all the help we need. The Aragon Council is formed of elements from the Popular Front so that all the forces upholding the government are represented in it.''

``Interviews with President Azaña, with President Companys, and with Largo Caballero,'' added a Generalidad statement of November 4, ``have destroyed any suspicions that might have arisen that the government which has been constituted [in Aragon] was of an extremist character, unrelated to the other governmental organs of the republic and opposed to the government of Catalonia.'' That day the anarchists took their seats in Caballero’s cabinet.