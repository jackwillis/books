\chapter{The Counter-Revolution and the Masses}

\lettrineI{t would be a libel} on the socialist and anarchist-led masses to think that they were not alarmed by the advance of the counter-revolution. Discontent, however, is not enough. It is necessary also to know the way out. Without a firm, well-developed strategy for repelling the counter-revolution and leading the masses to state power, discontent can accumulate indefinitely and only issue in sporadic, desperate lunges which are doomed to defeat. In other words, the masses require a revolutionary leadership.

Especially in the ranks of the \CNT\ and \FAI\ the discontent was enormous. It seeped out in hundreds of articles and letters in the anarchist press. Though the anarchist ministers in Valencia and the Generalidad voted for the governmental decrees or submitted to them without public protest, their press did not dare defend the governmental policies directly. As governmental repressions increased, the pressure of the \CNT\ workers on their leadership increased.

On March 27, the \CNT\ ministers withdrew from the Catalonian government. The ensuing ministerial crisis lasted three whole weeks.

``We cannot sacrifice the revolution to unity,'' declared the \CNT\ press.

``No more concessions to reformism.''

``Unity has been maintained until now on the basis of our concessions.''

``We can retreat no further.''

Precisely what the \CNT\ leadership now proposed, however, was a mystery. Companys neatly punctured their postures by a summary of the ministerial course since December, demonstrating that the \CNT\ ministers had voted for everything---the disarming of the workers, the army mobilization and reorganization decrees, dissolution of the workers’ patrols, etc. Stop this humbug and come back to work, Companys was saving. And as a matter of fact, the \CNT\ ministers were ready to come back at the end of the first week. At this point, however, the Stalinists demanded a further capitulation: the organizations providing ministers should sign a joint declaration pledging themselves to carry out a stated series of tasks. The \CNT\ ministers protested that the usual ministerial declaration after constituting the new cabinet would be sufficient---the Stalinist proposal would have left the \CNT\ ministers absolutely naked before the masses. Thus the ministerial crisis dragged out two more weeks.

There then ensued a little by-play which amounted to nothing more than a division of labour, whereby the \CNT\ leaders were bound more strongly than ever to the Generalidad. Companys assured the \CNT\ that he agreed with them and not with the Stalinists, and offered his services to ``force'' the Stalinists to relinquish their demand. At the same time, Premier Tarradellas, Companys’ lieutenant, defended the administration of the war industries (run by the \CNT) against an attack in the \PSUC\ organ, \emph{Treball}, which he termed the ``most arbitrary falsehoods.'' For these little services, the \CNT\ abjectly gave Companys unconditional political support:

\begin{quotation}
  We declare publicly that the \CNT\ is to be found at the side of the president of the Generalidad. Luis Companys, whom we have accorded whatever facilities have been required for the solution of the political crisis. We stand by the president who, without any kind of servile praise---a proceeding incompatible with the morale of our revolutionary movement---knows that he can count on our most profound respect and our most sincere support. (\emph{Solidaridad Obrera}, April 15, 1937, p. 12)
\end{quotation}

Companys, of course, managed to persuade the Stalinists to relinquish the demand for a pact, and on April 16, the ministerial crisis was ``resolved.'' The new cabinet, like its predecessor, provided a majority for the bourgeoisie and the Stalinists, and, of course, differed in nothing from the previous one.

The masses of the \CNT\ could not be so ``flexible.'' They had a heroic tradition of struggle to the death against capitalism. Even more insistently, the revival of the bourgeois state was taking place on their backs. Inflation and the uncontrolled manipulation of prices by the businessmen ``mediating'' between the peasantry and the city masses now led to perpendicular price rises. In this period the rise of prices is the leitmotif of all activity. The press is full of the problem. The condition of the masses was growing daily more intolerable, and the \CNT\ leaders showed them no way out.

Many voices now cried for a return to the traditional apoliticism of the \CNT. ``No More Governments!'' Local \CNT\ papers broke discipline and took up this refrain. It was counsel of unthinking despair.

Far more significant was the rise of the Friends of Durruti. In the name of the martyred leader, a movement rose which had assimilated the need for political life, but rejected collaboration with the bourgeoisie and reformists. The Friends of Durruti were organized to wrest leadership from the \CNT\ bureaucracy. In the last days of April, they plastered Barcelona with their slogans---an open break with the \CNT\ leadership. These slogans included the essential points of a revolutionary programme: all power to the working class, and democratic organs of the workers, peasants and combatants, as the expression of the workers’ power.

The Friends of Durruti represented a deep ferment in the libertarian movement. On April 1, a manifesto of the Libertarian Youth of Catalonia (\emph{Ruta}, April 1, 1937) had denounced the ``United Socialist Youth'' [Stalinists], who first assisted the revaluation of the Azaña stock-fallen so low in the first days of the revolution when he tried to flee the country-and who called to the Unified Catholic Youth and even to those who were sympathetic to fascism; stigmatized the bourgeois-Stalinist bloc as ``supporting openly all the intentions of the English and French governments to encircle the Spanish revolution;'' excoriated the counter-revolutionary assaults on the publishing houses and radio station of the \POUM\ in Madrid. 

It pointed out that:

\begin{quotation}
  \dots\ arms are denied to the Aragon front because it is definitely revolutionary, in order to be able afterward to throw mud at the columns operating on that front.
	
  \dots\ the Central Government boycotts Catalan economy in order to force us to renounce our revolutionary conquests.
	
  \dots\ the sons of the people are sent to the front, but for counter-revolutionary ends the uniformed forces are being kept in the rear.
  
  \dots\ they have gained ground for a dictatorship---not pro\-le\-ta\-ri\-an!---but bourgeois.
\end{quotation}

Clearly differentiating the Anarchist Youth from the \CNT\ ministers, the manifesto concluded:

\begin{quotation}
  We are firmly decided not to be responsible for the crimes and betrayals of which the working class is being made the object\dots\ We are ready to return, if that is necessary, to the underground struggle against the deceivers, against the tyrants of the people and the miserable merchants of politics.
\end{quotation}

An editorial in the same issue of Ruta declared:

\begin{quotation}
  Let not certain comrades come to us with appeasing words. We shall not renounce our struggle. Official automobiles and the sedentary life of the bureaucracy do not dazzle us.
\end{quotation} 

This from the official organization of the anarchist youth!

Not in a day or a month, however, does a regroupment take place. The \CNT\ had a long tradition and the discontent of its masses would evolve only at a slow pace into an organized struggle for a new leadership and a new programme. Particularly was this true because no revolutionary party existed to encourage this development.

\section{The \POUM’s Answer to the Counter-Revolution}

An abyss was opening up between the \CNT\ leaders and the masses within the \CNT\ movement. Would the \POUM\ step into the breach and place itself at the head of the militant masses?

The prevalence of a wide tendency in \CNT\ ranks to go back to traditional apoliticism was an annihilating criticism of the \POUM, which had done nothing to win these workers to revolutionary-political life. Also with no aid from the \POUM\ leadership, a genuinely revolutionary current was crystallizing in the Friends of Durruti and the Libertarian Youth. If the \POUM\ was ever to strike out independently of the \CNT\ leadership, this was the moment!

The \POUM\ did nothing of the sort. On the contrary, in the ministerial crisis of March 26--April 16, it revealed that it had learned nothing whatever from its earlier participation in the Generalidad. The Central Committee of the \POUM\ adopted a resolution declaring:

\begin{quotation}
  There is needed a government that would canalize the aspirations of the masses, giving a radical and concrete solution to all the problems by way of the creation of a new order that would be guarantor of the revolution and of the victory at the front. This Government can only be a government formed by representatives of all the political and trade union organizations of the working class which would propose as immediate tasks the realization of the following programme. (\emph{La Batalla}, March 30)
\end{quotation}

The proposed fifteen-point programme is not a bad one---for a revolutionary government. But the absurdity of proposing it to a government which by definition includes the Stalinists and the Esquerra-controlled Union of Rabassaires (independent peasants) is indicated immediately by the last point on the programme: the convocation of a congress of delegates of the unions, peasants and combatants which will in turn elect a permanent workers’ and peasants’ government.

For six months the \POUM\ had been saying that the Stalinists were organizing the counter-revolution. How, then, could the \POUM\ propose to collaborate with them in the government and in convoking the congress? From this proposal the workers could only conclude that the \POUM’s characterization of the Stalinists had been so much factional talk, and would henceforth take no \POUM\ charges against the Stalinists as being seriously meant.

And Companys and his Esquerra? A new cabinet must receive a mandate from Companys, and the \POUM\ proposed no break with this law. Was it conceivable that Companys would agree to a government which would convoke such a congress? Here, too, the masses could only draw the conclusion that the \POUM’s declaration of the necessarily counter-revolutionary role of the Esquerra of Companys was not seriously meant.

As a matter of fact, the workers could not feel that the \POUM\ attached fundamental importance to the congress. Much more important seemed the entry of the \POUM\ into the Generalidad. \emph{La Batalla} (March 30) published side by side two columns headed: ``Balance of Two Periods of Government.'' One, ``the government in which the \POUM\ participated,'' the other, ``the government in which the \POUM\ did not participate.''

The government of September 26--December 12 is described lyrically as a period of revolutionary construction. Thus, the \POUM\ still refused to understand how the government in which it had participated had taken the first giant steps in reviving the bourgeois state. From these tables, the worker could draw only one logical conclusion: all that was needed was that the \POUM\ should be re-admitted into the government.

Let us look more closely at the \POUM’s proposed congress of delegates of unions, peasants and combatants. It sounds \emph{almost} like soviets; and indeed it was proposed precisely to delude the restless left wing of the \POUM. But it has nothing whatever in common with the Leninist conception of soviets.

One must never forget---what the Stalinists have completely bu\-ried---that soviets do not begin as organs of state power. They arise in 1905, 1917, in Germany and Austria in 1918, rather as powerful strike committees and representatives of the masses in dealing with immediate concrete problems and with the government. Long before they can seize state power, they carry on as organs defending the workers’ daily interests. Long before the workers’, peasants’ and soldiers' deputies have united in an all-national congress, there must have been formed the city, village, regimental soviets which are later to be united in a national organ. The way to begin getting such a congress is to begin electing factory, peasants’ and combatants’ committees wherever the workers can be taught to function through their own committees. The example of a few committees in a few factories and regiments will win the masses to this form, the most democratic method of representation known to mankind. Then, only, can one organize an all-national congress in a bid for power.
\noclub

At that point, moreover, the congress will inevitably be a reflection, even if a more accurate one than other organs, of the political level of the masses. If the Stalinist, anarchist, and other reformist organizations are still powerful, then the congress will reflect their political line. There is, in a word, no magic in the soviet form: it is merely the most accurate; most quickly reflecting and responsively changing form of political representation of the masses.

The mere convocation of the congress would not solve the basic political task of the \POUM: \emph{to wrest from the Stalinists and the anarchists the political leadership of a majority of the working class}. The congress would concentrate the political thoughts and yearnings of the masses as no other organ could. It would provide the arena in which the revolutionary party can win the support of the working class---but only in the sharpest struggle against the false political lines of the reformists of all varieties.

Were the \POUM\ leadership serious about the proposed congress, it would not have asked the government to convoke it, but would immediately have sought election of committees wherever possible. But the \POUM\ did not even elect such committees in the factories or militias under its own control. Its ten thousand militiamen were controlled bureaucratically by officials appointed by the Central Committee of the party, election of soldiers’ committees being expressly forbidden. As the internal life of the party grew more intense, with the left-wing workers demanding a new course, more and more bureaucratic became the control of the leadership over the factories and militiamen. Here was scarcely an example to inspire the workers elsewhere to create elective committees!

The soviet form bases itself directly on the factories, by direct representation from each factory in the localities. This provides direct contact with the factories, enabling the soviet through recall and new elections to renew itself and reduce the time-lag of political development to a minimum. This characteristic of soviets also enables the revolutionists to converse directly with the factories, without the intervention of the trade union bureaucrats. Yet, precisely in this basic characteristic, the congress proposed by the \POUM\ differs from the soviet form: the \POUM\ proposes representation of the \emph{unions}. This was simply another concession to the prejudices of the \CNT\ leadership, who conceive the unions, and not the far broader workers’, peasants’ and soldiers’ soviets, as the governing form of industry in a socialist society and---incidentally!---object to the revolutionists reaching the workers in the factories.

Thus the utopian project of the \POUM\ was a fraud, a counterfeit, doomed to a paper existence---an empty concession to the left wing.

One seeks in vain, in the \POUM\ documents, for a systematic defence of its opportunist course. One finds only a paragraph here and there, which may be presumed to be the germ of a new theory. For example, Nin appeared to think that the only genuine form of the dictatorship of the proletariat must be based on the leadership of more than one workers’ party:

\begin{quotation}
  The dictatorship of the proletariat is not that of Russia, for that is a dictatorship of one party. The reformist workers’ parties within the soviets worked for armed struggle against the Bolsheviks and this created the circumstances for the taking of power by the Bolshevik party. In Spain, nobody can think of a dictatorship of a party, but of a government of full workers’ democracy\dots\ (\emph{La Batalla}, March 23, 1937)
\end{quotation}

Nin thus wipes out the soviet democracy of the first years after the October Revolution, and the history of the process of reaction, resulting from the isolation of the revolution from Europe, which in the end led in Russia, not to the dictatorship of a party but the dictatorship of a bureaucracy. If his words are to be taken seriously, Spain could not have a dictatorship of the proletariat, no matter how wide the influence of the \POUM\ became, unless other organizations (\FAI\ and \CNT) agreed to work for it; if they did not, then Spain is doomed to capitalist domination! Thus Nin rationalized his refusal to let go of the coat-tails of the \CNT\ leaders.

The crux of the matter is that Nin had abandoned the Leninist conception of soviets. This he did explicitly:

\begin{quotation}
  In Russia there was no democratic tradition. There did not exist a tradition of organization and of struggle in the proletariat We do have that. We have unions, parties, publications. A system of workers’ democracy.
  
  One understands, therefore, that in Russia, the soviets should have developed the importance that they did. The soviets were a spontaneous creation that in 1905 and 1917 took on an entirely political character.
  
  Our proletariat, however, had its unions, its parties, its own organizations. For this reason, the soviets have not risen among us. (``The Fundamental Problem of Power,'' \emph{La Batalla}, April 27, 1937)
\end{quotation}

Once embarked on a false, opportunist course, revolutionists will decompose politically at a fearful rate. Who would have believed, a few years ago, that Nin would be capable of speaking these lines? The gigantic ``tradition of organization and of struggle'' amassed by the Russian proletariat in the Revolution of 1905, the study and analysis of which developed the cadres which made the October Revolution, ``escapes'' him. What was peculiarly Russian about the soviet form? In 1918, in countries with a far richer proletarian tradition than Spain, Germany and Austria---the soviets rose. As a matter of simple fact, what were the factory committees, the militia committees, the village committees, the workers’ supply committees, the workers’ patrols, the investigation committees, etc., etc., which surged up in Spain in July 1936---were these not the foundation stones, which required only deeper politicalization and coordination, direct representation of the masses instead of the organizations, in order to form the soviet power? Nin’s rationalization is pitiful; it will not stand up for a moment; he had joined with the Stalinists and the bourgeoisie, in September, explicitly to abolish the soviet dual power as ``unnecessary duplication''---and nine months later could say, ``the soviets have not risen among us.''

Thus, the \POUM\ leadership stood at the tail of the \CNT. Instead of assimilating the lessons of Leninism, they denounced it as \dots\ Trotskyism. \emph{Why do the Stalinists call us Trotskyists?}---this is the perennial complaint of the \POUM\ leadership. The following is typical, from an article by Gorkin:

\begin{quotation}
  In any case Trotsky has given no basis for our being called Trotskyists. In 1931 he published two articles upon the then Workers and Peasants Bloc and its chief Maurin. For him [Trotsky] our political line was a ``mixture of petty-bourgeois prejudices, of ignorance, of provincial `science' and of political knavery.'' \dots
  
  With the Spanish Civil War, we have seen manifested once more the sectarianism of Trotsky \dots\ The representative today of the Fourth International in Spain, within two hours of arriving, and a quarter of an hour of talking with us, drew from his pocket a programme prepared \emph{a priori}, giving us advice concerning the tactic that we ought to apply. Courteously, we advised him to take a walk through Barcelona and to study a little better the situation. This citizen \dots\ is the perfect symbol of Trotskyism: of a sectarian doctrinairism, of a great sufficiency, certain that he possessed the revolutionary philosopher’s stone. (\emph{La Batalla}, April 24, 1937)
\end{quotation}

This provincial smugness, the heritage of Maurin, had not only been criticized by Trotsky. Nin himself, in August 1931, had declared that the greatest danger for the Workers and Peasants Bloc was Maurin’s contempt for the lessons of the Russian Revolution. In inheriting Maurin’s mantle, however, Nin had taken over this tradition of provincial blindness.

Not all those who had agreed with Nin in 1931 followed him in his renunciation of Leninism. Bearing the chief brunt of bourgeois-Stalinist repressions, the Madrid section of the \POUM\ gave an overwhelming majority to an oppositional programme based on the Leninist course. The main section of the party, Barcelona, voted for the immediate organization of soviets on April 15, 1937. Bureaucratic measures were resorted to by Nin and Gorkin to prevent the growth of the left wing. Dissidents were brought back from the front under guard, and expelled. Fraction organizations were forbidden. More important than the repressions by the leadership were those of the government, which fell most heavily, naturally, on those workers who stood out in the ranks and in the factories. The left-wing workers of the \POUM---those expelled constituting themselves the Spanish Bolshevik-Leninists (Fourth Internationalists)---established close contacts with the anarchist workers, especially the Friends of Durruti. But the regroupment took place too slowly. Before the revolutionary forces could come together and win the confidence of the masses, transform their discontent into the positive drive for power, substitute the objective strategy of leadership for the subjective desperation of the masses, the bitterness of the leaderless workers had already overflowed: the barricades went up on May 3.