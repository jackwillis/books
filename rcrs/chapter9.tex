\chapter{The Counter-Revolution and the Masses}

\lettrineI{t would be a libel} on the socialist and anarchist-led masses to think that they were not alarmed by the advance of the counter-revolution. Discontent, however, is not enough. It is necessary also to know the way out. Without a firm, well-developed strategy for repelling the counter-revolution and leading the masses to state power, discontent can accumulate indefinitely and only issue in sporadic, desperate lunges which are doomed to defeat. In other words, the masses require a revolutionary leadership.

Especially in the ranks of the \CNT\ and \FAI\ the discontent was enormous. It seeped out in hundreds of articles and letters in the anarchist press. Though the anarchist ministers in Valencia and the Generalidad voted for the governmental decrees or submitted to them without public protest, their press did not dare defend the governmental policies directly. As governmental repressions increased, the pressure of the \CNT\ workers on their leadership increased.

On March 27, the \CNT\ ministers withdrew from the Catalonian government. The ensuing ministerial crisis lasted three whole weeks.

``We cannot sacrifice the revolution to unity,'' declared the \CNT\ press.

``No more concessions to reformism.''

``Unity has been maintained until now on the basis of our concessions.''

``We can retreat no further.''

Precisely what the \CNT\ leadership now proposed, however, was a mystery. Companys neatly punctured their postures by a summary of the ministerial course since December, demonstrating that the \CNT\ ministers had voted for everything---the disarming of the workers, the army mobilization and reorganization decrees, dissolution of the workers’ patrols, etc. Stop this humbug and come back to work, Companys was saving. And as a matter of fact, the \CNT\ ministers were ready to come back at the end of the first week. At this point, however, the Stalinists demanded a further capitulation: the organizations providing ministers should sign a joint declaration pledging themselves to carry out a stated series of tasks. The \CNT\ ministers protested that the usual ministerial declaration after constituting the new cabinet would be sufficient---the Stalinist proposal would have left the \CNT\ ministers absolutely naked before the masses. Thus the ministerial crisis dragged out two more weeks.

There then ensued a little by-play which amounted to nothing more than a division of labour, whereby the \CNT\ leaders were bound more strongly than ever to the Generalidad. Companys assured the \CNT\ that he agreed with them and not with the Stalinists, and offered his services to ``force'' the Stalinists to relinquish their demand. At the same time, Premier Tarradellas, Companys’ lieutenant, defended the administration of the war industries (run by the \CNT) against an attack in the \PSUC\ organ, \emph{Treball}, which he termed the ``most arbitrary falsehoods.'' For these little services, the \CNT\ abjectly gave Companys unconditional political support:

\begin{quotation}
  We declare publicly that the \CNT\ is to be found at the side of the president of the Generalidad. Luis Companys, whom we have accorded whatever facilities have been required for the solution of the political crisis. We stand by the president who, without any kind of servile praise---a proceeding incompatible with the morale of our revolutionary movement---knows that he can count on our most profound respect and our most sincere support. (\emph{Solidaridad Obrera}, April 15, 1937, p. 12)
\end{quotation}

Companys, of course, managed to persuade the Stalinists to relinquish the demand for a pact, and on April 16, the ministerial crisis was ``resolved.'' The new cabinet, like its predecessor, provided a majority for the bourgeoisie and the Stalinists, and, of course, differed in nothing from the previous one.

The masses of the \CNT\ could not be so ``flexible.'' They had a heroic tradition of struggle to the death against capitalism. Even more insistently, the revival of the bourgeois state was taking place on their backs. Inflation and the uncontrolled manipulation of prices by the businessmen ``mediating'' between the peasantry and the city masses now led to perpendicular price rises. In this period the rise of prices is the leitmotif of all activity. The press is full of the problem. The condition of the masses was growing daily more intolerable, and the \CNT\ leaders showed them no way out.

Many voices now cried for a return to the traditional apoliticism of the \CNT. ``No More Governments!'' Local \CNT\ papers broke discipline and took up this refrain. It was counsel of unthinking despair.

Far more significant was the rise of the Friends of Durruti. In the name of the martyred leader, a movement rose which had assimilated the need for political life, but rejected collaboration with the bourgeoisie and reformists. The Friends of Durruti were organized to wrest leadership from the \CNT\ bureaucracy. In the last days of April, they plastered Barcelona with their slogans---an open break with the \CNT\ leadership. These slogans included the essential points of a revolutionary programme: all power to the working class, and democratic organs of the workers, peasants and combatants, as the expression of the workers’ power.

The Friends of Durruti represented a deep ferment in the libertarian movement. On April 1, a manifesto of the Libertarian Youth of Catalonia (\emph{Ruta}, April 1, 1937) had denounced the ``United Socialist Youth'' [Stalinists], who first assisted the revaluation of the Azaña stock-fallen so low in the first days of the revolution when he tried to flee the country-and who called to the Unified Catholic Youth and even to those who were sympathetic to fascism; stigmatized the bourgeois-Stalinist bloc as ``supporting openly all the intentions of the English and French governments to encircle the Spanish revolution;'' excoriated the counter-revolutionary assaults on the publishing houses and radio station of the \POUM\ in Madrid. 

It pointed out that:

\begin{quotation}
  \dots\ arms are denied to the Aragon front because it is definitely revolutionary, in order to be able afterward to throw mud at the columns operating on that front.
	
  \dots\ the Central Government boycotts Catalan economy in order to force us to renounce our revolutionary conquests.
	
  \dots\ the sons of the people are sent to the front, but for counter-revolutionary ends the uniformed forces are being kept in the rear.
  
  \dots\ they have gained ground for a dictatorship---not pro\-le\-ta\-ri\-an!---but bourgeois.
\end{quotation}

Clearly differentiating the Anarchist Youth from the \CNT\ ministers, the manifesto concluded:

\begin{quotation}
  We are firmly decided not to be responsible for the crimes and betrayals of which the working class is being made the object\dots\ We are ready to return, if that is necessary, to the underground struggle against the deceivers, against the tyrants of the people and the miserable merchants of politics.
\end{quotation}

An editorial in the same issue of Ruta declared:

\begin{quotation}
  Let not certain comrades come to us with appeasing words. We shall not renounce our struggle. Official automobiles and the sedentary life of the bureaucracy do not dazzle us.
\end{quotation} 

This from the official organization of the anarchist youth!

Not in a day or a month, however, does a regroupment take place. The \CNT\ had a long tradition and the discontent of its masses would evolve only at a slow pace into an organized struggle for a new leadership and a new programme. Particularly was this true because no revolutionary party existed to encourage this development.

\section*{The \POUM’s Answer to the Counter-Revolution}

An abyss was opening up between the \CNT\ leaders and the masses within the \CNT\ movement. Would the \POUM\ step into the breach and place itself at the head of the militant masses?

The prevalence of a wide tendency in \CNT\ ranks to go back to traditional apoliticism was an annihilating criticism of the \POUM, which had done nothing to win these workers to revolutionary-political life. Also with no aid from the \POUM\ leadership, a genuinely revolutionary current was crystallizing in the Friends of Durruti and the Libertarian Youth. If the \POUM\ was ever to strike out independently of the \CNT\ leadership, this was the moment!

The \POUM\ did nothing of the sort. On the contrary, in the ministerial crisis of March 26--April 16, it revealed that it had learned nothing whatever from its earlier participation in the Generalidad. The Central Committee of the \POUM\ adopted a resolution declaring:

\begin{quotation}
  There is needed a government that would canalize the aspirations of the masses, giving a radical and concrete solution to all the problems by way of the creation of a new order that would be guarantor of the revolution and of the victory at the front. This Government can only be a government formed by representatives of all the political and trade union organizations of the working class which would propose as immediate tasks the realization of the following programme. (\emph{La Batalla}, March 30)
\end{quotation}

The proposed fifteen-point programme is not a bad one---for a revolutionary government. But the absurdity of proposing it to a government which by definition includes the Stalinists and the Esquerra-controlled Union of Rabassaires (independent peasants) is indicated immediately by the last point on the programme: the convocation of a congress of delegates of the unions, peasants and combatants which will in turn elect a permanent workers’ and peasants’ government.

For six months the \POUM\ had been saying that the Stalinists were organizing the counter-revolution. How, then, could the \POUM\ propose to collaborate with them in the government and in convoking the congress? From this proposal the workers could only conclude that the \POUM’s characterization of the Stalinists had been so much factional talk, and would henceforth take no \POUM\ charges against the Stalinists as being seriously meant.

And Companys and his Esquerra? A new cabinet must receive a mandate from Companys, and the \POUM\ proposed no break with this law. Was it conceivable that Companys would agree to a government which would convoke such a congress? Here, too, the masses could only draw the conclusion that the \POUM’s declaration of the necessarily counter-revolutionary role of the Esquerra of Companys was not seriously meant.