\chapter{Why the Fascists Revolted}

\lettrineA{t dawn}, July 17, 1936, General~Franco{\indexFFranco} assumed command of the Moors and Legionnaires of Spanish Morocco, and issued a manifesto to the army and the nation to join him in establishing an authoritarian state in Spain. In the next three days, one by one, almost all of the fifty garrisons in Spain declared for fascism. The basic sections of the capitalists and landlords, having already participated in Franco’s conspiracy, fled into fascist-held territory or out of Spain either before or after the rising. It was clear immediately that this rising had nothing in common with the \emph{pronunciamento} movements whereby the Spanish army had so often supported one bourgeois faction against another. It was not a ``handful of generals,'' but the ruling class as a whole, which was directing its armed minions in an assault, above all, against the economic, political and cultural organizations of the working class.

Franco’s programme is identical in fundamentals with that of Mussolini and Hitler. Fascism\index{Fascism} is a special form of reaction, the product of the period of capitalist decline. To see this fully one has only to compare Franco’s regime with that of the monarchy. The last Alfonso’s record is a bloody account of massacres of peasants and workers, of terrorism and assassination of proletarian leaders. Yet side by side with the systematic measures of repression, the monarchy permitted a restricted existence to economic and political organizations of the working class and to municipal and national organs of parliamentary democracy. Even under the Primo~de~Rivera dictatorship (1923--1930), the Socialist~Party and the \textsc{ugt} led a legal existence; indeed, Largo~Caballero{\indexLCaballero}, head of the \textsc{ugt}, was a Councillor of State under Rivera. In other words, even the reactionary monarchy sought part of its mass support in the organized proletariat, through the mediation of reformist leaders like Prieto and Caballero. Similarly, a system of legal unions and social-democratic parties existed in the empires of Wilhelm and Franz Joseph. Even under Tsar~Nicholas there was a measure of legality for unions, cooperatives and the labor press, in which the Bolsheviks were able to work though themselves illegal: \emph{Pravda} had a circulation of 60,000 in 1912--1914.

As contrasted to these reactionary regimes, the special character of fascism consists in its extirpation of any and all independent organizations of the working class. Capitalism in decline finds impossible even the most elementary concessions to the masses. One by one, those capitalist countries which arrive at a complete impasse, take the road of fascism.

Italy, a ``victor'' in the World War, weakly developed in basic industries, could not compete with more advanced countries in the imperialist race for markets. Strangling in its economic contradictions, the capitalist class of Italy could find a way out only over the broken bones of the workers’ organizations. The hordes of the ``petty bourgeoisie gone mad,'' organized and uniformed by Mussolini,{\indexBMussolini} trained as hoodlums, were finally unleashed for the special task of crushing the workers’ organizations.

The bourgeoisie does not light-mindedly take to fascism. The Nazi movement of Germany had almost no bourgeois support in its putsch of 1923. In the ensuing decade, it secured financial support only from a few individual capitalists until 1932. The bourgeoisie of Germany hesitated for a long time before it accepted the instrumentality of Hitler{\indexAHitler}; for fifteen years it preferred to lean on the social-democratic leaders. But at the height of the world economic crisis, technically advanced Germany, handicapped by the Versailles Treaty in its imperialist conflicts with England, France and America, could ``solve'' its crisis temporarily, in capitalist terms, only by destroying the workers’ organization which had existed for three-fourths of a century.

Fascism is that special form of capitalist domination which the bourgeoisie finally resorts to when the continued existence of capitalism is incompatible with the existence of organized workers. Fascism is resorted to when the concessions, which are a product of the activities of trade unions and political parties of labor, become an intolerable burden on the capitalist rulers, hence intolerable to the further existence of capitalism. For the working class, at this point, the issue is inexorably posed for immediate solution: either fascism or socialism.

Spanish capitalism had arrived at this point when Franco revolted. His movement, though incorporating the remnants of Spain’s feudal aristocracy, is no more ``feudal'' in basic social character than was that of Mussolini or Hitler.

The chief industry of Spain, agriculture, accounting for over half the national income, almost two-thirds of exports and most, of the government’s internal revenue, with 70 per cent of the population living on the land, was in desperate straits. The division of the land was the worst in Europe: one-third owned by great landowners, in some cases in estates covering half a province; another third held by more numerous owners but also in large estates; only one-third owned by peasants\index{Peasantry}, and most of this in primitively equipped farms of five hectares\footnote{5 ha \( \approx \) 12 acres} or less of extraordinarily dry, poor land, insufficient to support their families, and necessitating day labor on the big estates to eke out an existence. Thus, most of the five million peasant families were dependent on sharecropping or employment on the big estates.

Spanish agriculture\index{Agricultural crisis} was conducted by primitive methods. Its yield per hectare was the lowest in Europe. Increased productivity required capital investment in machinery and fertilizers, employment of technicians, re-training of the peasants. From the landowners’ standpoint it was cheaper to continue primitive methods at the expense of the peasantry. The one recent period of good prices for produce, the war years, 1914--1918, which gave Spanish agriculture a temporary opportunity to profiteer in the word market, instead of being used to improve the land, was capitalized into cash via mortgages obtained by the landowners. Driven out of the world market after the war, Spanish agriculture collapsed. The general agricultural crisis, first preceding and then part of the world crisis, aggravated by the tariff barriers raised against Spanish agriculture by England and France, led to widespread unemployment and starvation.

Precisely at the depth of the crisis, in 1931, the rise of the republic gave a new impetus to the organization of agricultural workers’ unions. The resultant wage raises seem pitiful enough. A good wage was six pesetas (seventy-five cents) a day. But even this was a deadly menace to the profits of the Spanish landowners, in the epoch of the decline of European agriculture. The great plains of South America and Australia were providing wheat and beef to Europe at prices which were dealing European agriculture a blow incomparably more serious than that dealt by the produce of North America during the epoch of capitalist expansion. \emph{Thus, the existence of agricultural workers’ unions and peasants’ organizations was incompatible with the further existence of landed capitalism in Spain}.

The landowners got a breathing space during the \emph{bienio negro}{\indexbienionegro}, the ``two black years,'' of September 1933 to January 1936, when the reactionary governments of Lerroux--Gil Robles{\indexALerroux\indexGRobles} terrorized the masses and put down the revolt of October 1934. During that period, day wages on the land fell to two or three pesetas. But the masses soon rallied. Gil Robles’ attempt to build a mass fascist organization failed, both through his own ineptitude and under the blows of the workers. The Asturian Commune\index{Asturian Commune} of October 1934, though crushed by Moors and Legionnaires, became the inspiration of the masses, and Lerroux--Gil Robles gave way to the Popular Front in February 1936, rather than wait for a more decisive onslaught by the proletariat. The agrarian workers and peasants built even more formidable unions from February to July 1936, and the precarious condition of agricultural profits drove the landowners and their allies, the Catholic hierarchy and the banks, to a speedy resort to arms to destroy the workers’ organizations.

\dinkus

The capitalists in industry and transportation were likewise at an~impasse.

The era of expansion of Spanish industry had been short: 1898--1918. The very development of Spanish industry in the war years became a source of further difficulties. The end of the war meant that Spain’s industry, infantile and backed by no strong state power, soon fell behind in the imperialist race for markets. Even Spain’s internal market could not long be preserved for her own industry. Primo~de~Rivera’s attempt to preserve it by insurmountable tariff walls brought from France and England retaliation against Spanish agriculture. The resultant agricultural crisis caused the internal market for industry to collapse. In 1931, this country of twenty-four millions had nearly a million unemployed workers and peasants, heads of families; before the end of 1933 the number was a million and a half.

With the end of the \emph{bienio negro}, the workers’ economic struggles took on extraordinary scope. Conscious of having freed themselves from the domination of Gil Robles by their own efforts, the masses did not wait for Azaña to fulfill his promises. In the four days between the elections of February 1936 and Azaña’s hasty assumption of the premiership, the masses effectively carried out the amnesty by tearing open the jails. Nor did the workers wait for the government decree, and for the decision as to its constitutionality---which came from the Court of Constitutional Guarantees\index{Constitution of Spain} only on September 6, nearly two months after Franco’s revolt!---to get back the jobs of those dismissed after the October 1934 revolt. In the shops and factories the workers took along those dismissed, and put them back to work. Then, beginning with a general strike on April 17, 1936, in Madrid, there began a great movement of the masses, often including political demands, but primarily for better wages and conditions.

We can only roughly indicate the magnitude of the great strike wave. The strikes covered both the cities and the rural districts. Every city and province of any importance had at least one general strike between February and July 1936. Nearly a million were on strike on June 10, a half million on June 20, a million on June 24, over a million during the first days of July.

Spanish capitalism could scarcely hope to solve its problems by expanding its markets for manufactured goods. That road was closed to it externally by the great imperialist powers. Internally, the only way to expand was to create a prosperous landed peasantry, but that meant dividing the land. The city capitalist and landed proprietor were often one and the same person, or bound together by family ties. In any event, the summit of Spanish capitalism, the banks, were inextricably bound up with the interests of the landowners, whose mortgages they held. No real road of development was open to Spanish capitalism. But it could temporarily solve its problems in one way: by destroying the trade unions which were endangering its profits.

Bourgeois democracy is that form of capitalist state which leans on the support of the workers, secured through the reformist leaders. The capitalists of Spain concluded that democracy was intolerable, and that meant that bourgeois democracy and reformism were finished in Spain.

Mussolini declared he had saved Italy from Bolshevism. Unfortunately, the truth is that the workers’ post-war upsurge had already receded, thereby facilitating Mussolini’s assumption of power. Hitler said the same, at a time when the workers were hopelessly divided and disoriented. Franco had need of the same myth for justifying his resort to arms. What was true, in Italy, Germany and now in Spain, is that democracy could no longer exist. Precisely the fact that fascism had to seize power, even though there was no immediate danger of a proletarian revolution, is the most conclusive evidence that democracy was finished.

Franco’s rebellion left only two alternatives: \emph{either fascism would conquer, or the working class, rallying the peasantry by giving it the land, would destroy fascism and with it the capitalism in which it is rooted.}\index{Socialism or barbarism}

The Stalinists and social-democrats, seeking theoretical justification for their collaboration with the liberal bourgeoisie, declare that the roots of fascism in Spain are feudal.\index{Feudalism} For the Stalinists, this is an entirely new theory, concocted \emph{ad hoc}. Spanish fascism is no more feudal than is Italian. The backwardness of industry in both countries cannot be overcome within the capitalist framework, since neither can compete with the advanced industrial countries in the epoch of declining world markets. They could only secure temporary stabilization by cutting their labor costs below the European level, and to do this required crushing every form of labor organization. Spanish agriculture is backward and ``feudal'' in its operating methods. But land has been bought, sold and mortgaged, like any other commodity, for two centuries. Hence, the land question is a \emph{capitalist} question.
\noclub

The Stalinist journalist, Louis Fischer,{\indexLFischer} writes:

\begin{quotation}
  \sloppy
  
  Strangely enough, Spain’s small industrialist class supported the reactionary position taken by the landlords.\index{Landlords} The industrialists should have welcomed a land reform which would create a home market for their goods. But they believed that more than economics was involved. They feared that the granting of land to the peasantry would rob the owning classes of political power.
  
  The manufacturers, therefore, who should have encouraged the republic in its attempts to stage a peaceful revolution which would have enriched the country, actually leagued themselves with the backward looking landlords to prevent all amelioration and reform. (\emph{The War in Spain}, published by \emph{The Nation}.)
\end{quotation}

It does not occur to Fischer to wonder whether landlord and capitalist are not often one and the same, or of the same family, or whether the manufacturer, dependent on the banks, is not fearful for the banks’ mortgages on the land. But even as Fischer poses the problem, the answer is clear. The manufacturer fears the diminution of the political power of the owning classes. Why? Because the weakening of the police power permits the workers in his factory to organize and make inroads into his profits. Fischer’s own laboured explanation gives the show away. Spanish fascism is the weapon not of ``feudalism'' but of capitalism. It can be fought successfully by the working class and the peasantry, and by them alone.