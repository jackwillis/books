\chapter{The Conquest of Aragon}

\index{Battle}\index{Aragon}
\lettrineT{he fertile} province of Aragon was the living embodiment of victorious struggle against fascism. It was the only province actually invested by the fascists and then conquered from them by force of arms. It was especially the pride of the Catalan masses, for they had saved Aragon. Within three days of the victory in Barcelona, the \CNT\ and \POUM\ militias were off for Aragon. The \PSUC\ then was small and contributed little or nothing. Imperishable names of battles there---Montearagón, Estrecho Quinto, etc.---were associated solely with the \CNT\ and \POUM\ heroes who had won them. It was in the victorious conquest of Aragon that Durruti acquired his legendary fame as a military leader, and the forces he brought to the defense of Madrid in November were the picked troops whose victorious morale had been welded in Aragon victories.

\index{Fortress of the revolution@``Fortress of the revolution''}
Not the least of the reasons for the successes in Aragon had been that, under Durruti’s leadership, the militias marched as an army of social liberation. Every village wrested from the fascists was transformed into a fortress of the revolution. The militias sponsored elections of village committees, to which were turned over all the large estates and their equipment. Property titles, mortgages, etc., went into bonfires. Having thus transformed the world of the village, the \CNT--\POUM\ columns could go forward, secure in the knowledge that every village behind them would fight to the death for the land that was now theirs.

Backed by their success in freeing Aragon, the anarchists met with little resistance there from the bourgeois--Stalinist bloc in the first months. Aragon’s municipal councils were elected directly by the communities. The Council of Aragon was at first largely anarchist. When Caballero’s cabinet was formed, the anarchists agreed to give representation to the other anti-fascist groups in the Council, but up to the last days of its existence the masses of Aragon were grouped around the libertarian organizations. The Stalinists were a tiny and uninfluential group.

\index{Collectivization}\index{Cooperatives}
At least three-fourths of the land was tilled by collectives. Of four hundred collectives, only ten adhered to the \UGT. Peasants desiring to work the land individually were permitted to do so, provided they employed no hired labor. For family consumption, cattle were owned individually. Schools were subsidized by the community. Agricultural production increased in the region from thirty to fifty per cent over the previous year, as a result of collective labor. Enormous surpluses were voluntarily turned over to the government, free of charge, for use at the front.

\index{Anarchism}\index{Accounting}\index{Family wage}
Libertarian principles were attempted in the field of money and wages. Wages were paid by a system of coupons exchangeable for goods in the cooperatives. But this was merely pious genuflection to anarchist tradition, since the committees, carrying on sale of produce and purchase of goods with the rest of Spain, perforce used money in all transactions, so that the coupons were merely an internal accounting system based on the money held by the committees. Wages were based on the family unit: a single producer was paid the equivalent of 25~pesetas; a married couple with only one working, 35~pesetas, and four pesetas weekly additional for each child.

This system had a serious weakness, particularly while the rest of Spain operated on a system of great disparity in wages between manual and professional workers, since that prompted trained technicians to migrate from Aragon.

For the time being, however, ideological conviction, inspiring the many technicians and professionals in the libertarian organizations, more than made up for this weakness. Granted that, with stabilization of the revolution, a transitional period of higher wages for skilled and professional workers would have had to be instituted. But the Stalinists who had the effrontery to contrast the Aragon situation with the monstrous disparity of wages in the Soviet Union, appeared to have forgotten completely that the family wage---which is the essence of Marx’s ``to each according to his needs''---was a goal toward which to strive, from which the Soviet Union is infinitely further away under Stalin than under Lenin and Trotsky.

\index{Aragon!Council of Aragon}\index{Stalinism}
The anarchist majority in the Council of Aragon led in practice to the abandonment of the anarchist theory of the autonomy of economic administration. The Council acted as a centralizing agency. The opposition was in such a hopeless minority within Aragon, and the masses were so wedded to the new order, that there was no record of a single Stalinist mass meeting in Aragon in direct opposition to the Council. Many joint meetings were celebrated, with Stalinist participation, including one as late as July 7, 1937. Neither at these meetings nor elsewhere in Aragon did the Stalinists repeat the calumnies which the Stalinist press elsewhere was spreading, in order to prepare the ground for an invasion.

\index{Internationalism}
Many workers’ leaders from abroad saw Aragon and praised it: among them Carlo Rosselli, the Italian anti-fascist leader, serving as a commandant on the Aragon front (on leave in Paris when he and his brother were assassinated by the Italian fascists). The prominent French socialist, Juin, wrote strong praise of Aragon in \emph{Le Peuple.} \emph{Giustizia~e~Liberta,} the leading Italian anti-fascist organ, said of the Aragon collectives:

\begin{quotation}
  The manifest benefits of the new social system strengthened the spirit of solidarity among the peasants, arousing them to greater efforts and activity.
\end{quotation}

\begin{sloppypar}
The manifest benefits of social revolution, however, scarcely weighed in the balance against the grim necessities of the bourgeois--Stalinist programme for stabilizing a bourgeois regime and winning the favor of Anglo--French imperialism. The preconditions of such favor was destruction of every vestige of social revolution. But the masses of Aragon were united. The destruction must, therefore, come from outside. Once the Negr\'in government came to power, a terrific barrage of propaganda against Aragon was laid down in the bourgeois and Stalinist press. And, after three months of this preparation, the invasion was launched.
\end{sloppypar}

\index{Political repression}\index{Reprivatization}
On August~11, the government decreed the dissolution of the Council of Aragon. Instead was appointed a governor-general ``with the faculties which the prevailing legislation attributes to civil governors''~-- legislation from the days of reaction. The governor-general, Mantec\'on, proved only a figurehead, however. The real job was done by military forces under the leadership of the Stalinist, Enrique Lister.

\index{Censorship}\index{Ascaso, Joaquin@Ascaso, Joaqu\'in}\index{Political prisoners}
One of the manufactured heroes of the Stalinists,\kn\kn\footnote{`\CNT' published his picture with the title, ``Hero of many battles. We know it because the Communist Party has told us so''~-- irony was the only way of getting past the censor.} Lister marched his troops into the rear of Aragon. The municipal councils elected directly by the population were forcibly dissolved. Collectives were broken up and their leaders jailed. As with the \POUM\ prisoners in Catalonia, not even the Governor-General knew the whereabouts of the members of the \CNT\ Regional Committee arrested by Lister’s bands. They had, indeed, carried safe-conducts\index{Safe conduct} from the Governor-General but that did not save them.

Joaqu\'in Ascaso, President of the Council of Aragon, was jailed on the charge \dots\ of stealing jewels! The government censorship forbade the \CNT\ press to publish the news of Ascaso’s imprisonment, refused to divulge the place of his incarceration, and from their foully reactionary viewpoint, they were right. For Ascaso was flesh and blood of the masses, as the dead Durruti had been, and they would have torn the jail down with their bare hands.

Suffice it to say that the official \CNT\ press---none too anxious to arouse the masses---compared the assault on Aragon with the subjection of Asturias by L\'opez~Ochoa in October 1934.

\index{Libel}\index{Stalinism}\index{Aragon!Council of Aragon}
To justify the rape of Aragon, the Stalinist press published fantastic tales. \emph{Frente Rojo} wrote:
\begin{quotation}
  Under the regime of the extinguished Council of Aragon, neither the citizens, nor property, could count on the least guarantee\dots
  
  The government will find in Aragon gigantic arsenals of arms and thousands of bombs, hundreds of the latest model machine-guns, cannons and tanks, reserved there, not to fight fascism on the battle fronts, but the private property of those who wished to make of Aragon a bastion from which to fight the government of the republic\dots
  
  Not a peasant but had been forced to enter the collectives. He who resisted suffered on his body and his little property the sanctions of terror. Thousands of peasants have emigrated from the region, preferring to leave the land than to endure the thousand methods of torture of the Council\dots
  
  The land was confiscated, and rings, lockets, and even the earthen cooking pots were confiscated. Animals were confiscated, grain and even the cooked food and wine for home consumption\dots
  
  In the Municipal Councils there were installed known fascists and Falangist chiefs. Holding union cards they officiated as mayors and councillors, as agents of the public order of Aragon, bandits by origin making a profession and a government regime of banditry.
\end{quotation}

Was anybody expected seriously to believe this nonsense? The police mentality of the Stalinists was evident in the alibi that an insurrection was preparing. Unfortunately, it was not true. The arms? The Aragon front had come under complete government control on May~6, with a Stalinist party member, General Pozas,\index{Pozas, Sebastian} in supreme command. Prior to that the \CNT, \POUM, \FAI\ press from October~1936 on had abounded in long and precise complaints that the Aragon front was being deprived of arms, and that the armed guard of the Aragon collectives---actually, with the front irregular and shifting, part of the frontline defenses---were dangerously stripped of arms. For eight months these charges had been made, from press, platform, and radio and with it the charge that Russian aid was being conditioned by Stalinist control of the disposition of incoming arms. The Stalinists had met these specific charges with dead silence. Now, in the pogrom atmosphere of August~1937, their answer was that arms were there! No one was, nor could be expected to believe this poppycock, not even the party members.

But the charges do not require serious rebuttal. For on September~18, the man who presumably had been the chief culprit, who had terrorized, installed, fascists, etc., etc., etc., Joaqu\'in Ascaso, was released from prison. If the Stalinists were ready to prove their charges against Ascaso even in their corrupted courts, why did they not do so? The answer is: the charges were balderdash. What was terribly real, however, was the destruction of the Aragon collectives.

\index{Bates, Ralph}
After the bourgeois--Stalinist bloc conquered Aragon and the story of their invasion began to seep out to the world labor movement where the Stalinists did not dare to attempt to repeat their fantastic charges, they adopted a new tack which sought to move away from these charges to the assertion that the dissolution of the Council was required in order to re-organize the Aragon front. Thus Ralph Bates wrote,

\begin{quotation}
  There have been exaggerated charges against the Council~of~Aragon, but I think the following can be substantiated by detailed evidence: the wholesale application of extreme measures in land and social reform had confused and even antagonized non-anarchist peasantry and workers; anarchist control of village military committees had undoubtedly hampered efficient conduct of operations\dots
  
  The problem, therefore, was to bring this strip of Aragon under the control of the Valencia government, as part of a campaign to reform the Aragon military forces. (\emph{New Republic,} October 27, 1937).
\end{quotation}

This latest alibi had two functions: first, to get away from the preposterous charges on which the dissolution had first been justified; second, to cover up the fact that, although the central government had been in complete control of the Aragon front since May, its so-called offensives had been fiascos. The infinite infamy of all this will become apparent if we now turn to the military question itself and examine the Aragon front as part of the whole program of military strategy.