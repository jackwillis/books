\chapter{The Bourgeois “Allies” in the People's Front}

\lettrineT{he stake of} the workers’ parties and unions in the struggle against fascism was clear: their very existence was at stake. As Hitler and Mussolini before him, Franco would physically destroy the leadership and active cadres of their organizations and leave the workers in forced disunity, each an atom at the mercy of concentrated capital. The struggle against fascism was, therefore, a life and death question not only to the masses of the workers but also even to the reformist leaders of the workers. But this is not the same as to say that these leaders knew how to fight against fascism. Their most fatal error was their assumption that their bourgeois allies in the People’s Front were equally vitally concerned to fight against fascism.

Azaña's \emph{Republican Left}, Martinet Barrio's \emph{Republican Union}, Companys' \emph{Catalan Left}, had compacted with the Socialist and Communist parties and the \textsc{ugt}---with the tacit consent of the anarchists, whose masses voted for the joint tickets---in the elections of February 16, 1936. The Basque Nationalists had also joined. These four bourgeois groups, therefore, found themselves on the other side of the barricades from the big bourgeoisie on July 17. Could they be depended on to co-operate loyally in the struggle against fascism?

We said no, because no vital interest of the liberal bourgeoisie was menaced by the fascists. The workers were in danger of losing their trade unions, without which they would starve. What comparable loss was faced by the liberal bourgeoisie? Undoubtedly, in a totalitarian state, the professional politicians would have to find other professions; the liberal bourgeois press would go under (if we grant that the bourgeois politicians and journalists would not go over to Franco entirely). Both Italy and Germany have demonstrated that fascism refuses to become reconciled to individual democratic politicians; some are jailed, others must emigrate. But all these constitute minor inconveniences. The basic strata of the liberal bourgeoisie go on as before the advent of fascism. If they do not share the special favours extended by the fascist state to those capitalists who had joined the fascists before victory, they do share the advantages of low wages and curtailed social services. Only to the same extent as all other capitalists are they subject to the fascist exactions, via party or government, which are the stiff price which capitalism pays for the services of fascism. The liberal bourgeoisie of Spain had only to look at Germany and Italy to be reassured for the future. While the trade union officials have been extirpated, the liberal bourgeoisie has found elbow room in which to be assimilated. What is at work here is a class criterion: fascism is the enemy primarily of the working class. \emph{Hence, it is absolutely false and fatal to assume that the bourgeois elements in the Popular Front have a crucial stake in prosecuting seriously the struggle against fascism.}

Second, our proof that Azaña, Barrio, Companys and their ilk cannot be loyal allies of the working class, rested not merely on deductive analysis, but upon concrete experience---the record of these worthies. Since the socialists and Stalinists in the People’s Front have suppressed the facts about their allies, we must give some space to this question.

From 1931 to 1934, the Comintern called Azaña a fascist, which was certainly incorrect, although it accurately pointed to his systematic oppression of the masses. As late as January 1936, the Comintern said of him:

\begin{quotation}
  The Communist Party knows the danger of Azaña just as well as the socialists who collaborated with him when he was in power. They know that he is an enemy of the working class\ldots\ But they also know that the defeat of the \textsc{ceda} (Gil Robles) would automatically bring with it a certain amount of relief from the repression, for a time at least. (\emph{Inprecorr}, Vol. 15, p. 762.)
\end{quotation}

The last phrase is an admission that repression will come from the direction of Azaña himself. And it did come, as Josh Diaz, Secretary of the Communist Party, was compelled to admit just before the civil war broke out:

\begin{quotation}
  The government, which we are loyally supporting in the measure that it completes the pact of the Popular Front, is a government that is commencing to lose the confidence of the workers, and I say to the left republican government that its road is the wrong road of April 1931. (\emph{Mundo Obrero}, July 6, 1936.)
\end{quotation}

One must recall the ``wrong road of April 1931'' to realize what an admission the Stalinists were making after all their attempts to differentiate the coalition government of 1931 from the Popular Front government of 1936. The coalition of 1931 had promised land to the peasants and gave them none because the land could not be divided without undermining capitalism. The coalition of 1931 had refused the workers unemployment relief. Azaña, as Minister of War, had not touched the reactionary officer caste of the army, and had enforced the infamous law under which all criticism of the army by civilians was an offence against the state. As Premier, Azaña had left the swollen wealth and power of the church hierarchy intact. Azaña had left Morocco in the hands of the Legionnaires and Moorish mercenaries. Only against the workers and peasants had Azaña been stern. The annals of 1931--33 are the annals of his government’s repressions of the workers and peasants. Elsewhere\endnote{\emph{The Civil War in Spain}, September 1936, Pioneer Publishers.}\label{en:CivilWarReference} I have told this story at length.

Azaña, as \emph{Mundo Obrero} admitted, proved no better as head of the Popular Front government of February--July 1936. Again his regime rejected the idea of distribution of the land, and put down the peasantry when it attempted to seize it. Again the church remained in full control of its great wealth and power. Again Morocco remained in the hands of the Foreign Legionnaires until they finally took it over completely on July 17. Again strikes were declared illegal, modified martial law imposed, workers’ demonstrations and meetings broken up. Suffice it to say that in the critical last days, after the assassination of the fascist leader, Calvo Sotelo, the working class headquarters were ordered closed. \emph{The day before the fascist outbreak the labour press appeared with gaping white spaces where the government censorship had lifted out editorials and sections of articles warning against the \emph{coup d’\'etat}!}

In the last three months before July 17, in desperate attempts to stop the strike movement, hundreds of strikers were arrested en masse, local general strikes declared illegal and regional headquarters of the \textsc{ugt} and \textsc{cnt} closed for weeks at a time.

The most damning insight into Azaña was provided by his attitude toward the army. Its officer caste was disloyal to the core toward the republic. These pampered pets of the monarchy had seized every opportunity since 1931 to wreak bloody vengeance against the workers and peasants upon whom the republic rested. The atrocities they committed in crushing the revolt of October 1934 were so horrible that criminal punishment of those responsible was one of Azaña’s campaign promises. But he brought not a single officer to trial in the ensuing months. Mola, director of Public Safety of Madrid under the dictatorship of Berenguer-Mola who had fled on the heels of Alfonso while the streets echoed with the masses’ cries of ``¡Abajo Mola!''---this Mola was restored to his generalship in the army by Azaña, and despite his close association with Gil Robles in the \emph{bienio negro}, was military commander of Navarre at the moment of the fascist revolt, and became chief strategist of the Franco armies. Franco, Goded, Queipo de Llano---all had similarly malodorous records of disloyalty to the republic and yet Azaña left the army in their hands. More, he demanded that the masses submit to them.

Colonel Julio Mangada, now fighting in the anti-fascist forces, who had been court-martialled and driven out of the army by these generals because of his republicanism, is authority for the fact that he had repeatedly informed Azaña, Martinez Barrio, and other republican leaders of the plans of the generals. In April 1936, Mangada published a thoroughly documented pamphlet which not only exposed the fascist plot, but proved conclusively that President Azaña was fully informed of the plot when, on March 18, 1936, upon the demand of the general staff, his government gave the army a clean bill of health. Referring to ``rumours insistently circulating concerning the state of mind of the officers and subalterns of the army,'' ``the Government of the Republic has learned with sorrow and indignation of the unjust attacks to which the officers of the army have been subjected.'' Azaña’s cabinet not only repudiated these rumours, describing the military conspirators as ``remote from all political struggle, faithful servitors of the constituted power and guarantee of obedience to the popular will,'' but declared that only a criminal and tortuous desire to undermine it [the army] can explain the insults and verbal and written attacks which have been directed against it.'' And finally, ``the Government of the Republic applies and will apply the law against anyone who persists in such an unpatriotic attitude.''

No wonder that the reactionary leaders praised Azaña. On April 3, 1936, Azaña made a speech promising the reactionaries that he would stop the strikes and seizures of the land. Calvo Sotelo praised it: ``It was the expression of a true conservative. His declaration of respect for the law and the constitution should make a good impression on public opinion.''

``I support ninety per cent of the speech,'' declared the spokesman of Gil Robles’ organization. ``Azaña is the only man capable of offering the country security and defence of all legal rights,'' declared Ventosa, spokesman of the Catalan landowners. They praised Azaña, for he was preparing the way for them.

Though the army was ready to rebel in May 1936, many reactionaries doubted whether this was possible as yet. Azaña pressed upon them his solution: let the reformist leaders stop the strikes. His offer was accepted. Miguel Maura, representing the extreme right industrialists and landowners, called for a strong regime of ``all republicans and those socialists not contaminated by revolutionary lunacy.'' So, having been elevated to the presidency, Azaña offered the premiership to the right-wing socialist, Prieto. The Stalinists, the Catalan Esquerra, the Republican Union of Barrio, as well as the reactionary bourgeoisie, supported Azaña’s candidate.

The left socialists, however, prevented Prieto from accepting\ldots\ For the reactionary bourgeoisie, the Prieto premiership would have been, at most, a breathing space to prepare further. Having, failed to secure it, they plunged into civil war.

Such was the record of Azaña’s Republican Left. That of the other liberal-bourgeois parties was, if anything, worse. Companys’ Catalan Esquerra had ruled Catalonia since 1931. Its Catalonian nationalism served to hold in leash the more backward strata of peasants while Companys used armed force against the \textsc{cnt}. On the eve of the October 1934 revolt he had reduced the \textsc{cnt} to semi-legal status, with hundreds of its leaders jailed. It was this situation which had moved the \textsc{cnt} so unwisely to refuse to join the revolt against Lerroux-Gil Robles, declaring that Companys was just as much a tyrant; while Companys, faced with the choice between arming the workers and knuckling under to Gil Robles, had chosen the latter.\endnote{The \emph{Estat Catala}, a split-off from the Esquerra, combining extreme separatism with anti-labour hooliganism, had provided its khaki-shirt members for strike-breaking; had disarmed workers during the October 1934 revolt. This organization, too, after July 19, turned up in the ``anti-fascist'' camp!}

As for the Republican Union of Martinez Barrio, it was nothing more than reconstituted remnants of Gil Robles’ allies, Lerroux’s Radicals. Barrio himself had been Lerroux’s chief lieutenant, and had served as one of the premiers in the \emph{bienio negro}, putting down with great cruelty an anarchist rising in December 1933. He had perspicaciously left the sinking ship of the Radicals when it became clear that the crushing of the October 1934 revolt had failed to stem the masses, and made his debut as an ``anti-fascist'' in 1935 by signing a petition for amnesty for political prisoners. When Lerroux fell after a financial scandal his following turned to Barrio.

The fourth of the bourgeois parties, the Basque Nationalists, had collaborated closely with the extreme reactionaries of the rest of Spain until Lerroux had sought to curb ancient provincial privileges. Catholic, led by the big landowners and capitalists of the four Basque provinces, the Basque Nationalists had supported Gil Robles in crushing the Asturian Commune of October 1934. They were from the first uncomfortable in their alliance with the workers’ organizations. That they did not go over to the other side of the barricades immediately, is to be explained by the fact that the Biscayan region was a traditional sphere of influence of Anglo-French imperialism and as such hesitated to enter the alliance with Hitler and Mussolini.
\noclub

These, then, were the ``loyal,'' ``reliable,'' ``honourable'' allies of the Stalinist-reformist leaders in the struggle against fascism. If in peacetime the liberal bourgeoisie had refused to touch the land, the church or the army, because they did not want to undermine the foundations of private property, was it conceivable that now, with arms in hand, the liberal bourgeoisie would loyally support a war to the finish against reaction? If Franco’s army was crushed, what would happen to the liberal bourgeoisie which in the last analysis had maintained its privileges only because of the army? Precisely because of these considerations, the Franco forces moved boldly, taking it for granted that Azaña and Companys would come to heel. Precisely because of these considerations, Azaña and the liberal bourgeoisie \emph{did attempt to come to terms with Franco}.

The Stalinists and reformists, compromised by their People’s Front policy, have connived with the liberal bourgeoisie in suppressing almost completely from the outside world the cold facts which reveal the treachery of which Azaña and his associates were guilty in the first days of the revolt. But here are the indisputable facts.

On the morning of July 17, 1936, General Franco, having seized Morocco, radioed his manifesto to the Spanish garrisons, instructing them to seize the cities. Franco’s communications were received at the naval station near Madrid by a loyal operator and promptly revealed to the Minister of the Navy, Giral. But the government did not divulge the news in any form until the morning of the 18th and then it issued only a reassuring note:
\nowidow

\begin{quotation}
  The Government declares that the movement is exclusively limited to certain cities of the protectorate Zone [Morocco] and that nobody, absolutely nobody on the Peninsula [Spain], has added to such an absurd undertaking.
\end{quotation}

Later that day, at 3 \textsc{pm}, when the government had full and positive information of the scope of the rising, including the seizure of Seville, Navarre and Saragossa, it issued a note which said:

\begin{quotation}
  The Government speaks again in order to confirm the absolute tranquillity of the whole Peninsula. The Government acknowledges the offers of support which it has received [from the workers’ organizations] and, while being grateful for them, declares that the best aid that can be given to the Government is to guarantee the normality of daily life, in order to set a high example of serenity and of confidence in the means of the military strength of the State.
  
  Thanks to the foresighted means adopted by the authorities, a broad movement of aggression against the republic may be deemed to have been broken up; it has found no assistance in the Peninsula and has only succeeded in securing followers in a fraction of the army in Morocco\ldots
  
  These measures, together with the customary orders to the forces in Morocco who are labouring to overcome the rising, permit us to affirm that the action of the Government will be sufficient to re-establish normality. (\emph{Claridad}, July 18, 1936.)
\end{quotation}

This incredibly dishonest note was issued to justify the government’s refusal to arm the workers, as the trade unions had requested. But this was not all. At 5:20 and again at 7:20 \textsc{pm}, the government issued similar notes, the last declaring that ``in Seville\ldots there were acts of rebellion by the military elements that were repelled by the forces in the service of the government.'' Seville had then been in the hands of Queipo de Llano for most of the day.

Having deceived the workers about the true state of affairs, the cabinet went into an all-night conference. Azaña had Premier Casares Quiroga, a member of his own party, resign, and replaced him with the more ``respectable'' Barrio, and the night was spent hunting up bourgeois leaders outside the Popular Front who could be induced to enter the cabinet. With this rightist combination, Azaña made frantic attempts to contact the military leaders and come to an understanding with them. The fascist leaders, however, took the overtures as a sure sign of their victory, and refused Azaña any form of face-saving compromise. They demanded that the republicans step aside for an open military dictatorship. Even when this became known to Azaña and the cabinet ministers, they took no steps to organize resistance. Meanwhile, garrison after garrison, apprised of the government’s paralysis, took heart and unfurled the banner of rebellion.

Thus, for two decisive days, the rebels marched on while the government besought them to save its face. It made no move to declare the dissolution of rebellious regiments, to declare the soldiers absolved from obeying their officers. The workers, remembering the \emph{bienio negro}, remembering the fate of the proletariat of Italy and Germany, were clamouring for arms. Even the reformist leaders were knocking at the doors of the presidential palace, beseeching Azaña and Giral to arm the workers. In the vicinity of the garrisons, the unions had declared a general strike to paralyze the rebellion. But folded arms would not be sufficient to face the enemy. Grim silence enveloped the Montana barracks in Madrid. The officers there, in accordance with the plan of the rising, were waiting for the garrisons surrounding Madrid to reach the city, when they would join forces. Azaña and Giral and their associates waited helplessly for the blow to fall.

And, indeed could it be otherwise? The camp of Franco was saying: We, the serious masters of capital, the real spokesmen of bourgeois society, tell you that democracy must be finished, if capitalism is to live. Choose, Azaña, between democracy and capitalism. What was deeper in Azaña and the liberal bourgeoisie? Their ``democracy'' or their capitalism? They gave their answer by bowing their heads before the onward marching ranks of fascism.

On the afternoon of July 18, the chief worker-allies of the bourgeoisie, the National Committees of the Socialist and Communist parties, issued a joint declaration:

\begin{quotation}
  The moment is a difficult one, but by no means desperate. The Government is certain that it has sufficient resources to overcome the criminal attempt\ldots\ In the eventuality that the resources of the Government be not sufficient the republic has the solemn promise of the Popular Front, which gathers under its discipline the whole Spanish proletariat, resolved serenely and dispassionately to intervene in the struggle as quickly as its intervention is to be called for\ldots\ The Government commands and the Popular Front obeys.
\end{quotation}

But the government never gave the signal!

Fortunately, the workers did not wait for it.