\chapter{The Conquest of Catalonia}

\lettrineO{n May 5,} Catalan autonomy\index{Catalonia!Autonomy} had ceased to exist. The central government had taken over the Catalan Ministries of Public Order and Defense.\index{Ministries!Public Safety, Catalonia} Caballero’s{\indexLCaballero} delegate in Barcelona had broadcast:

\begin{quotation}
  From this moment, all the forces are at the orders of the central government\dots. These do not consider any union or antifascist organization as an enemy. There is no other enemy than the fascists.
\end{quotation}

But a week later, the Ministries of Defense and Public Order were surrendered by Caballero’s delegate to the representatives of Negr\'in--Stalin, and the pogrom began in earnest. The \POUM\indexPOUM\ went down with hardly a ripple. The \PSUC\ opened a monstrous campaign against it identical in language, slogans, etc., with the witch hunts of the Soviet bureaucracy before the Moscow trials. \emph{The Trotskyists in the \POUM\ have organized the latest insurrection on orders from the German and Italian secret police.} The \POUM’s answer to the \PSUC\ was \dots\ to institute a libel\index{Libel} suit against the Stalinist editors in a court filled with bourgeois and Stalinist judges and officials!

On May 28, \emph{La Batalla} was suppressed\index{Censorship} permanently and the \POUM\ radio seized. The Friends of Durruti\index{Friends of Durruti} headquarters were occupied and the organization outlawed.\index{Political repression} Simultaneously, the official anarchist press was put under iron political censorship. Yet the \POUM\ and \CNT\ did not join in a mass protest. ``We formulate no protest. We only make public the facts,\kn\kn'' wrote \emph{Solidaridad Obrera,} May 29.

\medskip

The \POUM\ youth organ, \emph{Juventud Comunista,} grandly remarked:

\begin{quotation}
  These are cries of panic and of impotence against a firmly revolutionary party\dots. The [libel] trial goes forward. The \PSUC\ organ must appear before the Popular Tribunals and they shall be revealed before national and international labor for what they are: vulgar calumniators. (June 3).
\end{quotation}
  
Naturally, on a technicality the trial was soon dismissed.

\medskip

\index{Political repression}
On the night of June 3, the Assault Guards\index{Assault Guards} attempted to disarm one of the remaining workers’ patrols. Shots were exchanged. There were dead and wounded on both sides. Here was the government’s opportunity to finish off the patrols. But here, also, was the \POUM’s opportunity to force the \CNT\ leaders to defend the workers’ elementary rights by demanding a united front for simple, concrete proposals~-- defense of free assembly, the press, the patrols, common defense of workers’ districts against the Stalinist hooligans, freedom of political prisoners, etc.\@ The anarchist leaders could have hardly rejected these proposals without compromising themselves irreparably before their membership. Even against the \CNT\ leaders’ will, united front committees could have been created in the localities to fight for such simple, concrete demands.

For the \POUM\ leaders, however, to raise such simple demands meant: we have been wrong in estimating the May Days as a defeat of the counterrevolution; it was a defeat of the workers and now we must fight for the most elementary democratic rights. Secondly, it meant: we have been wrong in leaning on the \CNT\ leaders, limiting ourselves to the general, abstract proposal of a ``revolutionary front''\index{Revolutionary Front} of \CNT-\FAI--\POUM\ which implies that the \CNT\ is a revolutionary organization with which we can have a common platform on fundamental policies.\kn\kn%
\endnote{Juan Andrade\index{Andrade, Juan} had justified the ``revolutionary front'' absurdity by the following argument:
	
\begin{quotation}
  The disillusioned worker, turning away from the democratic tendencies of the socialists and communists inclines to join a powerful organization, such as, \CNT-\FAI\indexCNT, which holds radical positions \emph{even if they are not applied in fact,} rather than join a minority party bothered by material difficulties. The workers already in the \CNT\ see no need of leaving it to join a revolutionary Marxist party because in contrasting the \emph{surface} revolutionary positions of the \CNT-\FAI\ with the simply democratic ones of the socialists and Stalinists, they believe the tactics of their organization still hold the guarantee for continued development of the revolution toward the building of a socialist economy. In this sense, all those who hold a strictly sectarian schematic concept of how a minority with a correct political line can rapidly become a decisive force, can learn a valuable lesson from the events in Spain\dots

  The difficulties in the way of the rapid development of a great mass party which would assume the effective leadership of the struggle can be largely resolved by the establishment of the Revolutionary Front between these two organizations.
\end{quotation}

In other words, it is impossible to build the party of the revolution; the Revolutionary Front is a substitute. But the \emph{main} obstacle to building the revolutionary party, beside the \POUM’s own false program, was that the \CNT's \emph{surface} radicalism was \emph{not} systematically criticized before the masses by the \POUM\kn. The \POUM\ had thus cut itself off from growth~-- and used its failure to justify continuing that failure.}
We must say openly\kn, that a united front for the most elementary workers’ rights is the most than can be expected of the anarchist leadership, if even that.

\emph{Not once} in the year had the \POUM\ called for a united front with the \CNT\ for concrete tasks of struggle! The whole policy of the \POUM\ leadership essentially consisted of nothing but trying to curry favor with the \CNT\ leadership. \emph{Not once} did they characterize the capitulatory policy of the \CNT\ leadership, not even when they expelled the Friends of Durruti and left them to the mercies of the Assault Guards!

In its darkest hour the \POUM\ was completely isolated. On June~16, Nin{\indexANin} was arrested in his office. The same night, widespread raids caught almost all the forty members of the Executive Committee. A few who escaped were forced to give themselves up because their wives were seized as hostages. The next morning the \POUM\ was outlawed.\index{Political repression}

The Regional Committee of the \CNT\ did not come to the defense of the \POUM\kn. \emph{La Noche} (\CNT) of June 22, published in bold face:

\begin{quotation}
  About the espionage service discovered in the last days. The principal ones implicated were found in the leading circles of the \POUM. Andr\'es Nin and other known persons arrested.
\end{quotation}

There followed some general reflections on slander, with copious references to Shakespeare, Gorki, Dostoevsky, and Freud\dots. If censorship was to be blamed, then where were the \CNT’s illegal leaflets? In Madrid, `\CNT' did come to the defense of the \POUM\kn, and was followed by \emph{Castilla Libre} and \emph{Frente Libertario,} militia organ. On June~28, the National Committee of the \CNT\ addressed a letter to the ministers and their organizations reminding them that Nin, Andrade, David Rey, Gorkin, etc., ``had acquired their prestige among the masses by long years of sacrifice.\kn\kn''

\begin{quotation}
  Let them solve their problem in the \USSR\ as they can or circumstances advise them. It is not possible to transplant to Spain the same struggle, prosecuted with blood and fire, internationally by means of the press and here by means of the law utilized as a weapon.
\end{quotation}

The letter indicated an entire lack of understanding of the significance of the persecutions.

\begin{quotation}
  Before all it imports us to declare that the \CNT\kn, by its intact and powerful strength, today perfectly organized and disciplined, is above all fear that tomorrow this method of elimination can overcome us. Placed above this semi-internal struggle\dots.
\end{quotation}

This pompous chest-beating meant that the \CNT\ masses would not be aroused by their leaders to the counterrevolutionary meaning of the persecutions.

Above all, the great masses had not been prepared to understand the Stalinist system of frameup and slander. Currying favor with Stalin,\index{Stalinism} the anarchist leaders had been guilty of such statements as that of Montseny: ``Lenin was not the true builder of Russia but rather Stalin with his practical realism.\kn\kn'' The anarchist press had preserved a dead silence about the Moscow trials and purges, publishing only the official news reports. The \CNT\ leaders even ceased to defend their anarchist comrades in Russia. When the anarchist Erich M\"uhsam was murdered by Hitler\kn, and his wife sought refuge in the Soviet Union\indexUSSR, only to be imprisoned shortly after her arrival, the \CNT\ leadership stifled the protest movement in the \CNT\ ranks. Even when the Red Generals were shot, the \CNT\ organs published only the official bulletins.

By mid-July\kn, the \POUM’s leaders and active cadres were all in jail!\index{Political prisoners} Over its buildings flew the violet-yellow-red flags of the bourgeoisie. The Lenin barracks were occupied by the Republican ``People’s Army.\kn\kn'' Its presses had been destroyed or given to the \PSUC. On the bulletin board of \emph{Batalla} was a copy of \emph{Julio,} \PSUC\ youth organ, headlining: ``Trotskyism is synonymous with counterrevolution.'' The \POUM’s dormitories, ex--Hotel Falcon, had become a prison for \POUM\ members and the headquarters of the Spanish \GPU\indexGPU\index{Political repression}. Its members were dispersed, disoriented, living in fear of nightly raids by the Assault Guards. ``Small groups work on their own hook,\kn\kn'' wrote an authoritative eyewitness early in July. ``It reminds one very much of the crumbling of the Communist Party of Germany in January 1933. The working class remains passive and permits anything to happen. The \CNT\ press prints only official notices. No protest! Nowhere even a word of protest! The \POUM\ has been swept away like a speck of dust. “Like under Hitler{\indexAHitler},\kn\kn” say the German comrades. The Russian Bolshevik--Leninists would add: “Almost like under Stalin.\kn\kn”\index{Stalinism}

In July\kn, the local \FAI\ committees began illegal propaganda.\index{Illegal propaganda} Unfortunately\kn, it did not center around rallying the workers to the concrete tasks of freeing the political prisoners. One typical leaflet recalled the German Social Democracy’s propaganda on the eve of Hitler\kn, demanding the help of the state~-- \emph{Staat, greif zu!}\index{Staat, greif zu@\emph{Staat, greif zu}}~-- against its own bands. Protesting Stalinist assaults on anarchist youth buildings, ``How long? It is time for the Government Council to speak, or lacking that, the Delegate General of Public Order and the Chief of Police,\kn\kn'' read one pathetic leaflet.\index{Leaflet}

Nor were the illegal \POUM\ leaflets, which now began to appear\kn, much better. They, who had always reproached the Bolshevik--Len\-in\-ists for seeing only Stalinism as the enemy, became themselves anti-Stalinist and nothing more. One typical leaflet, for example, addressed itself to everybody on the left and on the right: to the anarchists as well as to the ``young separatists'' of the Estat Catala.\index{Estat Catala} ``The men of the Left cannot betray their postulates. The Separatists cannot sell Catalonia by their silence.\kn\kn'' And the final slogan! ``Prevent the establishment of the dictatorship of a party behind the lines.\kn\kn'' What of the Estat Catala and Esquerra, Prieto and Azaña, accomplices of the Stalinists, indeed the main beneficiaries?

Thus, false policies facilitated the deadly advance of the counterrevolution. Only the small forces of the Bolshevik--Leninists,\index{Bolshevik-Leninists} who had been expelled as ``Trotskyists''\index{Trotskyism} from the \POUM\kn, and had formed their organization in the spring of 1937~-- only this small band, working under the three-fold illegality of the state, the Stalinists and the \CNT--\POUM\ leadership, clearly pointed the road for the workers. Not only the ultimate road of the workers’ state but the immediate task of defending the democratic rights of the workers. That the \CNT\ masses could be aroused was shown by the protection they accorded Bolshevik--Leninists distributing illegal leaflets.\index{Illegal propaganda} At one meeting (of the woodworkers’ union), lorries of Assault Guards arrived and attempted to arrest the distributors. The meeting declared that the leaflet distributors were under their protection and would repulse with arms any attempt to break in. The police were forced to leave without our comrades.

A Bolshevik--Leninist leaflet of July 19 points the road: the united front\index{United front} of struggle of the \CNT-\FAI, \POUM\kn, the Bolshevik--Leninists and the dissident anarchists (p.~\pageref{fig:bolshlenleaflet2}).\index{Leaflet}\index{Anarchism!Dissident anarchists}

\begin{figure}
\index{Leaflet}
\begin{oframed}
  \begin{quote}
  	\normalsize
  	\raggedright
  	
  	\medskip
  	
  	Workers---
  
    \smallskip
  	
  	Demand of your organization and your leaders a united front pact which must contain:
  
    \smallskip
  
    \begin{enumerate}[leftmargin=2em, labelindent=0.5em, itemindent=-1em]
      \item Struggle for the freedom of the workers’ press!
      
      Down with political censorship!\index{Censorship}
      
      \item Liberation of all revolutionary prisoners.\index{Political prisoners}
      
      For the liberation of Comrade~Nin,{\indexANin} transported to Valencia!
      
      \item Joint protection of all centers and enterprises in the possession of our organizations.
      
      \item Reconstitution of strengthened Workers’ Patrols.\index{Workers' militias}
      
      Cessation of disarming the working class.
      
      \item Equal pay for officers and soldiers.
      
      The return to the front of all the armed forces sent from Valencia.
      
      \index{General offensive}
      General offensive on all fronts.
      
      \item Control of prices and distribution through committees of working men and working women.\index{Price control}
      
      \item Arrest of the provocateurs of May~3: Rodr\'iguez Salas, Ayguad\'e, etc.
    \end{enumerate}
  
    \smallskip
  
    To achieve this, all workers form the united front!
    
    \smallskip
    
    \index{Workers' committees}
    Organize Committees of Workers, Peasants and Combatants in all enterprises, barracks and districts on the land and at the front!
    
    \medskip
  \end{quote}
\end{oframed}

\vspace{-0.5\baselineskip}

\caption{Leaflet distributed by Bolshevik--Leninists on July~19, 1937}
\label{fig:bolshlenleaflet2}
\end{figure}

But not in a day, or a month, does a new organization win the leadership of the masses. The road is long and hard~-- and yet the only road.

\dinkus

By July, according to the official \CNT\ figures, eight hundred of their members in Barcelona alone were imprisoned, and sixty had ``disappeared''~-- euphemism for assassination. The left socialist press reported scores of its leading militants everywhere seized and jailed.\index{Political prisoners}\index{Assassination}

One of the most repulsive phases of the counterrevolution was its merciless persecution of the foreign revolutionists\index{Foreign revolutionaries} who had come to Spain to fight in the ranks of the militias. A single report to the \CNT\ on July 24, counted 150 foreign revolutionists in a Valencia prison~-- arrested on the charge of ``illegally entering Spain.\kn\kn'' Hundreds were expelled from the country and the \CNT\ cabled the workers’ organizations in Paris, appealing to them to prevent the German, Italian, Polish exiles from being delivered into the hands of their consulates.
\nowidow

But the foreigners arrested and expelled did not meet the worst fate. Others among them were selected to complete the amalgam between the \POUM\ and the fascists. Maur\'in was in fascist hands, in danger of death. Nin, Andrade, Gorkin were too well known to the Spanish masses. The \POUM\ had too many thousands of its best men at the front. Too many of its leaders had died fighting fascism: Germinal Vidal, the Youth Secretary, at the taking of the Atarazañas barracks on July 19; his successor, Miquel Pedrola, commandant on the Huesca front; Etcheb\'eh\`ere, commandant at Sig\"uenza; Cahué and Adriano Nathan, commandants on the Aragon front; Jes\'us Blanco, commandant on the Pozuelo front, etc. Among the \POUM’s military figures were men like Rovira and Josh Alcantarilla, famed throughout Spain. A few unknown foreigners, fighting in the \POUM\ battalions, would serve to add to the credibility of the fantastic charges.

Georges Kopp, a Belgian ex-officer\kn, serving in the \POUM’s Lenin Division, had just returned to Barcelona from Valencia, where he had been given a major’s commission, the highest commission awarded to foreigners, when the Stalinists arrested him.
Then the Stalinist propaganda factory went to work.

Robert Minor, American Stalinist leader, announced that the pau\-ci\-ty of arms on the Aragon front (this was the first time the Stalinists admitted this charge of the \CNT) was now explained: ``The Trotskyist General Kopp had been carting enormous supplies of arms and ammunition across no man’s land to the fascists!'' (\emph{Daily Worker,} August~31 and October~5).\index{Framing}\index{Stalinism}

The choice of Kopp, however\kn, was a \GPU\ blunder of major proportions, comparable to the story of Romm’s meeting with Trotsky in Paris or Piatakov’s flight to Norway. For Georges Kopp, forty five years old, was a militant of long standing in the Belgian revolutionary movement. When the Spanish war broke out, he was chief engineer in a large firm in Belgium. It had been usual for him to experiment at night. He circulated the story that he was trying out a new machine, perfecting it by the actual process of manufacture. What he manufactured, however, were the ingredients for millions of rounds of cartridges. Left socialists organized illegal transport to Barcelona.

When Kopp discovered that he was under suspicion, he took leave of his four children and headed for the frontier. The very day he fled, the police raided his laboratory. \emph{In absentia,} Kopp was sentenced by the Belgian courts to fifteen years at hard labor: five for making explosives for a foreign power, five for leaving the country without permission while a reserve officer in the Belgian Army\kn, and five for joining a foreign army. Twice wounded on the Aragon front, he soon won the rank of commandant.\kn\endnote{The British \emph{New Leader}, August 13, 1937, published two detailed articles on Kopp’s record. [Kopp in fact survived and lived till 1951. See his biography by Don Bateman in \emph{Revolutionary History}, Vol.~4 Nos.~1/2, pp.~242--252. ---\ERC]}

Kopp cannot answer the Stalinist slanderers, for they have killed him. He was in a Barcelona jail with our American comrade, Harry Milton. In the middle of the night, Kopp was dragged out. That was in July\kn, and the last time he was ever seen.\index{Assassination}

On July~17\kn, a group of \kern -0.125pt\POUM\ members were released from prison in Valencia. It is a fact that most of them were extreme right-wingers, such as, Luis~Portela, editor of \emph{El Comunista}; Jorge~Arquer\kn, etc. Consequently their subsequent testimony was particularly cogent. Upon release, they went to Zugazagoitia, Minister of Interior\kn,\index{Ministries!Interior} who told them that Nin{\indexANin} had been taken from Barcelona to one of the private prisons of the Stalinists in Madrid. Arquer thereupon requested a safe-conduct\index{Safe conduct} to search for Nin. The minister, a Prieto man, told him: ``I guarantee you nothing; what is more, I advise you [not] to go to Madrid because, with my safe-conduct or without, you would endanger your life. These communists don’t respect me and they do as they please. And there would be nothing strange if you were seized and shot immediately by them.\kn\kn''

Publicly, however, Zugazagoitia was still saying that Nin was in a government prison. On July~19, however, Montseny, for the \CNT\kn, publicly charged that Nin had been murdered. Embarrassed by the numerous inquiries from abroad about specific prisoners’ whereabouts which the government was unable to answer for the simple reason that most of the prominent ones were in private Stalinist ``preventoriums,\kn\kn'' it was arranged that the prisoners be transferred from the Stalinist jails in Madrid and Valencia to the formal keeping of the Ministry of Justice. Nin was not among them. Irujo issued a statement that Nin was ``missing.\kn\kn'' Fled, the Stalinists said, toward the fascist lines. But the truth finally got out. On August~8, the \emph{New York Times} reported that ``nearly a month ago, a band of armed men kidnapped Nin from a Madrid prison. Although every effort has been made to hush up the affair\kn, it is now a matter of common knowledge that he was found dead on the outskirts of Madrid, a victim of assassination.\kn\kn''\index{Assassination}

As a personal friend of Nin and Andrade, the great Italian novelist, Ignazio Silone, had tried to save them. ``But,\kn\kn'' he warned, ``unless the revolutionary proletariat of other countries is watchful, the Stalinists are capable of every crime.\kn\kn'' \'Alvarez del Vayo, former Minister of Foreign Affairs\index{Ministries!Foreign Affairs} in Caballero’s cabinet, notoriously Stalin’s agent in the Caballero group, had the effrontery to tell the wife of Andrade that Nin had been murdered by his own comrades.\footnote{It is only just to add that del Vayo has since been excluded by the Socialist organization (Caballero-led) of Madrid.} Premier Prieto shrived his soul for this and other crimes by dismissing the Stalinist Police Chief, Ortega~-- and replaced him with the Stalinist, Mor\'on.

To cap the suppression of revolutionists with slander is scarcely a new invention.\index{Political repression}

When, in Paris, the June~1848 insurrection was drowned in blood, the National Assembly was assured by the left democrat, Flocon, that the insurrectionists had been bribed by monarchists and foreign governments. When the Spartacists were shot down, Ludendorff charged that they~-- and indeed, the social democrats who shot them too!~-- were agents of England. When the counter revolution got the upper hand in Petrograd, after the July Days, Lenin and Trotsky were indicted as agents of the Kaiser\kn. The destruction of the generation of 1917 is now carried out by Stalin under the charge that they have sold themselves to the Gestapo.

The parallel goes further. While Kerensky\index{Kerensky, Alexander} was shouting that Lenin and Trotsky were German agents, Tseretelli and Lieber in the soviets were, under questioning, dissociating themselves from the charge, limiting themselves to demanding the outlawry of the Bolsheviks for planning an insurrection. But, profiting from Kerensky’s charge, the Mensheviks did not ascend the housetops to proclaim the Bolsheviks’~innocence.

So, too, in Spain. The Stalinists were not even as successful as Kerensky: the indictment handed down against the \POUM\ leaders made no mention of collaboration with Franco or the Gestapo. The charge was based on the May Days and similar subversive and oppositional deeds. Prieto and other collaborators of the Stalinists told the \textsc{ilp} delegation they did not believe the Stalinist linking of \POUM\ to the fascists. They ``merely'' did not come to the defense of the \POUM\kn. Companys not only disavowed belief in the charges but made the fact public. Thus there was a division of labor: if you don’t believe the slanders, then you must believe the \POUM\ was organizing an insurrection, i.e., they were either counterrevolutionists or revolutionists, whichever you preferred. A narrower division of labor was that between the world Stalinist press, which repeated the ``Trotskyist-fascist'' slanders, and the anti--\POUM-\CNT\ propaganda of Louis Fischer,{\indexLFischer\index{Bates, Ralph}} Ralph Bates, Ernest Hemingway, Herbert Matthews, etc., who ``merely'' repeated such myths as that the \POUM\ militias played football in no man’s land with the fascists.

\dinkus

Already by the end of June, Catalan autonomy, though guaranteed by statute, was completely suppressed.\index{Catalonia!Autonomy} The authorities distrusted anybody who had any tie with the Catalan masses, no matter how tenuous. With the exception of the most reactionary sector, the old Civil Guards,\index{Civil Guards} all the police in Catalonia were transferred to other parts of the country.\index{Police} Even the firemen were transferred to Madrid. Parades were forbidden, and union meetings could be held only by permission of the delegate of public order on three days’ notice~-- as under the monarchy!\index{Political repression}
\nowidow

The workers’ patrols\index{Workers' militias} had been wiped out, their most active members imprisoned, their chiefs ``disappeared.''

Having done all this with the aid of the screen provided by \CNT\ ministers still sitting in the Generalidad, the bourgeois--Stalinist bloc now dispensed with their services.
A June~7 bulletin of the \FAI\ published a Stalinist communication which had been intercepted:\indexCNT

\begin{quotation}
  Based on the provisional composition of the government, our party will demand the presidency. The new government will have the same characteristics as that of Valencia; a strong government of the People’s Front \emph{whose chief task} will be to calm the spirits and demand punishment for the authors of the last counterrevolutionary movement. The anarchists will be offered posts in this government but in such a manner that they will be compelled to refuse to collaborate, and in this manner we shall be able to present ourselves to the public as the only ones willing to collaborate with all sections.
\end{quotation}

The anarchists challenged the \PSUC\ to deny the authenticity of this document, but there were no challengers.

At the end of June came the ministerial crisis. The \CNT\ agreed to whatever demands were made, and the new ministry was formed. Publication of the ministerial list on June~29, however\kn, revealed to the \CNT\ that, without their knowledge, a minister without portfolio had been added~-- an ``independent'' named Dr.~Pedro Gimpera, a notorious reactionary and anarchist-baiter. Companys blandly refused to withdraw him. The \CNT\ at last withdrew, leaving a government of the Stalinists and the bourgeoisie. {\indexCNT}

The only difference between the Stalinist bulletin exposed by the \FAI\ and the actual course of the ministerial crisis was that the Stalinists had not asked for the presidency. But six weeks later\kn, without any previous hint, the Stalinists clashed with President Companys.

In November~1936, when the \CNT\ intelligence service had seized Reberter, the Chief of Police, and had him tried and shot for organizing a \emph{coup d’\'etat,} the investigation had implicated Casanovas,\index{Casanovas i Maristany, Joan} President of the Catalan Parliament.\index{Catalonia!Parliament} But the Stalinists had supported Companys in prevailing upon the \CNT\ to let Casanovas leave the country and Casanovas had fled to Paris. After the May Days, he had returned to Barcelona with impunity. He spent the next three months pleasantly reestablishing himself in political life. During all these nine months, he had not been subject to a word of condemnation from the Stalinists.\kn\kn\footnote{Stalin has employed this method systematically in Russia: a bureaucrat is involved in a misdeed; he is permitted to go on because all the more servile for knowing that his crime is detected, then~-- sometimes years later~-- Stalin needs a scapegoat and the wretch is pilloried.} On August~18, the Catalan parliament opened. Without a previous word of warning to their allies~-- it could obviously have been settled behind closed doors~-- the \PSUC\ delegation of four publicly attacked Casanovas as a traitor. The Esquerra had been tricked into a position where it had to reject Casanovas’ offer to resign. With this excellent little whip, the Stalinists began to drive the Esquerra as they pleased, ending with the announcement of Companys’ early resignation from the presidency, after the Stalinists had boycotted the October~1 session of the Catalan parliament.

Why did the Stalinists break with Companys? He had done their bidding in so much! Why\kn, then, was Companys now slated to go?\kp\kp\footnote{The speed with which the fascists broke through the Aragon front disrupted the Stalinists’ plans, and Companys was not dropped.}

He had made one unforgivable break with the Stalinists. Companys had publicly declared that he had known nothing of the plans for outlawing the \POUM\kn; had protested against the transfer of the prisoners from Barcelona; and had sent to Madrid the chief of the Catalan press bureau, Jaime Miravittlles, to see the Stalinist Police Chief, Ortega, on behalf of Nin.

When Ortega showed him the ``crushing proofs''~-- a document ``found'' in a fascist center\kn, linking one ‘N’ to a spy ring~-- Miravittlles, by his own account, had burst out laughing, and declared the document was such an obvious forgery no one would dream of taking it seriously. Companys had then written to the Valencia government that Catalan public opinion could not believe Nin was a fascist spy.

\index{Framing}
Not that Companys was going to fight for the \POUM\ prisoners. Having salved his conscience~-- and made the record for any future overturn!~-- Companys relapsed into silence. That his silence did not save him from attack indicated that the Stalinists could not forgive any ally who exposed their frameups: the frameup is the very foundation stone of Stalinism today.

But there was a more profound reason for the break with the Esquerra. The Nin incident merely indicated that Companys was not hardened enough for the future moves of the Stalinists. He was, after all, a nationalist, desiring a return to Catalan autonomy. And for Stalinism,\index{Stalinism} Spain and Catalonia were merely pawns which they were ready to sacrifice, with which they were ready to do anything that Anglo--French imperialism dictated, in return for a military alliance for Stalin in the coming war. That is why there had to be a selection even from among the Prieto socialists and the Azaña republicans: only the most brutalized, most corrupt, most cynical could weather the coming storms created by the Stalinists, and remain in collaboration with them.

The economic counterrevolution in Catalonia advanced against the collectives.\index{Workers' militias} To the honor of the local sections of the libertarian movement, they stood their ground. For example, the strong anarchist \index{Anarchism!Dissident anarchists} movement in Bajo Llobregat (heart of armed struggles against the monarchy and the republic) declared in its weekly\kn, \emph{Ideas,} on May~20:

\begin{quotation}
  Here is what we must do, workers! You have the opportunity to be free. For the first time in our social history the arms are in our hands: don’t drop them. Workers and peasants! When you hear that the government, or anybody else, tells you that the arms should be at the front, answer them that that is certainly so, that the thousands of rifles, machine guns, mortars, etc., that are kept in the barracks, that are used by the Carabineros, Assault and National Guards,\index{Assault Guards} etc., should be sent to the front because to defend your fields and factories, nobody can do that better than you.\index{Battle}
  
  Remember always that airplanes, cannons and tanks are what are needed at the front quickly to crush fascism \dots\ what the politicians are looking for is to disarm the workers, to have them at their mercy, and to take away from them what has cost so much proletarian blood and lives. Let nobody permit the disarming of anybody; let no village allow another to be disarmed; let us all disarm those who try to disarm us. This should be, must be, the revolutionary slogan of the hour.
\end{quotation}

The gap between the pusillanimity of the central organs of the \CNT\ and the fighting spirit of the local papers, close to the masses, was as wide as that between craven cowards and revolutionary workers.

But tens of thousands of Assault Guards, concentrated behind the lines, struck systematically at the collectives. Without centralized direction, the villages were overpowered, one by one. \emph{Libertad,} one of the dissident illegal anarchist papers of Barcelona (incidentally\kn, it paid its contemptuous respects to \emph{Solidaridad Obrera,} which had denounced the illegal organs), described the situation in the countryside in its issue of August~1:

\begin{quotation}
  \index{Censorship}
  It is useless that the censorship, in the power of one party, prevents a word being said about the thousands of blows inflicted on the workers’ organizations, the peasant collectives. In vain that they prohibit mention of that terrible word, counterrevolution. The working masses know perfectly that the thing exists, that the counterrevolution advances under the protection of the government, and that the black beasts of reaction, the disguised fascists, the old \emph{caciques,} are again raising their heads.
  
  \index{Reprivatization}
  And how should they not know it, if there is not a village in Catalonia where the punitive expeditions of the Assault Guards have not been, where they have not assaulted the \CNT\ workers, destroying their branch organizations or what is worse, destroyed those portentous works of the revolution, the collectives of the peasants, in order to return the land to the old proprietors, almost always known as fascists, ex-\emph{caciques} of the black epoch of Gil-Robles, Lerroux or Primo de~Rivera?
  
  \index{Collectivization}
  The peasants took the goods of the bosses~-- which in justice did not belong to them~-- to place them at the service of collective labor, permitting the old bosses to dignify themselves, if they wished, by work. They believed, the peasants, that so noble a work was guaranteed by its own efficiency, if fascism were not triumphant, and it could not triumph. Scarcely did they suspect that in the midst of war against the terrible enemy, with the government being men of the left, the public forces [police] would come to destroy that which had been created with such fatigue and joy. For this inconceivable thing to happen, there had to come to power\kn, by dirty means, those called communists. And the workers, ready always to make the greatest sacrifices to defeat fascism do not end wondering how it is possible that they be attacked from behind, that they be humiliated and betrayed, when there still is so much lacking for conquering the common enemy\dots.
  
  \index{Political repression}
  The technique of repression is always the same. Lorries of Assault Guards that enter the village like conquerors. Sinister registrations in the branch organizations of the \CNT. Annulment of municipal councils where the \CNT\ is represented. Plundering searches and arrests. Seizure of the food of the collectives. Return of the land to their old proprietors.
\end{quotation}

\index{Assassination}
This movingly simple description was followed by a long list of villages, the dates on which they were assaulted, the names of those arrested or killed~-- and in the ensuing months the list grew longer and longer\kn.

In industry and commerce, the juridical basis of the collectivized establishments rested on the insecure foundation of the collectivization decree of October~24, 1936. But immediately after the May Days, the Generalidad repudiated the decree! The occasion was the attempt of the \CNT\ to release the factories from the stranglehold of the customs officials, without whose certification of ownership of export goods arriving abroad were being sequestered under claims of emigrated former owners. The anarchist-led Council of Economy (of the Ministry of Industry)\index{Ministries!Industry} adopted on May~15 a proposed decree to record the collectivized establishments as the official owners in the Mercantile Register. But the bourgeois--Stalinist majority in the Generalidad rejected the proposal on the ground that the October~24 collectivization decree was ``dictated without competency by the Generalidad,\kn\kn'' because ``there was not, nor is there yet, legislation of the [Spanish] State to apply,\kn\kn'' and ``Article~44 of the [Spanish] Constitution\index{Constitution of Spain} declares expropriation\index{Collectivization} and socialization are functions of the [Spanish] State,\kn\kn'' i.e., the Catalan autonomy\index{Catalonia!Autonomy} statute had been exceeded! The Generalidad would now await action by Valencia. But Companys had signed the October Decree! That was during the revolution\dots.

\dinkus

The chief agency of economic counterrevolution was the \GEPCI, the long-established businessman’s organization taken bodily into the Catalan section of the \UGT\ by the Stalinists but repudiated by the \UGT\ nationally. With union cards in their pockets, these men did with impunity what they would never have dared before July~19 against the organized workers. Many of them were now no longer petty manufacturers but great entrepreneurs. They received ``pref\-er\-en\-tial consideration in securing financial credits, raw materials, export services, etc.,\kn\kn'' as against the factory collectives. One little item will destroy the Stalinist myth that these were petty little storekeepers, one-man establishments. In June~1937\kn, the \UGT\ clothing workers drafted a scale of wages, identical with those in the clothing collectives, and sought to negotiate with the capitalist-owned clothing factories. The employers rejected the demands. But who were the employers? Members, to a man, of \GEPCI\indexGEPCI, that is, like the employees whom they were refusing wage rises, they were members of the \UGT\ in Catalonia! (\emph{Solidaridad Obrera,} June 10). Would the most reactionary trade union bureaucrat, of the stamp of Bill~Green or Ernest~Bevin, propose that bosses and workers be in one ``union?\kp'' No, that vast step backward could come only from the Stalinists, aping Fascist Italy and Nazi Germany.

In June, under the slogan of ``municipalization,\kn\kn''\index{Municipalization} the \PSUC\ launch\-ed a campaign to wrest the transportation,\index{Ministries!Utilities} electric, gas, and other key industries from workers’ control. On June 3, the \PSUC\ delegation formally proposed, in the Barcelona Municipal Council,\index{Barcelona} that it municipalize public services. On the morrow, of course, the \CNT\ councillors would be thrown out, and the Stalinists would have the public services in their hands for the next step in returning them to their former owners. But this time they were confronted, not merely by the temporizing \CNT\ leaders~-- who proposed that municipalization was ``premature'' in this field, one ought to begin with housing~-- but with the mass response of the workers involved. The Transport Workers Union plastered every block of the city with huge~posters:

\medskip

\begin{oframed}
  \centering
  \setlength{\parskip}{0.5\baselineskip}
  
  \medskip
  The revolutionary conquests belong to the workers.
  
  The workers’ collectives are the product of these conquests.
  
  We must defend them\dots.
  
  To municipalize the urban public services, yes~--
  
  but only when the municipalities belong to the workers
  
  and not to the politicians.
  \medskip
\end{oframed}

\medskip

The posters demonstrated that since the workers had taken control,\index{Collectivization} there had been a thirty percent increase in plant facilities, lowering of fares, additional workers employed, big donations to the agricultural collectives, subventions to the harbor workers, social insurance to families of deceased or wounded workers, etc. For the moment, the Stalinist advance was beaten in this field.

But the Stalinists continued toward their goal of destroying the worker-controlled factories. The Catalan Generalidad\index{Catalonia!Generalidad} set September~15 as the deadline for proving the legality of collectivized factories. Since much of the collectivization was done overnight to speed the civil war against the fascists, few factories had established any juridical procedure. What, indeed, were the legalities involved in the expropriations? The original decree of October~24, 1936, we have discussed in our chapter on the first Generalidad cabinet. It was designed precisely to provide entering wedges for the future. And the Generalidad had now repudiated it! At leisure and at will, the Generalidad would now examine the legal title of the social revolution and find it undoubtedly full of legal flaws. What a preposterous business! But a tragic one.

It was in the food industries, distribution, markets, etc., that the Stalinists had got their first grip, holding the Ministry of Supplies\index{Ministries!Supplies} in the Generalidad since December\kn, when they had promptly dissolved the workers’ supply committees, which then had been provisioning the cities under controlled prices. Even through the temporizing of the \CNT\ press and the opacity of the censorship, the accounts now reflected what was happening here:

\begin{quotation}
  Collectives, socialized undertakings and cooperatives, embracing both members of the \UGT\ and \CNT\kn, have been made the target of attack on the part of those who hid in desertion on the 19th of duly\dots.
  \index{Reprivatization}
  The dairymen of both unions are being arrested right and left. The cows and the dairy farms, organized legally on a cooperative basis, are being confiscated, although their statutes have been officially approved by the Generalidad for several months. These cows and dairy establishments are being hand\-ed over to their former owners\dots.
  The same thing is happening, although still on a small scale, in the bread industry\dots.
  Our markets, the central fish market, etc., although collectivized legally\kn, are also suffering from these vicious attacks by the former bourgeoisie. They are being encouraged by the poisonous campaigns conducted daily in the press of the party that has constituted itself the Champion for the Defense of the \GEPCI\ (Corporations and Units of Petty Merchants and Manufacturers). It is no longer merely a fight against the \CNT\ collectives, but against all the revolutionary conquests of the \UGT--\CNT\dots.
  A hard fist against the fascists and counterrevolutionaries hiding behind a trade union card! (\emph{Solidaridad Obrera,} June~29).\indexSolidaridadObrera
\end{quotation}

``Is the Ministry of Supplies at the service of the people, or has it been transformed into a bigger merchant?\kp'' asked the \CNT\ press. ``The basic articles of food are: rice, string beans, sugar\kn, milk, etc. Why are these not included among those items that the Committee of Distribution,\index{Ministries!Distribution} recently formed by the \UGT--\CNT, distributed equally among all the stores of Barcelona, regardless of the organization to which they belong?\kp'' Instead, these basic articles were uncontrolled, left to the mercy of the \GEPCI.

\emph{La Noche} (June 26) responding to the bitterness of the masses: ``The death penalty for thieves! Scandalous abuses of the merchants at the expense of the people.\kn\kn'' And, after showing, from the official statistics, the precipitous rise of food prices between June~1936 and February~1937\kn, \emph{La Noche} said: ``Nor would it have been so bad if the prices had remained at that level! One can speak to the housekeepers about the increase in the cost of living since February. It is reaching inaccessible figures\dots. We must create some form of protection for the interests of the people against the egoism of the merchants who are carrying on with full impunity.\kn\kn''

Yes, it was in food supplies that the Stalinists had their grip long\-est. And the result: hunger\kn, yes, actual hunger\index{Hunger} stalked Catalonia. The bitterness of the masses breaks through in \emph{Solidaridad Obrera\indexSolidaridadObrera} (September~19):

\begin{quotation}
  Proletarian mothers with sons at the front here suffer stoically of hunger together with their innocent little ones\dots.
  We say that sacrifices ought to be by all and it is an inconceivable situation that in actuality there are places where, by paying prices outside the reach of any worker, it is possible to obtain all kinds of food. These luxurious restaurants are veritable foci of provocation and should disappear, as ought to disappear all privileges of any sector. Flagrant inequality, privilege, is in such cases a terrible dissolvent of popular cohesion. It must be eliminated at all costs\dots. Protected \dots\ there has entered into action a repugnant caste of speculators and profiteers who traffic in the hunger of the people\dots.
  \index{Inequality}
  We repeat that our people do not fear sacrifices but do not tolerate monstrous inequality\dots. Respect the proletariat that fights and suffers!
\end{quotation}

Yes, the masses do not fear sacrifices. The workers of Petrograd suffered the most extreme privations~-- not even running water in the city during the civil war. But what there was belonged to all equally. It is not the bare pangs of hunger that contort the faces of the Barcelona workers and their women and children. It is that while they hunger, the bourgeoisie eat luxuriously~-- and this in the midst of civil war against fascism! But that is the inevitable consequence of not finishing with bourgeois ``democracy.\kn\kn''

To those who have been impressed by Stalinist ``common sense'' in modestly fighting for democracy: Do you begin to understand what it meant in the concrete, in the seared souls of the Spanish people?
