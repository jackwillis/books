\chapter{The Conquest of Catalonia}

\lettrineO{n May 5}, Catalan autonomy\index{Catalonia!Autonomy} had ceased to exist. The central government had taken over the Catalan Ministries of Public Order and Defense.\index{Ministries!Public Safety, Catalonia} Caballero’s delegate in Barcelona had broadcast: ``From this moment, all the forces are at the orders of the central government \dots\ These do not consider any union or anti-fascist organization as an enemy. There is no other enemy than the fascists.''

But a week later the Ministries of Defence and Public Order were surrendered by Caballero’s delegate to the representatives of Negrin--Stalin, and the pogrom began in earnest. The \POUM\ went down with hardly a ripple. The \PSUC\ opened a monstrous campaign against it identical in language, slogans, etc., with the witch-hunts of the Soviet bureaucracy before the Moscow trials. ``The Trotskyists in the \POUM\ have organized the latest insurrection on orders from the German and Italian secret police.'' The \POUM’s answer to the \PSUC\ was---to institute a libel\index{Libel} suit against the Stalinist editors in a court filled with bourgeois and Stalinist judges and officials!

On May 28, \emph{La Batalla} was suppressed\index{Censorship} permanently and the \POUM\ radio seized. The Friends of Durruti headquarters were occupied and the organization outlawed. Simultaneously, the official anarchist press was put under iron political censorship. Yet the \POUM\ and \CNT\ did not join in a mass protest. ``We formulate no protest. We only make public the facts,'' wrote \emph{Solidaridad Obrera}, May 29.

The \POUM\ youth organ, \emph{Juventud Comunista}, grandly remarked: ``These are cries of panic and of impotence against a firmly revolutionary party\dots'' (June 3). And: ``The [libel] trial goes forward. The \PSUC\ organ must appear before the Popular Tribunals and they shall be revealed before national and international labour for what they are: vulgar calumniators.'' Naturally, on a technicality the trial was soon dismissed.

On the night of June 3, the Assault Guards attempted to disarm one of the remaining workers’ patrols. Shots were exchanged. There were dead and wounded on both sides. Here was the government’s opportunity to finish off the patrols. But here, also, was the \POUM’s opportunity to force the \CNT\ leaders to defend the workers’ elementary rights by demanding a united front for simple, concrete proposals---defence of free assembly, the press, the patrols, common defence of workers’ districts against the Stalinist hooligans, freedom of political prisoners, etc.\@ The anarchist leaders could have hardly rejected these proposals without compromising themselves irreparably before their membership. Even against the \CNT\ leaders’ will, united front committees could have been created in the localities to fight for such simple, concrete demands.

For the \POUM\ leaders, however, to raise such simple demands meant: we have been wrong in estimating the May Days as a defeat of the counter-revolution; it was a defeat of the workers and now we must fight for the most elementary democratic rights. Secondly, it meant: we have been wrong in leaning on the \CNT\ leaders, limiting ourselves to the general, abstract proposal of a ``revolutionary front''\index{Revolutionary Front} of \CNT-\FAI--\POUM\ which implies that the \CNT\ is a revolutionary organization with which we can have a common platform on fundamental policies.%
\endnote{Juan Andrade\index{Andrade, Juan} had justified the ``revolutionary front'' absurdity by the following argument:
	
\begin{quotation}
  The disillusioned worker, turning away from the democratic tendencies of the socialists and communists inclines to join a powerful organization, such as, \CNT-\FAI, which holds radical positions even if they are not applied in fact, rather than join a minority party bothered by material difficulties. The workers already in the \CNT\ see no need of leaving it to join a revolutionary Marxist party because in contrasting the surface revolutionary positions of the \CNT-\FAI\ with the simply democratic ones of the socialists and Stalinists, they believe the tactics of their organization still hold the guarantee for continued development of the revolution toward the building of a socialist economy. In this sense, all those who hold a strictly sectarian schematic concept of how a minority with a correct political line can rapidly become a decisive force, can learn a valuable lesson from the events in Spain\dots

  The difficulties in the way of the rapid development of a great mass party which would assume the effective leadership of the struggle can be largely resolved by the establishment of the Revolutionary Front between these two organizations.
\end{quotation}

In other words, it is impossible to build the party of the revolution; the Revolutionary Front is a substitute. But the main obstacle to building the revolutionary party, beside the \POUM’s own false programme, was that the \CNT's surface radicalism was not systematically criticized before the masses by the \POUM. The \POUM\ had thus cut itself off from growth---and used its failure to justify continuing that failure.}
We must say openly, that a united front for the most elementary workers’ rights is the most than can be expected of the anarchist leadership, if even that.