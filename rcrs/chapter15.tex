\chapter{Military Struggle Under Giral and Caballero}

\index{Dual power}\index{Battle}
\lettrineM{ilitary warfare} is merely the continuation of politics by forcible means. A proclamation dropped over the enemy lines, expressing the aspirations of the landless peasants, is also an instrument of warfare. A successfully incited revolt behind enemy lines may be infinitely more efficacious than a frontal attack. Maintenance of the morale of the troops is as important as equipping them. To guard against treacherous officers is as important as training efficient officers. In sum, the creation of a workers’ and peasants’ government for which the masses will work and die like heroes is the best political adjunct of military struggle against the fascist enemy in civil war\kn.

\index{Russia!Civil War}
By these methods, the workers and peasants of Russia defeated imperialist intervention and White Guard armies on twenty-two fronts, despite the most rigid economic blockade ever imposed on any nation. In the organization and direction of the Red Armies under these adverse conditions, Trotsky seemed to perform miracles, but these were miracles compounded of revolutionary politics, of the capacities for sacrifice, labor and heroism of a class defending its newly-won freedom.

That reactionary political policies determined the false military policies of the Loyalist government can be demonstrated by surveying the course of the military struggle.

\indexIPrieto
From July~19 until September~4, 1936~-- seven decisive weeks~-- the Giral cabinet of the People’s Front was at the helm, with the unconditional political support of the Stalinists and the Prieto Socialists (Prieto, indeed, was unofficially part of the ministry, establishing an office in the government on July 20).

\indexJGiral\indexGRobles\index{Workers' militias}
The Giral government had about \textsc{\$600,000,000} in gold at its disposal. Recall that the real embargo on the sale of munitions to Spain was not established until August~19, when the British Board of Trade revoked all licences for export of arms and planes to Spain. Thus, the Giral regime had at least a month in which to purchase stores of arms~-- but the damning fact is that it bought almost nothing! The story of the treacherous attempt of Azaña--Giral to reach a compromise with the fascists has already been told. One further fact: Franco and his friends waited six days before forming their own government. Gil-Robles later revealed that they waited in expectation of a satisfactory arrangement with the Madrid government. By that time the militias had emerged from the ranks of the workers and Giral no longer had the power to meet Franco’s demands.

The most important gains of the first seven weeks were the successful march on Aragon by the Catalan militias, using socialization of the land as much as they used their rifles; and the attack of the loyalist warships on Franco’s transportation of troops from Morocco to the mainland.

\medskip

Wrote two Stalinists at the time,

\begin{quotation}
  The loyalty of a large part of the navy decisively prevented Franco from transporting large numbers of Moroccan troops to the mainland in the first two weeks of the war. The naval patrol in the south made transport by sea extremely hazardous. Franco was forced to resort to airplane passage, but this was slow work. In this respect, again, the government was given a chance to organize defense and take stock. (\emph{Spain in Revolt,} Gannes and Repard, p.~119).
\end{quotation}

\index{Sailors}\index{Ministries!Navy}
What they failed to add was that the warships were under the command of elected sailors’ committees, which, like the militias, had no faith in the Giral government and carried on operations, despite the passivity of the government. The significance of this fact will become apparent when we come to the naval policy of the Caballero--Stalinist--Prieto cabinet.

The terrible defeats of Badajoz and Irun finished the Giral cabinet. Why Irun fell was told in a moving dispatch by Pierre~van~Paassen:

\index{Battle}\index{Irun}
\begin{quotation}
  They fought to the last cartridge, the men of Irun. When they had no more ammunition, they hurled packets of dynamite. When dynamite was gone, they rushed forward barehanded and tackled each their man, while the sixty times stronger enemy butchered them with bayonets. A girl held two armored cars at bay for half an hour by hurling glycerin bombs. Then the Moroccans stormed the barricade of which she was the last living defender and tore her to pieces. The men of Fort~Martial held three hundred foreign legionnaires at a distance for half a day by rolling rocks down the hill on which the old fort is perched.
\end{quotation}

\index{Central Committee of Anti-Fascist Militias of Catalonia}
Irun fell because the Giral government had made no attempt to provide its defenders with ammunition. The Central Committee of Anti-Fascist Militias of Catalonia, having already transformed the available factories into munitions works, had sent several carloads of ammunition to Irun by way of the regular railroad from Catalonia to Irun. But that railroad runs partway through French territory. And the government of ``Comrade'' Blum, the ally of Stalin, had held up the cars at Behobia, just across the frontier\kn, for days \dots\ they rumbled across the bridge into Irun after the fascists had won.

\indexLCaballero\indexIPrieto\index{Popular Front}
The Giral cabinet gave way to the ``real, complete'' People’s Front government of Caballero--Prieto--Stalin. Undoubtedly, it had the confidence of a large part of the masses. The militias and the sailors’ committees obeyed its orders from the first. There were three major military campaigns that the new government had to undertake. There were, of course, other tasks, but these were the most important, the most pressing, and essentially, the simplest.

\section{Morocco and Algeciras}

\indexFFranco\index{Spanish Legion}\index{Morocco}
Franco’s military base during the first six months was Spanish Morocco. From here he had to bring his Moors and Legionnaires and military stores.

\index{Sailors}\index{Workers' militias}\index{Battle!Naval}\index{Foreign intervention!Italy}
The first successes of the Loyalist navy under the sailors’ committees in harassing Franco’s communication lines with Morocco were followed by others. On August~4, the Loyalist cruiser \emph{Libertad} effectively shelled the fascist fortress at Tarrifa in Morocco. It was a deadly blow at Franco. So deadly that it was answered by the first open Italian act of intervention: an Italian plane proceeded to bomb the \emph{Libertad.} When Loyalist warships steamed into position for a large-scale shelling of Ceuta in Morocco while fascist transports were loading, the German battleship \emph{Deutschland} brazenly steamed back and forth between the Loyalist warships and Ceuta to prevent the bombardment. A week later a Spanish cruiser stopped the German freighter \emph{Kamerun,} found her loaded to the decks with arms for Franco, and prevented her from landing at Cadiz. Whereupon Portugal openly went over to the fascists, permitting the \emph{Kamerun} to unload at a Portuguese port and forwarding the munitions by rail to Franco. German naval commanders received orders to fire on any Spanish vessel attempting to stop German munition shipments. The Loyalist naval operations, if continued, were fatal to Franco, and his allies had to unmask completely to save him.

\indexLCaballero\indexIPrieto\index{Ministries!Navy}
At this point the Caballero cabinet was formed, and Prieto, now in the closest collaboration with the Stalinists, and always ``France’s man,\kn\kn'' became head of the Naval Ministry. He put an end to naval operations off Morocco and the straits of Gibraltar\kn, and recalled the Loyalist forces which had held Majorca.

\index{Spanish Legion}\index{Andalusia!Algeciras}
The task of the hour was to prevent the Moors and Legionnaires from landing at Algeciras and constituting that army which was soon to make that fearful march from Badajoz straight through to Toledo, through Toledo and Talavera de~la~Reina to the gates of Madrid. The first line in that task belonged to the navy.

\smallskip

It was not used for this purpose.

\smallskip

\index{Basque Country!Biscay}\index{Battle!Naval}\index{Military supplies!Battleships}\index{Andalusia!Algeciras}
Instead, in mid-September, almost the whole fleet~-- including the battleship \emph{Jaime I,} the cruisers \emph{Cervantes} and \emph{Libertad,} and three destroyers~-- were ordered to leave Malaga and go all the way around the peninsula to the Biscayan coast! They left behind the destroyer \emph{Ferrandiz} and the cruiser \emph{Gravina.} On September~29, two fascist cruisers sank the Ferrandiz, after having shelled and driven away the Gravina. What considerations determined that the naval forces go off to the Biscayan coast, while news dispatches reported~-- to quote but one instance~-- an armed trawler conveying Moroccan troops from Ceuta and escorted by the \emph{Canarias,} the \emph{Cervera} and a destroyer and torpedo boat crossed the straits this evening. The convoy landed the troops at Algeciras without hindrance. It transported from Morocco a supply of field and other guns and abundant supplies of ammunition. (\emph{New~York~Times,} September~29).

What considerations? Certainly not military ones, for the forces sent to Biscay were more than sufficient to hold their own with the fascists’ armed convoy; and certainly barring communications from Morocco was the chief task of the navy.

\medskip

The American military expert, Hanson~W.~Baldwin, writing (in the \emph{New~York~Times} of November~21) on the naval question in Spain said:

\indexNYT\index{Ministries!Navy}
\begin{quotation}
  The Spanish navy has been to a great extent neglected, particularly in the recent troubled years of the republic’s history\kn, and it has never been properly led or manned. But with efficient, well-drilled crews, Spain’s handful of cruisers and destroyers could be a force to be reckoned with, particularly in the narrow basin of the Mediterranean, \emph{where well-handled ships long ago could have cut General Franco’s line of communication to his reservoir of manpower in Africa\dots.}
  
  Judging from the somewhat obscure reports, most of the ships~-- despite the officers’ efforts~-- continued to fly the red, yellow and mauve of [Loyalist] Spain or else hoisted the Red flags at their gaffs \dots\ but altogether the role of the navy in the civil war has to date been a minor one. The occasional engagements in which the ships have participated have had in most instances an \emph{op\'era bouffe} quality and have attested to the poor marksmanship and seamanship of the crews.
\end{quotation}

\index{Basque Country}\index{Foreign revolutionaries}
The Loyalist operations of September~27\kn, at Zumaia near Bilbao, however\kn, demonstrated accurate firing. The real point, however\kn, was that it would have been a simple matter to equip the Loyalist warships with able crews. Toulon, Brest, and Marseilles were filled with thousands of socialist and communist sailors, veterans of the navy\kn, including skilled gunners and officers. They could more than have manned the fleet, and other ships that could have been built at the main construction docks, in Cartagena in Loyalist hands.

\index{Andalusia!Algeciras}\index{Murcia!Cartagena}\index{Battle!Naval}
Returning eventually from the northern coast, the fleet was anchored far from the strait, at Cartagena~-- and there it stayed, except for a few pointless trips down the coast. That it existed at all, one learned on November~22, when foreign submarines entered the port of Cartagena and loosed torpedoes, one damaging the Cervantes. The same day the Ministry of Marine announced reorganization of the fleet to combat attempted blockades~-- and that was the last heard of that project. Franco’s transports moved at will from Ceuta to Algeciras, bringing tens of thousands of troops and the armaments they required.

\index{Anarchism}
In a letter to Montseny\kn, demanding that the anarchist ministers publicly fight against the false governmental policies, Camillo Ber\-neri said of the navy:
\nowidow

\indexIPrieto\index{Morocco}\index{Berneri, Camillo}
\begin{quotation}
  The concentration of the forces coming from Morocco, the piracy in the Canaries and the Balearics, the taking of M\'alaga, are the consequences of this inactivity. If Prieto is incapable and inactive, why tolerate him? If Prieto is bound by a policy which paralyzes the fleet, why not denounce this policy?
\end{quotation}

\index{Foreign influence!England}\index{Foreign influence!France}\index{Morocco}\index{Counterfactual}\index{Ministries!Navy}
Why did Prieto and the governmental bloc follow this suicidal policy? It was simply one factor of the whole policy which rested on securing the goodwill of England and France. What these were seeking is clear. An aggressive Loyalist naval policy\kn, as the August incidents off Morocco had shown, would have precipitated the decisive stage of the civil war. It would have threatened to crush Franco immediately. Germany and Italy, their prestige involved in supporting Franco, would perhaps have been driven to desperate steps in his defense, such as, the open resort to use of the Italian and German navies in sweeping the Loyalists out of the straits. But England and France could not have tolerated Italo--German control of the straits (it might be retained thenceforward). That open war would thus begin was, of course, not a certainty. Especially prior to November~9, 1936, when Germany and Italy formally recognized the Burgos regime, Germany and Italy might have retreated before precipitating war. Had revolutionists been at the helm and boldly moved in August and September to a systematic naval campaign, and succeeded in cutting off Morocco from Spain, the probability was that Italy and Germany would have retreated as gracefully as they could.

Anglo--French imperialism, however, was not interested in a Loyal\-ist victory but in staving off a war crisis while resisting encroachments on their imperialist interests in the Mediterranean. And they had their way due to the Anglo--French orientation of the Loyalist government. Each month that passed thereafter\kn, involving Germany and Italy more deeply\kn, rendered more and more likely an international explosion if the Loyalist navy were activized. \emph{It simply ceased to exist as a Loyalist weapon.}

Here is the first terrible instance of how anti-revolutionary politics hamstrung the military struggle.

\index{Andalusia!Algeciras}\index{Andalusia!Malaga@Málaga}\index{Battle}\index{Nationalist army}\indexCNT\index{Ministries!Navy}
The same Anglo--French orientation explains the failure to strike by land at Algeciras, the Spanish port at which the fascist forces were landing from Morocco. M\'alaga was strategically located to be the spearhead for this drive. Instead, M\'alaga itself was left defenseless. Chiefly defended by \CNT\ forces, who pleaded in vain from August until February for the necessary equipment, Malaga was invaded by an Italian landing force, while the fleet which could have stopped them rode at anchor in Cartagena. M\'alaga fell on February~8. For two days before that, the militias had received no instructions from the military headquarters and then, the day before M\'alaga fell, they discovered that the headquarters had already been abandoned without a word to the defending militiamen. It was not a military defeat, but a betrayal. The base treachery was not the last-minute desertion by the general staff but the political policy which dictated the inactivity of the navy and disuse of M\'alaga as a base against Algeciras.%
\endnote{%
  On February~21, Undersecretary of~War, José~Asenio, was dismissed, and soon arrested together with Colonel Villalba, for the betrayal of Malaga. War Commissar Bolivar, a Stalinist, who had joined Villalba in abandoning headquarters, was not arrested. Nor was a word breathed~-- until the National Committee of the \CNT\ got really desperate for the moment~-- at the Stalinists’ assaults~-- that Antonio Guerra, Stalinist representative in the Military Command of M\'alaga, stayed behind and went over to the fascists. (\emph{\CNT\ Boletin Valencia,} August 26, 1937).

  The day Gij\'on fell~-- eight months later~-- the government announced it would try the M\'alaga traitors~-- Asensio, his chief of staff, Cabrera, and another general. Why try these and not those guilty at Bilbao, Santander, etc., etc.? Because M\'alaga fell under Caballero, while the far more brazen betrayals in the north took place under Negr\'in\dots.

  \index{Stalinism}\indexCNT\index{Andalusia!Malaga@Málaga}
}

\medskip

If not by sea or by land, there was still another way of striking at Franco’s Moroccan base. We quote Camillo Berneri:

\index{Berneri, Camillo}\index{Morocco!Autonomy}\index{Foreign influence!Arab world}
\begin{quotation}
  The fascist army’s base of operations is in Morocco. We should intensify the propaganda in favor of Moroccan autonomy on every sector of Pan-Islamic influence. Madrid should make unequivocal declarations, announcing the abandonment of Morocco and the protection of Moroccan autonomy. France views with concern the possibility of repercussions and insurrections in North Africa and Syria; England sees the agitation for Egyptian autonomy being reinforced as well as that of the Arabs in Palestine. It is necessary to make the most of such fears by adopting a policy which threatens to unloose revolt in the Islamic world.
  
  For such a policy money is needed and speed to send agitators and organizers to all centres of Arab emigration, to all the frontier zones of~French Morocco. (\emph{Guerra di Classe,} October 24, 1936).
\end{quotation}

\index{Morocco}\index{Foreign influence!England}\index{Foreign influence!France}\index{Ministries!Foreign Affairs}
But the Loyalist government, far from arousing French and English fears by inciting insurrection in Spanish Morocco, proceeded to offer them concessions in Morocco! On February~9, 1937, Foreign Minister del~Vayo delivered to France and England a note, the exact text of which was never revealed but was stated later, without denial by the cabinet, to include the following points:

\index{Foreign intervention!Italy}\index{Foreign intervention!Germany}
\begin{enumerate}
  \item Proposing to base its European policy on active collaboration with Great Britain and France, the Spanish government proposes the modification of the African situation.
  
  \item Desiring a rapid end to the civil war, now susceptible of being prolonged by German and Italian aid, the government is disposed to make certain sacrifices in the Spanish zone of Morocco, if the British and French governments should take steps to prevent Italo--German intervention in Spanish affairs.
\end{enumerate}

The first inkling of the existence of this shameful note came a month after its dispatch, in the French and English press on March~19, when Eden made a passing reference to it. The \CNT\ ministers swore they had not been consulted in its dispatch. Berneri addressed them bitingly:

\index{Berneri, Camillo}\indexCNT\index{Reconciliation}
\begin{quotation}
  You are in a government which has offered France and England advantages in Morocco, while from July, 1936, it should have been obligatory on us to proclaim officially the political autonomy of Morocco\dots.
  The hour has come to make known that you, Montseny, and the other anarchist ministers are not in agreement with the nature and purport of such proposals\dots.
  It goes without saying that one cannot guarantee English and French interests in Morocco and at the same time agitate for an insurrection there\dots.
  
  But this policy must change. And to change it, a clear and strong statement of our own intentions must be made~-- because at Valencia, some influences are at work to make peace with Franco. (\emph{Guerra di~Classe,} April~14, 1937). 
\end{quotation}

But the anarchist leaders remained silent, and Morocco remained undisturbed in Franco’s power.%
\endnote{%
  The only official \CNT\ pamphlet dealing with Morocco that I have been able to find is: ``What Spain Could Have Done in Morocco and What It Has Done,\kn\kn'' a speech by Gonzalo de~Reparaz on January~17, 1936, telling how he tried to get the monarchy and the republic to organize things efficiently in Morocco, and how they did not! Not a hint that the only advice a revolutionist can give on the colonial question is: get out of Morocco.
}

\section[The Aragon Offensive against Zaragoza and Huesca]{The Aragon Offensive against Zaragoza \\ and Huesca}

\index{Aragon!Aragon Offensive}\index{Aragon!Huesca}\index{Aragon!Zaragoza}\index{Battle}
Thumb through the Spanish or the French or American press of August to November 1936 and note the sharp contrast between the Loyalist defeats on the central and western fronts and the victories on the Aragon front. The \CNT\kn, \FAI, and \POUM\ troops predominated in Aragon. They obeyed the military orders of the bourgeois officers sent by the government but kept them under surveillance. By the end of October, having captured the surrounding heights of Montearagón and Estrecho Quinto, the Aragon militias were in position to take Huesca, the gateway to Zaragoza.

\indexCNT\index{Treason}
The importance of taking Zaragoza can be readily understood by a glance at the map. It lies athwart the road from Catalonia and Aragon to Navarre, the heart of the fascist movement. Saragossa taken, the rear of the fascist army facing the Basque province would be endangered, as well as the rear of the forces converging on Madrid from the north. An offensive on this front, therefore, would have enabled the initiative in the military struggle to pass to the Loyalists. Moreover, Zaragoza had been one of the great strongholds of the \CNT\kn, and had only fallen to the fascists because of the outright treachery of the civil governor, a member of Azaña’s party and his appointee.

\index{Strikes!General strike}
As late as the end of September, a general strike of the workers was still going on in Zaragoza, though its leaders were being killed by slow torture for refusing, to call it off. A strong attack on Zaragoza would have been accompanied by an uprising of the workers within, as the anarchists pledged.

\index{Military supplies!Planes}\index{Military supplies!Artillery}
To take the strongly fortified cities of Huesca and Zaragoza, however\kn, required planes and heavy artillery.

\index{Aragon!Aragon Offensive!Government boycott}\index{Workers' militias!Liquidation of militias}\index{Political repression}\index{Bourgeois--Stalinist bloc}
But from September onward, there developed a systematic boycott, conducted by the government against the Aragon front. The artillery and planes which arrived from abroad, beginning in October, were sent only to the Stalinist-controlled centers. Even in the matter of rifles, machine guns and ammunition, the boycott was imposed. The Catalonian munition plants, dependent on the central government for financing, were compelled to surrender their product to such destinations as the government chose. The \CNT\kn, \FAI, and \POUM\ press charged that the brazen discrimination against the Aragon front was dictated by the Stalinists, backed by the Soviet representatives. (Caballero’s friends now admit this.) The government plans for liquidating the militias into a bourgeois army could not be carried out so long as the \CNT\ militias had the prestige of a string of victories to their credit. Ergo, the Aragon front must be held back.

\index{Durruti, Buenaventura}\index{Oliver, Juan Garcia@Oliver, Juan Garc\'ia}\index{Workers' militias}\indexCNT
This situation, among others, drove the \CNT\ leaders into the central government. The two principal figures of Spanish anarchism, Garc\'ia Oliver and Buenaventura Durruti, transferred their activities to Madrid, Durruti bringing the pick of the troops of the Aragon front. But the boycott of the Aragon front continued despite all anarchist concessions. For it was fundamental to the strategy of the bourgeois--Stalinist bloc to break down the prestige and power of the \CNT\kn, no matter what the cost. Six months of anarchist and \POUM\ press complaints and demands for an offensive on the Aragon front, met with dead silence from the bourgeois--Stalinist press. Then the Stalinists proceeded to slander the \CNT\ militiamen’s inactivity on that front, and to adduce that fact as proof of the need for a bourgeois army. The \CNT--\POUM\ counter-proposal for a unified command and a disciplined army under workers’ control was defeated.

\index{Framing}\indexPOUM\index{Trotskyism}\index{Military supplies}
For many months the Stalinists denied to the outside world their sabotage of the Aragon front. But when the fact became too well known, the Stalinist produced an alibi: there were arms aplenty sent to the Aragon front but the ``Trotskyists'' diverted them across no man’s land to the fascists. (\emph{Daily Worker,} October 5, 1937).

Like all Stalinist frame-up stories, this one carried its inherent falsity on its face. The \POUM~-- the alleged Trotskyists~-- had at most ten thousand men on this front. The dominant force here was the \CNT\kn. Were they~-- with their press clamoring for these arms~-- such dolts as not to see what the \POUM\ was doing? Or is this story merely a preparation for the day when the Stalinists will accuse the \CNT\ of having connived with the \POUM\ in diverting arms to the fascists?
\indexCNT

\index{Orwell, George}\index{Military supplies}\index{Foreign revolutionaries}
The pitiful armament on the Aragon front has been described by the English author, George Orwell, who fought there in the \textsc{ilp}~Battalion. The infantry ``were far worse armed than an English public school Officers’ Training Corps'' with ``worn-out Mauser rifles which usually jammed after five shots; approximately one machine gun to fifty men; and one pistol or revolver to about thirty men. These weapons, so necessary in trench warfare, were not issued by the government and could be bought only illegally and with the greatest difficulty.\kn\kn''

``A government which sends boys of fifteen to the front with rifles forty years old and keeps its biggest men and newest weapons in the rear,\kn\kn'' concluded Orwell, ``is manifestly more afraid of the revolution than of the fascists. Hence the feeble war policy of the past six months, and hence the compromise with which the war will almost certainly end.\kn\kn'' (\emph{Controversy,} August 1937).

Thus, the government surrendered the opportunity provided in Aragon to wrest the initiative and carry the war into fascist territory.

\section{The Northern Front}

\index{Basque Country!Biscay!Bilbao}\index{Castile and León!Burgos}\index{Aragon!Aragon Offensive}\index{War in the North}
Bilbao, and the industrial towns and iron and coal mines surrounding it, constituted a concentrated industrial area second only to Catalonia. For war purposes, it was even superior to the Catalonian area, which had to build its metallurgical plants up out of nothing when the civil war began. Bilbao should have become the center of Spain’s greatest munitions source. From this material base, the northern armies should have driven sharply south toward Burgos and east against Navarre, to effect a junction with the troops from the Aragon front. The strategy dictated was of the most elementary kind.

\index{Bourgeoisie}\index{Foreign influence!England}\index{Private property}
The Basque capitalists, however\kn, were the masters in the Biscayan region. As an English sphere of influence for centuries, it had no enthusiasm for joining Franco and his Italo--German allies. Neither\kn, however\kn, had the Basque bourgeoisie any intention of fighting to the death against Franco. Thanks to the support of the socialist and communist parties, the Basque capitalists had not had their factories seized by the workers after July~19. But they had no guarantee that a Loyalist victory over Franco would not be followed by seizure of their factories also.

\index{Military surrender}\index{Collaboration}\indexCNT\index{Basque Country!San Sebastian@San Sebasti\'an}\index{Battle}\index{Bourgeoisie}\index{Private property}
The property question determined the military conduct of the Basque regional government. This was seen as early as mid-Sep\-tem\-ber 1936 when the fascists advanced on San~Sebasti\'an. Before the attack was well launched, San~Sebasti\'an surrendered. Before the Basque bourgeoisie retreated, they drove out of the city the \CNT\ militiamen who wanted to destroy factory equipment and other useful materials, to prevent them from falling into the hands of the fascists. As a further precaution, fifty armed Basque guards were left behind to protect the buildings. Thus, the city was delivered intact to Franco. The bourgeoisie reasoned; \emph{destroyed property is gone forever; but if we eventually make peace with Franco, he may give us back our property\dots.}

\index{War in the North}\index{Battle}
When this happened, I wrote, on September 22, 1936: ``\kp The northern front has been betrayed.\kn\kn'' The anarchist ministers have since revealed that this was the opinion in the Caballero cabinet. What delayed the completion of outright betrayal for six months, however\kn, was the stupidity of Franco’s officers who took over San~Sebasti\'an. The fifty guards left behind to protect the buildings were shot; bourgeois proprietors who had remained behind to make their peace with Franco were imprisoned, some of them executed. The inhabitants were terrorized. The Basque front stiffened~-- for a little while.
\nowidow

\medskip

\index{Military surrender}\index{Basque Country!Government}\index{Collaboration}
By December\kn, however\kn, the Basque government was again feeling its way to an armistice. At a time when Madrid was still rejecting all negotiations, for exchange of prisoners, the Basques negotiated such an agreement:

\begin{quotation}
  \indexNYT\index{Military surrender}
  
  The fact that the Basque group was negotiating in San~Sebasti\'an became known only yesterday. The writer learned, however, that the delegation had left Bilbao more than a week ago \dots\ proceeded to Barcelona but its mission there ended unsatisfactorily. The Basque delegates expressed their disappointment with the state of affairs in the Catalan capital \dots\ and it is thought that they also took offense at the attitude of the Catalans toward the church.
  
  In any case, the result has been that they decided to sound out the San~Sebasti\'an leaders in the hope of arranging some sort of compromise and perhaps ultimately a truce.
  
  It is known that during the last month or two the northern front has been quiet, with much fraternizing between those in the opposing front lines. (Hendaye frontier dispatch, \emph{New York Times,} December 17, 1937).
\end{quotation}

\index{Foreign influence!England}
Any doubt as to the substance of this report was dispelled the same day by ``Augur,\kn\kn'' the ``unofficial'' voice of the British Foreign Office.

\index{Basque Country!Biscay!Bilbao}\index{Military surrender}
\begin{quotation}
The British have been working to promote local armistices between the rebels and loyalists. The offer of the Basque regional government at Bilbao to conclude a Christmas truce was directly due to discreet intervention by British agents who hope this may lead to a complete suspension of hostilities.
\end{quotation}

\noindent
\index{Foreign influence!France}\indexNYT
The French, added ``Augur,\kn\kn''

\begin{quotation}
\noindent
—\,are exercising similar influence in Barcelona where their success is less marked because the desires of President Companys to end the bloodshed have been overawed by the communists and anarchists. (\emph{New York Times,} December 17, 1936).
\end{quotation}

\index{Censorship}
Nothing of this, of course, appeared in the Loyalist press, where the censorship was now in full blast. Such circumstantial accounts, particularly one bearing the name of \kp ``\kn Augur\kn,\kn\kn'' appearing in papers of the stature of the \emph{New York Times} and the \emph{Times} of London, required at the last, a formal denial, if denial could be made. Neither the government nor the Stalinist press, however\kn, dared deny the facts: for they were true.

\index{Bourgeoisie}\index{Collaboration}\indexCNT\index{Basque Country!Government}\index{Political repression}\index{Basque Country!Defense Junta}
The Basque bourgeoisie simply had no basic stake in fighting fascism. If the struggle involved serious sacrifice, they were ready to withdraw. One of the factors which gave them pause, however, was the growing \CNT\ movement in the Basque regions. Here the Stalinists and right-wing socialists, sitting in the regional government with the bourgeoisie (the \CNT\ had been dropped when the Defense Junta gave way to the government), facilitated the betrayal. On the flimsiest pretext imaginable~-- the Basque government invited the \CNT\ militiamen to join in celebrating Easter Week and the \CNT\ Regional Committee and press indignantly denounced the religious ceremonial~-- the whole regional committee and the editorial staff of `\CNT\ del~Norte' were imprisoned March~26, and the printing presses turned over to the Stalinists! Systematic persecution of the \CNT\ thereafter paved the way for going over to Franco.
\noclub

\index{Basque Country!Biscay!Bilbao}\index{Mola, Emilio}\index{Military supplies!Manufacturing}
The Loyalist government was aware of the danger\kn, aware of Bilbao’s failure to transform her plants for munitions purposes, aware of the criminal inactivity of the Basque front which enabled Mola to shift his troops southward to join the encirclement of Madrid. Why did the government do nothing about it? Of course, the cabinet sent numerous emissaries to Bilbao, flattered the Basques, went out of the way to please them, sent generals to collaborate with the Basque leaders~-- Llano de~Encomienda, just freed by a court martial in Barcelona from charges of complicity in the uprising became commander-in-chief of the North!~-- but these measures naturally proved fruitless.

\index{United front}
There was only one way to save the northern front: by confronting the Basque bourgeoisie with a powerful united front of the proletarian forces in the region, ready to take power if the bourgeoisie faltered, and to prepare for this by ideological criticism of the Basque capitalists. That way\kn, however\kn, was alien to this government which, above all, feared to arouse the masses to political initiative.

\index{Asturias}\indexCNT\indexUGT\index{Battle}
But there was one sector of the northern front which was active, Asturias. We have seen how, within forty-eight hours of news of the rising, five thousand Asturian miners had arrived in Madrid. Within a few weeks, they had cleared out the fascists, except in well-fortified Oviedo, which had been the seat of a strong praetorian garrison since the crushing of the Asturian Commune of October 1934. Every miner in Asturias would have given his life to take Oviedo. Armed with little more than rifles and crude dynamite bombs, the miners besieged Oviedo, soon taking its suburbs. The fall of Oviedo would have freed them for an offensive against Old Castille. Spokesmen for the Asturians begged in Valencia for a few planes and the necessary artillery to batter down Oviedo’s defences. They were sent away empty-handed.

\index{Asturias!Asturian Commune}\index{Collectivization}
What was their crime? The Asturian workers abolished private property in land, collectivized housing and industry. The strong \CNT\ movement, hand in hand with the \UGT~-- here revolutionary in tendency, as was demonstrated by its organ, \emph{Avance,} under the editorship of Javier Bueno~-- exclusively controlled production and consumption. It was known that they intended, when Oviedo was theirs, to proclaim there again, as in 1934, the Commune of~Asturias\dots.

\index{Battle}
The government invited them to shed their blood everywhere except for the commune. Tens of thousands of them, for want of any other course, joined the Loyalists on all the fronts. Their fighting prowess became legendary. But enough of them remained before Oviedo, penning in the garrison, until the very end\dots.

\section{Why Madrid Became the Key Front}

With Morocco and its communications lines with the mainland un\-dis\-turbed, with the northern front quiescent thanks to Basque passivity\kn, and with governmental sabotage of the Aragon front, Franco was in a position to dictate the course of the war\kn, to choose his points of offensive at will. He never once lost the initiative to the Loyalists, who had to accept battle when and as the enemy willed.

\index{Madrid}\indexFFranco\index{Foreign intervention!Germany}\index{Foreign intervention!Italy}\index{Battle}
Thus, Franco was enabled to throw his main forces against Madrid. By October the encirclement of Madrid was well on the way. Franco wanted the nation’s capital in order to provide his German and Italian allies with a plausible basis for recognizing his regime. And, indeed, by all accounts, German and Italian recognition were extended on November~9, 1936, in the belief that Madrid was about to fall and that recognition would provide the added incentive to make its fall speedy. By all accounts, too, Franco made his major strategical blunder here, when he attempted, in his haste to take Madrid by frontal attack instead of completing its encirclement by cutting the Valencia road. The fascists stubbornly clung to this strategy for months, giving the Loyalists the opportunity to fortify the area sufficiently to withstand the flank attacks when they came in February and March.

\index{Madrid!Defense Junta}\index{Communist International}\index{Stalinism}
The significant fact to note in the defense of Madrid was \emph{the use of revolutionary political methods.} If Madrid fell, the jig was up for the Stalinists. In Spain, their prestige was bound up with the Fifth Regiment of Madrid~-- in reality an army of over a hundred thousand men~-- and the Defense Junta which from October~11 was responsible for Madrid’s defense and which was Stalinist-controlled. Internationally\kn, the prestige of the Comintern and the Soviet Union would have collapsed irrevocably with the fall of Madrid. The retreat to Valencia and Catalonia would have found a new relationship of forces, with the Stalinists taking a back seat. Out of that new phase might have come a resort to a revolutionary war against fascism, which would have put an end to all the schemes of Eden, Delbos and Stalin. Madrid absolutely had to be held. In dire necessity\kn, the Stalinists abandoned purely bourgeois methods~-- but only for a time and only within the confines of Madrid.

\indexCNT\index{Leaflet}
Methods of defense which, in other cities, were proposed by the local \POUM\kn, \FAI\ and \CNT\ organizations and denounced as adventuristic, as alienating the liberal bourgeoisie, were sanctioned here by the Stalinists themselves, on November~7\kn, when the fascist drive reached the city’s suburbs.

\medskip

A \CNT\ leaflet of that week is worth quoting:

\index{Fifth column}\index{Madrid}\index{Battle}
\begin{quotation}
  Yesterday\kn, we warned the people of Madrid that the enemy was at the gates of the city\kn, and we advised them to fill bottles with petrol and attach fuses to them to be lighted and thrown onto rebel tanks when they enter the city.
  
  \index{Fortress of the revolution@``Fortress of the revolution''}
  Today, we suggest other precautions. Every house and apartment in the district known to be inhabited by fascist sympathizers should be thoroughly searched for arms. Para\-pets and barricades should be set up in all streets leading to the city’s center.
  
  \index{Military supplies!Molotov cocktails}\index{Military supplies!Machine guns}
  Every house in Madrid in which antifascists live must constitute itself a fortress, and every obstacle should be offered as invaders try to pass through the capital’s streets. Fire on them from the upper stories of buildings, against which the fire of their machine guns will be ineffective. Above all we must purge Madrid of the fifth column of hidden fascists.
\end{quotation}

\index{Mola, Emilio}\index{Dual power}\index{Popular Front}\index{Stalinism}
One of Mola’s boasts~-- that four columns were converging on Madrid, with a fifth secretly forming inside~-- had given the workers the splendid slogan: smash the fifth column. Gone now were the governmental (and Stalinist) strictures against ``illegal searches,\kn\kn'' ``unauthorized seizures and arrests,\kn\kn'' etc., etc. Over five hundred Assault Guards were arrested and imprisoned in those days as fascist suspects~-- the first and last time the Stalinists sanctioned such a purge of bourgeois elements. The Stalinists were on record for ``all power to the government of the people’s front,\kn\kn'' and, therefore, hostile to workers’ committees in the factories and districts. For once, however, desperation caused them to abandon this. The Stalinist-controlled Fifth Regiment issued a manifesto which, among other things, called for the masses to elect street and house committees for vigilance against the fifth column within the city!\kp%
\endnote{%
Ralph Bates mentions this fact (\emph{New Republic,} October 27, 1937) as if to imply it was typical of Stalinist policy. I challenge him to find a single other instance subsequently in which the Stalinists made a similar proposal.
  
\index{Bates, Ralph}
}

\index{Workers' committees!Street and house committees}\index{Madrid!Defense Junta}\index{Military supplies!Manufacturing}\index{Political repression}\index{Battle}\index{Madrid}\index{Madrid!People in arms@``The people in arms''}
Workers’ committees went through the streets, impressing all able-bodied men into building barricades and trenches. The Defense Junta organized separate councils for food, munitions, etc., each of which swelled daily into a mass organization. Women’s committees organized kitchens and laundries for the militiamen. Means were found in this non-industrial city to begin~-- this too on initiative from below~-- the manufacture of ammunition. The Stalinists did not forget to continue their persecution of the \POUM\kn, but even this slowed down and the \POUM\ militants were given room to function in the defense of the city. Those were glorious months, though laden with death: November, December, January. What was this? ``\kp The people in arms.\kn\kn''

\indexCNT\index{Durruti, Buenaventura}\index{Aragon}
The Stalinists were even so desperate as to welcome the triumphal entry into Madrid of the picked troops from the \CNT\ Aragon Front columns, whose heroic conduct destroyed the slanderous myth, already being propagated by the Stalinists, about the Aragon militias. Shortly after bringing these troops, however, the greatest military figure produced by the war\kn, the anarchist Durruti, was killed and the spotlight was turned on Miaja.

\index{Madrid}
But the political methods pursued on the southern, northern and Aragon fronts, remained the same. The incessant campaign of the \CNT\kn, the \POUM\kn, and sections of the \UGT\ for an offensive on all fronts as the best way to help Madrid, and the only way to lift the siege of the city, was ignored.

\index{Madrid!People in arms@``The people in arms''}\index{Bourgeois--Stalinist bloc}\index{Political repression}\index{Censorship}
Nor did the ``people in arms'' remain the defenders of Madrid. By January the immediate danger was over, and the Stalinist--bourgeois bloc reverted to ``normal.\kn\kn'' The house-to-house searches for fascists and arms by the workers’ committees were discouraged and then suppressed. Soldiers replaced workers at the street barricades. The work of the women’s committees was taken over by the army. Mass initiative was no longer invited. The current ran now the other way, although the siege of Madrid had not been lifted. `\POUM,\kn\kn' a weekly\kn, was permanently suspended in January. In February\kn, the Junta seized the \POUM\ radio and the presses of \emph{El~Combatiente Rojo.}

\index{Madrid!Defense Junta}\index{Political prisoners}\index{Ministries!Prisons}\indexCNT
The Stalinist, José Cazorla, the Junta’s police commissioner\kn, organized the repression both legally and illegally. If his arrests of workers were not sanctioned by the Popular Tribunals, he took ``said acquitted parties to secret jails or sent them into communist militia battalions in advanced positions'' to be used as ``fortifications.\kn\kn'' Simultaneously, the bars were let down on the right, and Cazorla released many fascists and reactionaries. These charges were made by Rodrigues, Special Commissioner of Prisons (\emph{Solidaridad Obrera,} April~20, 1937), and the \CNT\ demand for an investigation was refused. The dissolution of the Junta completed the move to bourgeois bureaucratic methods of conducting the defense of Madrid.
\index{Political repression}

\indexLCaballero\index{Foreign intervention!Italy}\index{Madrid}
The sole military victory of the Caballero cabinet was the Guadalajara defeat of the Italian divisions in March~-- an unexpected victory as was indicated by the lack of preparations of reserves and materials to complete the rout of the Italians. The failure to coordinate the Madrid fighting with offensives on the other fronts, for the political reasons we have outlined, thus by default made Madrid the key front and simultaneously made impossible lifting of the siege of Madrid.