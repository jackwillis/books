\chapter{Military Struggle Under Giral and Caballero}

\index{Dual power}\index{Battle}
\lettrineM{ilitary warfare is} merely the continuation of politics by forcible means. A proclamation dropped over the enemy lines, expressing the aspirations of the landless peasants, is also an instrument of warfare. A successfully incited revolt behind enemy lines may be infinitely more efficacious than a frontal attack. Maintenance of the morale of the troops is as important as equipping them. To guard against treacherous officers is as important as training efficient officers. In sum, the creation of a workers’ and peasants’ government for which the masses will work and die like heroes is the best political adjunct of military struggle against the fascist enemy in civil war.

\index{Russia!Civil War}
By these methods, the workers and peasants of Russia defeated imperialist intervention and White Guard armies on twenty-two fronts, despite the most rigid economic blockade ever imposed on any nation. In the organization and direction of the Red Armies under these adverse conditions, Trotsky seemed to perform miracles, but these were miracles compounded of revolutionary politics, of the capacities for sacrifice, labor and heroism of a class defending its newly-won freedom.

That reactionary political policies determined the false military policies of the Loyalist government can be demonstrated by surveying the course of the military struggle.

\indexIPrieto
From July~19 until September~4, 1936---seven decisive weeks---the Giral cabinet of the People’s Front was at the helm, with the unconditional political support of the Stalinists and the Prieto Socialists (Prieto, indeed, was unofficially part of the ministry, establishing an office in the government on July 20).

\indexJGiral\indexGRobles\index{Workers' militias}
The Giral government had about \$600,000,000 in gold at its disposal. Recall that the real embargo on the sale of munitions to Spain was not established until August~19, when the British Board of Trade revoked all licences for export of arms and planes to Spain. Thus, the Giral regime had at least a month in which to purchase stores of arms---but the damning fact is that it bought almost nothing! The story of the treacherous attempt of Azaña--Giral to reach a compromise with the fascists has already been told. One further fact: Franco and his friends waited six days before forming their own government. Gil~Robles later revealed that they waited in expectation of a satisfactory arrangement with the Madrid government. By that time the militias had emerged from the ranks of the workers and Giral no longer had the power to meet Franco’s demands.

The most important gains of the first seven weeks were the successful march on Aragon by the Catalan militias, using socialization of the land as much as they used their rifles; and the attack of the loyalist warships on Franco’s transportation of troops from Morocco to the mainland.

Wrote two Stalinists at the time,

\begin{quotation}
  The loyalty of a large part of the navy decisively prevented Franco from transporting large numbers of Moroccan troops to the mainland in the first two weeks of the war. The naval patrol in the south made transport by sea extremely hazardous. Franco was forced to resort to airplane passage, but this was slow work. In this respect, again, the government was given a chance to organize defense and take stock. (\emph{Spain in Revolt,} Gannes and Repard, p.~119).
\end{quotation}

\index{Sailors}\index{Ministries!Navy}
What they failed to add was that the warships were under the command of elected sailors’ committees, which, like the militias, had no faith in the Giral government and carried on operations, despite the passivity of the government. The significance of this fact will become apparent when we come to the naval policy of the Caballero--Stalinist--Prieto cabinet.

The terrible defeats of Badajoz and Irun finished the Giral cabinet. Why Irun fell was told in a moving dispatch by Pierre~van~Paassen:

\index{Battle}\index{Irun}
\begin{quotation}
  They fought to the last cartridge, the men of Irun. When they had no more ammunition, they hurled packets of dynamite. When dynamite was gone, they rushed forward barehanded and tackled each their man, while the sixty times stronger enemy butchered them with bayonets. A girl held two armored cars at bay for half an hour by hurling glycerin bombs. Then the Moroccans stormed the barricade of which she was the last living defender and tore her to pieces. The men of Fort~Martial held three hundred foreign legionnaires at a distance for half a day by rolling rocks down the hill on which the old fort is perched.
\end{quotation}

\index{Central Committee of Anti-Fascist Militias of Catalonia}
Irun fell because the Giral government had made no attempt to provide its defenders with ammunition. The Central Committee of Anti-Fascist Militias of Catalonia, having already transformed the available factories into munitions works, had sent several carloads of ammunition to Irun by way of the regular railroad from Catalonia to Irun. But that railroad runs partway through French territory. And the government of ``Comrade'' Blum, the ally of Stalin, had held up the cars at Behobia, just across the frontier, for days \dots\ they rumbled across the bridge into Irun after the fascists had won.

\indexLCaballero\indexIPrieto\index{Popular Front}
The Giral cabinet gave way to the ``real, complete'' People’s Front government of Caballero--Prieto--Stalin. Undoubtedly, it had the confidence of a large part of the masses. The militias and the sailors’ committees obeyed its orders from the first. There were three major military campaigns that the new government had to undertake. There were, of course, other tasks, but these were the most important, the most pressing, and essentially, the simplest.

\subsection*{Morocco and Algeciras}

\indexFFranco\index{Spanish Legion}\index{Morocco}
Franco’s military base during the first six months was Spanish Morocco. From here he had to bring his Moors and Legionnaires and military stores.

\index{Sailors}\index{Workers' militias}\index{Battle!Naval}\index{Foreign intervention!Italy}
The first successes of the Loyalist navy under the sailors’ committees in harassing Franco’s communication lines with Morocco were followed by others. On August~4, the Loyalist cruiser \emph{Libertad} effectively shelled the fascist fortress at Tarrifa in Morocco. It was a deadly blow at Franco. So deadly that it was answered by the first open Italian act of intervention: an Italian plane proceeded to bomb the \emph{Libertad.} When Loyalist warships steamed into position for a large-scale shelling of Ceuta in Morocco while fascist transports were loading, the German battleship \emph{Deutschland} brazenly steamed back and forth between the Loyalist warships and Ceuta to prevent the bombardment. A week later a Spanish cruiser stopped the German freighter \emph{Kamerun,} found her loaded to the decks with arms for Franco, and prevented her from landing at Cadiz. Whereupon Portugal openly went over to the fascists, permitting the \emph{Kamerun} to unload at a Portuguese port and forwarding the munitions by rail to Franco. German naval commanders received orders to fire on any Spanish vessel attempting to stop German munition shipments. The Loyalist naval operations, if continued, were fatal to Franco, and his allies had to unmask completely to save him.

\indexLCaballero\indexIPrieto\index{Ministries!Navy}
At this point the Caballero cabinet was formed, and Prieto, now in the closest collaboration with the Stalinists, and always ``France’s man,'' became head of the Naval Ministry. He put an end to naval operations off Morocco and the straits of Gibraltar, and recalled the Loyalist forces which had held Majorca.

\index{Spanish Legion}\index{Algeciras}
The task of the hour was to prevent the Moors and Legionnaires from landing at Algeciras and constituting that army which was soon to make that fearful march from Badajoz straight through to Toledo, through Toledo and Talavera~de~la~Reina to the gates of Madrid. The first line in that task belonged to the navy. \emph{It was not used for this purpose.}
\nowidow

sdf