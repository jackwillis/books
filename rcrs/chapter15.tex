\chapter{Military Struggle Under Giral and Caballero}

\index{Dual power}\index{Battle}
\lettrineM{ilitary warfare is} merely the continuation of politics by forcible means. A proclamation dropped over the enemy lines, expressing the aspirations of the landless peasants, is also an instrument of warfare. A successfully incited revolt behind enemy lines may be infinitely more efficacious than a frontal attack. Maintenance of the morale of the troops is as important as equipping them. To guard against treacherous officers is as important as training efficient officers. In sum, the creation of a workers’ and peasants’ government for which the masses will work and die like heroes is the best political adjunct of military struggle against the fascist enemy in civil war.

\index{Russia!Civil War}
By these methods, the workers and peasants of Russia defeated imperialist intervention and White Guard armies on twenty-two fronts, despite the most rigid economic blockade ever imposed on any nation. In the organization and direction of the Red Armies under these adverse conditions, Trotsky seemed to perform miracles, but these were miracles compounded of revolutionary politics, of the capacities for sacrifice, labor and heroism of a class defending its newly-won freedom.

That reactionary political policies determined the false military policies of the Loyalist government can be demonstrated by surveying the course of the military struggle.

\indexIPrieto
From July~19 until September~4, 1936---seven decisive weeks---the Giral cabinet of the People’s Front was at the helm, with the unconditional political support of the Stalinists and the Prieto Socialists (Prieto, indeed, was unofficially part of the ministry, establishing an office in the government on July 20).

\indexJGiral\indexGRobles\index{Workers' militias}
The Giral government had about \$600,000,000 in gold at its disposal. Recall that the real embargo on the sale of munitions to Spain was not established until August~19, when the British Board of Trade revoked all licences for export of arms and planes to Spain. Thus, the Giral regime had at least a month in which to purchase stores of arms---but the damning fact is that it bought almost nothing! The story of the treacherous attempt of Azaña--Giral to reach a compromise with the fascists has already been told. One further fact: Franco and his friends waited six days before forming their own government. Gil~Robles later revealed that they waited in expectation of a satisfactory arrangement with the Madrid government. By that time the militias had emerged from the ranks of the workers and Giral no longer had the power to meet Franco’s demands.

The most important gains of the first seven weeks were the successful march on Aragon by the Catalan militias, using socialization of the land as much as they used their rifles; and the attack of the loyalist warships on Franco’s transportation of troops from Morocco to the mainland.

