\chapter{Military Struggle Under Giral and Caballero}

\index{Dual power}\index{Battle}
\lettrineM{ilitary warfare is} merely the continuation of politics by forcible means. A proclamation dropped over the enemy lines, expressing the aspirations of the landless peasants, is also an instrument of warfare. A successfully incited revolt behind enemy lines may be infinitely more efficacious than a frontal attack. Maintenance of the morale of the troops is as important as equipping them. To guard against treacherous officers is as important as training efficient officers. In sum, the creation of a workers’ and peasants’ government for which the masses will work and die like heroes is the best political adjunct of military struggle against the fascist enemy in civil war.

\index{Russia!Civil War}
By these methods, the workers and peasants of Russia defeated imperialist intervention and White Guard armies on twenty-two fronts, despite the most rigid economic blockade ever imposed on any nation. In the organization and direction of the Red Armies under these adverse conditions, Trotsky seemed to perform miracles, but these were miracles compounded of revolutionary politics, of the capacities for sacrifice, labor and heroism of a class defending its newly-won freedom.

That reactionary political policies determined the false military policies of the Loyalist government can be demonstrated by surveying the course of the military struggle.

\indexIPrieto
From July~19 until September~4, 1936---seven decisive weeks---the Giral cabinet of the People’s Front was at the helm, with the unconditional political support of the Stalinists and the Prieto Socialists (Prieto, indeed, was unofficially part of the ministry, establishing an office in the government on July 20).

\indexJGiral\indexGRobles\index{Workers' militias}
The Giral government had about \$600,000,000 in gold at its disposal. Recall that the real embargo on the sale of munitions to Spain was not established until August~19, when the British Board of Trade revoked all licences for export of arms and planes to Spain. Thus, the Giral regime had at least a month in which to purchase stores of arms---but the damning fact is that it bought almost nothing! The story of the treacherous attempt of Azaña--Giral to reach a compromise with the fascists has already been told. One further fact: Franco and his friends waited six days before forming their own government. Gil~Robles later revealed that they waited in expectation of a satisfactory arrangement with the Madrid government. By that time the militias had emerged from the ranks of the workers and Giral no longer had the power to meet Franco’s demands.

The most important gains of the first seven weeks were the successful march on Aragon by the Catalan militias, using socialization of the land as much as they used their rifles; and the attack of the loyalist warships on Franco’s transportation of troops from Morocco to the mainland.

Wrote two Stalinists at the time,

\begin{quotation}
  The loyalty of a large part of the navy decisively prevented Franco from transporting large numbers of Moroccan troops to the mainland in the first two weeks of the war. The naval patrol in the south made transport by sea extremely hazardous. Franco was forced to resort to airplane passage, but this was slow work. In this respect, again, the government was given a chance to organize defense and take stock. (\emph{Spain in Revolt,} Gannes and Repard, p.~119).
\end{quotation}

\index{Sailors}\index{Ministries!Navy}
What they failed to add was that the warships were under the command of elected sailors’ committees, which, like the militias, had no faith in the Giral government and carried on operations, despite the passivity of the government. The significance of this fact will become apparent when we come to the naval policy of the Caballero--Stalinist--Prieto cabinet.

The terrible defeats of Badajoz and Irun finished the Giral cabinet. Why Irun fell was told in a moving dispatch by Pierre~van~Paassen:

\index{Battle}\index{Irun}
\begin{quotation}
  They fought to the last cartridge, the men of Irun. When they had no more ammunition, they hurled packets of dynamite. When dynamite was gone, they rushed forward barehanded and tackled each their man, while the sixty times stronger enemy butchered them with bayonets. A girl held two armored cars at bay for half an hour by hurling glycerin bombs. Then the Moroccans stormed the barricade of which she was the last living defender and tore her to pieces. The men of Fort~Martial held three hundred foreign legionnaires at a distance for half a day by rolling rocks down the hill on which the old fort is perched.
\end{quotation}

\index{Central Committee of Anti-Fascist Militias of Catalonia}
Irun fell because the Giral government had made no attempt to provide its defenders with ammunition. The Central Committee of Anti-Fascist Militias of Catalonia, having already transformed the available factories into munitions works, had sent several carloads of ammunition to Irun by way of the regular railroad from Catalonia to Irun. But that railroad runs partway through French territory. And the government of ``Comrade'' Blum, the ally of Stalin, had held up the cars at Behobia, just across the frontier, for days \dots\ they rumbled across the bridge into Irun after the fascists had won.

\indexLCaballero\indexIPrieto\index{Popular Front}
The Giral cabinet gave way to the ``real, complete'' People’s Front government of Caballero--Prieto--Stalin. Undoubtedly, it had the confidence of a large part of the masses. The militias and the sailors’ committees obeyed its orders from the first. There were three major military campaigns that the new government had to undertake. There were, of course, other tasks, but these were the most important, the most pressing, and essentially, the simplest.

\section{Morocco and Algeciras}

\indexFFranco\index{Spanish Legion}\index{Morocco}
Franco’s military base during the first six months was Spanish Morocco. From here he had to bring his Moors and Legionnaires and military stores.

\index{Sailors}\index{Workers' militias}\index{Battle!Naval}\index{Foreign intervention!Italy}
The first successes of the Loyalist navy under the sailors’ committees in harassing Franco’s communication lines with Morocco were followed by others. On August~4, the Loyalist cruiser \emph{Libertad} effectively shelled the fascist fortress at Tarrifa in Morocco. It was a deadly blow at Franco. So deadly that it was answered by the first open Italian act of intervention: an Italian plane proceeded to bomb the \emph{Libertad.} When Loyalist warships steamed into position for a large-scale shelling of Ceuta in Morocco while fascist transports were loading, the German battleship \emph{Deutschland} brazenly steamed back and forth between the Loyalist warships and Ceuta to prevent the bombardment. A week later a Spanish cruiser stopped the German freighter \emph{Kamerun,} found her loaded to the decks with arms for Franco, and prevented her from landing at Cadiz. Whereupon Portugal openly went over to the fascists, permitting the \emph{Kamerun} to unload at a Portuguese port and forwarding the munitions by rail to Franco. German naval commanders received orders to fire on any Spanish vessel attempting to stop German munition shipments. The Loyalist naval operations, if continued, were fatal to Franco, and his allies had to unmask completely to save him.

\indexLCaballero\indexIPrieto\index{Ministries!Navy}
At this point the Caballero cabinet was formed, and Prieto, now in the closest collaboration with the Stalinists, and always ``France’s man,'' became head of the Naval Ministry. He put an end to naval operations off Morocco and the straits of Gibraltar, and recalled the Loyalist forces which had held Majorca.

\index{Spanish Legion}\index{Andalusia!Algeciras}
The task of the hour was to prevent the Moors and Legionnaires from landing at Algeciras and constituting that army which was soon to make that fearful march from Badajoz straight through to Toledo, through Toledo and Talavera~de~la~Reina to the gates of Madrid. The first line in that task belonged to the navy.

\smallskip

\emph{It was not used for this purpose.}

\smallskip

\index{Basque Country!Biscay}\index{Battle!Naval}\index{Battleships}
Instead, in mid-September, almost the whole fleet---including the battleship \emph{Jaime I,} the cruisers \emph{Cervantes} and \emph{Libertad,} and three destroyers---were ordered to leave Malaga and go all the way around the peninsula to the Biscayan coast! They left behind the destroyer \emph{Ferrandiz} and the cruiser \emph{Gravina.} On September~29, two fascist cruisers sank the Ferrandiz, after having shelled and driven away the Gravina. What considerations determined that the naval forces go off to the Biscayan coast, while news dispatches reported---to quote but one instance---an armed trawler conveying Moroccan troops from Ceuta and escorted by the \emph{Canarias,} the \emph{Cervera} and a destroyer and torpedo boat crossed the straits this evening. The convoy landed the troops at Algeciras without hindrance. It transported from Morocco a supply of field and other guns and abundant supplies of ammunition. (\emph{New~York~Times,} September~29).

What considerations? Certainly not military ones, for the forces sent to Biscay were more than sufficient to hold their own with the fascists’ armed convoy; and certainly barring communications from Morocco was the chief task of the navy.

The American military expert, Hanson~W.~Baldwin, writing (in the \emph{New~York~Times} of November~21) on the naval question in Spain said:

\indexNYT\index{Ministries!Navy}
\begin{quotation}
  The Spanish navy has been to a great extent neglected, particularly in the recent troubled years of the republic’s history, and it has never been properly led or manned. But with efficient, well-drilled crews, Spain’s handful of cruisers and destroyers could be a force to be reckoned with, particularly in the narrow basin of the Mediterranean, \emph{where well-handled ships long ago could have cut General Franco’s line of communication to his reservoir of manpower in Africa\dots}
  
  Judging from the somewhat obscure reports, most of the ships---despite the officers’ efforts---continued to fly the red, yellow and mauve of [Loyalist] Spain or else hoisted the Red flags at their gaffs\dots
  
  \dots\ but altogether the role of the navy in the civil war has to date been a minor one. The occasional engagements in which the ships have participated have had in most instances an \emph{op\'era bouffe} quality and have attested to the poor marksmanship and seamanship of the crews.
\end{quotation}

\index{Basque Country}\index{Foreign revolutionaries}
The Loyalist operations of September~27, at Zumaia near Bilbao, however, demonstrated accurate firing. The real point, however, was that it would have been a simple matter to equip the Loyalist warships with able crews. Toulon, Brest, and Marseilles were filled with thousands of socialist and communist sailors, veterans of the navy, including skilled gunners and officers. They could more than have manned the fleet, and other ships that could have been built at the main construction docks, in Cartagena in Loyalist hands.

\index{Andalusia!Algeciras}\index{Murcia!Cartagena}\index{Battle!Naval}
Returning eventually from the northern coast, the fleet was anchored far from the strait, at Cartagena---and there it stayed, except for a few pointless trips down the coast. That it existed at all, one learned on November~22, when foreign submarines entered the port of Cartagena and loosed torpedoes, one damaging the Cervantes. The same day the Ministry of Marine announced reorganization of the fleet to combat attempted blockades---and that was the last heard of that project. Franco’s transports moved at will from Ceuta to Algeciras, bringing tens of thousands of troops and the armaments they required.

\index{Anarchism}
In a letter to Montseny, demanding that the anarchist ministers publicly fight against the false governmental policies, Camillo~Ber\-neri said of the navy:

\indexIPrieto\index{Morocco}\index{Berneri, Camillo}
\begin{quotation}
  The concentration of the forces coming from Morocco, the piracy in the Canaries and the Balearics, the taking of Malaga, are the consequences of this inactivity. If Prieto is incapable and inactive, why tolerate him? If Prieto is bound by a policy which paralyses the fleet, why not denounce this policy?
\end{quotation}

\index{Foreign influence!England and France}\index{Morocco}\index{Counterfactual}\index{Ministries!Navy}
Why did Prieto and the governmental bloc follow this suicidal policy? It was simply one factor of the whole policy which rested on securing the goodwill of England and France. What these were seeking is clear. An aggressive Loyalist naval policy, as the August incidents off Morocco had shown, would have precipitated the decisive stage of the civil war. It would have threatened to crush Franco immediately. Germany and Italy, their prestige involved in supporting Franco, would perhaps have been driven to desperate steps in his defense, such as, the open resort to use of the Italian and German navies in sweeping the Loyalists out of the straits. But England and France could not have tolerated Italo--German control of the straits (it might be retained thenceforward). That open war would thus begin was, of course, not a certainty. Especially prior to November~9, 1936, when Germany and Italy formally recognized the Burgos regime, Germany and Italy might have retreated before precipitating war. Had revolutionists been at the helm and boldly moved in August and September to a systematic naval campaign, and succeeded in cutting off Morocco from Spain, the probability was that Italy and Germany would have retreated as gracefully as they could.

Anglo-French imperialism, however, was not interested in a Loyalist victory but in staving off a war crisis while resisting encroachments on their imperialist interests in the Mediterranean. And they had their way due to the Anglo--French orientation of the Loyalist government. Each month that passed thereafter, involving Germany and Italy more deeply, rendered more and more likely an international explosion if the Loyalist navy were activized. \emph{It simply ceased to exist as a Loyalist weapon.}

Here is the first terrible instance of how anti-revolutionary politics hamstrung the military struggle.

\index{Andalusia!Algeciras}\index{Andalusia!Malaga@Málaga}\index{Battle}\index{Nationalist army}\indexCNT\index{Ministries!Navy}
The same Anglo--French orientation explains the failure to strike by land at Algeciras, the Spanish port at which the fascist forces were landing from Morocco. Malaga was strategically located to be the spearhead for this drive. Instead, Malaga itself was left defenseless. Chiefly defended by \CNT\ forces, who pleaded in vain from August until February for the necessary equipment, Malaga was invaded by an Italian landing force, while the fleet which could have stopped them rode at anchor in Cartagena. Malaga fell on February~8. For two days before that, the militias had received no instructions from the military headquarters and then, the day before Malaga fell, they discovered that the headquarters had already been abandoned without a word to the defending militiamen. It was not a military defeat, but a betrayal. The base treachery was not the last-minute desertion by the general staff but the political policy which dictated the inactivity of the navy and disuse of Malaga as a base against Algeciras.%
\endnote{%
  On February~21, Undersecretary~of~War, José~Asenio, was dismissed, and soon arrested together with Colonel~Villalba, for the betrayal of Malaga. War~Commissar~Bolivar, a Stalinist, who had joined Villalba in abandoning headquarters, was not arrested. Nor was a word breathed---until the National Committee of the \CNT\ got really desperate for the moment---at the Stalinists’ assaults---that Antonio Guerra, Stalinist representative in the Military Command of Malaga, stayed behind and went over to the fascists. (\emph{\CNT\ Boletin Valencia,} August 26, 1937).

  The day Gijon fell---eight months later---the government announced it would try the Malaga traitors---Asensio, his chief of staff, Cabrera, and another general. Why try these and not those guilty at Bilbao, Santander, etc., etc.? Because Malaga fell under Caballero, while the far more brazen betrayals in the north took place under Negrin\dots

  \index{Stalinism}\indexCNT\index{Andalusia!Malaga@Málaga}
}

If not by sea or by land, there was still another way of striking at Franco’s Moroccan base. We quote Camillo~Berneri:

\index{Berneri, Camillo}\index{Morocco!Autonomy}\index{Foreign influence!Arab world}
\begin{quotation}
  The fascist army’s base of operations is in Morocco. We should intensify the propaganda in favor of Moroccan autonomy on every sector of Pan-Islamic influence. Madrid should make unequivocal declarations, announcing the abandonment of Morocco and the protection of Moroccan autonomy. France views with concern the possibility of repercussions and insurrections in North Africa and Syria; England sees the agitation for Egyptian autonomy being reinforced as well as that of the Arabs in Palestine. It is necessary to make the most of such fears by adopting a policy which threatens to unloose revolt in the Islamic world.
  
  For such a policy money is needed and speed to send agitators and organizers to all centres of Arab emigration, to all the frontier zones of~French Morocco. (\emph{Guerra di Classe,} October 24, 1936).
\end{quotation}

\index{Morocco}\index{Foreign influence!England and France}\index{Ministries!Foreign Affairs}
But the Loyalist government, far from arousing French and English fears by inciting insurrection in Spanish Morocco, proceeded to offer them concessions in Morocco! On February~9, 1937, Foreign Minister del~Vayo delivered to France and England a note, the exact text of which was never revealed but was stated later, without denial by the cabinet, to include the following points:

\index{Foreign intervention!Italy}\index{Foreign intervention!Germany}
\begin{enumerate}
  \item Proposing to base its European policy on active collaboration with Great~Britain and France, the Spanish government proposes the modification of the African situation.
  
  \item Desiring a rapid end to the civil war, now susceptible of being prolonged by German and Italian aid, the government is disposed to make certain sacrifices in the Spanish zone of Morocco, if the British and French governments should take steps to prevent Italo--German intervention in Spanish affairs.
\end{enumerate}

The first inkling of the existence of this shameful note came a month after its dispatch, in the French and English press on March~19, when Eden made a passing reference to it. The \CNT\ ministers swore they had not been consulted in its dispatch. Berneri addressed them bitingly:

\index{Berneri, Camillo}\indexCNT\index{Reconciliation}
\begin{quotation}
  You are in a government which has offered France and England advantages in Morocco, while from July, 1936, it should have been obligatory on us to proclaim officially the political autonomy of Morocco.\ \dots
  
  The hour has come to make known that you, Montseny, and the other anarchist ministers are not in agreement with the nature and purport of such proposals.\ \dots
  
  It goes without saying that one cannot guarantee English and French interests in Morocco and at the same time agitate for an insurrection there.\ \dots
  
  But this policy must change. And to change it, a clear and strong statement of our own intentions must be made -- because at Valencia, some influences are at work to make peace with Franco. (\emph{Guerra~di~Classe,} April~14, 1937). 
\end{quotation}

But the anarchist leaders remained silent, and Morocco remained undisturbed in Franco’s power.%
\endnote{%
  The only official \CNT\ pamphlet dealing with Morocco that I have been able to find is: ``What Spain Could Have Done in Morocco and What It Has Done,'' a speech by Gonzalo~de~Reparaz on January~17, 1936, telling how he tried to get the monarchy and the republic to organize things efficiently in Morocco, and how they did not! Not a hint that the only advice a revolutionist can give on the colonial question is: get out of Morocco.
}

\section{The Aragon Offensive against Saragossa and Huesca}

\index{Aragon!Aragon Offensive}\index{Aragon!Huesca}\index{Aragon!Zaragoza}\index{Battle}
Thumb through the Spanish or the French or American press of August--November~1936 and note the sharp contrast between the Loyalist defeats on the central and western fronts and the victories on the Aragon front. The \CNT, \FAI, and \POUM\ troops predominated in Aragon. They obeyed the military orders of the bourgeois officers sent by the government but kept them under surveillance. By the end of October, having captured the surrounding heights of Montearagón and Estrecho~Quinto, the Aragon militias were in position to take Huesca, the gateway to Saragossa.