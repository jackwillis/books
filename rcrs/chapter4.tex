\chapter{Toward a Coalition with the Bourgeoisie}

\lettrineI{n every other} period of dual power\index{Dual power}---Russia\index{Russian Revolution} of February to November 1917 and Germany of 1918 to 1919\index{German Revolution (1918--19)} being the most im\-por\-tant---the bourgeois government continued to exist, thanks only to the entry into it by representatives of the reformist workers’ organizations, who thereby became the main prop for the bourgeoisie. The Mensheviks\index{Mensheviks} and Social Revolutionaries\index{Social Revolutionaries} not only defended the Provisional Government within the Soviets but also sat with the bourgeois ministers in the government. Ebert and Scheidemann wielded a majority in the Soldiers’ and Workers’ Councils but simultaneously sat in the government. In Spain, however, for seven critical weeks, no workers’ representative entered the cabinet.

Not that the bourgeoisie did not want them there, nor that workers’ leaders were not available and willing! On the evening of July~19, when full confirmation of the workers’ conquest of Barcelona arrived, Azaña finally abandoned the attempt to form a ``peace cabinet''\index{Peace cabinet} under Barrio. Giral became premier. Azaña and Giral asked Prieto and Caballero to enter the cabinet. Prieto was more than willing. Caballero refused Giral’s proposal, and Prieto dared not enter without him.

In Catalonia, during the last days of July, Companys took three Stalinist leaders into his cabinet. But in three days they were forced to resign, at the demand of the anarchists, who denounced their entry as disruption of the leading role of the Central Committee of~Militias.

Thus, for seven weeks, the bourgeois governments remained isolated from the masses, unprotected by reformist ministers. Nor did the conduct of the republicans enhance their prestige. The more cowardly functionaries fled to Paris. The \textsc{cnt} \emph{Solidaridad Obrera} published day after day a ``Gallery of Illustrious Men,'' of republicans who had fled. The government had in its possession one of the largest gold reserves outside those of the big imperialist powers---over six hundred millions in dollars---yet made no effort during those first two months to purchase arms abroad. It praised France’s attempts to organize ``non-intervention.'' It cried out against the workers’ seizure of the factories and organization of war production. It denounced the district committees and worker-patrols which were cleansing the rear-guard of reactionaries.

The Catalan bourgeois regime, led by the astute Companys{\indexLCompanys} (who had once been a lawyer for the \textsc{cnt} and had a keen knowledge of the workers’ movement), riding a revolutionary upsurge far more intense than that in Madrid, behaved much more cleverly than Azaña--Giral. In the first red weeks it sanctioned without question all steps taken by the workers. But it was even more isolated in Barcelona than was the Madrid cabinet.

The Madrid and Barcelona governments lacked the indispensable instrument of sovereignty: armed force. The regular army was with Franco. The regular police\index{Police} no longer had any real independent existence, having been swallowed up in the flood of armed workers. Though itself denuded of its police, most of whom had either volunteered or been sent, under workers’ pressure, to the front, the Madrid bourgeoisie had looked askance at the official status conceded to the worker leadership of the militias by the Catalan government. The discreet explanation offered by the Esquerra leader,{\indexEsquerra} Jaume Miravittles, tells volumes:

\begin{quotation}
  The Central Committee of Militias was born two or three days after the [subversive] movement, in the absence of any regular public force and when there was no army in Barcelona. For another thing, there were no longer any Civil or Assault Guards. For all of them had fought so arduously, united with the forces of the people, that now they formed part of the same mass and had remained mixed up with that. In these circumstances, weeks went by without it being possible to reunite and regroup the dispersed forces of the Assault and Civil Guards. (\emph{Heraldo de Madrid}, September 4, 1936.)
\end{quotation}

Yet, the fact is, despite the rise of dual power, despite the scope of the power of the proletariat in the militias and their control of economic life, the workers’ state remained embryonic, atomized, scattered in the various militias and factory committees and local anti-fascist defense committees jointly constituted by the various organizations. It never became centralized in nationwide Soldiers’ and Workers’ Councils, as it had been in Russia in 1917, in Germany in 1918–19. Only when dual power assumes such organizational proportions is there put on the order of the day the choice between the prevailing regime and a new revolutionary order of which the Councils become the state form. The Spanish revolution never rose to this point despite the fact that the real power of the proletariat was far greater than the power wielded by the workers in the German revolution or, indeed, than that wielded by the Russian workers before November. Locally and in each militia column, the workers ruled; but at the top there was only the government! This paradox has a simple explanation: there was no revolutionary party in Spain, ready to drive through the organization of soviets boldly and single-mindedly.

But isn’t it a far cry from the failure to create the organs to overthrow the bourgeoisie, to the acceptance of the role of class collaboration with the bourgeoisie? Not at all. In a revolutionary period the alternatives are poised on a razor-edge: either one or the other. Every day is as a decade in peacetime.\index{Socialism or barbarism} Today’s ``realism'' becomes tomorrow’s avenue to collaboration with the bourgeoisie. Civil war is raging. The liberal bourgeoisie offers to cooperate in fighting the fascists. It is obvious that the workers should accept their aid. What are the limits of such cooperation? The ``sectarian'' Bolsheviks, in the struggle against Kornilov, set exceedingly sharp limits. Above all, they gathered power in the hands of the soviets.
\nowidow

In the very heat of the struggle against the Kornilov\index{Kornilov, Lavr} counterrevolution in September 1917, when Kerensky and the other bourgeois ministers in the coalition government were certainly shouting for smashing Kornilov, as much as Azaña and Companys were declaiming against Franco, the Bolsheviks warned the workers that the Provisional Government was impotent and that only the soviets could defeat Kornilov. In a special letter to the Central Committee of the Bolsheviks, Lenin\index{Lenin, Vladimir} castigated those who uttered ``phrases about the defense of the country, about supporting the Provisional~Government.''
``We will fight, we are fighting against Kornilov, even as Kerensky’s troops do, but we do not support Kerensky,'' said Lenin. ``On the contrary, we expose his weakness. There is the difference. It is rather a subtle difference, but it is highly essential and one must not forget~it.''

And there was not the slightest thought of waiting until the struggle against Kornilov was over before taking state power. On the contrary, declared Lenin, ``even tomorrow events may put power into our hands, and then we shall not relinquish it.''\endnote{V.~I.~Lenin, \emph{Works}, Vol. XXI, Book I, p. 137}
Lenin was ready to collaborate with Kerensky himself in a military-technical union. But with this pre-condition, already existing: the masses organized in class organs, democratically elected, where the Bolsheviks could contend for a majority.

Without developing soviets---workers’ councils---it was inevitable that even the anarchists and the \textsc{poum} would drift into governmental collaboration with the bourgeoisie. For what does it mean, in practice, to refuse to build soviets in the midst of civil war?\index{Dual power} It means to recognize the right of the liberal bourgeoisie to \emph{govern} the struggle, i.e., to dictate its social and political limits.

Thus it was that all the workers’ organizations, without exception, drifted closer and closer to the liberal bourgeoisie. In the intervening weeks, Azaña and Companys recovered their nerve, as they saw that the inroads of the workers were not to be consummated by the overturn of the state power. Azaña gathered together every officer who, caught behind the lines, proclaimed himself for the republic. At first the officers could deal with the militias only through the militia committees. But the Bolshevik method of using the technical knowledge of the officers without giving them power over the soldiers, can be employed only during the height of the transition from dual power to a workers’ state, or by a soviet regime. Little by little the officers pushed their way to direct command.

The government’s control of the treasury\index{Ministries!Treasury} and of the banks---for the workers, including the anarchists, had stopped short at the banks, merely instituting a form of workers’ control which was little more than guarding against disbursements to fascists and encouraging capital loans\index{Finance capital} to collectivized factories---gave it a powerful lever in encouraging the considerable number of foreign-owned enterprises (which had not been seized), in placing governmental representatives in the factories, in intervening in foreign trade, in providing room for quick growth to small factories and shops and traders that had been spared from collectivization. Madrid, controlling the gold\index{Gold} reserves, used them as an unansswerable argument in Catalonia in instances where Companys proved powerless.

Under contemporary capitalism, finance capital dominates manufacturing and transportation. This law of economics was not abrogated because the workers had seized the factories and railroads. All that the workers had done in seizing these enterprises was to transform them into producers’ cooperatives\index{Cooperatives}, still subject to the laws of capitalist economics. Before they could be freed from these laws, all industry and land, together with bank capital and gold and silver reserves, would have to become the property of a workers’ state. But this required overthrowing the bourgeois state.

The manipulation of finance capital to curb the workers’ movement is a phase of the Spanish struggle which will deserve the most careful and detailed study, and undoubtedly will provide new insights into the nature of the bourgeois state. This weapon was openly unleashed in its full force much later, but even in the first seven weeks its guarded use enabled the regime to recover much lost ground.
\nowidow

In the very first weeks, the government, feeling its way, returned to the use of one of the instruments of state power most hated by the workers, the press censorship. It was hated particularly because of the government’s use of it during the last days before the fascist rebellion, when socialist and anarchist warnings against the imminent civil war were deleted. Azaña hastened to assure the press that the censorship would be limited to military news; but this was merely a bridge to general censorship. The unreserved supporters of the Popular Front, the Stalinists and Prieto Socialists, agreed without a murmur. An objectionable feature in the Stalinist \emph{Mundo Obrero}{\indexMundoObrero} of August~20 led to suppression of the issue. Caballero’s \emph{Claridad}{\indexClaridad} grumblingly acceded. The anarchists and the \textsc{poum} followed. Only the Madrid organ of the Anarchist~Youth refused entry to the censor. But censorship was not a separate problem: it would inevitably be the prerogative of the state power.

In August, the \textsc{cnt}{\indexCNT} entered the Basque Defense Junta,\index{Basque Defense Junta} which was not a military organization at all, but a regional government in which the Basque big bourgeois party held the posts of finance and industry. This, the first time in history that anarchists participated in a government, was reported by the anarchist press without explanation. A great opportunity was presented to the \textsc{poum} to win the \textsc{cnt} workers to struggle for a workers’ state, but the \textsc{poum} made no issue of the Basque government---for the \textsc{poum} acted identically in~Valencia.

The Popular Executive\index{Popular Executive}, with bourgeois participation, was constituted in Valencia as a regional government, and here the \textsc{poum} entered too. In those days the \textsc{poum}’s central organ, \emph{La Batalla}{\indexLaBatalla}, was calling for an all-workers’ government in Madrid and Barcelona: the contradiction between this slogan and the Valencia step was passed by without comment.

Formed within two days of the uprising as a military centre, the Central Committee of the Catalan Militias\index{Central Committee of the Catalan Militias} began to undertake collaboration with the bourgeoisie in economic activities as well. Transformation of the Central Committee into a body of democratically elected delegates from the factories and militia columns would have given it more power and authority and, at the same time, would have reduced the role of the bourgeoisie to its actual strength in the militias and factories. This was the only way out of the dilemma. But the \textsc{cnt} was blind to the problem, and the \textsc{poum} kept silent.

Finally, on August 11, the Council of Economy\index{Council of Economy} was formed on the initiative of Companys to centralize economic activity. Here it was, despite the bait of a radical economic program, an undisguised question of socio-economic collaboration under the hegemony of the bourgeoisie. But the \textsc{cnt} and \textsc{poum} entered it.

Thus, in every sphere, the bourgeoisie edged its way back. Thus, the workers were carried, step by step, toward governmental coalition with the bourgeoisie.

To understand this process clearly, we must now examine more closely the political conceptions of the workers’ organizations.