\documentclass[
  12pt,
  pagesize,
  paper = 6in:9in,
  DIV = 12,
  openany
]{scrbook}

\usepackage[T1]{fontenc}
\usepackage[utf8]{inputenc}
\usepackage[english]{babel}
\usepackage[osf]{Alegreya}
\usepackage[osf]{AlegreyaSans}
\usepackage[tracking = true, babel = true]{microtype}

\title{The Spanish Revolution \\ 1931--37}
\author{Ted Grant \and Peter Taaffe}
\date{}

% Acronym
\newcommand{\ac}[1]{\textsc{\lowercase{#1}}}

\begin{document}

\maketitle

\tableofcontents

\frontmatter

\chapter{Preface}

Events of the last eight years have served, to further confirm the
analysis of this pamphlet which was originally published in 1977.
The `Communist' Party has paid the full price for its reformist programme,
being riven by splits and reduced from 15\% to only 4\% of
the vote in the 1982 elections. The \ac{PSOE} (Socialist Party) leaders
have degenerated beyond the pre-war ‘popular frontists'. Elected to
government in 1982 with a promise to create 800,000 new jobs, they
are now unloading the full weight of the crisis onto the backs of the
working class, presiding over 15\% unemployment, and now introducing
in the face of massive opposition from the workers, a plan to
restructure heavy industry with the loss of 200,000 jobs.
All the lessons of the 1930s retain their full force today, especially
the impossibility, whether by the Popular Front of the 1930s or a
Socialist majority government today, of solving the fundamental problems
in Spain---unemployment, near starvation among landless
labourers, the National Question etc.,---except on the basis of the immediate
socialist transformation in Spain and internationally.

\begin{flushright}
  May 1985
\end{flushright}

\mainmatter

\chapter{Introduction \\ \textsc{Peter Taaffe,} 1977}

\emph{The Spanish Revolution 1931--37} by Ted Grant was first published
as a lengthy article in the \emph{Militant International Review,} No.7
(Autumn 1973). We are republishing it now as a separate pamphlet
for a number of reasons. Firstly, July is the 41st anniversary of Franco’s
fascist uprising and the revolutionary answer it received from
the immortal Spanish working class. Secondly, the Spanish working
class stands on the eve of the revolutionary storms which will
put in the shade even the magnificent movement of their grandfathers
and fathers between 1931--37. At the same time the leaders
of the Spanish Communist Party---and their apologists in Britain---are
attempting to cover up their disastrous role in 1936. Their policies
of Popular Frontism paved the way for Franco’s victory and 40 years
of fascist slavery for the Spanish working class. Moreover, these
leaders, together with the leaders of the \ac{PSOE,} the Socialist Party,
are putting forward a new and worse version of the Popular Front,
which if followed, will result in a catastrophe for the Spanish workers
in the future.

Therefore, this study is particularly relevant for the new generation
of Spanish workers, who because of illegal conditions, have not
had ready access to material which accurately recorded the perfidious
role of the CP leaders during the Spanish revolution. Even now, with
the loosening of the grip of the dictatorship and the legalisation of
the Socialist and Communist Parties, this material is not being made
available. On the contrary a monstrous re-writing of history---casting
the Spanish CP leaders as palladins of the revolution---is underway.
Outstanding amongst these are two recent works, a book by Santiago
Carrillo, the present leader of the Spanish CP, which takes the
form of a lengthy interview between himself, Regis Debray and Max
Gallo and is called \emph{Dialogue on Spain.} The other is an article by Monty
Johnstone of the British CP, in the Young Communist Journal
\emph{Cogito,} in which a section is devoted to refuting Trotsky’s writings
on Spain between 1931--39.

All the arguments which Monty Johnstone deploys against Trotsky
are answered in advance, in this article by Ted Grant. We would
advise the reader to go over Johnstone’s article and then read this
pamphlet. Johnstone seeks to justify the ‘progressive’ role of the
Popular Front in the Spanish revolution as a means of ‘winning the
middle class’ He asserts that the bloc between the bourgeois
republicans in Spain and the workers’ parties was merely an extension
of the ‘united front’ between the workers’ parties themselves.
He even drags in by the hair a quote from Lenin to justify this position
\dots\ “to renounce in advance \dots\ any conciliation or compromise
with possible allies (even if they are only temporary, unstable, or
vacillating or conditional allies) is this not ridiculous in the extreme.''
He couples this with a denunciation of Trotsky for his statement that
the Popular Front was “lulling the workers and peasants with
Parliamentary illusions \dots\ [and] \dots\ paralysing their will to struggle.''
After all this bloc with the republican ‘allies’ was justified by the
fact that “In February the Popular Front \dots\ had beaten the right
and centre parties'' asserts Monty. Ted Grant shows that the
masses---highly suspicious of the bourgeois republicans---voted for
the Popular Front, not because of the inclusion of the republicans
in the Front, but despite this. This was shown by the fact that the
masses refused to wait for the government to legislate but acted
within hours to put their demands into practice. They tore open the
jails and released the workers who had been imprisoned after the
crushing of the Asturian Commune in October 1934. The Popular
Front government got round to legislating their release in September
1936, two months after Franco’s revolt!

Moreover the above remarks of Trotsky were made not before
February 1936, but in July of that year on the eve of the fascist uprising.
What was the Republican government doing if not lulling the
workers to sleep? Not only did they refuse to arm the workers, but
threatened to shoot anyone who attempted to do so! While the fascist
generals plotted their revolt the Popular Front government condemned
as slanderers anyone who dared suggest that this was what was
actually taking place. Thus the Republican Premier, Casares Quiroga
on the 14th July---just three days before the fascist revolt---when
challenged that Mola, chief conspirator with Franco, was planning
an uprising declared: “Mola is a general loyal to the Republic”! Martinez
Barrio, who replaced Quiroga on the 19th July, on assuming
office phoned Mola, then in the thick of the fascist uprising and the
murderer of thousands of workers, and offered him and Franco the
Ministry of the Interior and Defence; the two positions which Hitler
received and enabled him to consolidate power ‘peacefully’.

Thus if these “allies of the workers” had succeeded Franco would
have taken power three years earlier than he actually did! The
conspiracy failed because the fascist generals politely refused the
offer---not believing that the Republicans would be able to sell it to the working
class. Mola regretfully told Barrio: “If you and I should reach
agreement, both of us will have betrayed our ideals and our men.”
In reality the bourgeois republicans were entirely loyal to their
ideals---the defence of private property. They were prepared to deliver
the workers, bound hand and foot into the arms of Franco and Mola.
Are these the actions of ‘allies’---even the unstable and vacillating
variety---Monty? If the bourgeois republicans in Spain were not
outright enemies of the Spanish working class and of the social
revolution then what will enemies look like?

Moreover, Lenin never at any time justified a programmatic bloc,
with the leaders of middle class parties as a means of winning the
little men of town and country to the side of the working class. On
the contrary, the history of Bolshevism is a history of a war against
such notions, not just in Russia either, as Monty Johnstone suggests.
When Millerand, the French Socialist Party leader, formed a bloc
with the leaders of the Radical Socialist Party at the turn of the century,
he was condemned by Lenin. The Radical Socialist Party was
characterised by Lenin as “the most vicious and consummate
representatives of finance capital, the political exploiter of the
peasants and middle class”. The way to win the middle class, said
Lenin, was not in a coalition with these “political exploiters” but
by unmasking them before their followers and demonstrating in action,
that only the working class was capable of solving their problems.
In Russia in 1917 this policy---implacably opposed to the Menshevik
and Social Revolutionary versions of the Popular Frontsucceeded
in winning the peasantry to the side of the working class.
In Spain in 1936 the `strike breaking conspiracy' of the Popular Front
succeeded only in pushing the peasantry and the middle class into
indifference and opposition.

\end{document}
